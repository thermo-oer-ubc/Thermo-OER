%% Generated by Sphinx.
\def\sphinxdocclass{jupyterBook}
\documentclass[letterpaper,10pt,english]{jupyterBook}
\ifdefined\pdfpxdimen
   \let\sphinxpxdimen\pdfpxdimen\else\newdimen\sphinxpxdimen
\fi \sphinxpxdimen=.75bp\relax
\ifdefined\pdfimageresolution
    \pdfimageresolution= \numexpr \dimexpr1in\relax/\sphinxpxdimen\relax
\fi
%% let collapsible pdf bookmarks panel have high depth per default
\PassOptionsToPackage{bookmarksdepth=5}{hyperref}
%% turn off hyperref patch of \index as sphinx.xdy xindy module takes care of
%% suitable \hyperpage mark-up, working around hyperref-xindy incompatibility
\PassOptionsToPackage{hyperindex=false}{hyperref}
%% memoir class requires extra handling
\makeatletter\@ifclassloaded{memoir}
{\ifdefined\memhyperindexfalse\memhyperindexfalse\fi}{}\makeatother

\PassOptionsToPackage{warn}{textcomp}

\catcode`^^^^00a0\active\protected\def^^^^00a0{\leavevmode\nobreak\ }
\usepackage{cmap}
\usepackage{fontspec}
\defaultfontfeatures[\rmfamily,\sffamily,\ttfamily]{}
\usepackage{amsmath,amssymb,amstext}
\usepackage{polyglossia}
\setmainlanguage{english}



\setmainfont{FreeSerif}[
  Extension      = .otf,
  UprightFont    = *,
  ItalicFont     = *Italic,
  BoldFont       = *Bold,
  BoldItalicFont = *BoldItalic
]
\setsansfont{FreeSans}[
  Extension      = .otf,
  UprightFont    = *,
  ItalicFont     = *Oblique,
  BoldFont       = *Bold,
  BoldItalicFont = *BoldOblique,
]
\setmonofont{FreeMono}[
  Extension      = .otf,
  UprightFont    = *,
  ItalicFont     = *Oblique,
  BoldFont       = *Bold,
  BoldItalicFont = *BoldOblique,
]



\usepackage[Bjarne]{fncychap}
\usepackage[,numfigreset=1,mathnumfig]{sphinx}

\fvset{fontsize=\small}
\usepackage{geometry}


% Include hyperref last.
\usepackage{hyperref}
% Fix anchor placement for figures with captions.
\usepackage{hypcap}% it must be loaded after hyperref.
% Set up styles of URL: it should be placed after hyperref.
\urlstyle{same}

\addto\captionsenglish{\renewcommand{\contentsname}{Acknowledgements}}

\usepackage{sphinxmessages}



        % Start of preamble defined in sphinx-jupyterbook-latex %
         \usepackage[Latin,Greek]{ucharclasses}
        \usepackage{unicode-math}
        % fixing title of the toc
        \addto\captionsenglish{\renewcommand{\contentsname}{Contents}}
        \hypersetup{
            pdfencoding=auto,
            psdextra
        }
        % End of preamble defined in sphinx-jupyterbook-latex %
        

\title{OER Engineering Thermodynamics}
\date{Dec 14, 2023}
\release{}
\author{Faculty of Applied Science, UBC licensed under <a href="https://creativecommons.org/licenses/by\sphinxhyphen{}sa/4.0/">{[}CC\sphinxhyphen{}BY\sphinxhyphen{}SA 4.0{]}</a>}
\newcommand{\sphinxlogo}{\vbox{}}
\renewcommand{\releasename}{}
\makeindex
\begin{document}

\pagestyle{empty}
\sphinxmaketitle
\pagestyle{plain}
\sphinxtableofcontents
\pagestyle{normal}
\phantomsection\label{\detokenize{intro::doc}}


\sphinxAtStartPar
This is a project that extends the original e\sphinxhyphen{}textbook written by Dr. Claire Yu Yan in UBC Okanagan campus:\sphinxhref{https://pressbooks.bccampus.ca/thermo1}{Introduction to Engineering Thermodynamics}. The main goals for this project are as follows:

\sphinxAtStartPar
(i) To provide students with a list of questions that can be done in their laptops with the help of some Python packages such as CoolProp, numpy and matplotlib.

\sphinxAtStartPar
(ii) To reinforce the concepts learnt from e\sphinxhyphen{}textbook and to get a better appreciation of how to use digital tools to do perform thermodynamic calculations.

\sphinxAtStartPar
(iii) To provide a starting point for Thermodynamics courses in other disciplines to adopt a more computational approach for their courses.
\begin{itemize}
\item {} 
\sphinxAtStartPar
Acknowledgements

\begin{itemize}
\item {} 
\sphinxAtStartPar
{\hyperref[\detokenize{notebooks/ack/ack::doc}]{\sphinxcrossref{Acknowledgements}}}

\end{itemize}
\end{itemize}
\begin{itemize}
\item {} 
\sphinxAtStartPar
Getting Started

\begin{itemize}
\item {} 
\sphinxAtStartPar
{\hyperref[\detokenize{notebooks/getting_started/getting_started::doc}]{\sphinxcrossref{Getting Started: Introduction to Jupyter and Python!}}}

\item {} 
\sphinxAtStartPar
{\hyperref[\detokenize{notebooks/getting_started/basic_math::doc}]{\sphinxcrossref{Basic operations inside Jupyter}}}

\item {} 
\sphinxAtStartPar
{\hyperref[\detokenize{notebooks/getting_started/Table_G1_Properties_of_ideal_gases::doc}]{\sphinxcrossref{Defining variables: Properties of Ideal Gases as use case}}}

\end{itemize}
\end{itemize}
\begin{itemize}
\item {} 
\sphinxAtStartPar
Modules

\begin{itemize}
\item {} 
\sphinxAtStartPar
{\hyperref[\detokenize{notebooks/Chapter1/chapter1::doc}]{\sphinxcrossref{1. Basic Concepts and Definitions}}}

\item {} 
\sphinxAtStartPar
{\hyperref[\detokenize{notebooks/Chapter2/thermo-properties::doc}]{\sphinxcrossref{2. Thermodynamic Properties of a Pure Substance}}}

\item {} 
\sphinxAtStartPar
{\hyperref[\detokenize{notebooks/Chapter3/chapter3::doc}]{\sphinxcrossref{3. Ideal and Real Gases}}}

\item {} 
\sphinxAtStartPar
{\hyperref[\detokenize{notebooks/Chapter4/first-law::doc}]{\sphinxcrossref{4. The First Law of Thermodynamics for Closed Systems}}}

\item {} 
\sphinxAtStartPar
{\hyperref[\detokenize{notebooks/Chapter5/chapter5::doc}]{\sphinxcrossref{5. The First Law of Thermodynamics for a Control Volume}}}

\item {} 
\sphinxAtStartPar
{\hyperref[\detokenize{notebooks/Chapter6/chapter6::doc}]{\sphinxcrossref{6. Entropy and the Second Law of Thermodynamics}}}

\end{itemize}
\end{itemize}

\sphinxstepscope


\part{Acknowledgements}

\sphinxstepscope


\chapter{Acknowledgements}
\label{\detokenize{notebooks/ack/ack:acknowledgements}}\label{\detokenize{notebooks/ack/ack::doc}}

\section{Funding acknowledgements}
\label{\detokenize{notebooks/ack/ack:funding-acknowledgements}}
\sphinxAtStartPar
This project was made possible by 2023/24 \sphinxhref{https://oerfund.open.ubc.ca/oer-affordability-grants/}{OER Affordability Grant} (formerly known as OER Implementation Grant) at UBC Vancouver.


\section{Contributions}
\label{\detokenize{notebooks/ack/ack:contributions}}
\sphinxAtStartPar
Content for various chapers was created by the following students:

\sphinxAtStartPar
\sphinxhref{https://ca.linkedin.com/in/ali-doustahadi-652251169}{Ali Doustahadi}, MASc candidate in the Department of Materials Engineering, UBC Vancouver, created material for Chapter 3 and Chapter 5.

\sphinxAtStartPar
\sphinxhref{https://ca.linkedin.com/in/drcswang}{Dr.Chun\sphinxhyphen{}Sheng Wang}, Postdoctoral research fellow in the Department of Mechanical Engineering, UBC Okanagan, created material for Chapter 1 and 6.

\sphinxAtStartPar
\sphinxhref{https://www.linkedin.com/in/hariharan-hari-u-a25788254/}{Hariharan Umashankar}, PhD candidate in the Department of Materials Engineering, UBC Vancouver, created material for Chapter 2 and Chapter 4 and hosted this website on Github pages.

\sphinxAtStartPar
The following faculty members were responsible for providing feedback along the course of this project and acquiring the project funds:

\sphinxAtStartPar
\sphinxhref{https://mtrl.ubc.ca/casey-keulen/}{Dr. Casey Keulen}, Assistant Professor of Teaching, Department of Materials Engineering, UBC Vancouver.

\sphinxAtStartPar
\sphinxhref{https://mtrl.ubc.ca/amir-m-dehkhoda/}{Dr. Amir M. Dehkhoda}, Assistant Professor of Teaching, Department of Materials Engineering, UBC Vancouver.

\sphinxAtStartPar
\sphinxhref{https://apsc.ubc.ca/profile/yu-claire-yan}{Dr. Claire (Yu) Yan}, Associate Professor of Teaching, Department of Mechanical Engineering, UBC Okanagan.

\sphinxstepscope


\part{Getting Started}

\sphinxstepscope


\chapter{Getting Started: Introduction to Jupyter and Python!}
\label{\detokenize{notebooks/getting_started/getting_started:getting-started-introduction-to-jupyter-and-python}}\label{\detokenize{notebooks/getting_started/getting_started::doc}}
\sphinxAtStartPar
Hey! Welcome to learning about Jupyter notebooks and using it for doing basic thermodynamic calculations.

\sphinxAtStartPar
In this section, we will use the Jupyter notebook interface to show how to add, subtract, multiply and divide numbers. Also, an introduction about python packages will be provided in here.

\sphinxstepscope


\chapter{Basic operations inside Jupyter}
\label{\detokenize{notebooks/getting_started/basic_math:basic-operations-inside-jupyter}}\label{\detokenize{notebooks/getting_started/basic_math::doc}}
\sphinxAtStartPar
Adding a cell above the current one: ‘a’

\sphinxAtStartPar
Adding a cell below the current one: ‘b’

\sphinxAtStartPar
Delete cell entirely: ‘d,d’ (press d twice)

\sphinxAtStartPar
To execute/run a cell: Ctrl + Enter

\sphinxAtStartPar
To execute/run a cell and move to the next one: Shift + enter

\sphinxAtStartPar
Each cell has a type associated with it: “Code”, “Markdown”, “Raw”


\section{Import libraries}
\label{\detokenize{notebooks/getting_started/basic_math:import-libraries}}
\begin{sphinxuseclass}{cell}\begin{sphinxVerbatimInput}

\begin{sphinxuseclass}{cell_input}
\begin{sphinxVerbatim}[commandchars=\\\{\}]
\PYG{k+kn}{import} \PYG{n+nn}{numpy} \PYG{k}{as} \PYG{n+nn}{np}
\PYG{k+kn}{import} \PYG{n+nn}{matplotlib}\PYG{n+nn}{.}\PYG{n+nn}{pyplot} \PYG{k}{as} \PYG{n+nn}{plt}
\end{sphinxVerbatim}

\end{sphinxuseclass}\end{sphinxVerbatimInput}

\end{sphinxuseclass}

\section{Basic math}
\label{\detokenize{notebooks/getting_started/basic_math:basic-math}}
\begin{sphinxuseclass}{cell}\begin{sphinxVerbatimInput}

\begin{sphinxuseclass}{cell_input}
\begin{sphinxVerbatim}[commandchars=\\\{\}]
\PYG{n}{b} \PYG{o}{=} \PYG{l+m+mi}{1} \PYG{o}{+} \PYG{l+m+mi}{1}
\PYG{n+nb}{print}\PYG{p}{(}\PYG{n}{b}\PYG{p}{)}
\end{sphinxVerbatim}

\end{sphinxuseclass}\end{sphinxVerbatimInput}
\begin{sphinxVerbatimOutput}

\begin{sphinxuseclass}{cell_output}
\begin{sphinxVerbatim}[commandchars=\\\{\}]
2
\end{sphinxVerbatim}

\end{sphinxuseclass}\end{sphinxVerbatimOutput}

\end{sphinxuseclass}
\begin{sphinxuseclass}{cell}\begin{sphinxVerbatimInput}

\begin{sphinxuseclass}{cell_input}
\begin{sphinxVerbatim}[commandchars=\\\{\}]
\PYG{n}{c} \PYG{o}{=} \PYG{l+m+mi}{4} \PYG{o}{*} \PYG{l+m+mi}{4} \PYG{c+c1}{\PYGZsh{} multiply}
\PYG{n+nb}{print}\PYG{p}{(}\PYG{n}{c}\PYG{p}{)}
\PYG{n}{c} \PYG{o}{=} \PYG{l+m+mi}{4} \PYG{o}{*}\PYG{o}{*}\PYG{l+m+mi}{2} \PYG{c+c1}{\PYGZsh{} exponentiate}
\PYG{n+nb}{print}\PYG{p}{(}\PYG{n}{c}\PYG{p}{)}
\end{sphinxVerbatim}

\end{sphinxuseclass}\end{sphinxVerbatimInput}
\begin{sphinxVerbatimOutput}

\begin{sphinxuseclass}{cell_output}
\begin{sphinxVerbatim}[commandchars=\\\{\}]
16
16
\end{sphinxVerbatim}

\end{sphinxuseclass}\end{sphinxVerbatimOutput}

\end{sphinxuseclass}

\section{Plotting using \sphinxstyleliteralintitle{\sphinxupquote{matplotlib}} package}
\label{\detokenize{notebooks/getting_started/basic_math:plotting-using-matplotlib-package}}
\begin{sphinxuseclass}{cell}\begin{sphinxVerbatimInput}

\begin{sphinxuseclass}{cell_input}
\begin{sphinxVerbatim}[commandchars=\\\{\}]
\PYG{n}{plt}\PYG{o}{.}\PYG{n}{plot}\PYG{p}{(}\PYG{p}{[}\PYG{l+m+mi}{1}\PYG{p}{,}\PYG{l+m+mi}{1}\PYG{p}{]}\PYG{p}{,}\PYG{p}{[}\PYG{l+m+mi}{1}\PYG{p}{,}\PYG{l+m+mi}{2}\PYG{p}{]}\PYG{p}{,}\PYG{l+s+s1}{\PYGZsq{}}\PYG{l+s+s1}{\PYGZhy{}*b}\PYG{l+s+s1}{\PYGZsq{}}\PYG{p}{)} \PYG{c+c1}{\PYGZsh{}\PYGZsh{} x array, y array and line or marker specifications}
\PYG{n}{plt}\PYG{o}{.}\PYG{n}{xlabel}\PYG{p}{(}\PYG{l+s+s1}{\PYGZsq{}}\PYG{l+s+s1}{X values}\PYG{l+s+s1}{\PYGZsq{}}\PYG{p}{)}
\PYG{n}{plt}\PYG{o}{.}\PYG{n}{ylabel}\PYG{p}{(}\PYG{l+s+s1}{\PYGZsq{}}\PYG{l+s+s1}{Y values}\PYG{l+s+s1}{\PYGZsq{}}\PYG{p}{)}
\end{sphinxVerbatim}

\end{sphinxuseclass}\end{sphinxVerbatimInput}
\begin{sphinxVerbatimOutput}

\begin{sphinxuseclass}{cell_output}
\begin{sphinxVerbatim}[commandchars=\\\{\}]
Text(0, 0.5, \PYGZsq{}Y values\PYGZsq{})
\end{sphinxVerbatim}

\noindent\sphinxincludegraphics{{d4ca1d9555daa33a4bf01dc1887639d8261fc8ad172a196ab1955df7461d4613}.png}

\end{sphinxuseclass}\end{sphinxVerbatimOutput}

\end{sphinxuseclass}

\section{Numerical operation using \sphinxstyleliteralintitle{\sphinxupquote{numpy}} package}
\label{\detokenize{notebooks/getting_started/basic_math:numerical-operation-using-numpy-package}}
\begin{sphinxuseclass}{cell}\begin{sphinxVerbatimInput}

\begin{sphinxuseclass}{cell_input}
\begin{sphinxVerbatim}[commandchars=\\\{\}]
\PYG{n}{a} \PYG{o}{=} \PYG{n}{np}\PYG{o}{.}\PYG{n}{linspace}\PYG{p}{(}\PYG{l+m+mi}{10}\PYG{p}{,}\PYG{l+m+mi}{50}\PYG{p}{,}\PYG{n}{num}\PYG{o}{=}\PYG{l+m+mi}{50}\PYG{p}{)} \PYG{c+c1}{\PYGZsh{}\PYGZsh{} create a regular\PYGZhy{}spaced array from start to end and number of datapoints}
\end{sphinxVerbatim}

\end{sphinxuseclass}\end{sphinxVerbatimInput}

\end{sphinxuseclass}
\begin{sphinxuseclass}{cell}\begin{sphinxVerbatimInput}

\begin{sphinxuseclass}{cell_input}
\begin{sphinxVerbatim}[commandchars=\\\{\}]
\PYG{n}{b} \PYG{o}{=} \PYG{n}{np}\PYG{o}{.}\PYG{n}{linspace}\PYG{p}{(}\PYG{l+m+mi}{0}\PYG{p}{,}\PYG{l+m+mf}{0.5}\PYG{p}{,}\PYG{n}{num}\PYG{o}{=}\PYG{l+m+mi}{50}\PYG{p}{)} \PYG{c+c1}{\PYGZsh{}\PYGZsh{} create a regular\PYGZhy{}spaced array from start to stop (excluding) with step size as input}
\end{sphinxVerbatim}

\end{sphinxuseclass}\end{sphinxVerbatimInput}

\end{sphinxuseclass}
\begin{sphinxuseclass}{cell}\begin{sphinxVerbatimInput}

\begin{sphinxuseclass}{cell_input}
\begin{sphinxVerbatim}[commandchars=\\\{\}]
\PYG{n}{np}\PYG{o}{.}\PYG{n}{max}\PYG{p}{(}\PYG{n}{a}\PYG{p}{)}
\end{sphinxVerbatim}

\end{sphinxuseclass}\end{sphinxVerbatimInput}
\begin{sphinxVerbatimOutput}

\begin{sphinxuseclass}{cell_output}
\begin{sphinxVerbatim}[commandchars=\\\{\}]
50.0
\end{sphinxVerbatim}

\end{sphinxuseclass}\end{sphinxVerbatimOutput}

\end{sphinxuseclass}
\begin{sphinxuseclass}{cell}\begin{sphinxVerbatimInput}

\begin{sphinxuseclass}{cell_input}
\begin{sphinxVerbatim}[commandchars=\\\{\}]
\PYG{n}{np}\PYG{o}{.}\PYG{n}{max}\PYG{p}{(}\PYG{n}{b}\PYG{p}{)}
\end{sphinxVerbatim}

\end{sphinxuseclass}\end{sphinxVerbatimInput}
\begin{sphinxVerbatimOutput}

\begin{sphinxuseclass}{cell_output}
\begin{sphinxVerbatim}[commandchars=\\\{\}]
0.5
\end{sphinxVerbatim}

\end{sphinxuseclass}\end{sphinxVerbatimOutput}

\end{sphinxuseclass}
\begin{sphinxuseclass}{cell}\begin{sphinxVerbatimInput}

\begin{sphinxuseclass}{cell_input}
\begin{sphinxVerbatim}[commandchars=\\\{\}]
\PYG{n}{plt}\PYG{o}{.}\PYG{n}{plot}\PYG{p}{(}\PYG{n}{a}\PYG{p}{,}\PYG{n}{b}\PYG{p}{,}\PYG{l+s+s1}{\PYGZsq{}}\PYG{l+s+s1}{\PYGZhy{}*b}\PYG{l+s+s1}{\PYGZsq{}}\PYG{p}{)}
\end{sphinxVerbatim}

\end{sphinxuseclass}\end{sphinxVerbatimInput}
\begin{sphinxVerbatimOutput}

\begin{sphinxuseclass}{cell_output}
\begin{sphinxVerbatim}[commandchars=\\\{\}]
[\PYGZlt{}matplotlib.lines.Line2D at 0x7f6285497d60\PYGZgt{}]
\end{sphinxVerbatim}

\noindent\sphinxincludegraphics{{642f4881db6488054865e83c11b32bece169375a4c3acd18f6d5f7e6c17d193b}.png}

\end{sphinxuseclass}\end{sphinxVerbatimOutput}

\end{sphinxuseclass}
\begin{sphinxuseclass}{cell}\begin{sphinxVerbatimInput}

\begin{sphinxuseclass}{cell_input}
\begin{sphinxVerbatim}[commandchars=\\\{\}]
\PYG{n}{b} \PYG{o}{=} \PYG{n}{np}\PYG{o}{.}\PYG{n}{linspace}\PYG{p}{(}\PYG{l+m+mi}{12}\PYG{p}{,}\PYG{l+m+mi}{60}\PYG{p}{,}\PYG{n}{num}\PYG{o}{=}\PYG{l+m+mi}{50}\PYG{p}{)}
\PYG{n}{plt}\PYG{o}{.}\PYG{n}{plot}\PYG{p}{(}\PYG{n}{a}\PYG{p}{,}\PYG{n}{b}\PYG{p}{,}\PYG{l+s+s1}{\PYGZsq{}}\PYG{l+s+s1}{*b}\PYG{l+s+s1}{\PYGZsq{}}\PYG{p}{)}
\end{sphinxVerbatim}

\end{sphinxuseclass}\end{sphinxVerbatimInput}
\begin{sphinxVerbatimOutput}

\begin{sphinxuseclass}{cell_output}
\begin{sphinxVerbatim}[commandchars=\\\{\}]
[\PYGZlt{}matplotlib.lines.Line2D at 0x7f628541f9a0\PYGZgt{}]
\end{sphinxVerbatim}

\noindent\sphinxincludegraphics{{d3bd21d0785762a9f629ceb393d28346185b9b98338d39ba03c9381b3596c28d}.png}

\end{sphinxuseclass}\end{sphinxVerbatimOutput}

\end{sphinxuseclass}
\begin{sphinxuseclass}{cell}\begin{sphinxVerbatimInput}

\begin{sphinxuseclass}{cell_input}
\begin{sphinxVerbatim}[commandchars=\\\{\}]
\PYG{n}{np}\PYG{o}{.}\PYG{n}{random}\PYG{o}{.}\PYG{n}{rand}\PYG{p}{(}\PYG{l+m+mi}{15}\PYG{p}{)}
\end{sphinxVerbatim}

\end{sphinxuseclass}\end{sphinxVerbatimInput}
\begin{sphinxVerbatimOutput}

\begin{sphinxuseclass}{cell_output}
\begin{sphinxVerbatim}[commandchars=\\\{\}]
array([0.89342128, 0.22409828, 0.97702249, 0.5533418 , 0.72261383,
       0.63191313, 0.2455794 , 0.588028  , 0.94449222, 0.81295154,
       0.85673411, 0.98242762, 0.70420093, 0.73077149, 0.63828771])
\end{sphinxVerbatim}

\end{sphinxuseclass}\end{sphinxVerbatimOutput}

\end{sphinxuseclass}
\begin{sphinxuseclass}{cell}\begin{sphinxVerbatimInput}

\begin{sphinxuseclass}{cell_input}
\begin{sphinxVerbatim}[commandchars=\\\{\}]
\PYG{n}{np}\PYG{o}{.}\PYG{n}{random}\PYG{o}{.}\PYG{n}{randint}\PYG{p}{(}\PYG{l+m+mi}{50}\PYG{p}{,}\PYG{n}{high}\PYG{o}{=}\PYG{l+m+mi}{100}\PYG{p}{,}\PYG{n}{size}\PYG{o}{=}\PYG{l+m+mi}{12}\PYG{p}{)}
\end{sphinxVerbatim}

\end{sphinxuseclass}\end{sphinxVerbatimInput}
\begin{sphinxVerbatimOutput}

\begin{sphinxuseclass}{cell_output}
\begin{sphinxVerbatim}[commandchars=\\\{\}]
array([59, 83, 54, 95, 99, 89, 88, 75, 55, 65, 76, 68])
\end{sphinxVerbatim}

\end{sphinxuseclass}\end{sphinxVerbatimOutput}

\end{sphinxuseclass}
\begin{sphinxuseclass}{cell}\begin{sphinxVerbatimInput}

\begin{sphinxuseclass}{cell_input}
\begin{sphinxVerbatim}[commandchars=\\\{\}]
\PYG{c+c1}{\PYGZsh{}\PYGZsh{} did some plotting with sinx, cosx}
\PYG{n}{x} \PYG{o}{=} \PYG{n}{np}\PYG{o}{.}\PYG{n}{linspace}\PYG{p}{(}\PYG{o}{\PYGZhy{}}\PYG{n}{np}\PYG{o}{.}\PYG{n}{pi}\PYG{p}{,} \PYG{n}{np}\PYG{o}{.}\PYG{n}{pi}\PYG{p}{,} \PYG{l+m+mi}{100}\PYG{p}{)}
\PYG{n}{y} \PYG{o}{=} \PYG{n}{np}\PYG{o}{.}\PYG{n}{sin}\PYG{p}{(}\PYG{n}{x}\PYG{p}{)}
\PYG{n}{plt}\PYG{o}{.}\PYG{n}{plot}\PYG{p}{(}\PYG{n}{x}\PYG{p}{,} \PYG{n}{y}\PYG{p}{,}\PYG{n}{label}\PYG{o}{=}\PYG{l+s+s1}{\PYGZsq{}}\PYG{l+s+s1}{Sin(x)}\PYG{l+s+s1}{\PYGZsq{}}\PYG{p}{)}
\PYG{n}{y} \PYG{o}{=} \PYG{n}{np}\PYG{o}{.}\PYG{n}{cos}\PYG{p}{(}\PYG{n}{x}\PYG{p}{)}
\PYG{n}{plt}\PYG{o}{.}\PYG{n}{plot}\PYG{p}{(}\PYG{n}{x}\PYG{p}{,} \PYG{n}{y}\PYG{p}{,} \PYG{n}{label}\PYG{o}{=}\PYG{l+s+s1}{\PYGZsq{}}\PYG{l+s+s1}{Cos(x)}\PYG{l+s+s1}{\PYGZsq{}}\PYG{p}{)}
\PYG{n}{plt}\PYG{o}{.}\PYG{n}{legend}\PYG{p}{(}\PYG{p}{)}
\end{sphinxVerbatim}

\end{sphinxuseclass}\end{sphinxVerbatimInput}
\begin{sphinxVerbatimOutput}

\begin{sphinxuseclass}{cell_output}
\begin{sphinxVerbatim}[commandchars=\\\{\}]
\PYGZlt{}matplotlib.legend.Legend at 0x7f62b86e7f70\PYGZgt{}
\end{sphinxVerbatim}

\noindent\sphinxincludegraphics{{23c86f31aa0456a60db1fc341134cdbbe773436b95e1f84a1552570b328da3e9}.png}

\end{sphinxuseclass}\end{sphinxVerbatimOutput}

\end{sphinxuseclass}
\begin{sphinxuseclass}{cell}\begin{sphinxVerbatimInput}

\begin{sphinxuseclass}{cell_input}
\begin{sphinxVerbatim}[commandchars=\\\{\}]
\PYG{c+c1}{\PYGZsh{}\PYGZsh{} did some plotting with polynomials}
\PYG{n}{x} \PYG{o}{=} \PYG{n}{np}\PYG{o}{.}\PYG{n}{linspace}\PYG{p}{(}\PYG{o}{\PYGZhy{}}\PYG{l+m+mi}{10}\PYG{p}{,}\PYG{l+m+mi}{10}\PYG{p}{,}\PYG{l+m+mi}{100}\PYG{p}{)}
\PYG{n}{y} \PYG{o}{=} \PYG{n}{x} \PYG{o}{*}\PYG{o}{*} \PYG{l+m+mi}{2}
\PYG{n}{plt}\PYG{o}{.}\PYG{n}{plot}\PYG{p}{(}\PYG{n}{x}\PYG{p}{,} \PYG{n}{y}\PYG{p}{,} \PYG{n}{label} \PYG{o}{=}\PYG{l+s+s1}{\PYGZsq{}}\PYG{l+s+s1}{x\PYGZca{}2}\PYG{l+s+s1}{\PYGZsq{}}\PYG{p}{)}
\PYG{n}{y} \PYG{o}{=} \PYG{n}{x} \PYG{o}{*}\PYG{o}{*} \PYG{l+m+mi}{5} \PYG{o}{\PYGZhy{}} \PYG{n}{x} \PYG{o}{*}\PYG{o}{*}\PYG{l+m+mi}{2} \PYG{o}{+} \PYG{l+m+mi}{1}
\PYG{n}{plt}\PYG{o}{.}\PYG{n}{plot}\PYG{p}{(}\PYG{n}{x}\PYG{p}{,} \PYG{n}{y}\PYG{p}{,} \PYG{n}{label} \PYG{o}{=} \PYG{l+s+s1}{\PYGZsq{}}\PYG{l+s+s1}{x\PYGZca{}5 \PYGZhy{} x\PYGZca{}2 + 1}\PYG{l+s+s1}{\PYGZsq{}}\PYG{p}{)}

\PYG{c+c1}{\PYGZsh{}\PYGZsh{} something that we did not cover in class:}
\PYG{n}{plt}\PYG{o}{.}\PYG{n}{xlim}\PYG{p}{(}\PYG{p}{[}\PYG{o}{\PYGZhy{}}\PYG{l+m+mi}{10}\PYG{p}{,}\PYG{l+m+mi}{10}\PYG{p}{]}\PYG{p}{)} \PYG{c+c1}{\PYGZsh{}\PYGZsh{} restricts the x\PYGZhy{}axis limit from \PYGZhy{}10 to +10}
\PYG{n}{plt}\PYG{o}{.}\PYG{n}{ylim}\PYG{p}{(}\PYG{p}{[}\PYG{o}{\PYGZhy{}}\PYG{l+m+mi}{10}\PYG{p}{,}\PYG{l+m+mi}{10}\PYG{p}{]}\PYG{p}{)} \PYG{c+c1}{\PYGZsh{}\PYGZsh{} restricts the y\PYGZhy{}axis limit from \PYGZhy{}10 to + 10}
\end{sphinxVerbatim}

\end{sphinxuseclass}\end{sphinxVerbatimInput}
\begin{sphinxVerbatimOutput}

\begin{sphinxuseclass}{cell_output}
\begin{sphinxVerbatim}[commandchars=\\\{\}]
(\PYGZhy{}10.0, 10.0)
\end{sphinxVerbatim}

\noindent\sphinxincludegraphics{{736f143bbe7d458ae4ef553f471bda9900c8c0b0035c475322fd610e68859e3e}.png}

\end{sphinxuseclass}\end{sphinxVerbatimOutput}

\end{sphinxuseclass}

\section{Installing CoolProp and calculating Enthalpy of Vaporization}
\label{\detokenize{notebooks/getting_started/basic_math:installing-coolprop-and-calculating-enthalpy-of-vaporization}}
\sphinxAtStartPar
install CoolProp by using:

\sphinxAtStartPar
!pip install CoolProp

\sphinxAtStartPar
or

\sphinxAtStartPar
import sys
!\{sys.executable\} \sphinxhyphen{}m pip install CoolProp

\begin{sphinxuseclass}{cell}\begin{sphinxVerbatimInput}

\begin{sphinxuseclass}{cell_input}
\begin{sphinxVerbatim}[commandchars=\\\{\}]
\PYG{k+kn}{from} \PYG{n+nn}{CoolProp}\PYG{n+nn}{.}\PYG{n+nn}{CoolProp} \PYG{k+kn}{import} \PYG{n}{PropsSI}
\end{sphinxVerbatim}

\end{sphinxuseclass}\end{sphinxVerbatimInput}

\end{sphinxuseclass}
\begin{sphinxuseclass}{cell}\begin{sphinxVerbatimInput}

\begin{sphinxuseclass}{cell_input}
\begin{sphinxVerbatim}[commandchars=\\\{\}]
\PYG{c+c1}{\PYGZsh{} enthalpy of vaporization example}
\PYG{n}{H\PYGZus{}L} \PYG{o}{=} \PYG{n}{PropsSI}\PYG{p}{(}\PYG{l+s+s1}{\PYGZsq{}}\PYG{l+s+s1}{H}\PYG{l+s+s1}{\PYGZsq{}}\PYG{p}{,} \PYG{l+s+s1}{\PYGZsq{}}\PYG{l+s+s1}{P}\PYG{l+s+s1}{\PYGZsq{}}\PYG{p}{,} \PYG{l+m+mi}{101325}\PYG{p}{,} \PYG{l+s+s1}{\PYGZsq{}}\PYG{l+s+s1}{Q}\PYG{l+s+s1}{\PYGZsq{}}\PYG{p}{,} \PYG{l+m+mi}{0}\PYG{p}{,} \PYG{l+s+s1}{\PYGZsq{}}\PYG{l+s+s1}{water}\PYG{l+s+s1}{\PYGZsq{}}\PYG{p}{)}
\PYG{n}{H\PYGZus{}V} \PYG{o}{=} \PYG{n}{PropsSI}\PYG{p}{(}\PYG{l+s+s1}{\PYGZsq{}}\PYG{l+s+s1}{H}\PYG{l+s+s1}{\PYGZsq{}}\PYG{p}{,} \PYG{l+s+s1}{\PYGZsq{}}\PYG{l+s+s1}{P}\PYG{l+s+s1}{\PYGZsq{}}\PYG{p}{,} \PYG{l+m+mi}{101325}\PYG{p}{,} \PYG{l+s+s1}{\PYGZsq{}}\PYG{l+s+s1}{Q}\PYG{l+s+s1}{\PYGZsq{}}\PYG{p}{,} \PYG{l+m+mi}{1}\PYG{p}{,} \PYG{l+s+s1}{\PYGZsq{}}\PYG{l+s+s1}{water}\PYG{l+s+s1}{\PYGZsq{}}\PYG{p}{)}
\end{sphinxVerbatim}

\end{sphinxuseclass}\end{sphinxVerbatimInput}

\end{sphinxuseclass}
\begin{sphinxuseclass}{cell}\begin{sphinxVerbatimInput}

\begin{sphinxuseclass}{cell_input}
\begin{sphinxVerbatim}[commandchars=\\\{\}]
\PYG{n+nb}{print}\PYG{p}{(}\PYG{l+s+s2}{\PYGZdq{}}\PYG{l+s+s2}{the enthalpy of vaporization at P = 1 atm is:}\PYG{l+s+s2}{\PYGZdq{}}\PYG{p}{,}\PYG{n+nb}{round}\PYG{p}{(}\PYG{p}{(}\PYG{n}{H\PYGZus{}V} \PYG{o}{\PYGZhy{}} \PYG{n}{H\PYGZus{}L}\PYG{p}{)}\PYG{o}{/}\PYG{l+m+mi}{1000}\PYG{p}{)}\PYG{p}{,} \PYG{l+s+s1}{\PYGZsq{}}\PYG{l+s+s1}{kJ/kg}\PYG{l+s+s1}{\PYGZsq{}}\PYG{p}{)}
\end{sphinxVerbatim}

\end{sphinxuseclass}\end{sphinxVerbatimInput}
\begin{sphinxVerbatimOutput}

\begin{sphinxuseclass}{cell_output}
\begin{sphinxVerbatim}[commandchars=\\\{\}]
the enthalpy of vaporization at P = 1 atm is: 2256 kJ/kg
\end{sphinxVerbatim}

\end{sphinxuseclass}\end{sphinxVerbatimOutput}

\end{sphinxuseclass}

\section{}
\label{\detokenize{notebooks/getting_started/basic_math:id1}}
\sphinxstepscope


\chapter{Defining variables: Properties of Ideal Gases as use case}
\label{\detokenize{notebooks/getting_started/Table_G1_Properties_of_ideal_gases:defining-variables-properties-of-ideal-gases-as-use-case}}\label{\detokenize{notebooks/getting_started/Table_G1_Properties_of_ideal_gases::doc}}
\sphinxAtStartPar
From: \sphinxurl{https://pressbooks.bccampus.ca/thermo1/back-matter/properties-of-various-substances/\#TG1}

\sphinxAtStartPar
Use the above table to look\sphinxhyphen{}up the following properties of Air:

\sphinxAtStartPar
Gas constant (R)

\sphinxAtStartPar
C\_p (heat capacity at const. pressure)

\sphinxAtStartPar
C\_v (heat capacity at const. volume)

\sphinxAtStartPar
k (ratio of Cp and Cv)

\begin{sphinxuseclass}{cell}\begin{sphinxVerbatimInput}

\begin{sphinxuseclass}{cell_input}
\begin{sphinxVerbatim}[commandchars=\\\{\}]
\PYG{c+c1}{\PYGZsh{}Air}
\PYG{n}{R\PYGZus{}air} \PYG{o}{=} \PYG{l+m+mf}{0.287} \PYG{c+c1}{\PYGZsh{} kJ/kgK}
\PYG{n}{Cp\PYGZus{}air} \PYG{o}{=} \PYG{l+m+mf}{1.005} \PYG{c+c1}{\PYGZsh{}kJ/kgK}
\PYG{n}{Cv\PYGZus{}air} \PYG{o}{=} \PYG{l+m+mf}{0.718} \PYG{c+c1}{\PYGZsh{}kJ/kgK}
\PYG{n}{k\PYGZus{}air} \PYG{o}{=} \PYG{l+m+mf}{1.4}
\end{sphinxVerbatim}

\end{sphinxuseclass}\end{sphinxVerbatimInput}

\end{sphinxuseclass}
\begin{sphinxuseclass}{cell}\begin{sphinxVerbatimInput}

\begin{sphinxuseclass}{cell_input}
\begin{sphinxVerbatim}[commandchars=\\\{\}]
\PYG{c+c1}{\PYGZsh{}Argon}

\PYG{c+c1}{\PYGZsh{}\PYGZsh{}\PYGZsh{} Please fill in here from looking into the above URL \PYGZsh{}\PYGZsh{}\PYGZsh{}}
\end{sphinxVerbatim}

\end{sphinxuseclass}\end{sphinxVerbatimInput}

\end{sphinxuseclass}
\sphinxAtStartPar
R\_argon = \# kJ/kgK

\sphinxAtStartPar
Cp\_argon = \#kJ/kgK

\sphinxAtStartPar
Cv\_argon =  \#kJ/kgK

\sphinxAtStartPar
k\_argon =   \# unitless

\sphinxstepscope


\part{Modules}

\sphinxstepscope


\chapter{1. Basic Concepts and Definitions}
\label{\detokenize{notebooks/Chapter1/chapter1:basic-concepts-and-definitions}}\label{\detokenize{notebooks/Chapter1/chapter1::doc}}
\sphinxAtStartPar
Let’s do some problems based on concepts on \sphinxhref{https://pressbooks.bccampus.ca/thermo1/chapter/introduction-and-learning-objectives/}{Chapter 1} from the e\sphinxhyphen{}textbook.

\sphinxstepscope

\begin{sphinxuseclass}{cell}\begin{sphinxVerbatimInput}

\begin{sphinxuseclass}{cell_input}
\begin{sphinxVerbatim}[commandchars=\\\{\}]
\PYG{k+kn}{import} \PYG{n+nn}{numpy} \PYG{k}{as} \PYG{n+nn}{np}
\end{sphinxVerbatim}

\end{sphinxuseclass}\end{sphinxVerbatimInput}

\end{sphinxuseclass}
\sphinxstepscope


\chapter{2. Thermodynamic Properties of a Pure Substance}
\label{\detokenize{notebooks/Chapter2/thermo-properties:thermodynamic-properties-of-a-pure-substance}}\label{\detokenize{notebooks/Chapter2/thermo-properties::doc}}
\sphinxAtStartPar
This is a problem set based on the concepts from the Chapter here: \sphinxurl{https://pressbooks.bccampus.ca/thermo1/chapter/2-0-chapter-introduction-and-learning-objectives/}

\sphinxstepscope


\section{Pressure in a Tank}
\label{\detokenize{notebooks/Chapter2/Problem_1_Pressure_in_a_Tank:pressure-in-a-tank}}\label{\detokenize{notebooks/Chapter2/Problem_1_Pressure_in_a_Tank::doc}}

\subsection{Problem Statement:}
\label{\detokenize{notebooks/Chapter2/Problem_1_Pressure_in_a_Tank:problem-statement}}
\sphinxAtStartPar
A cylindrical tank contains air at a gauge pressure of 3 atm and an ambient atmospheric pressure of 1 atm.
Calculate the absolute pressure inside the tank.

\begin{sphinxuseclass}{cell}\begin{sphinxVerbatimInput}

\begin{sphinxuseclass}{cell_input}
\begin{sphinxVerbatim}[commandchars=\\\{\}]
\PYG{c+c1}{\PYGZsh{}\PYGZsh{} Solution:}

\PYG{c+c1}{\PYGZsh{} Given values}
\PYG{n}{P\PYGZus{}gauge} \PYG{o}{=} \PYG{l+m+mi}{3} \PYG{o}{*} \PYG{l+m+mi}{101325}  \PYG{c+c1}{\PYGZsh{} Gauge pressure in Pascals (1 atm = 101325 Pa)}
\PYG{n}{P\PYGZus{}atm} \PYG{o}{=} \PYG{l+m+mi}{1} \PYG{o}{*} \PYG{l+m+mi}{101325}  \PYG{c+c1}{\PYGZsh{} Atmospheric pressure in Pascals}

\PYG{c+c1}{\PYGZsh{} Absolute pressure}
\PYG{n}{P\PYGZus{}abs} \PYG{o}{=} \PYG{n}{P\PYGZus{}gauge} \PYG{o}{+} \PYG{n}{P\PYGZus{}atm}

\PYG{c+c1}{\PYGZsh{} Output the result}
\PYG{n+nb}{print}\PYG{p}{(}\PYG{l+s+sa}{f}\PYG{l+s+s2}{\PYGZdq{}}\PYG{l+s+s2}{Absolute pressure inside the tank: }\PYG{l+s+si}{\PYGZob{}}\PYG{n+nb}{round}\PYG{p}{(}\PYG{n}{P\PYGZus{}abs}\PYG{o}{/}\PYG{l+m+mf}{1e3}\PYG{p}{,}\PYG{l+m+mi}{1}\PYG{p}{)}\PYG{l+s+si}{\PYGZcb{}}\PYG{l+s+s2}{ kPa}\PYG{l+s+s2}{\PYGZdq{}}\PYG{p}{)}
\end{sphinxVerbatim}

\end{sphinxuseclass}\end{sphinxVerbatimInput}
\begin{sphinxVerbatimOutput}

\begin{sphinxuseclass}{cell_output}
\begin{sphinxVerbatim}[commandchars=\\\{\}]
Absolute pressure inside the tank: 405.3 kPa
\end{sphinxVerbatim}

\end{sphinxuseclass}\end{sphinxVerbatimOutput}

\end{sphinxuseclass}
\sphinxstepscope


\section{Density of a Gas}
\label{\detokenize{notebooks/Chapter2/Problem_2_Density_of_a_Gas:density-of-a-gas}}\label{\detokenize{notebooks/Chapter2/Problem_2_Density_of_a_Gas::doc}}

\subsection{Problem Statement:}
\label{\detokenize{notebooks/Chapter2/Problem_2_Density_of_a_Gas:problem-statement}}
\sphinxAtStartPar
A container with a volume of 2 m³ holds 5 kg of nitrogen gas.
Calculate the density of the gas.

\begin{sphinxuseclass}{cell}\begin{sphinxVerbatimInput}

\begin{sphinxuseclass}{cell_input}
\begin{sphinxVerbatim}[commandchars=\\\{\}]
\PYG{c+c1}{\PYGZsh{}\PYGZsh{} Solution:}

\PYG{c+c1}{\PYGZsh{} Given values}
\PYG{n}{m} \PYG{o}{=} \PYG{l+m+mi}{5}  \PYG{c+c1}{\PYGZsh{} Mass of nitrogen in kg}
\PYG{n}{V} \PYG{o}{=} \PYG{l+m+mi}{2}  \PYG{c+c1}{\PYGZsh{} Volume in m\PYGZca{}3}

\PYG{c+c1}{\PYGZsh{} Density calculation}
\PYG{n}{density} \PYG{o}{=} \PYG{n}{m} \PYG{o}{/} \PYG{n}{V}

\PYG{c+c1}{\PYGZsh{} Output the result}
\PYG{n+nb}{print}\PYG{p}{(}\PYG{l+s+sa}{f}\PYG{l+s+s2}{\PYGZdq{}}\PYG{l+s+s2}{Density of the nitrogen gas: }\PYG{l+s+si}{\PYGZob{}}\PYG{n}{density}\PYG{l+s+si}{\PYGZcb{}}\PYG{l+s+s2}{ kg/m\PYGZca{}3}\PYG{l+s+s2}{\PYGZdq{}}\PYG{p}{)}
\end{sphinxVerbatim}

\end{sphinxuseclass}\end{sphinxVerbatimInput}
\begin{sphinxVerbatimOutput}

\begin{sphinxuseclass}{cell_output}
\begin{sphinxVerbatim}[commandchars=\\\{\}]
Density of the nitrogen gas: 2.5 kg/m\PYGZca{}3
\end{sphinxVerbatim}

\end{sphinxuseclass}\end{sphinxVerbatimOutput}

\end{sphinxuseclass}
\sphinxstepscope


\section{Energy in a Moving Vehicle}
\label{\detokenize{notebooks/Chapter2/Problem_3_Energy_in_a_Moving_Vehicle:energy-in-a-moving-vehicle}}\label{\detokenize{notebooks/Chapter2/Problem_3_Energy_in_a_Moving_Vehicle::doc}}

\subsection{Problem Statement:}
\label{\detokenize{notebooks/Chapter2/Problem_3_Energy_in_a_Moving_Vehicle:problem-statement}}
\sphinxAtStartPar
A car with a mass of 1200 kg is traveling at a speed of 60 km/h.
If the height above the ground is 50 m, calculate the total stored energy of the car.

\begin{sphinxuseclass}{cell}\begin{sphinxVerbatimInput}

\begin{sphinxuseclass}{cell_input}
\begin{sphinxVerbatim}[commandchars=\\\{\}]
\PYG{c+c1}{\PYGZsh{}\PYGZsh{} Solution:}

\PYG{c+c1}{\PYGZsh{} Given values}
\PYG{n}{m} \PYG{o}{=} \PYG{l+m+mi}{1200}  \PYG{c+c1}{\PYGZsh{} Mass of the car in kg}
\PYG{n}{V} \PYG{o}{=} \PYG{l+m+mi}{60} \PYG{o}{*} \PYG{p}{(}\PYG{l+m+mi}{1000} \PYG{o}{/} \PYG{l+m+mi}{3600}\PYG{p}{)}  \PYG{c+c1}{\PYGZsh{} Speed in m/s (60 km/h)}
\PYG{n}{g} \PYG{o}{=} \PYG{l+m+mf}{9.81}  \PYG{c+c1}{\PYGZsh{} Acceleration due to gravity in m/s\PYGZca{}2}
\PYG{n}{z} \PYG{o}{=} \PYG{l+m+mi}{50}  \PYG{c+c1}{\PYGZsh{} Height in meters}

\PYG{c+c1}{\PYGZsh{} Total stored energy calculation}
\PYG{n}{E} \PYG{o}{=} \PYG{n}{m} \PYG{o}{*} \PYG{p}{(}\PYG{n}{V}\PYG{o}{*}\PYG{o}{*}\PYG{l+m+mi}{2} \PYG{o}{/} \PYG{l+m+mi}{2} \PYG{o}{+} \PYG{n}{g} \PYG{o}{*} \PYG{n}{z}\PYG{p}{)}

\PYG{c+c1}{\PYGZsh{} Output the result}
\PYG{n+nb}{print}\PYG{p}{(}\PYG{l+s+sa}{f}\PYG{l+s+s2}{\PYGZdq{}}\PYG{l+s+s2}{Total stored energy of the car: }\PYG{l+s+si}{\PYGZob{}}\PYG{n+nb}{round}\PYG{p}{(}\PYG{n}{E}\PYG{o}{/}\PYG{l+m+mf}{1e3}\PYG{p}{)}\PYG{l+s+si}{\PYGZcb{}}\PYG{l+s+s2}{ kJ}\PYG{l+s+s2}{\PYGZdq{}}\PYG{p}{)}
\end{sphinxVerbatim}

\end{sphinxuseclass}\end{sphinxVerbatimInput}
\begin{sphinxVerbatimOutput}

\begin{sphinxuseclass}{cell_output}
\begin{sphinxVerbatim}[commandchars=\\\{\}]
Total stored energy of the car: 755 kJ
\end{sphinxVerbatim}

\end{sphinxuseclass}\end{sphinxVerbatimOutput}

\end{sphinxuseclass}
\sphinxstepscope


\section{Enthalpy of a Steam}
\label{\detokenize{notebooks/Chapter2/Problem_4_Enthalpy_of_a_Steam:enthalpy-of-a-steam}}\label{\detokenize{notebooks/Chapter2/Problem_4_Enthalpy_of_a_Steam::doc}}

\subsection{Problem Statement:}
\label{\detokenize{notebooks/Chapter2/Problem_4_Enthalpy_of_a_Steam:problem-statement}}
\sphinxAtStartPar
Calculate the specific enthalpy of steam at a pressure of 200 kPa and a specific volume of 0.15 m³/kg.

\begin{sphinxuseclass}{cell}\begin{sphinxVerbatimInput}

\begin{sphinxuseclass}{cell_input}
\begin{sphinxVerbatim}[commandchars=\\\{\}]
\PYG{c+c1}{\PYGZsh{}\PYGZsh{} Solution:}

\PYG{k+kn}{import} \PYG{n+nn}{CoolProp}\PYG{n+nn}{.}\PYG{n+nn}{CoolProp} \PYG{k}{as} \PYG{n+nn}{CP}

\PYG{c+c1}{\PYGZsh{} Given values}
\PYG{n}{P} \PYG{o}{=} \PYG{l+m+mi}{200000}  \PYG{c+c1}{\PYGZsh{} Pressure in Pascals}
\PYG{n}{v} \PYG{o}{=} \PYG{l+m+mf}{0.15}  \PYG{c+c1}{\PYGZsh{} Specific volume in m\PYGZca{}3/kg}

\PYG{c+c1}{\PYGZsh{} Specific enthalpy calculation}
\PYG{n}{h} \PYG{o}{=} \PYG{n}{CP}\PYG{o}{.}\PYG{n}{PropsSI}\PYG{p}{(}\PYG{l+s+s1}{\PYGZsq{}}\PYG{l+s+s1}{H}\PYG{l+s+s1}{\PYGZsq{}}\PYG{p}{,} \PYG{l+s+s1}{\PYGZsq{}}\PYG{l+s+s1}{P}\PYG{l+s+s1}{\PYGZsq{}}\PYG{p}{,} \PYG{n}{P}\PYG{p}{,} \PYG{l+s+s1}{\PYGZsq{}}\PYG{l+s+s1}{D}\PYG{l+s+s1}{\PYGZsq{}}\PYG{p}{,} \PYG{l+m+mi}{1}\PYG{o}{/}\PYG{n}{v}\PYG{p}{,} \PYG{l+s+s1}{\PYGZsq{}}\PYG{l+s+s1}{Water}\PYG{l+s+s1}{\PYGZsq{}}\PYG{p}{)}

\PYG{c+c1}{\PYGZsh{} Output the result}
\PYG{n+nb}{print}\PYG{p}{(}\PYG{l+s+sa}{f}\PYG{l+s+s2}{\PYGZdq{}}\PYG{l+s+s2}{Specific enthalpy of steam: }\PYG{l+s+si}{\PYGZob{}}\PYG{n+nb}{round}\PYG{p}{(}\PYG{n}{h}\PYG{o}{/}\PYG{l+m+mf}{1e3}\PYG{p}{,}\PYG{l+m+mi}{1}\PYG{p}{)}\PYG{l+s+si}{\PYGZcb{}}\PYG{l+s+s2}{ kJ/kg}\PYG{l+s+s2}{\PYGZdq{}}\PYG{p}{)}
\end{sphinxVerbatim}

\end{sphinxuseclass}\end{sphinxVerbatimInput}
\begin{sphinxVerbatimOutput}

\begin{sphinxuseclass}{cell_output}
\begin{sphinxVerbatim}[commandchars=\\\{\}]
Specific enthalpy of steam: 875.4 kJ/kg
\end{sphinxVerbatim}

\end{sphinxuseclass}\end{sphinxVerbatimOutput}

\end{sphinxuseclass}
\sphinxstepscope


\section{Quality of a Vapor Mixture}
\label{\detokenize{notebooks/Chapter2/Problem_5_Quality_of_a_Vapor_Mixture:quality-of-a-vapor-mixture}}\label{\detokenize{notebooks/Chapter2/Problem_5_Quality_of_a_Vapor_Mixture::doc}}

\subsection{Problem Statement:}
\label{\detokenize{notebooks/Chapter2/Problem_5_Quality_of_a_Vapor_Mixture:problem-statement}}
\sphinxAtStartPar
In a vapor mixture at 150°C, the mass of the vapor phase is 3 kg, and the total mass of the mixture is 5 kg.
Determine the quality of the mixture.

\begin{sphinxuseclass}{cell}\begin{sphinxVerbatimInput}

\begin{sphinxuseclass}{cell_input}
\begin{sphinxVerbatim}[commandchars=\\\{\}]
\PYG{c+c1}{\PYGZsh{}\PYGZsh{} Solution:}

\PYG{c+c1}{\PYGZsh{} Given values}
\PYG{n}{m\PYGZus{}vapor} \PYG{o}{=} \PYG{l+m+mi}{3}  \PYG{c+c1}{\PYGZsh{} Mass of vapor phase in kg}
\PYG{n}{m\PYGZus{}total} \PYG{o}{=} \PYG{l+m+mi}{5}  \PYG{c+c1}{\PYGZsh{} Total mass of the mixture in kg}

\PYG{c+c1}{\PYGZsh{} Quality calculation}
\PYG{n}{x} \PYG{o}{=} \PYG{n}{m\PYGZus{}vapor} \PYG{o}{/} \PYG{n}{m\PYGZus{}total}

\PYG{c+c1}{\PYGZsh{} Output the result}
\PYG{n+nb}{print}\PYG{p}{(}\PYG{l+s+sa}{f}\PYG{l+s+s2}{\PYGZdq{}}\PYG{l+s+s2}{Quality of the mixture: }\PYG{l+s+si}{\PYGZob{}}\PYG{n}{x}\PYG{l+s+si}{\PYGZcb{}}\PYG{l+s+s2}{\PYGZdq{}}\PYG{p}{)}
\end{sphinxVerbatim}

\end{sphinxuseclass}\end{sphinxVerbatimInput}
\begin{sphinxVerbatimOutput}

\begin{sphinxuseclass}{cell_output}
\begin{sphinxVerbatim}[commandchars=\\\{\}]
Quality of the mixture: 0.6
\end{sphinxVerbatim}

\end{sphinxuseclass}\end{sphinxVerbatimOutput}

\end{sphinxuseclass}
\sphinxstepscope


\section{Compressed Liquid Approximation}
\label{\detokenize{notebooks/Chapter2/Problem_6_Compressed_Liquid_Approximation:compressed-liquid-approximation}}\label{\detokenize{notebooks/Chapter2/Problem_6_Compressed_Liquid_Approximation::doc}}

\subsection{Problem Statement:}
\label{\detokenize{notebooks/Chapter2/Problem_6_Compressed_Liquid_Approximation:problem-statement}}
\sphinxAtStartPar
For water at 50°C under high pressure, approximate the specific volume, internal energy, enthalpy, and entropy
assuming it behaves as a compressed liquid.

\begin{sphinxuseclass}{cell}\begin{sphinxVerbatimInput}

\begin{sphinxuseclass}{cell_input}
\begin{sphinxVerbatim}[commandchars=\\\{\}]
\PYG{c+c1}{\PYGZsh{}\PYGZsh{} Solution:}

\PYG{k+kn}{import} \PYG{n+nn}{CoolProp}\PYG{n+nn}{.}\PYG{n+nn}{CoolProp} \PYG{k}{as} \PYG{n+nn}{CP}

\PYG{c+c1}{\PYGZsh{} Given values}
\PYG{n}{T} \PYG{o}{=} \PYG{l+m+mi}{50} \PYG{o}{+} \PYG{l+m+mf}{273.15}  \PYG{c+c1}{\PYGZsh{} Temperature in Kelvin}

\PYG{c+c1}{\PYGZsh{} Approximations for compressed liquid}
\PYG{n}{v} \PYG{o}{=} \PYG{n}{CP}\PYG{o}{.}\PYG{n}{PropsSI}\PYG{p}{(}\PYG{l+s+s1}{\PYGZsq{}}\PYG{l+s+s1}{D}\PYG{l+s+s1}{\PYGZsq{}}\PYG{p}{,} \PYG{l+s+s1}{\PYGZsq{}}\PYG{l+s+s1}{T}\PYG{l+s+s1}{\PYGZsq{}}\PYG{p}{,} \PYG{n}{T}\PYG{p}{,} \PYG{l+s+s1}{\PYGZsq{}}\PYG{l+s+s1}{Q}\PYG{l+s+s1}{\PYGZsq{}}\PYG{p}{,} \PYG{l+m+mi}{0}\PYG{p}{,} \PYG{l+s+s1}{\PYGZsq{}}\PYG{l+s+s1}{Water}\PYG{l+s+s1}{\PYGZsq{}}\PYG{p}{)}  \PYG{c+c1}{\PYGZsh{} Specific volume}
\PYG{n}{u} \PYG{o}{=} \PYG{n}{CP}\PYG{o}{.}\PYG{n}{PropsSI}\PYG{p}{(}\PYG{l+s+s1}{\PYGZsq{}}\PYG{l+s+s1}{U}\PYG{l+s+s1}{\PYGZsq{}}\PYG{p}{,} \PYG{l+s+s1}{\PYGZsq{}}\PYG{l+s+s1}{T}\PYG{l+s+s1}{\PYGZsq{}}\PYG{p}{,} \PYG{n}{T}\PYG{p}{,} \PYG{l+s+s1}{\PYGZsq{}}\PYG{l+s+s1}{Q}\PYG{l+s+s1}{\PYGZsq{}}\PYG{p}{,} \PYG{l+m+mi}{0}\PYG{p}{,} \PYG{l+s+s1}{\PYGZsq{}}\PYG{l+s+s1}{Water}\PYG{l+s+s1}{\PYGZsq{}}\PYG{p}{)}\PYG{o}{/}\PYG{l+m+mf}{1e3}  \PYG{c+c1}{\PYGZsh{} Specific internal energy}
\PYG{n}{h} \PYG{o}{=} \PYG{n}{CP}\PYG{o}{.}\PYG{n}{PropsSI}\PYG{p}{(}\PYG{l+s+s1}{\PYGZsq{}}\PYG{l+s+s1}{H}\PYG{l+s+s1}{\PYGZsq{}}\PYG{p}{,} \PYG{l+s+s1}{\PYGZsq{}}\PYG{l+s+s1}{T}\PYG{l+s+s1}{\PYGZsq{}}\PYG{p}{,} \PYG{n}{T}\PYG{p}{,} \PYG{l+s+s1}{\PYGZsq{}}\PYG{l+s+s1}{Q}\PYG{l+s+s1}{\PYGZsq{}}\PYG{p}{,} \PYG{l+m+mi}{0}\PYG{p}{,} \PYG{l+s+s1}{\PYGZsq{}}\PYG{l+s+s1}{Water}\PYG{l+s+s1}{\PYGZsq{}}\PYG{p}{)}\PYG{o}{/}\PYG{l+m+mf}{1e3}  \PYG{c+c1}{\PYGZsh{} Specific enthalpy}
\PYG{n}{s} \PYG{o}{=} \PYG{n}{CP}\PYG{o}{.}\PYG{n}{PropsSI}\PYG{p}{(}\PYG{l+s+s1}{\PYGZsq{}}\PYG{l+s+s1}{S}\PYG{l+s+s1}{\PYGZsq{}}\PYG{p}{,} \PYG{l+s+s1}{\PYGZsq{}}\PYG{l+s+s1}{T}\PYG{l+s+s1}{\PYGZsq{}}\PYG{p}{,} \PYG{n}{T}\PYG{p}{,} \PYG{l+s+s1}{\PYGZsq{}}\PYG{l+s+s1}{Q}\PYG{l+s+s1}{\PYGZsq{}}\PYG{p}{,} \PYG{l+m+mi}{0}\PYG{p}{,} \PYG{l+s+s1}{\PYGZsq{}}\PYG{l+s+s1}{Water}\PYG{l+s+s1}{\PYGZsq{}}\PYG{p}{)}\PYG{o}{/}\PYG{l+m+mf}{1e3}  \PYG{c+c1}{\PYGZsh{} Specific entropy}

\PYG{c+c1}{\PYGZsh{} Output the results}
\PYG{n+nb}{print}\PYG{p}{(}\PYG{l+s+sa}{f}\PYG{l+s+s2}{\PYGZdq{}}\PYG{l+s+s2}{Specific volume: }\PYG{l+s+si}{\PYGZob{}}\PYG{n+nb}{round}\PYG{p}{(}\PYG{n}{v}\PYG{p}{,}\PYG{l+m+mi}{2}\PYG{p}{)}\PYG{l+s+si}{\PYGZcb{}}\PYG{l+s+s2}{ m\PYGZca{}3/kg}\PYG{l+s+s2}{\PYGZdq{}}\PYG{p}{)}
\PYG{n+nb}{print}\PYG{p}{(}\PYG{l+s+sa}{f}\PYG{l+s+s2}{\PYGZdq{}}\PYG{l+s+s2}{Specific internal energy: }\PYG{l+s+si}{\PYGZob{}}\PYG{n+nb}{round}\PYG{p}{(}\PYG{n}{u}\PYG{p}{,}\PYG{l+m+mi}{2}\PYG{p}{)}\PYG{l+s+si}{\PYGZcb{}}\PYG{l+s+s2}{ kJ/kg}\PYG{l+s+s2}{\PYGZdq{}}\PYG{p}{)}
\PYG{n+nb}{print}\PYG{p}{(}\PYG{l+s+sa}{f}\PYG{l+s+s2}{\PYGZdq{}}\PYG{l+s+s2}{Specific enthalpy: }\PYG{l+s+si}{\PYGZob{}}\PYG{n+nb}{round}\PYG{p}{(}\PYG{n}{h}\PYG{p}{,}\PYG{l+m+mi}{2}\PYG{p}{)}\PYG{l+s+si}{\PYGZcb{}}\PYG{l+s+s2}{ kJ/kg}\PYG{l+s+s2}{\PYGZdq{}}\PYG{p}{)}
\PYG{n+nb}{print}\PYG{p}{(}\PYG{l+s+sa}{f}\PYG{l+s+s2}{\PYGZdq{}}\PYG{l+s+s2}{Specific entropy: }\PYG{l+s+si}{\PYGZob{}}\PYG{n+nb}{round}\PYG{p}{(}\PYG{n}{s}\PYG{p}{,}\PYG{l+m+mi}{2}\PYG{p}{)}\PYG{l+s+si}{\PYGZcb{}}\PYG{l+s+s2}{ kJ/kg*K}\PYG{l+s+s2}{\PYGZdq{}}\PYG{p}{)}
\end{sphinxVerbatim}

\end{sphinxuseclass}\end{sphinxVerbatimInput}
\begin{sphinxVerbatimOutput}

\begin{sphinxuseclass}{cell_output}
\begin{sphinxVerbatim}[commandchars=\\\{\}]
Specific volume: 988.0 m\PYGZca{}3/kg
Specific internal energy: 209.33 kJ/kg
Specific enthalpy: 209.34 kJ/kg
Specific entropy: 0.7 kJ/kg*K
\end{sphinxVerbatim}

\end{sphinxuseclass}\end{sphinxVerbatimOutput}

\end{sphinxuseclass}
\sphinxstepscope


\section{Conversion of Temperature}
\label{\detokenize{notebooks/Chapter2/Problem_7_Conversion_of_Temperature:conversion-of-temperature}}\label{\detokenize{notebooks/Chapter2/Problem_7_Conversion_of_Temperature::doc}}

\subsection{Problem Statement:}
\label{\detokenize{notebooks/Chapter2/Problem_7_Conversion_of_Temperature:problem-statement}}
\sphinxAtStartPar
Convert a temperature of 20°C to Kelvin.

\begin{sphinxuseclass}{cell}\begin{sphinxVerbatimInput}

\begin{sphinxuseclass}{cell_input}
\begin{sphinxVerbatim}[commandchars=\\\{\}]
\PYG{c+c1}{\PYGZsh{}\PYGZsh{} Solution:}

\PYG{c+c1}{\PYGZsh{} Given value}
\PYG{n}{T\PYGZus{}Celsius} \PYG{o}{=} \PYG{l+m+mi}{20}  \PYG{c+c1}{\PYGZsh{} Temperature in Celsius}

\PYG{c+c1}{\PYGZsh{} Conversion to Kelvin}
\PYG{n}{T\PYGZus{}Kelvin} \PYG{o}{=} \PYG{n}{T\PYGZus{}Celsius} \PYG{o}{+} \PYG{l+m+mf}{273.15}

\PYG{c+c1}{\PYGZsh{} Output the result}
\PYG{n+nb}{print}\PYG{p}{(}\PYG{l+s+sa}{f}\PYG{l+s+s2}{\PYGZdq{}}\PYG{l+s+s2}{Temperature in Kelvin: }\PYG{l+s+si}{\PYGZob{}}\PYG{n}{T\PYGZus{}Kelvin}\PYG{l+s+si}{\PYGZcb{}}\PYG{l+s+s2}{ K}\PYG{l+s+s2}{\PYGZdq{}}\PYG{p}{)}
\end{sphinxVerbatim}

\end{sphinxuseclass}\end{sphinxVerbatimInput}
\begin{sphinxVerbatimOutput}

\begin{sphinxuseclass}{cell_output}
\begin{sphinxVerbatim}[commandchars=\\\{\}]
Temperature in Kelvin: 293.15 K
\end{sphinxVerbatim}

\end{sphinxuseclass}\end{sphinxVerbatimOutput}

\end{sphinxuseclass}
\sphinxstepscope


\section{Specific Entropy Calculation}
\label{\detokenize{notebooks/Chapter2/Problem_8_Specific_Entropy_Calculation:specific-entropy-calculation}}\label{\detokenize{notebooks/Chapter2/Problem_8_Specific_Entropy_Calculation::doc}}

\subsection{Problem Statement:}
\label{\detokenize{notebooks/Chapter2/Problem_8_Specific_Entropy_Calculation:problem-statement}}
\sphinxAtStartPar
Calculate the specific entropy of a system with a total entropy of 2500 J/K and a mass of 10 kg.

\begin{sphinxuseclass}{cell}\begin{sphinxVerbatimInput}

\begin{sphinxuseclass}{cell_input}
\begin{sphinxVerbatim}[commandchars=\\\{\}]
\PYG{c+c1}{\PYGZsh{}\PYGZsh{} Solution:}

\PYG{c+c1}{\PYGZsh{} Given values}
\PYG{n}{S\PYGZus{}total} \PYG{o}{=} \PYG{l+m+mi}{2500}  \PYG{c+c1}{\PYGZsh{} Total entropy in J/K}
\PYG{n}{m} \PYG{o}{=} \PYG{l+m+mi}{10}  \PYG{c+c1}{\PYGZsh{} Mass in kg}

\PYG{c+c1}{\PYGZsh{} Specific entropy calculation}
\PYG{n}{s} \PYG{o}{=} \PYG{n}{S\PYGZus{}total} \PYG{o}{/} \PYG{n}{m}

\PYG{c+c1}{\PYGZsh{} Output the result}
\PYG{n+nb}{print}\PYG{p}{(}\PYG{l+s+sa}{f}\PYG{l+s+s2}{\PYGZdq{}}\PYG{l+s+s2}{Specific entropy: }\PYG{l+s+si}{\PYGZob{}}\PYG{n+nb}{round}\PYG{p}{(}\PYG{n}{s}\PYG{o}{/}\PYG{l+m+mf}{1e3}\PYG{p}{,}\PYG{l+m+mi}{1}\PYG{p}{)}\PYG{l+s+si}{\PYGZcb{}}\PYG{l+s+s2}{ kJ/(kg*K)}\PYG{l+s+s2}{\PYGZdq{}}\PYG{p}{)}
\end{sphinxVerbatim}

\end{sphinxuseclass}\end{sphinxVerbatimInput}
\begin{sphinxVerbatimOutput}

\begin{sphinxuseclass}{cell_output}
\begin{sphinxVerbatim}[commandchars=\\\{\}]
Specific entropy: 0.2 kJ/(kg*K)
\end{sphinxVerbatim}

\end{sphinxuseclass}\end{sphinxVerbatimOutput}

\end{sphinxuseclass}
\sphinxstepscope


\section{Complete the following table for Water:}
\label{\detokenize{notebooks/Chapter2/complete-table-water:complete-the-following-table-for-water}}\label{\detokenize{notebooks/Chapter2/complete-table-water::doc}}

\begin{savenotes}\sphinxattablestart
\centering
\begin{tabulary}{\linewidth}[t]{|T|T|T|T|}
\hline
\sphinxstyletheadfamily 
\sphinxAtStartPar
T, degC
&\sphinxstyletheadfamily 
\sphinxAtStartPar
P, kPa
&\sphinxstyletheadfamily 
\sphinxAtStartPar
u, kJ/kg
&\sphinxstyletheadfamily 
\sphinxAtStartPar
Phase description
\\
\hline
\sphinxAtStartPar

&
\sphinxAtStartPar
560
&
\sphinxAtStartPar
1470
&
\sphinxAtStartPar

\\
\hline
\sphinxAtStartPar
240
&
\sphinxAtStartPar

&
\sphinxAtStartPar

&
\sphinxAtStartPar
Saturated vapor
\\
\hline
\sphinxAtStartPar
150
&
\sphinxAtStartPar
2600
&
\sphinxAtStartPar

&
\sphinxAtStartPar

\\
\hline
\sphinxAtStartPar

&
\sphinxAtStartPar
4200
&
\sphinxAtStartPar
3100
&
\sphinxAtStartPar

\\
\hline
\end{tabulary}
\par
\sphinxattableend\end{savenotes}

\sphinxAtStartPar
Lookup tables from the e\sphinxhyphen{}textbook: \sphinxurl{https://pressbooks.bccampus.ca/thermo1/chapter/thermodynamic-tables/}

\begin{sphinxuseclass}{cell}\begin{sphinxVerbatimInput}

\begin{sphinxuseclass}{cell_input}
\begin{sphinxVerbatim}[commandchars=\\\{\}]
\PYG{k+kn}{import} \PYG{n+nn}{CoolProp}\PYG{n+nn}{.}\PYG{n+nn}{CoolProp} \PYG{k}{as} \PYG{n+nn}{CP}
\end{sphinxVerbatim}

\end{sphinxuseclass}\end{sphinxVerbatimInput}

\end{sphinxuseclass}
\begin{sphinxuseclass}{cell}\begin{sphinxVerbatimInput}

\begin{sphinxuseclass}{cell_input}
\begin{sphinxVerbatim}[commandchars=\\\{\}]
\PYG{c+c1}{\PYGZsh{}\PYGZsh{}\PYGZsh{}========== (a)================\PYGZsh{}\PYGZsh{}\PYGZsh{}}
\PYG{n}{P} \PYG{o}{=} \PYG{l+m+mf}{0.56e6} \PYG{c+c1}{\PYGZsh{} in Pa}
\PYG{n}{u} \PYG{o}{=} \PYG{l+m+mi}{1470} \PYG{c+c1}{\PYGZsh{} in kJ/kg}
\PYG{n}{fluid} \PYG{o}{=} \PYG{l+s+s2}{\PYGZdq{}}\PYG{l+s+s2}{water}\PYG{l+s+s2}{\PYGZdq{}}
\PYG{n}{uf} \PYG{o}{=} \PYG{n}{CP}\PYG{o}{.}\PYG{n}{PropsSI}\PYG{p}{(}\PYG{l+s+s2}{\PYGZdq{}}\PYG{l+s+s2}{U}\PYG{l+s+s2}{\PYGZdq{}}\PYG{p}{,} \PYG{l+s+s2}{\PYGZdq{}}\PYG{l+s+s2}{Q}\PYG{l+s+s2}{\PYGZdq{}}\PYG{p}{,} \PYG{l+m+mi}{0}\PYG{p}{,} \PYG{l+s+s2}{\PYGZdq{}}\PYG{l+s+s2}{P}\PYG{l+s+s2}{\PYGZdq{}}\PYG{p}{,} \PYG{n}{P}\PYG{p}{,}\PYG{n}{fluid}\PYG{p}{)}\PYG{o}{/}\PYG{l+m+mf}{1e3}
\PYG{n}{ug} \PYG{o}{=} \PYG{n}{CP}\PYG{o}{.}\PYG{n}{PropsSI}\PYG{p}{(}\PYG{l+s+s2}{\PYGZdq{}}\PYG{l+s+s2}{U}\PYG{l+s+s2}{\PYGZdq{}}\PYG{p}{,} \PYG{l+s+s2}{\PYGZdq{}}\PYG{l+s+s2}{Q}\PYG{l+s+s2}{\PYGZdq{}}\PYG{p}{,} \PYG{l+m+mi}{1}\PYG{p}{,} \PYG{l+s+s2}{\PYGZdq{}}\PYG{l+s+s2}{P}\PYG{l+s+s2}{\PYGZdq{}}\PYG{p}{,} \PYG{n}{P}\PYG{p}{,} \PYG{n}{fluid}\PYG{p}{)}\PYG{o}{/}\PYG{l+m+mf}{1e3}
\PYG{n+nb}{print}\PYG{p}{(}\PYG{l+s+s2}{\PYGZdq{}}\PYG{l+s+s2}{Uf at given pressure: }\PYG{l+s+si}{\PYGZob{}\PYGZcb{}}\PYG{l+s+s2}{ kJ/kg}\PYG{l+s+s2}{\PYGZdq{}}\PYG{o}{.}\PYG{n}{format}\PYG{p}{(}\PYG{n+nb}{round}\PYG{p}{(}\PYG{n}{uf}\PYG{p}{,}\PYG{l+m+mi}{2}\PYG{p}{)}\PYG{p}{)}\PYG{p}{)}
\PYG{n+nb}{print}\PYG{p}{(}\PYG{l+s+s2}{\PYGZdq{}}\PYG{l+s+s2}{Ug at given pressure: }\PYG{l+s+si}{\PYGZob{}\PYGZcb{}}\PYG{l+s+s2}{ kJ/kg}\PYG{l+s+s2}{\PYGZdq{}}\PYG{o}{.}\PYG{n}{format}\PYG{p}{(}\PYG{n+nb}{round}\PYG{p}{(}\PYG{n}{ug}\PYG{p}{,}\PYG{l+m+mi}{2}\PYG{p}{)}\PYG{p}{)}\PYG{p}{)}
\PYG{c+c1}{\PYGZsh{}\PYGZsh{} we are in saturated mixture region; since u\PYGZus{}given \PYGZgt{} uf and u\PYGZus{}given \PYGZlt{} ug!}
\PYG{n}{T} \PYG{o}{=} \PYG{n}{CP}\PYG{o}{.}\PYG{n}{PropsSI}\PYG{p}{(}\PYG{l+s+s2}{\PYGZdq{}}\PYG{l+s+s2}{T}\PYG{l+s+s2}{\PYGZdq{}}\PYG{p}{,}\PYG{l+s+s2}{\PYGZdq{}}\PYG{l+s+s2}{P}\PYG{l+s+s2}{\PYGZdq{}}\PYG{p}{,}\PYG{n}{P}\PYG{p}{,}\PYG{l+s+s2}{\PYGZdq{}}\PYG{l+s+s2}{Q}\PYG{l+s+s2}{\PYGZdq{}}\PYG{p}{,}\PYG{l+m+mi}{0}\PYG{p}{,}\PYG{n}{fluid}\PYG{p}{)}
\PYG{n+nb}{print}\PYG{p}{(}\PYG{l+s+s2}{\PYGZdq{}}\PYG{l+s+s2}{Phase description: Saturated mixture}\PYG{l+s+s2}{\PYGZdq{}}\PYG{p}{)}
\PYG{n+nb}{print}\PYG{p}{(}\PYG{l+s+s2}{\PYGZdq{}}\PYG{l+s+s2}{Temperature: }\PYG{l+s+si}{\PYGZob{}\PYGZcb{}}\PYG{l+s+s2}{\PYGZdq{}}\PYG{o}{.}\PYG{n}{format}\PYG{p}{(}\PYG{n+nb}{round}\PYG{p}{(}\PYG{n}{T} \PYG{o}{\PYGZhy{}} \PYG{l+m+mf}{273.15}\PYG{p}{,}\PYG{l+m+mi}{2}\PYG{p}{)}\PYG{p}{)}\PYG{p}{,}\PYG{l+s+s2}{\PYGZdq{}}\PYG{l+s+s2}{deg C}\PYG{l+s+s2}{\PYGZdq{}}\PYG{p}{)}
\end{sphinxVerbatim}

\end{sphinxuseclass}\end{sphinxVerbatimInput}
\begin{sphinxVerbatimOutput}

\begin{sphinxuseclass}{cell_output}
\begin{sphinxVerbatim}[commandchars=\\\{\}]
Uf at given pressure: 658.15 kJ/kg
Ug at given pressure: 2564.51 kJ/kg
Phase description: Saturated mixture
Temperature: 156.15 deg C
\end{sphinxVerbatim}

\end{sphinxuseclass}\end{sphinxVerbatimOutput}

\end{sphinxuseclass}
\begin{sphinxuseclass}{cell}\begin{sphinxVerbatimInput}

\begin{sphinxuseclass}{cell_input}
\begin{sphinxVerbatim}[commandchars=\\\{\}]
\PYG{c+c1}{\PYGZsh{}\PYGZsh{}\PYGZsh{}========== (b)================\PYGZsh{}\PYGZsh{}\PYGZsh{}}
\PYG{n}{fluid} \PYG{o}{=} \PYG{l+s+s2}{\PYGZdq{}}\PYG{l+s+s2}{water}\PYG{l+s+s2}{\PYGZdq{}}
\PYG{n}{T} \PYG{o}{=} \PYG{l+m+mi}{240} \PYG{o}{+} \PYG{l+m+mf}{273.15} \PYG{c+c1}{\PYGZsh{} Kelvin}
\PYG{n}{uf} \PYG{o}{=} \PYG{n}{CP}\PYG{o}{.}\PYG{n}{PropsSI}\PYG{p}{(}\PYG{l+s+s2}{\PYGZdq{}}\PYG{l+s+s2}{U}\PYG{l+s+s2}{\PYGZdq{}}\PYG{p}{,} \PYG{l+s+s2}{\PYGZdq{}}\PYG{l+s+s2}{Q}\PYG{l+s+s2}{\PYGZdq{}}\PYG{p}{,} \PYG{l+m+mi}{0}\PYG{p}{,} \PYG{l+s+s2}{\PYGZdq{}}\PYG{l+s+s2}{T}\PYG{l+s+s2}{\PYGZdq{}}\PYG{p}{,} \PYG{n}{T}\PYG{p}{,}\PYG{n}{fluid}\PYG{p}{)}\PYG{o}{/}\PYG{l+m+mf}{1e3}
\PYG{n}{ug} \PYG{o}{=} \PYG{n}{CP}\PYG{o}{.}\PYG{n}{PropsSI}\PYG{p}{(}\PYG{l+s+s2}{\PYGZdq{}}\PYG{l+s+s2}{U}\PYG{l+s+s2}{\PYGZdq{}}\PYG{p}{,} \PYG{l+s+s2}{\PYGZdq{}}\PYG{l+s+s2}{Q}\PYG{l+s+s2}{\PYGZdq{}}\PYG{p}{,} \PYG{l+m+mi}{1}\PYG{p}{,} \PYG{l+s+s2}{\PYGZdq{}}\PYG{l+s+s2}{T}\PYG{l+s+s2}{\PYGZdq{}}\PYG{p}{,} \PYG{n}{T}\PYG{p}{,} \PYG{n}{fluid}\PYG{p}{)}\PYG{o}{/}\PYG{l+m+mf}{1e3}
\PYG{n+nb}{print}\PYG{p}{(}\PYG{l+s+s2}{\PYGZdq{}}\PYG{l+s+s2}{U at given temperature : }\PYG{l+s+si}{\PYGZob{}\PYGZcb{}}\PYG{l+s+s2}{ kJ/kg}\PYG{l+s+s2}{\PYGZdq{}}\PYG{o}{.}\PYG{n}{format}\PYG{p}{(}\PYG{n+nb}{round}\PYG{p}{(}\PYG{n}{ug}\PYG{p}{,}\PYG{l+m+mi}{1}\PYG{p}{)}\PYG{p}{)}\PYG{p}{)}
\PYG{n}{pressure} \PYG{o}{=} \PYG{n}{CP}\PYG{o}{.}\PYG{n}{PropsSI}\PYG{p}{(}\PYG{l+s+s2}{\PYGZdq{}}\PYG{l+s+s2}{P}\PYG{l+s+s2}{\PYGZdq{}}\PYG{p}{,} \PYG{l+s+s2}{\PYGZdq{}}\PYG{l+s+s2}{T}\PYG{l+s+s2}{\PYGZdq{}}\PYG{p}{,} \PYG{n}{T}\PYG{p}{,} \PYG{l+s+s2}{\PYGZdq{}}\PYG{l+s+s2}{Q}\PYG{l+s+s2}{\PYGZdq{}}\PYG{p}{,} \PYG{l+m+mi}{1}\PYG{p}{,} \PYG{n}{fluid}\PYG{p}{)}
\PYG{n+nb}{print}\PYG{p}{(}\PYG{l+s+s2}{\PYGZdq{}}\PYG{l+s+s2}{Pressure: }\PYG{l+s+si}{\PYGZob{}\PYGZcb{}}\PYG{l+s+s2}{ kPa}\PYG{l+s+s2}{\PYGZdq{}}\PYG{o}{.}\PYG{n}{format}\PYG{p}{(}\PYG{n+nb}{round}\PYG{p}{(}\PYG{n}{pressure} \PYG{o}{/}\PYG{l+m+mf}{1e3}\PYG{p}{,}\PYG{l+m+mi}{1}\PYG{p}{)}\PYG{p}{)}\PYG{p}{)}
\end{sphinxVerbatim}

\end{sphinxuseclass}\end{sphinxVerbatimInput}
\begin{sphinxVerbatimOutput}

\begin{sphinxuseclass}{cell_output}
\begin{sphinxVerbatim}[commandchars=\\\{\}]
U at given temperature : 2603.1 kJ/kg
Pressure: 3346.9 kPa
\end{sphinxVerbatim}

\end{sphinxuseclass}\end{sphinxVerbatimOutput}

\end{sphinxuseclass}
\begin{sphinxuseclass}{cell}\begin{sphinxVerbatimInput}

\begin{sphinxuseclass}{cell_input}
\begin{sphinxVerbatim}[commandchars=\\\{\}]
\PYG{c+c1}{\PYGZsh{}\PYGZsh{}\PYGZsh{}========== (c) ================\PYGZsh{}\PYGZsh{}\PYGZsh{}}
\PYG{n}{fluid} \PYG{o}{=} \PYG{l+s+s2}{\PYGZdq{}}\PYG{l+s+s2}{water}\PYG{l+s+s2}{\PYGZdq{}}
\PYG{n}{T} \PYG{o}{=} \PYG{l+m+mi}{150} \PYG{o}{+} \PYG{l+m+mf}{273.15} \PYG{c+c1}{\PYGZsh{} Kelvin}
\PYG{n}{P} \PYG{o}{=} \PYG{l+m+mf}{2600e3} \PYG{c+c1}{\PYGZsh{} Pa}
\PYG{n}{Psat} \PYG{o}{=} \PYG{n}{CP}\PYG{o}{.}\PYG{n}{PropsSI}\PYG{p}{(}\PYG{l+s+s2}{\PYGZdq{}}\PYG{l+s+s2}{P}\PYG{l+s+s2}{\PYGZdq{}}\PYG{p}{,} \PYG{l+s+s2}{\PYGZdq{}}\PYG{l+s+s2}{T}\PYG{l+s+s2}{\PYGZdq{}}\PYG{p}{,} \PYG{n}{T}\PYG{p}{,} \PYG{l+s+s2}{\PYGZdq{}}\PYG{l+s+s2}{Q}\PYG{l+s+s2}{\PYGZdq{}}\PYG{p}{,}\PYG{l+m+mi}{0}\PYG{p}{,}\PYG{n}{fluid}\PYG{p}{)}\PYG{o}{/}\PYG{l+m+mf}{1e3}
\PYG{c+c1}{\PYGZsh{}\PYGZsh{} P \PYGZgt{} Psat; hence compressed fluid}
\PYG{n+nb}{print}\PYG{p}{(}\PYG{l+s+s2}{\PYGZdq{}}\PYG{l+s+s2}{Phase description: Compressed liquid}\PYG{l+s+s2}{\PYGZdq{}}\PYG{p}{)}
\PYG{n}{u\PYGZus{}state} \PYG{o}{=} \PYG{n}{CP}\PYG{o}{.}\PYG{n}{PropsSI}\PYG{p}{(}\PYG{l+s+s2}{\PYGZdq{}}\PYG{l+s+s2}{U}\PYG{l+s+s2}{\PYGZdq{}}\PYG{p}{,} \PYG{l+s+s2}{\PYGZdq{}}\PYG{l+s+s2}{T}\PYG{l+s+s2}{\PYGZdq{}}\PYG{p}{,} \PYG{n}{T}\PYG{p}{,} \PYG{l+s+s2}{\PYGZdq{}}\PYG{l+s+s2}{P}\PYG{l+s+s2}{\PYGZdq{}}\PYG{p}{,} \PYG{n}{P}\PYG{p}{,} \PYG{n}{fluid}\PYG{p}{)}
\PYG{n}{uf} \PYG{o}{=} \PYG{n}{CP}\PYG{o}{.}\PYG{n}{PropsSI}\PYG{p}{(}\PYG{l+s+s2}{\PYGZdq{}}\PYG{l+s+s2}{U}\PYG{l+s+s2}{\PYGZdq{}}\PYG{p}{,} \PYG{l+s+s2}{\PYGZdq{}}\PYG{l+s+s2}{T}\PYG{l+s+s2}{\PYGZdq{}}\PYG{p}{,} \PYG{n}{T}\PYG{p}{,} \PYG{l+s+s2}{\PYGZdq{}}\PYG{l+s+s2}{Q}\PYG{l+s+s2}{\PYGZdq{}}\PYG{p}{,} \PYG{l+m+mi}{0}\PYG{p}{,} \PYG{n}{fluid}\PYG{p}{)}
\PYG{n+nb}{print}\PYG{p}{(}\PYG{l+s+s2}{\PYGZdq{}}\PYG{l+s+s2}{u = }\PYG{l+s+si}{\PYGZob{}\PYGZcb{}}\PYG{l+s+s2}{ kJ/kg}\PYG{l+s+s2}{\PYGZdq{}}\PYG{o}{.}\PYG{n}{format}\PYG{p}{(}\PYG{n+nb}{round}\PYG{p}{(}\PYG{n}{u\PYGZus{}state}\PYG{o}{/}\PYG{l+m+mf}{1e3}\PYG{p}{,}\PYG{l+m+mi}{2}\PYG{p}{)}\PYG{p}{)}\PYG{p}{)}
\PYG{n+nb}{print}\PYG{p}{(}\PYG{l+s+s2}{\PYGZdq{}}\PYG{l+s+s2}{uf  = }\PYG{l+s+si}{\PYGZob{}\PYGZcb{}}\PYG{l+s+s2}{ kJ/kg}\PYG{l+s+s2}{\PYGZdq{}}\PYG{o}{.}\PYG{n}{format}\PYG{p}{(}\PYG{n+nb}{round}\PYG{p}{(}\PYG{n}{uf}\PYG{o}{/}\PYG{l+m+mf}{1e3}\PYG{p}{,}\PYG{l+m+mi}{2}\PYG{p}{)}\PYG{p}{)}\PYG{p}{)}
\end{sphinxVerbatim}

\end{sphinxuseclass}\end{sphinxVerbatimInput}
\begin{sphinxVerbatimOutput}

\begin{sphinxuseclass}{cell_output}
\begin{sphinxVerbatim}[commandchars=\\\{\}]
Phase description: Compressed liquid
u = 630.66 kJ/kg
uf  = 631.66 kJ/kg
\end{sphinxVerbatim}

\end{sphinxuseclass}\end{sphinxVerbatimOutput}

\end{sphinxuseclass}
\begin{sphinxuseclass}{cell}\begin{sphinxVerbatimInput}

\begin{sphinxuseclass}{cell_input}
\begin{sphinxVerbatim}[commandchars=\\\{\}]
\PYG{c+c1}{\PYGZsh{}\PYGZsh{}\PYGZsh{}========== (d)================\PYGZsh{}\PYGZsh{}\PYGZsh{}}
\PYG{n}{fluid} \PYG{o}{=} \PYG{l+s+s2}{\PYGZdq{}}\PYG{l+s+s2}{water}\PYG{l+s+s2}{\PYGZdq{}}
\PYG{n}{pressure} \PYG{o}{=} \PYG{l+m+mf}{4200e3} \PYG{c+c1}{\PYGZsh{} Pa}
\PYG{n}{u\PYGZus{}given\PYGZus{}state} \PYG{o}{=} \PYG{l+m+mf}{3100e3} \PYG{c+c1}{\PYGZsh{} in J/kg}
\PYG{c+c1}{\PYGZsh{} calculating u at q = 0 (sat. liquid)}
\PYG{c+c1}{\PYGZsh{}\PYGZsh{} add comments to help the audience to help them to understand the arguments that you supply}
\PYG{n}{uf} \PYG{o}{=} \PYG{n}{CP}\PYG{o}{.}\PYG{n}{PropsSI}\PYG{p}{(}\PYG{l+s+s2}{\PYGZdq{}}\PYG{l+s+s2}{U}\PYG{l+s+s2}{\PYGZdq{}}\PYG{p}{,} \PYG{l+s+s2}{\PYGZdq{}}\PYG{l+s+s2}{Q}\PYG{l+s+s2}{\PYGZdq{}}\PYG{p}{,} \PYG{l+m+mi}{0}\PYG{p}{,} \PYG{l+s+s2}{\PYGZdq{}}\PYG{l+s+s2}{P}\PYG{l+s+s2}{\PYGZdq{}}\PYG{p}{,} \PYG{n}{pressure}\PYG{p}{,}\PYG{n}{fluid}\PYG{p}{)}\PYG{o}{/}\PYG{l+m+mf}{1e3}
\PYG{n}{ug} \PYG{o}{=} \PYG{n}{CP}\PYG{o}{.}\PYG{n}{PropsSI}\PYG{p}{(}\PYG{l+s+s2}{\PYGZdq{}}\PYG{l+s+s2}{U}\PYG{l+s+s2}{\PYGZdq{}}\PYG{p}{,} \PYG{l+s+s2}{\PYGZdq{}}\PYG{l+s+s2}{Q}\PYG{l+s+s2}{\PYGZdq{}}\PYG{p}{,} \PYG{l+m+mi}{1}\PYG{p}{,} \PYG{l+s+s2}{\PYGZdq{}}\PYG{l+s+s2}{P}\PYG{l+s+s2}{\PYGZdq{}}\PYG{p}{,} \PYG{n}{pressure}\PYG{p}{,} \PYG{n}{fluid}\PYG{p}{)}\PYG{o}{/}\PYG{l+m+mf}{1e3}
\PYG{n+nb}{print}\PYG{p}{(}\PYG{l+s+s2}{\PYGZdq{}}\PYG{l+s+s2}{Uf at given pressure: }\PYG{l+s+si}{\PYGZob{}\PYGZcb{}}\PYG{l+s+s2}{ kJ/kg}\PYG{l+s+s2}{\PYGZdq{}}\PYG{o}{.}\PYG{n}{format}\PYG{p}{(}\PYG{n+nb}{round}\PYG{p}{(}\PYG{n}{uf}\PYG{p}{,}\PYG{l+m+mi}{1}\PYG{p}{)}\PYG{p}{)}\PYG{p}{)}
\PYG{n+nb}{print}\PYG{p}{(}\PYG{l+s+s2}{\PYGZdq{}}\PYG{l+s+s2}{Ug at given pressure: }\PYG{l+s+si}{\PYGZob{}\PYGZcb{}}\PYG{l+s+s2}{ kJ/kg}\PYG{l+s+s2}{\PYGZdq{}}\PYG{o}{.}\PYG{n}{format}\PYG{p}{(}\PYG{n+nb}{round}\PYG{p}{(}\PYG{n}{ug}\PYG{p}{,}\PYG{l+m+mi}{1}\PYG{p}{)}\PYG{p}{)}\PYG{p}{)}
\PYG{c+c1}{\PYGZsh{}\PYGZsh{} we are in ssuperheated vapor region; u \PYGZgt{} ug!}
\PYG{n}{T} \PYG{o}{=} \PYG{n}{CP}\PYG{o}{.}\PYG{n}{PropsSI}\PYG{p}{(}\PYG{l+s+s2}{\PYGZdq{}}\PYG{l+s+s2}{T}\PYG{l+s+s2}{\PYGZdq{}}\PYG{p}{,} \PYG{l+s+s2}{\PYGZdq{}}\PYG{l+s+s2}{P}\PYG{l+s+s2}{\PYGZdq{}}\PYG{p}{,} \PYG{n}{pressure}\PYG{p}{,} \PYG{l+s+s2}{\PYGZdq{}}\PYG{l+s+s2}{U}\PYG{l+s+s2}{\PYGZdq{}}\PYG{p}{,} \PYG{n}{u\PYGZus{}given\PYGZus{}state}\PYG{p}{,}\PYG{n}{fluid}\PYG{p}{)}
\PYG{n+nb}{print}\PYG{p}{(}\PYG{l+s+s2}{\PYGZdq{}}\PYG{l+s+s2}{Temperature at the final state in deg C: }\PYG{l+s+si}{\PYGZob{}\PYGZcb{}}\PYG{l+s+s2}{\PYGZdq{}}\PYG{o}{.}\PYG{n}{format}\PYG{p}{(}\PYG{n+nb}{round}\PYG{p}{(}\PYG{n}{T}\PYG{o}{\PYGZhy{}}\PYG{l+m+mf}{273.15}\PYG{p}{,}\PYG{l+m+mi}{2}\PYG{p}{)}\PYG{p}{)}\PYG{p}{)}
\end{sphinxVerbatim}

\end{sphinxuseclass}\end{sphinxVerbatimInput}
\begin{sphinxVerbatimOutput}

\begin{sphinxuseclass}{cell_output}
\begin{sphinxVerbatim}[commandchars=\\\{\}]
Uf at given pressure: 1096.4 kJ/kg
Ug at given pressure: 2601.0 kJ/kg
Temperature at the final state in deg C: 500.8
\end{sphinxVerbatim}

\end{sphinxuseclass}\end{sphinxVerbatimOutput}

\end{sphinxuseclass}

\subsection{Completed table:}
\label{\detokenize{notebooks/Chapter2/complete-table-water:completed-table}}

\begin{savenotes}\sphinxattablestart
\centering
\begin{tabulary}{\linewidth}[t]{|T|T|T|T|}
\hline
\sphinxstyletheadfamily 
\sphinxAtStartPar
T, degC
&\sphinxstyletheadfamily 
\sphinxAtStartPar
P, kPa
&\sphinxstyletheadfamily 
\sphinxAtStartPar
u, kJ/kg
&\sphinxstyletheadfamily 
\sphinxAtStartPar
Phase description
\\
\hline
\sphinxAtStartPar
\sphinxstylestrong{156.15}
&
\sphinxAtStartPar
560
&
\sphinxAtStartPar
1470
&
\sphinxAtStartPar
\sphinxstylestrong{Saturated liq\sphinxhyphen{}vapour mixture}
\\
\hline
\sphinxAtStartPar
240
&
\sphinxAtStartPar
\sphinxstylestrong{3346.92}
&
\sphinxAtStartPar
\sphinxstylestrong{2603.12}
&
\sphinxAtStartPar
Saturated vapor
\\
\hline
\sphinxAtStartPar
150
&
\sphinxAtStartPar
2600
&
\sphinxAtStartPar
\sphinxstylestrong{630.66}
&
\sphinxAtStartPar
\sphinxstylestrong{Compressed Liquid}
\\
\hline
\sphinxAtStartPar
\sphinxstylestrong{500.8}
&
\sphinxAtStartPar
4200
&
\sphinxAtStartPar
3100
&
\sphinxAtStartPar
\sphinxstylestrong{Superheated vapor}
\\
\hline
\end{tabulary}
\par
\sphinxattableend\end{savenotes}

\sphinxstepscope


\section{Complete the table: R\sphinxhyphen{}134a}
\label{\detokenize{notebooks/Chapter2/complete-table-R134a:complete-the-table-r-134a}}\label{\detokenize{notebooks/Chapter2/complete-table-R134a::doc}}

\begin{savenotes}\sphinxattablestart
\centering
\begin{tabulary}{\linewidth}[t]{|T|T|T|T|}
\hline
\sphinxstyletheadfamily 
\sphinxAtStartPar
T, degC
&\sphinxstyletheadfamily 
\sphinxAtStartPar
P, kPa
&\sphinxstyletheadfamily 
\sphinxAtStartPar
u, kJ/kg
&\sphinxstyletheadfamily 
\sphinxAtStartPar
Phase description
\\
\hline
\sphinxAtStartPar

&
\sphinxAtStartPar
800
&
\sphinxAtStartPar
275
&
\sphinxAtStartPar

\\
\hline
\sphinxAtStartPar
\sphinxhyphen{}10
&
\sphinxAtStartPar

&
\sphinxAtStartPar

&
\sphinxAtStartPar
Saturated vapor
\\
\hline
\sphinxAtStartPar
30
&
\sphinxAtStartPar
1200
&
\sphinxAtStartPar

&
\sphinxAtStartPar

\\
\hline
\sphinxAtStartPar

&
\sphinxAtStartPar
2000
&
\sphinxAtStartPar
250
&
\sphinxAtStartPar

\\
\hline
\end{tabulary}
\par
\sphinxattableend\end{savenotes}

\begin{sphinxuseclass}{cell}\begin{sphinxVerbatimInput}

\begin{sphinxuseclass}{cell_input}
\begin{sphinxVerbatim}[commandchars=\\\{\}]
\PYG{k+kn}{from} \PYG{n+nn}{CoolProp} \PYG{k+kn}{import} \PYG{n}{CoolProp} \PYG{k}{as} \PYG{n}{CP}
\PYG{c+c1}{\PYGZsh{}\PYGZsh{}\PYGZsh{}========== (a)================\PYGZsh{}\PYGZsh{}\PYGZsh{}}
\PYG{n}{P} \PYG{o}{=} \PYG{l+m+mf}{0.8e6} \PYG{c+c1}{\PYGZsh{} in Pa}
\PYG{n}{u} \PYG{o}{=} \PYG{l+m+mi}{275} \PYG{c+c1}{\PYGZsh{} in kJ/kg}
\PYG{n}{fluid} \PYG{o}{=} \PYG{l+s+s2}{\PYGZdq{}}\PYG{l+s+s2}{R134a}\PYG{l+s+s2}{\PYGZdq{}}
\PYG{n}{uf} \PYG{o}{=} \PYG{n}{CP}\PYG{o}{.}\PYG{n}{PropsSI}\PYG{p}{(}\PYG{l+s+s2}{\PYGZdq{}}\PYG{l+s+s2}{U}\PYG{l+s+s2}{\PYGZdq{}}\PYG{p}{,} \PYG{l+s+s2}{\PYGZdq{}}\PYG{l+s+s2}{Q}\PYG{l+s+s2}{\PYGZdq{}}\PYG{p}{,} \PYG{l+m+mi}{0}\PYG{p}{,} \PYG{l+s+s2}{\PYGZdq{}}\PYG{l+s+s2}{P}\PYG{l+s+s2}{\PYGZdq{}}\PYG{p}{,} \PYG{n}{P}\PYG{p}{,}\PYG{n}{fluid}\PYG{p}{)}\PYG{o}{/}\PYG{l+m+mf}{1e3}
\PYG{n}{ug} \PYG{o}{=} \PYG{n}{CP}\PYG{o}{.}\PYG{n}{PropsSI}\PYG{p}{(}\PYG{l+s+s2}{\PYGZdq{}}\PYG{l+s+s2}{U}\PYG{l+s+s2}{\PYGZdq{}}\PYG{p}{,} \PYG{l+s+s2}{\PYGZdq{}}\PYG{l+s+s2}{Q}\PYG{l+s+s2}{\PYGZdq{}}\PYG{p}{,} \PYG{l+m+mi}{1}\PYG{p}{,} \PYG{l+s+s2}{\PYGZdq{}}\PYG{l+s+s2}{P}\PYG{l+s+s2}{\PYGZdq{}}\PYG{p}{,} \PYG{n}{P}\PYG{p}{,} \PYG{n}{fluid}\PYG{p}{)}\PYG{o}{/}\PYG{l+m+mf}{1e3}
\PYG{n+nb}{print}\PYG{p}{(}\PYG{l+s+s2}{\PYGZdq{}}\PYG{l+s+s2}{Uf at given pressure: }\PYG{l+s+si}{\PYGZob{}\PYGZcb{}}\PYG{l+s+s2}{ kJ/kg}\PYG{l+s+s2}{\PYGZdq{}}\PYG{o}{.}\PYG{n}{format}\PYG{p}{(}\PYG{n+nb}{round}\PYG{p}{(}\PYG{n}{uf}\PYG{p}{,}\PYG{l+m+mi}{1}\PYG{p}{)}\PYG{p}{)}\PYG{p}{)}
\PYG{n+nb}{print}\PYG{p}{(}\PYG{l+s+s2}{\PYGZdq{}}\PYG{l+s+s2}{Ug at given pressure: }\PYG{l+s+si}{\PYGZob{}\PYGZcb{}}\PYG{l+s+s2}{ kJ/kg}\PYG{l+s+s2}{\PYGZdq{}}\PYG{o}{.}\PYG{n}{format}\PYG{p}{(}\PYG{n+nb}{round}\PYG{p}{(}\PYG{n}{ug}\PYG{p}{,}\PYG{l+m+mi}{1}\PYG{p}{)}\PYG{p}{)}\PYG{p}{)}
\PYG{c+c1}{\PYGZsh{}\PYGZsh{} we are in saturated mixture region; since u\PYGZus{}given \PYGZgt{} uf and u\PYGZus{}given \PYGZlt{} ug!}
\PYG{n}{T} \PYG{o}{=} \PYG{n}{CP}\PYG{o}{.}\PYG{n}{PropsSI}\PYG{p}{(}\PYG{l+s+s2}{\PYGZdq{}}\PYG{l+s+s2}{T}\PYG{l+s+s2}{\PYGZdq{}}\PYG{p}{,}\PYG{l+s+s2}{\PYGZdq{}}\PYG{l+s+s2}{P}\PYG{l+s+s2}{\PYGZdq{}}\PYG{p}{,}\PYG{n}{P}\PYG{p}{,}\PYG{l+s+s2}{\PYGZdq{}}\PYG{l+s+s2}{Q}\PYG{l+s+s2}{\PYGZdq{}}\PYG{p}{,}\PYG{l+m+mi}{0}\PYG{p}{,}\PYG{n}{fluid}\PYG{p}{)}
\PYG{n+nb}{print}\PYG{p}{(}\PYG{l+s+s2}{\PYGZdq{}}\PYG{l+s+s2}{Phase description: Saturated mixture}\PYG{l+s+s2}{\PYGZdq{}}\PYG{p}{)}
\PYG{n+nb}{print}\PYG{p}{(}\PYG{l+s+s2}{\PYGZdq{}}\PYG{l+s+s2}{Temperature: }\PYG{l+s+si}{\PYGZob{}\PYGZcb{}}\PYG{l+s+s2}{\PYGZdq{}}\PYG{o}{.}\PYG{n}{format}\PYG{p}{(}\PYG{n+nb}{round}\PYG{p}{(}\PYG{n}{T} \PYG{o}{\PYGZhy{}} \PYG{l+m+mf}{273.15}\PYG{p}{,}\PYG{l+m+mi}{2}\PYG{p}{)}\PYG{p}{)}\PYG{p}{,}\PYG{l+s+s2}{\PYGZdq{}}\PYG{l+s+s2}{deg C}\PYG{l+s+s2}{\PYGZdq{}}\PYG{p}{)}
\end{sphinxVerbatim}

\end{sphinxuseclass}\end{sphinxVerbatimInput}
\begin{sphinxVerbatimOutput}

\begin{sphinxuseclass}{cell_output}
\begin{sphinxVerbatim}[commandchars=\\\{\}]
Uf at given pressure: 243.0 kJ/kg
Ug at given pressure: 395.0 kJ/kg
Phase description: Saturated mixture
Temperature: 31.33 deg C
\end{sphinxVerbatim}

\end{sphinxuseclass}\end{sphinxVerbatimOutput}

\end{sphinxuseclass}
\begin{sphinxuseclass}{cell}\begin{sphinxVerbatimInput}

\begin{sphinxuseclass}{cell_input}
\begin{sphinxVerbatim}[commandchars=\\\{\}]
\PYG{c+c1}{\PYGZsh{}\PYGZsh{}\PYGZsh{}========== (b)================\PYGZsh{}\PYGZsh{}\PYGZsh{}}
\PYG{n}{T} \PYG{o}{=} \PYG{o}{\PYGZhy{}}\PYG{l+m+mi}{10} \PYG{o}{+} \PYG{l+m+mf}{273.15} \PYG{c+c1}{\PYGZsh{}\PYGZsh{} K}
\PYG{n}{fluid} \PYG{o}{=} \PYG{l+s+s2}{\PYGZdq{}}\PYG{l+s+s2}{R134a}\PYG{l+s+s2}{\PYGZdq{}}
\PYG{n}{press1} \PYG{o}{=} \PYG{n}{CP}\PYG{o}{.}\PYG{n}{PropsSI}\PYG{p}{(}\PYG{l+s+s2}{\PYGZdq{}}\PYG{l+s+s2}{P}\PYG{l+s+s2}{\PYGZdq{}}\PYG{p}{,}\PYG{l+s+s2}{\PYGZdq{}}\PYG{l+s+s2}{T}\PYG{l+s+s2}{\PYGZdq{}}\PYG{p}{,}\PYG{n}{T}\PYG{p}{,}\PYG{l+s+s2}{\PYGZdq{}}\PYG{l+s+s2}{Q}\PYG{l+s+s2}{\PYGZdq{}}\PYG{p}{,}\PYG{l+m+mi}{1}\PYG{p}{,}\PYG{n}{fluid}\PYG{p}{)}\PYG{o}{/}\PYG{l+m+mf}{1e3}
\PYG{n}{ug} \PYG{o}{=} \PYG{n}{CP}\PYG{o}{.}\PYG{n}{PropsSI}\PYG{p}{(}\PYG{l+s+s2}{\PYGZdq{}}\PYG{l+s+s2}{U}\PYG{l+s+s2}{\PYGZdq{}}\PYG{p}{,} \PYG{l+s+s2}{\PYGZdq{}}\PYG{l+s+s2}{Q}\PYG{l+s+s2}{\PYGZdq{}}\PYG{p}{,} \PYG{l+m+mi}{1}\PYG{p}{,} \PYG{l+s+s2}{\PYGZdq{}}\PYG{l+s+s2}{T}\PYG{l+s+s2}{\PYGZdq{}}\PYG{p}{,} \PYG{n}{T}\PYG{p}{,} \PYG{n}{fluid}\PYG{p}{)}\PYG{o}{/}\PYG{l+m+mf}{1e3}
\PYG{n+nb}{print}\PYG{p}{(}\PYG{l+s+s2}{\PYGZdq{}}\PYG{l+s+s2}{U = Ug at given temperature: }\PYG{l+s+si}{\PYGZob{}\PYGZcb{}}\PYG{l+s+s2}{ kJ/kg}\PYG{l+s+s2}{\PYGZdq{}}\PYG{o}{.}\PYG{n}{format}\PYG{p}{(}\PYG{n+nb}{round}\PYG{p}{(}\PYG{n}{ug}\PYG{p}{,}\PYG{l+m+mi}{2}\PYG{p}{)}\PYG{p}{)}\PYG{p}{)}
\PYG{n+nb}{print}\PYG{p}{(}\PYG{l+s+s2}{\PYGZdq{}}\PYG{l+s+s2}{Pressure = }\PYG{l+s+si}{\PYGZob{}\PYGZcb{}}\PYG{l+s+s2}{ kPa}\PYG{l+s+s2}{\PYGZdq{}}\PYG{o}{.}\PYG{n}{format}\PYG{p}{(}\PYG{n+nb}{round}\PYG{p}{(}\PYG{n}{press1}\PYG{p}{,}\PYG{l+m+mi}{1}\PYG{p}{)}\PYG{p}{)}\PYG{p}{)}
\end{sphinxVerbatim}

\end{sphinxuseclass}\end{sphinxVerbatimInput}
\begin{sphinxVerbatimOutput}

\begin{sphinxuseclass}{cell_output}
\begin{sphinxVerbatim}[commandchars=\\\{\}]
U = Ug at given temperature: 372.69 kJ/kg
Pressure = 200.6 kPa
\end{sphinxVerbatim}

\end{sphinxuseclass}\end{sphinxVerbatimOutput}

\end{sphinxuseclass}
\begin{sphinxuseclass}{cell}\begin{sphinxVerbatimInput}

\begin{sphinxuseclass}{cell_input}
\begin{sphinxVerbatim}[commandchars=\\\{\}]
\PYG{c+c1}{\PYGZsh{}\PYGZsh{}\PYGZsh{}========== (c)================\PYGZsh{}\PYGZsh{}\PYGZsh{}}
\PYG{n}{T} \PYG{o}{=} \PYG{l+m+mi}{30} \PYG{o}{+} \PYG{l+m+mf}{273.15} \PYG{c+c1}{\PYGZsh{}\PYGZsh{} K}
\PYG{n}{P} \PYG{o}{=} \PYG{l+m+mf}{1200e3} \PYG{c+c1}{\PYGZsh{} Pa}
\PYG{n}{fluid} \PYG{o}{=} \PYG{l+s+s2}{\PYGZdq{}}\PYG{l+s+s2}{R134a}\PYG{l+s+s2}{\PYGZdq{}}
\PYG{n}{press\PYGZus{}at\PYGZus{}given\PYGZus{}temp} \PYG{o}{=} \PYG{n}{CP}\PYG{o}{.}\PYG{n}{PropsSI}\PYG{p}{(}\PYG{l+s+s2}{\PYGZdq{}}\PYG{l+s+s2}{P}\PYG{l+s+s2}{\PYGZdq{}}\PYG{p}{,}\PYG{l+s+s2}{\PYGZdq{}}\PYG{l+s+s2}{T}\PYG{l+s+s2}{\PYGZdq{}}\PYG{p}{,}\PYG{n}{T}\PYG{p}{,}\PYG{l+s+s2}{\PYGZdq{}}\PYG{l+s+s2}{Q}\PYG{l+s+s2}{\PYGZdq{}}\PYG{p}{,}\PYG{l+m+mi}{0}\PYG{p}{,}\PYG{n}{fluid}\PYG{p}{)} \PYG{o}{/}\PYG{l+m+mf}{1e3}
\PYG{n+nb}{print}\PYG{p}{(}\PYG{l+s+s2}{\PYGZdq{}}\PYG{l+s+s2}{Pressure = }\PYG{l+s+si}{\PYGZob{}\PYGZcb{}}\PYG{l+s+s2}{ kPa}\PYG{l+s+s2}{\PYGZdq{}}\PYG{o}{.}\PYG{n}{format}\PYG{p}{(}\PYG{n+nb}{round}\PYG{p}{(}\PYG{n}{press\PYGZus{}at\PYGZus{}given\PYGZus{}temp}\PYG{p}{,}\PYG{l+m+mi}{1}\PYG{p}{)}\PYG{p}{)}\PYG{p}{)}
\PYG{n}{u}\PYG{o}{=} \PYG{n}{CP}\PYG{o}{.}\PYG{n}{PropsSI}\PYG{p}{(}\PYG{l+s+s2}{\PYGZdq{}}\PYG{l+s+s2}{U}\PYG{l+s+s2}{\PYGZdq{}}\PYG{p}{,} \PYG{l+s+s2}{\PYGZdq{}}\PYG{l+s+s2}{P}\PYG{l+s+s2}{\PYGZdq{}}\PYG{p}{,} \PYG{n}{P}\PYG{p}{,} \PYG{l+s+s2}{\PYGZdq{}}\PYG{l+s+s2}{T}\PYG{l+s+s2}{\PYGZdq{}}\PYG{p}{,} \PYG{n}{T}\PYG{p}{,} \PYG{n}{fluid}\PYG{p}{)}\PYG{o}{/}\PYG{l+m+mf}{1e3}
\PYG{n+nb}{print}\PYG{p}{(}\PYG{l+s+s2}{\PYGZdq{}}\PYG{l+s+s2}{U = }\PYG{l+s+si}{\PYGZob{}\PYGZcb{}}\PYG{l+s+s2}{ kJ/kg}\PYG{l+s+s2}{\PYGZdq{}}\PYG{o}{.}\PYG{n}{format}\PYG{p}{(}\PYG{n+nb}{round}\PYG{p}{(}\PYG{n}{u}\PYG{p}{,}\PYG{l+m+mi}{2}\PYG{p}{)}\PYG{p}{)}\PYG{p}{)}
\PYG{n}{uf}\PYG{o}{=} \PYG{n}{CP}\PYG{o}{.}\PYG{n}{PropsSI}\PYG{p}{(}\PYG{l+s+s2}{\PYGZdq{}}\PYG{l+s+s2}{U}\PYG{l+s+s2}{\PYGZdq{}}\PYG{p}{,} \PYG{l+s+s2}{\PYGZdq{}}\PYG{l+s+s2}{Q}\PYG{l+s+s2}{\PYGZdq{}}\PYG{p}{,}\PYG{l+m+mi}{0}\PYG{p}{,} \PYG{l+s+s2}{\PYGZdq{}}\PYG{l+s+s2}{T}\PYG{l+s+s2}{\PYGZdq{}}\PYG{p}{,} \PYG{n}{T}\PYG{p}{,} \PYG{n}{fluid}\PYG{p}{)}\PYG{o}{/}\PYG{l+m+mf}{1e3}
\PYG{n+nb}{print}\PYG{p}{(}\PYG{l+s+s2}{\PYGZdq{}}\PYG{l+s+s2}{Uf = }\PYG{l+s+si}{\PYGZob{}\PYGZcb{}}\PYG{l+s+s2}{ kJ/kg}\PYG{l+s+s2}{\PYGZdq{}}\PYG{o}{.}\PYG{n}{format}\PYG{p}{(}\PYG{n+nb}{round}\PYG{p}{(}\PYG{n}{uf}\PYG{p}{,}\PYG{l+m+mi}{2}\PYG{p}{)}\PYG{p}{)}\PYG{p}{)}
\end{sphinxVerbatim}

\end{sphinxuseclass}\end{sphinxVerbatimInput}
\begin{sphinxVerbatimOutput}

\begin{sphinxuseclass}{cell_output}
\begin{sphinxVerbatim}[commandchars=\\\{\}]
Pressure = 770.2 kPa
U = 240.7 kJ/kg
Uf = 241.07 kJ/kg
\end{sphinxVerbatim}

\end{sphinxuseclass}\end{sphinxVerbatimOutput}

\end{sphinxuseclass}
\begin{sphinxuseclass}{cell}\begin{sphinxVerbatimInput}

\begin{sphinxuseclass}{cell_input}
\begin{sphinxVerbatim}[commandchars=\\\{\}]
\PYG{c+c1}{\PYGZsh{}\PYGZsh{}\PYGZsh{}========== (d)================\PYGZsh{}\PYGZsh{}\PYGZsh{}}
\PYG{n}{P} \PYG{o}{=} \PYG{l+m+mf}{2000e3} \PYG{c+c1}{\PYGZsh{} Pa}
\PYG{n}{fluid} \PYG{o}{=} \PYG{l+s+s2}{\PYGZdq{}}\PYG{l+s+s2}{R134a}\PYG{l+s+s2}{\PYGZdq{}}
\PYG{n}{T\PYGZus{}sat} \PYG{o}{=} \PYG{n}{CP}\PYG{o}{.}\PYG{n}{PropsSI}\PYG{p}{(}\PYG{l+s+s2}{\PYGZdq{}}\PYG{l+s+s2}{T}\PYG{l+s+s2}{\PYGZdq{}}\PYG{p}{,} \PYG{l+s+s2}{\PYGZdq{}}\PYG{l+s+s2}{P}\PYG{l+s+s2}{\PYGZdq{}}\PYG{p}{,} \PYG{n}{P}\PYG{p}{,}\PYG{l+s+s2}{\PYGZdq{}}\PYG{l+s+s2}{Q}\PYG{l+s+s2}{\PYGZdq{}}\PYG{p}{,}\PYG{l+m+mi}{0}\PYG{p}{,}\PYG{n}{fluid}\PYG{p}{)}
\PYG{n+nb}{print}\PYG{p}{(}\PYG{l+s+s2}{\PYGZdq{}}\PYG{l+s+s2}{Temperature = }\PYG{l+s+si}{\PYGZob{}\PYGZcb{}}\PYG{l+s+s2}{ C}\PYG{l+s+s2}{\PYGZdq{}}\PYG{o}{.}\PYG{n}{format}\PYG{p}{(}\PYG{n+nb}{round}\PYG{p}{(}\PYG{n}{T\PYGZus{}sat}\PYG{o}{\PYGZhy{}}\PYG{l+m+mf}{273.15}\PYG{p}{,}\PYG{l+m+mi}{1}\PYG{p}{)}\PYG{p}{)}\PYG{p}{)}
\PYG{n}{uf}\PYG{o}{=} \PYG{n}{CP}\PYG{o}{.}\PYG{n}{PropsSI}\PYG{p}{(}\PYG{l+s+s2}{\PYGZdq{}}\PYG{l+s+s2}{U}\PYG{l+s+s2}{\PYGZdq{}}\PYG{p}{,} \PYG{l+s+s2}{\PYGZdq{}}\PYG{l+s+s2}{P}\PYG{l+s+s2}{\PYGZdq{}}\PYG{p}{,} \PYG{n}{P}\PYG{p}{,} \PYG{l+s+s2}{\PYGZdq{}}\PYG{l+s+s2}{Q}\PYG{l+s+s2}{\PYGZdq{}}\PYG{p}{,}\PYG{l+m+mi}{0}\PYG{p}{,} \PYG{n}{fluid}\PYG{p}{)}\PYG{o}{/}\PYG{l+m+mf}{1e3}
\PYG{n}{ug}\PYG{o}{=} \PYG{n}{CP}\PYG{o}{.}\PYG{n}{PropsSI}\PYG{p}{(}\PYG{l+s+s2}{\PYGZdq{}}\PYG{l+s+s2}{U}\PYG{l+s+s2}{\PYGZdq{}}\PYG{p}{,} \PYG{l+s+s2}{\PYGZdq{}}\PYG{l+s+s2}{P}\PYG{l+s+s2}{\PYGZdq{}}\PYG{p}{,} \PYG{n}{P}\PYG{p}{,} \PYG{l+s+s2}{\PYGZdq{}}\PYG{l+s+s2}{Q}\PYG{l+s+s2}{\PYGZdq{}}\PYG{p}{,}\PYG{l+m+mi}{1}\PYG{p}{,} \PYG{n}{fluid}\PYG{p}{)}\PYG{o}{/}\PYG{l+m+mf}{1e3}

\PYG{n+nb}{print}\PYG{p}{(}\PYG{l+s+s2}{\PYGZdq{}}\PYG{l+s+s2}{Uf = }\PYG{l+s+si}{\PYGZob{}\PYGZcb{}}\PYG{l+s+s2}{ kJ/kg}\PYG{l+s+s2}{\PYGZdq{}}\PYG{o}{.}\PYG{n}{format}\PYG{p}{(}\PYG{n+nb}{round}\PYG{p}{(}\PYG{n}{uf}\PYG{p}{,}\PYG{l+m+mi}{2}\PYG{p}{)}\PYG{p}{)}\PYG{p}{)}
\PYG{n+nb}{print}\PYG{p}{(}\PYG{l+s+s2}{\PYGZdq{}}\PYG{l+s+s2}{Ug = }\PYG{l+s+si}{\PYGZob{}\PYGZcb{}}\PYG{l+s+s2}{ kJ/kg}\PYG{l+s+s2}{\PYGZdq{}}\PYG{o}{.}\PYG{n}{format}\PYG{p}{(}\PYG{n+nb}{round}\PYG{p}{(}\PYG{n}{ug}\PYG{p}{,}\PYG{l+m+mi}{2}\PYG{p}{)}\PYG{p}{)}\PYG{p}{)}
\PYG{c+c1}{\PYGZsh{}\PYGZsh{} given U is below u\PYGZus{}f; hence compressed liquid}
\PYG{n}{T} \PYG{o}{=} \PYG{n}{CP}\PYG{o}{.}\PYG{n}{PropsSI}\PYG{p}{(}\PYG{l+s+s2}{\PYGZdq{}}\PYG{l+s+s2}{T}\PYG{l+s+s2}{\PYGZdq{}}\PYG{p}{,}\PYG{l+s+s2}{\PYGZdq{}}\PYG{l+s+s2}{U}\PYG{l+s+s2}{\PYGZdq{}}\PYG{p}{,}\PYG{l+m+mf}{250e3}\PYG{p}{,}\PYG{l+s+s2}{\PYGZdq{}}\PYG{l+s+s2}{P}\PYG{l+s+s2}{\PYGZdq{}}\PYG{p}{,}\PYG{n}{P}\PYG{p}{,}\PYG{n}{fluid}\PYG{p}{)} \PYG{o}{\PYGZhy{}} \PYG{l+m+mf}{273.15}
\PYG{n+nb}{print}\PYG{p}{(}\PYG{l+s+s2}{\PYGZdq{}}\PYG{l+s+s2}{Temperature = }\PYG{l+s+si}{\PYGZob{}\PYGZcb{}}\PYG{l+s+s2}{ C}\PYG{l+s+s2}{\PYGZdq{}}\PYG{o}{.}\PYG{n}{format}\PYG{p}{(}\PYG{n+nb}{round}\PYG{p}{(}\PYG{n}{T}\PYG{p}{,}\PYG{l+m+mi}{2}\PYG{p}{)}\PYG{p}{)}\PYG{p}{)}
\end{sphinxVerbatim}

\end{sphinxuseclass}\end{sphinxVerbatimInput}
\begin{sphinxVerbatimOutput}

\begin{sphinxuseclass}{cell_output}
\begin{sphinxVerbatim}[commandchars=\\\{\}]
Temperature = 67.5 C
Uf = 297.98 kJ/kg
Ug = 409.7 kJ/kg
Temperature = 36.92 C
\end{sphinxVerbatim}

\end{sphinxuseclass}\end{sphinxVerbatimOutput}

\end{sphinxuseclass}

\subsection{completed table:}
\label{\detokenize{notebooks/Chapter2/complete-table-R134a:completed-table}}

\begin{savenotes}\sphinxattablestart
\centering
\begin{tabulary}{\linewidth}[t]{|T|T|T|T|}
\hline
\sphinxstyletheadfamily 
\sphinxAtStartPar
T, degC
&\sphinxstyletheadfamily 
\sphinxAtStartPar
P, kPa
&\sphinxstyletheadfamily 
\sphinxAtStartPar
u, kJ/kg
&\sphinxstyletheadfamily 
\sphinxAtStartPar
Phase description
\\
\hline
\sphinxAtStartPar
\sphinxstylestrong{31.33}
&
\sphinxAtStartPar
800
&
\sphinxAtStartPar
275
&
\sphinxAtStartPar
\sphinxstylestrong{Saturated liq\sphinxhyphen{}vapour mixture}
\\
\hline
\sphinxAtStartPar
\sphinxhyphen{}10
&
\sphinxAtStartPar
\sphinxstylestrong{200.6}
&
\sphinxAtStartPar
\sphinxstylestrong{372.687}
&
\sphinxAtStartPar
Saturated vapor
\\
\hline
\sphinxAtStartPar
30
&
\sphinxAtStartPar
1200
&
\sphinxAtStartPar
\sphinxstylestrong{240.7}
&
\sphinxAtStartPar
\sphinxstylestrong{Compressed liquid}
\\
\hline
\sphinxAtStartPar
\sphinxstylestrong{36.92}
&
\sphinxAtStartPar
2000
&
\sphinxAtStartPar
250
&
\sphinxAtStartPar
\sphinxstylestrong{Compressed Liquid}
\\
\hline
\end{tabulary}
\par
\sphinxattableend\end{savenotes}

\sphinxstepscope


\section{Piston\sphinxhyphen{}cylinder: Isobaric process}
\label{\detokenize{notebooks/Chapter2/piston-cylinder:piston-cylinder-isobaric-process}}\label{\detokenize{notebooks/Chapter2/piston-cylinder::doc}}

\subsection{Problem statement:}
\label{\detokenize{notebooks/Chapter2/piston-cylinder:problem-statement}}
\sphinxAtStartPar
A piston–cylinder device contains 0.9 kg of steam at 300°C and 0.8 MPa. Steam is cooled at constant pressure until one\sphinxhyphen{}third of the mass condenses.
(a) Show the process on a T\sphinxhyphen{}v diagram and a P\sphinxhyphen{}v diagram
(b) Find the final temperature.
(c) Determine the volume change

\sphinxAtStartPar
\sphinxincludegraphics{{piston-cylinder}.png}


\subsection{A function definition to help plot P\sphinxhyphen{}v diagram}
\label{\detokenize{notebooks/Chapter2/piston-cylinder:a-function-definition-to-help-plot-p-v-diagram}}
\sphinxAtStartPar
The arguments of this function are supplied inside the paranthesis:

\sphinxAtStartPar
fluid: “Fluid of interest”

\sphinxAtStartPar
state1: {[}P,v{]} list

\sphinxAtStartPar
state2: {[}P,v{]} list

\begin{sphinxuseclass}{cell}\begin{sphinxVerbatimInput}

\begin{sphinxuseclass}{cell_input}
\begin{sphinxVerbatim}[commandchars=\\\{\}]
\PYG{k}{def} \PYG{n+nf}{plot\PYGZus{}p\PYGZus{}v\PYGZus{}diagram}\PYG{p}{(}\PYG{n}{fluid}\PYG{p}{,} \PYG{n}{state1}\PYG{p}{,} \PYG{n}{state2}\PYG{p}{,} \PYG{n}{plot\PYGZus{}const\PYGZus{}press\PYGZus{}line}\PYG{o}{=}\PYG{k+kc}{True}\PYG{p}{)}\PYG{p}{:}
    \PYG{c+c1}{\PYGZsh{} import the libraries we\PYGZsq{}ll need}
    \PYG{k+kn}{import} \PYG{n+nn}{CoolProp}\PYG{n+nn}{.}\PYG{n+nn}{CoolProp} \PYG{k}{as} \PYG{n+nn}{CP}
    \PYG{k+kn}{import} \PYG{n+nn}{numpy} \PYG{k}{as} \PYG{n+nn}{np}
    \PYG{k+kn}{import} \PYG{n+nn}{matplotlib}\PYG{n+nn}{.}\PYG{n+nn}{pyplot} \PYG{k}{as} \PYG{n+nn}{plt}

    \PYG{c+c1}{\PYGZsh{} define variables}
    \PYG{n}{fluid} \PYG{o}{=} \PYG{n}{fluid}  
    \PYG{n}{T\PYGZus{}min} \PYG{o}{=} \PYG{n}{CP}\PYG{o}{.}\PYG{n}{PropsSI}\PYG{p}{(}\PYG{l+s+s2}{\PYGZdq{}}\PYG{l+s+s2}{Tmin}\PYG{l+s+s2}{\PYGZdq{}}\PYG{p}{,} \PYG{n}{fluid}\PYG{p}{)} 
    \PYG{n}{T\PYGZus{}max} \PYG{o}{=} \PYG{n}{CP}\PYG{o}{.}\PYG{n}{PropsSI}\PYG{p}{(}\PYG{l+s+s2}{\PYGZdq{}}\PYG{l+s+s2}{Tcrit}\PYG{l+s+s2}{\PYGZdq{}}\PYG{p}{,} \PYG{n}{fluid}\PYG{p}{)} 
    \PYG{n}{T\PYGZus{}vals} \PYG{o}{=} \PYG{n}{np}\PYG{o}{.}\PYG{n}{linspace}\PYG{p}{(}\PYG{n}{T\PYGZus{}min}\PYG{p}{,} \PYG{n}{T\PYGZus{}max}\PYG{p}{,} \PYG{l+m+mi}{1000}\PYG{p}{)} 
    \PYG{n}{Q} \PYG{o}{=} \PYG{l+m+mi}{1}

    \PYG{n}{P\PYGZus{}saturated\PYGZus{}vapor} \PYG{o}{=} \PYG{p}{[}\PYG{n}{CP}\PYG{o}{.}\PYG{n}{PropsSI}\PYG{p}{(}\PYG{l+s+s2}{\PYGZdq{}}\PYG{l+s+s2}{P}\PYG{l+s+s2}{\PYGZdq{}}\PYG{p}{,} \PYG{l+s+s2}{\PYGZdq{}}\PYG{l+s+s2}{T}\PYG{l+s+s2}{\PYGZdq{}}\PYG{p}{,} \PYG{n}{T}\PYG{p}{,} \PYG{l+s+s2}{\PYGZdq{}}\PYG{l+s+s2}{Q}\PYG{l+s+s2}{\PYGZdq{}}\PYG{p}{,} \PYG{n}{Q}\PYG{p}{,} \PYG{n}{fluid}\PYG{p}{)} \PYG{k}{for} \PYG{n}{T} \PYG{o+ow}{in} \PYG{n}{T\PYGZus{}vals}\PYG{p}{]}
    \PYG{n}{vol\PYGZus{}vapor} \PYG{o}{=} \PYG{l+m+mi}{1} \PYG{o}{/} \PYG{n}{np}\PYG{o}{.}\PYG{n}{array}\PYG{p}{(}\PYG{p}{[}\PYG{n}{CP}\PYG{o}{.}\PYG{n}{PropsSI}\PYG{p}{(}\PYG{l+s+s2}{\PYGZdq{}}\PYG{l+s+s2}{D}\PYG{l+s+s2}{\PYGZdq{}}\PYG{p}{,} \PYG{l+s+s2}{\PYGZdq{}}\PYG{l+s+s2}{T}\PYG{l+s+s2}{\PYGZdq{}}\PYG{p}{,} \PYG{n}{T}\PYG{p}{,} \PYG{l+s+s2}{\PYGZdq{}}\PYG{l+s+s2}{Q}\PYG{l+s+s2}{\PYGZdq{}}\PYG{p}{,} \PYG{n}{Q}\PYG{p}{,} \PYG{n}{fluid}\PYG{p}{)} \PYG{k}{for} \PYG{n}{T} \PYG{o+ow}{in} \PYG{n}{T\PYGZus{}vals}\PYG{p}{]}\PYG{p}{)}
    \PYG{n}{plt}\PYG{o}{.}\PYG{n}{figure}\PYG{p}{(}\PYG{l+m+mi}{2}\PYG{p}{)}
    \PYG{n}{plt}\PYG{o}{.}\PYG{n}{plot}\PYG{p}{(}\PYG{n}{vol\PYGZus{}vapor}\PYG{p}{,} \PYG{n}{P\PYGZus{}saturated\PYGZus{}vapor}\PYG{p}{,} \PYG{l+s+s2}{\PYGZdq{}}\PYG{l+s+s2}{\PYGZhy{}b}\PYG{l+s+s2}{\PYGZdq{}}\PYG{p}{,} \PYG{n}{label}\PYG{o}{=}\PYG{l+s+s2}{\PYGZdq{}}\PYG{l+s+s2}{Saturated Vapor}\PYG{l+s+s2}{\PYGZdq{}}\PYG{p}{)}
    \PYG{n}{plt}\PYG{o}{.}\PYG{n}{xscale}\PYG{p}{(}\PYG{l+s+s2}{\PYGZdq{}}\PYG{l+s+s2}{log}\PYG{l+s+s2}{\PYGZdq{}}\PYG{p}{)}

    \PYG{n}{Q} \PYG{o}{=} \PYG{l+m+mi}{0}

    \PYG{n}{P\PYGZus{}saturated\PYGZus{}liquid} \PYG{o}{=} \PYG{p}{[}\PYG{n}{CP}\PYG{o}{.}\PYG{n}{PropsSI}\PYG{p}{(}\PYG{l+s+s2}{\PYGZdq{}}\PYG{l+s+s2}{P}\PYG{l+s+s2}{\PYGZdq{}}\PYG{p}{,} \PYG{l+s+s2}{\PYGZdq{}}\PYG{l+s+s2}{T}\PYG{l+s+s2}{\PYGZdq{}}\PYG{p}{,} \PYG{n}{T}\PYG{p}{,} \PYG{l+s+s2}{\PYGZdq{}}\PYG{l+s+s2}{Q}\PYG{l+s+s2}{\PYGZdq{}}\PYG{p}{,} \PYG{n}{Q}\PYG{p}{,} \PYG{n}{fluid}\PYG{p}{)} \PYG{k}{for} \PYG{n}{T} \PYG{o+ow}{in} \PYG{n}{T\PYGZus{}vals}\PYG{p}{]}
    \PYG{n}{vol\PYGZus{}liquid} \PYG{o}{=} \PYG{l+m+mi}{1} \PYG{o}{/} \PYG{n}{np}\PYG{o}{.}\PYG{n}{array}\PYG{p}{(}\PYG{p}{[}\PYG{n}{CP}\PYG{o}{.}\PYG{n}{PropsSI}\PYG{p}{(}\PYG{l+s+s2}{\PYGZdq{}}\PYG{l+s+s2}{D}\PYG{l+s+s2}{\PYGZdq{}}\PYG{p}{,} \PYG{l+s+s2}{\PYGZdq{}}\PYG{l+s+s2}{T}\PYG{l+s+s2}{\PYGZdq{}}\PYG{p}{,} \PYG{n}{T}\PYG{p}{,} \PYG{l+s+s2}{\PYGZdq{}}\PYG{l+s+s2}{Q}\PYG{l+s+s2}{\PYGZdq{}}\PYG{p}{,} \PYG{n}{Q}\PYG{p}{,} \PYG{n}{fluid}\PYG{p}{)} \PYG{k}{for} \PYG{n}{T} \PYG{o+ow}{in} \PYG{n}{T\PYGZus{}vals}\PYG{p}{]}\PYG{p}{)}

    \PYG{n}{plt}\PYG{o}{.}\PYG{n}{plot}\PYG{p}{(}\PYG{n}{vol\PYGZus{}liquid}\PYG{p}{,} \PYG{n}{P\PYGZus{}saturated\PYGZus{}liquid}\PYG{p}{,} \PYG{l+s+s2}{\PYGZdq{}}\PYG{l+s+s2}{\PYGZhy{}g}\PYG{l+s+s2}{\PYGZdq{}}\PYG{p}{,} \PYG{n}{label}\PYG{o}{=}\PYG{l+s+s2}{\PYGZdq{}}\PYG{l+s+s2}{Saturated Liquid}\PYG{l+s+s2}{\PYGZdq{}}\PYG{p}{)}
    \PYG{n}{plt}\PYG{o}{.}\PYG{n}{xscale}\PYG{p}{(}\PYG{l+s+s2}{\PYGZdq{}}\PYG{l+s+s2}{log}\PYG{l+s+s2}{\PYGZdq{}}\PYG{p}{)}

    \PYG{n}{plt}\PYG{o}{.}\PYG{n}{ylabel}\PYG{p}{(}\PYG{l+s+s2}{\PYGZdq{}}\PYG{l+s+s2}{Pressure [Pa]}\PYG{l+s+s2}{\PYGZdq{}}\PYG{p}{)}  
    \PYG{n}{plt}\PYG{o}{.}\PYG{n}{xlabel}\PYG{p}{(}\PYG{l+s+s2}{\PYGZdq{}}\PYG{l+s+s2}{Specific Volume (m\PYGZca{}3/kg)}\PYG{l+s+s2}{\PYGZdq{}}\PYG{p}{)}  
    
    \PYG{c+c1}{\PYGZsh{} Specific states}
    \PYG{n}{P1} \PYG{o}{=} \PYG{n}{state1}\PYG{p}{[}\PYG{l+m+mi}{0}\PYG{p}{]}
    \PYG{n}{P2} \PYG{o}{=} \PYG{n}{state2}\PYG{p}{[}\PYG{l+m+mi}{0}\PYG{p}{]}
    
    \PYG{n}{plt}\PYG{o}{.}\PYG{n}{plot}\PYG{p}{(}\PYG{n}{state1}\PYG{p}{[}\PYG{l+m+mi}{1}\PYG{p}{]}\PYG{p}{,} \PYG{n}{P1}\PYG{p}{,} \PYG{l+s+s2}{\PYGZdq{}}\PYG{l+s+s2}{ok}\PYG{l+s+s2}{\PYGZdq{}}\PYG{p}{,} \PYG{n}{label}\PYG{o}{=}\PYG{l+s+s2}{\PYGZdq{}}\PYG{l+s+s2}{State 1}\PYG{l+s+s2}{\PYGZdq{}}\PYG{p}{)}
    \PYG{n}{plt}\PYG{o}{.}\PYG{n}{plot}\PYG{p}{(}\PYG{n}{state2}\PYG{p}{[}\PYG{l+m+mi}{1}\PYG{p}{]}\PYG{p}{,} \PYG{n}{P2}\PYG{p}{,} \PYG{l+s+s2}{\PYGZdq{}}\PYG{l+s+s2}{or}\PYG{l+s+s2}{\PYGZdq{}}\PYG{p}{,} \PYG{n}{label}\PYG{o}{=}\PYG{l+s+s2}{\PYGZdq{}}\PYG{l+s+s2}{State 2}\PYG{l+s+s2}{\PYGZdq{}}\PYG{p}{)}

    \PYG{k}{if} \PYG{n}{plot\PYGZus{}const\PYGZus{}press\PYGZus{}line}\PYG{p}{:}
        \PYG{n}{plt}\PYG{o}{.}\PYG{n}{axhline}\PYG{p}{(}\PYG{n}{y}\PYG{o}{=}\PYG{n}{P1}\PYG{p}{,} \PYG{n}{color}\PYG{o}{=}\PYG{l+s+s1}{\PYGZsq{}}\PYG{l+s+s1}{k}\PYG{l+s+s1}{\PYGZsq{}}\PYG{p}{,} \PYG{n}{linestyle}\PYG{o}{=}\PYG{l+s+s1}{\PYGZsq{}}\PYG{l+s+s1}{\PYGZhy{}\PYGZhy{}}\PYG{l+s+s1}{\PYGZsq{}}\PYG{p}{,} \PYG{n}{label}\PYG{o}{=}\PYG{l+s+s2}{\PYGZdq{}}\PYG{l+s+s2}{P = constant (}\PYG{l+s+si}{\PYGZob{}\PYGZcb{}}\PYG{l+s+s2}{ kPa)}\PYG{l+s+s2}{\PYGZdq{}}\PYG{o}{.}\PYG{n}{format}\PYG{p}{(}\PYG{n+nb}{round}\PYG{p}{(}\PYG{n}{P1}\PYG{o}{/}\PYG{l+m+mf}{1e3}\PYG{p}{,}\PYG{l+m+mi}{2}\PYG{p}{)}\PYG{p}{)}\PYG{p}{)}
    
    \PYG{n}{plt}\PYG{o}{.}\PYG{n}{legend}\PYG{p}{(}\PYG{p}{)}
    \PYG{n}{plt}\PYG{o}{.}\PYG{n}{grid}\PYG{p}{(}\PYG{p}{)}
    \PYG{n}{plt}\PYG{o}{.}\PYG{n}{show}\PYG{p}{(}\PYG{p}{)}
\end{sphinxVerbatim}

\end{sphinxuseclass}\end{sphinxVerbatimInput}

\end{sphinxuseclass}

\subsection{Solution:}
\label{\detokenize{notebooks/Chapter2/piston-cylinder:solution}}
\begin{sphinxuseclass}{cell}\begin{sphinxVerbatimInput}

\begin{sphinxuseclass}{cell_input}
\begin{sphinxVerbatim}[commandchars=\\\{\}]
\PYG{k+kn}{import} \PYG{n+nn}{CoolProp}\PYG{n+nn}{.}\PYG{n+nn}{CoolProp} \PYG{k}{as} \PYG{n+nn}{CP}
\PYG{k+kn}{import} \PYG{n+nn}{numpy} \PYG{k}{as} \PYG{n+nn}{np}
\PYG{k+kn}{import} \PYG{n+nn}{matplotlib}\PYG{n+nn}{.}\PYG{n+nn}{pyplot} \PYG{k}{as} \PYG{n+nn}{plt}
\PYG{k}{def} \PYG{n+nf}{plot\PYGZus{}T\PYGZus{}v\PYGZus{}diagram}\PYG{p}{(}\PYG{n}{fluid}\PYG{p}{,}\PYG{n}{state1}\PYG{p}{,}\PYG{n}{state2}\PYG{p}{,}\PYG{n}{plot\PYGZus{}const\PYGZus{}press\PYGZus{}line}\PYG{o}{=}\PYG{k+kc}{True}\PYG{p}{)}\PYG{p}{:}
    \PYG{c+c1}{\PYGZsh{} import the libraries we\PYGZsq{}ll need}
    \PYG{k+kn}{import} \PYG{n+nn}{CoolProp}\PYG{n+nn}{.}\PYG{n+nn}{CoolProp} \PYG{k}{as} \PYG{n+nn}{CP}
    \PYG{k+kn}{import} \PYG{n+nn}{numpy} \PYG{k}{as} \PYG{n+nn}{np}
    \PYG{k+kn}{import} \PYG{n+nn}{matplotlib}\PYG{n+nn}{.}\PYG{n+nn}{pyplot} \PYG{k}{as} \PYG{n+nn}{plt}


    \PYG{c+c1}{\PYGZsh{} define variables}
    \PYG{n}{fluid} \PYG{o}{=} \PYG{n}{fluid}  \PYG{c+c1}{\PYGZsh{} define the fluid or material of interest, for full list see CP.Fluidslist()}
    \PYG{n}{T\PYGZus{}min} \PYG{o}{=} \PYG{n}{CP}\PYG{o}{.}\PYG{n}{PropsSI}\PYG{p}{(}\PYG{l+s+s2}{\PYGZdq{}}\PYG{l+s+s2}{Tmin}\PYG{l+s+s2}{\PYGZdq{}}\PYG{p}{,} \PYG{n}{fluid}\PYG{p}{)}  \PYG{c+c1}{\PYGZsh{} this is the min temp we can get data for water}
    \PYG{n}{T\PYGZus{}max} \PYG{o}{=} \PYG{n}{CP}\PYG{o}{.}\PYG{n}{PropsSI}\PYG{p}{(}\PYG{l+s+s2}{\PYGZdq{}}\PYG{l+s+s2}{Tcrit}\PYG{l+s+s2}{\PYGZdq{}}\PYG{p}{,} \PYG{n}{fluid}\PYG{p}{)}  \PYG{c+c1}{\PYGZsh{} this is the max temp we can get data for water}
    \PYG{n}{T\PYGZus{}vals} \PYG{o}{=} \PYG{n}{np}\PYG{o}{.}\PYG{n}{linspace}\PYG{p}{(}
        \PYG{n}{T\PYGZus{}min}\PYG{p}{,} \PYG{n}{T\PYGZus{}max}\PYG{p}{,} \PYG{l+m+mi}{1000}
    \PYG{p}{)}  \PYG{c+c1}{\PYGZsh{} define an array of values from T\PYGZus{}min to T\PYGZus{}max}
    \PYG{n}{Q} \PYG{o}{=} \PYG{l+m+mi}{1}  \PYG{c+c1}{\PYGZsh{} define the steam quality as 1, which is 100\PYGZpc{} vapor}


    \PYG{n}{density} \PYG{o}{=} \PYG{p}{[}
        \PYG{n}{CP}\PYG{o}{.}\PYG{n}{PropsSI}\PYG{p}{(}\PYG{l+s+s2}{\PYGZdq{}}\PYG{l+s+s2}{D}\PYG{l+s+s2}{\PYGZdq{}}\PYG{p}{,} \PYG{l+s+s2}{\PYGZdq{}}\PYG{l+s+s2}{T}\PYG{l+s+s2}{\PYGZdq{}}\PYG{p}{,} \PYG{n}{T}\PYG{p}{,} \PYG{l+s+s2}{\PYGZdq{}}\PYG{l+s+s2}{Q}\PYG{l+s+s2}{\PYGZdq{}}\PYG{p}{,} \PYG{n}{Q}\PYG{p}{,} \PYG{n}{fluid}\PYG{p}{)} \PYG{k}{for} \PYG{n}{T} \PYG{o+ow}{in} \PYG{n}{T\PYGZus{}vals}
    \PYG{p}{]}  \PYG{c+c1}{\PYGZsh{} call for density values using CoolProp}
    \PYG{n}{vol} \PYG{o}{=} \PYG{l+m+mi}{1} \PYG{o}{/} \PYG{n}{np}\PYG{o}{.}\PYG{n}{array}\PYG{p}{(}\PYG{n}{density}\PYG{p}{)}  \PYG{c+c1}{\PYGZsh{} convert density into specific volume}

    \PYG{n}{plt}\PYG{o}{.}\PYG{n}{figure}\PYG{p}{(}\PYG{l+m+mi}{1}\PYG{p}{)}
    \PYG{n}{plt}\PYG{o}{.}\PYG{n}{plot}\PYG{p}{(}\PYG{n}{vol}\PYG{p}{,} \PYG{n}{T\PYGZus{}vals}\PYG{p}{,} \PYG{l+s+s2}{\PYGZdq{}}\PYG{l+s+s2}{\PYGZhy{}b}\PYG{l+s+s2}{\PYGZdq{}}\PYG{p}{,} \PYG{n}{label}\PYG{o}{=}\PYG{l+s+s2}{\PYGZdq{}}\PYG{l+s+s2}{Saturated Vapor}\PYG{l+s+s2}{\PYGZdq{}}\PYG{p}{)}  \PYG{c+c1}{\PYGZsh{} plot temp vs specific vol}
    \PYG{n}{plt}\PYG{o}{.}\PYG{n}{xscale}\PYG{p}{(}\PYG{l+s+s2}{\PYGZdq{}}\PYG{l+s+s2}{log}\PYG{l+s+s2}{\PYGZdq{}}\PYG{p}{)}  \PYG{c+c1}{\PYGZsh{} use log scale on x axis}


    \PYG{n}{Q} \PYG{o}{=} \PYG{l+m+mi}{0}  \PYG{c+c1}{\PYGZsh{} define the steam quality as 0, which is 100\PYGZpc{} liquid}


    \PYG{n}{density} \PYG{o}{=} \PYG{p}{[}
        \PYG{n}{CP}\PYG{o}{.}\PYG{n}{PropsSI}\PYG{p}{(}\PYG{l+s+s2}{\PYGZdq{}}\PYG{l+s+s2}{D}\PYG{l+s+s2}{\PYGZdq{}}\PYG{p}{,} \PYG{l+s+s2}{\PYGZdq{}}\PYG{l+s+s2}{T}\PYG{l+s+s2}{\PYGZdq{}}\PYG{p}{,} \PYG{n}{T}\PYG{p}{,} \PYG{l+s+s2}{\PYGZdq{}}\PYG{l+s+s2}{Q}\PYG{l+s+s2}{\PYGZdq{}}\PYG{p}{,} \PYG{n}{Q}\PYG{p}{,} \PYG{n}{fluid}\PYG{p}{)} \PYG{k}{for} \PYG{n}{T} \PYG{o+ow}{in} \PYG{n}{T\PYGZus{}vals}
    \PYG{p}{]}  \PYG{c+c1}{\PYGZsh{} call for density values using CoolProp}
    \PYG{n}{vol} \PYG{o}{=} \PYG{l+m+mi}{1} \PYG{o}{/} \PYG{n}{np}\PYG{o}{.}\PYG{n}{array}\PYG{p}{(}\PYG{n}{density}\PYG{p}{)}  \PYG{c+c1}{\PYGZsh{} convert density into specific volume}


    \PYG{n}{plt}\PYG{o}{.}\PYG{n}{plot}\PYG{p}{(}\PYG{n}{vol}\PYG{p}{,} \PYG{n}{T\PYGZus{}vals}\PYG{p}{,} \PYG{l+s+s2}{\PYGZdq{}}\PYG{l+s+s2}{\PYGZhy{}g}\PYG{l+s+s2}{\PYGZdq{}}\PYG{p}{,} \PYG{n}{label}\PYG{o}{=}\PYG{l+s+s2}{\PYGZdq{}}\PYG{l+s+s2}{Saturated Liquid}\PYG{l+s+s2}{\PYGZdq{}}\PYG{p}{)}  \PYG{c+c1}{\PYGZsh{} plot temp vs specific vol}
    \PYG{n}{plt}\PYG{o}{.}\PYG{n}{xscale}\PYG{p}{(}\PYG{l+s+s2}{\PYGZdq{}}\PYG{l+s+s2}{log}\PYG{l+s+s2}{\PYGZdq{}}\PYG{p}{)}  \PYG{c+c1}{\PYGZsh{} use log scale on x axis}


    \PYG{n}{plt}\PYG{o}{.}\PYG{n}{ylabel}\PYG{p}{(}\PYG{l+s+s2}{\PYGZdq{}}\PYG{l+s+s2}{Temperature [K]}\PYG{l+s+s2}{\PYGZdq{}}\PYG{p}{)}  \PYG{c+c1}{\PYGZsh{} give y axis a label}
    \PYG{n}{plt}\PYG{o}{.}\PYG{n}{xlabel}\PYG{p}{(}\PYG{l+s+s2}{\PYGZdq{}}\PYG{l+s+s2}{Specific Volume (m\PYGZca{}3/kg)}\PYG{l+s+s2}{\PYGZdq{}}\PYG{p}{)}  \PYG{c+c1}{\PYGZsh{} give x axis a label}
    \PYG{n}{plt}\PYG{o}{.}\PYG{n}{grid}\PYG{p}{(}\PYG{p}{)}
    \PYG{c+c1}{\PYGZsh{} plot various points on the T\PYGZhy{}v diagram:}

    \PYG{n}{x} \PYG{o}{=} \PYG{p}{[}\PYG{n}{state1}\PYG{p}{[}\PYG{l+m+mi}{0}\PYG{p}{]}\PYG{p}{,} \PYG{n}{state2}\PYG{p}{[}\PYG{l+m+mi}{0}\PYG{p}{]}\PYG{p}{]}  \PYG{c+c1}{\PYGZsh{} specific volume in m3/kg}
    \PYG{n}{y} \PYG{o}{=} \PYG{p}{[}\PYG{n}{state1}\PYG{p}{[}\PYG{l+m+mi}{1}\PYG{p}{]}\PYG{p}{,} \PYG{n}{state2}\PYG{p}{[}\PYG{l+m+mi}{1}\PYG{p}{]}\PYG{p}{]}  \PYG{c+c1}{\PYGZsh{} temperature in K}
    \PYG{n}{plt}\PYG{o}{.}\PYG{n}{plot}\PYG{p}{(}\PYG{n}{x}\PYG{p}{[}\PYG{l+m+mi}{0}\PYG{p}{]}\PYG{p}{,} \PYG{n}{y}\PYG{p}{[}\PYG{l+m+mi}{0}\PYG{p}{]}\PYG{p}{,} \PYG{l+s+s2}{\PYGZdq{}}\PYG{l+s+s2}{ok}\PYG{l+s+s2}{\PYGZdq{}}\PYG{p}{,} \PYG{n}{label}\PYG{o}{=}\PYG{l+s+s2}{\PYGZdq{}}\PYG{l+s+s2}{State 1}\PYG{l+s+s2}{\PYGZdq{}}\PYG{p}{)}
    \PYG{n}{plt}\PYG{o}{.}\PYG{n}{plot}\PYG{p}{(}\PYG{n}{x}\PYG{p}{[}\PYG{l+m+mi}{1}\PYG{p}{]}\PYG{p}{,} \PYG{n}{y}\PYG{p}{[}\PYG{l+m+mi}{1}\PYG{p}{]}\PYG{p}{,} \PYG{l+s+s2}{\PYGZdq{}}\PYG{l+s+s2}{or}\PYG{l+s+s2}{\PYGZdq{}}\PYG{p}{,} \PYG{n}{label}\PYG{o}{=}\PYG{l+s+s2}{\PYGZdq{}}\PYG{l+s+s2}{State 2}\PYG{l+s+s2}{\PYGZdq{}}\PYG{p}{)}
    \PYG{k}{if} \PYG{n}{plot\PYGZus{}const\PYGZus{}press\PYGZus{}line} \PYG{o}{==} \PYG{k+kc}{True}\PYG{p}{:}
        \PYG{c+c1}{\PYGZsh{} Plotting the constant pressure line for the given pressure:}
        \PYG{n}{P\PYGZus{}const} \PYG{o}{=} \PYG{n}{CP}\PYG{o}{.}\PYG{n}{PropsSI}\PYG{p}{(}\PYG{l+s+s2}{\PYGZdq{}}\PYG{l+s+s2}{P}\PYG{l+s+s2}{\PYGZdq{}}\PYG{p}{,} \PYG{l+s+s2}{\PYGZdq{}}\PYG{l+s+s2}{T}\PYG{l+s+s2}{\PYGZdq{}}\PYG{p}{,} \PYG{n}{state1}\PYG{p}{[}\PYG{l+m+mi}{1}\PYG{p}{]}\PYG{p}{,} \PYG{l+s+s2}{\PYGZdq{}}\PYG{l+s+s2}{D}\PYG{l+s+s2}{\PYGZdq{}}\PYG{p}{,} \PYG{l+m+mi}{1}\PYG{o}{/}\PYG{n}{state1}\PYG{p}{[}\PYG{l+m+mi}{0}\PYG{p}{]}\PYG{p}{,} \PYG{n}{fluid}\PYG{p}{)}
        \PYG{n}{v\PYGZus{}vals\PYGZus{}constP} \PYG{o}{=} \PYG{p}{[}\PYG{l+m+mi}{1} \PYG{o}{/} \PYG{n}{CP}\PYG{o}{.}\PYG{n}{PropsSI}\PYG{p}{(}\PYG{l+s+s2}{\PYGZdq{}}\PYG{l+s+s2}{D}\PYG{l+s+s2}{\PYGZdq{}}\PYG{p}{,} \PYG{l+s+s2}{\PYGZdq{}}\PYG{l+s+s2}{T}\PYG{l+s+s2}{\PYGZdq{}}\PYG{p}{,} \PYG{n}{T}\PYG{p}{,} \PYG{l+s+s2}{\PYGZdq{}}\PYG{l+s+s2}{P}\PYG{l+s+s2}{\PYGZdq{}}\PYG{p}{,} \PYG{n}{P\PYGZus{}const}\PYG{p}{,} \PYG{n}{fluid}\PYG{p}{)} \PYG{k}{for} \PYG{n}{T} \PYG{o+ow}{in} \PYG{n}{T\PYGZus{}vals}\PYG{p}{]}
        \PYG{n}{plt}\PYG{o}{.}\PYG{n}{plot}\PYG{p}{(}\PYG{n}{v\PYGZus{}vals\PYGZus{}constP}\PYG{p}{,} \PYG{n}{T\PYGZus{}vals}\PYG{p}{,} \PYG{l+s+s2}{\PYGZdq{}}\PYG{l+s+s2}{\PYGZhy{}\PYGZhy{}k}\PYG{l+s+s2}{\PYGZdq{}}\PYG{p}{,} \PYG{n}{label}\PYG{o}{=}\PYG{l+s+s2}{\PYGZdq{}}\PYG{l+s+s2}{P = constant (}\PYG{l+s+si}{\PYGZob{}\PYGZcb{}}\PYG{l+s+s2}{ kPa)}\PYG{l+s+s2}{\PYGZdq{}}\PYG{o}{.}\PYG{n}{format}\PYG{p}{(}\PYG{n+nb}{round}\PYG{p}{(}\PYG{n}{P\PYGZus{}const}\PYG{o}{/}\PYG{l+m+mf}{1e3}\PYG{p}{,}\PYG{l+m+mi}{2}\PYG{p}{)}\PYG{p}{)}\PYG{p}{)}
        \PYG{n}{plt}\PYG{o}{.}\PYG{n}{legend}\PYG{p}{(}\PYG{p}{)}
    \PYG{k}{else}\PYG{p}{:}
        \PYG{n}{plt}\PYG{o}{.}\PYG{n}{legend}\PYG{p}{(}\PYG{p}{)}
        
\end{sphinxVerbatim}

\end{sphinxuseclass}\end{sphinxVerbatimInput}

\end{sphinxuseclass}
\begin{sphinxuseclass}{cell}\begin{sphinxVerbatimInput}

\begin{sphinxuseclass}{cell_input}
\begin{sphinxVerbatim}[commandchars=\\\{\}]
\PYG{c+c1}{\PYGZsh{}\PYGZsh{}\PYGZsh{}========== (a) Show the process on a T\PYGZhy{}v diagram: ================\PYGZsh{}\PYGZsh{}\PYGZsh{}}
\PYG{n}{m} \PYG{o}{=} \PYG{l+m+mf}{0.9}  \PYG{c+c1}{\PYGZsh{} kg}
\PYG{n}{T} \PYG{o}{=} \PYG{l+m+mi}{300} \PYG{o}{+} \PYG{l+m+mf}{273.15}  \PYG{c+c1}{\PYGZsh{} Kelvin}
\PYG{n}{pressure} \PYG{o}{=} \PYG{l+m+mf}{0.8e6}  \PYG{c+c1}{\PYGZsh{} Pa}
\PYG{n}{fluid} \PYG{o}{=} \PYG{l+s+s2}{\PYGZdq{}}\PYG{l+s+s2}{Water}\PYG{l+s+s2}{\PYGZdq{}}
\PYG{n}{final\PYGZus{}temp} \PYG{o}{=} \PYG{n}{CP}\PYG{o}{.}\PYG{n}{PropsSI}\PYG{p}{(}\PYG{l+s+s2}{\PYGZdq{}}\PYG{l+s+s2}{T}\PYG{l+s+s2}{\PYGZdq{}}\PYG{p}{,} \PYG{l+s+s2}{\PYGZdq{}}\PYG{l+s+s2}{Q}\PYG{l+s+s2}{\PYGZdq{}}\PYG{p}{,} \PYG{l+m+mi}{0}\PYG{p}{,} \PYG{l+s+s2}{\PYGZdq{}}\PYG{l+s+s2}{P}\PYG{l+s+s2}{\PYGZdq{}}\PYG{p}{,} \PYG{n}{pressure}\PYG{p}{,} \PYG{l+s+s2}{\PYGZdq{}}\PYG{l+s+s2}{water}\PYG{l+s+s2}{\PYGZdq{}}\PYG{p}{)}
\PYG{c+c1}{\PYGZsh{}\PYGZsh{} quality }
\PYG{n}{q} \PYG{o}{=} \PYG{l+m+mi}{1}\PYG{o}{/}\PYG{l+m+mi}{3}
\PYG{c+c1}{\PYGZsh{} const pressure process:}
\PYG{n}{vol\PYGZus{}state1} \PYG{o}{=} \PYG{l+m+mi}{1} \PYG{o}{/} \PYG{n}{CP}\PYG{o}{.}\PYG{n}{PropsSI}\PYG{p}{(}\PYG{l+s+s2}{\PYGZdq{}}\PYG{l+s+s2}{D}\PYG{l+s+s2}{\PYGZdq{}}\PYG{p}{,} \PYG{l+s+s2}{\PYGZdq{}}\PYG{l+s+s2}{T}\PYG{l+s+s2}{\PYGZdq{}}\PYG{p}{,} \PYG{n}{T}\PYG{p}{,} \PYG{l+s+s2}{\PYGZdq{}}\PYG{l+s+s2}{P}\PYG{l+s+s2}{\PYGZdq{}}\PYG{p}{,} \PYG{n}{pressure}\PYG{p}{,} \PYG{l+s+s2}{\PYGZdq{}}\PYG{l+s+s2}{water}\PYG{l+s+s2}{\PYGZdq{}}\PYG{p}{)}
\PYG{n}{vol\PYGZus{}state2} \PYG{o}{=} \PYG{l+m+mi}{1} \PYG{o}{/} \PYG{n}{CP}\PYG{o}{.}\PYG{n}{PropsSI}\PYG{p}{(}\PYG{l+s+s2}{\PYGZdq{}}\PYG{l+s+s2}{D}\PYG{l+s+s2}{\PYGZdq{}}\PYG{p}{,} \PYG{l+s+s2}{\PYGZdq{}}\PYG{l+s+s2}{P}\PYG{l+s+s2}{\PYGZdq{}}\PYG{p}{,} \PYG{n}{pressure}\PYG{p}{,} \PYG{l+s+s2}{\PYGZdq{}}\PYG{l+s+s2}{Q}\PYG{l+s+s2}{\PYGZdq{}}\PYG{p}{,} \PYG{n}{q}\PYG{p}{,} \PYG{l+s+s2}{\PYGZdq{}}\PYG{l+s+s2}{water}\PYG{l+s+s2}{\PYGZdq{}}\PYG{p}{)}
\PYG{n}{state1} \PYG{o}{=} \PYG{p}{[}\PYG{n}{vol\PYGZus{}state1}\PYG{p}{,} \PYG{n}{T}\PYG{p}{]}
\PYG{n}{state2} \PYG{o}{=} \PYG{p}{[}\PYG{n}{vol\PYGZus{}state2}\PYG{p}{,} \PYG{n}{final\PYGZus{}temp}\PYG{p}{]}
\PYG{c+c1}{\PYGZsh{}\PYGZsh{} use the following function:}
\PYG{n}{plot\PYGZus{}T\PYGZus{}v\PYGZus{}diagram}\PYG{p}{(}\PYG{n}{fluid}\PYG{p}{,}\PYG{n}{state1}\PYG{p}{,}\PYG{n}{state2}\PYG{p}{,}\PYG{n}{plot\PYGZus{}const\PYGZus{}press\PYGZus{}line}\PYG{o}{=}\PYG{k+kc}{True}\PYG{p}{)}
\PYG{n}{plot\PYGZus{}p\PYGZus{}v\PYGZus{}diagram}\PYG{p}{(}\PYG{n}{fluid}\PYG{p}{,} \PYG{p}{[}\PYG{n}{pressure}\PYG{p}{,}\PYG{n}{vol\PYGZus{}state1}\PYG{p}{]}\PYG{p}{,} \PYG{p}{[}\PYG{n}{pressure}\PYG{p}{,}\PYG{n}{vol\PYGZus{}state2}\PYG{p}{]}\PYG{p}{)}
\PYG{c+c1}{\PYGZsh{}\PYGZsh{}\PYGZsh{}========== (b) final temperature ================\PYGZsh{}\PYGZsh{}\PYGZsh{}}
\PYG{n+nb}{print}\PYG{p}{(}\PYG{l+s+s2}{\PYGZdq{}}\PYG{l+s+s2}{(b) Final temperature : }\PYG{l+s+si}{\PYGZob{}\PYGZcb{}}\PYG{l+s+s2}{ °C}\PYG{l+s+s2}{\PYGZdq{}}\PYG{o}{.}\PYG{n}{format}\PYG{p}{(}\PYG{n+nb}{round}\PYG{p}{(}\PYG{n}{final\PYGZus{}temp}\PYG{o}{\PYGZhy{}}\PYG{l+m+mf}{273.15}\PYG{p}{,}\PYG{l+m+mi}{2}\PYG{p}{)}\PYG{p}{)}\PYG{p}{)}

\PYG{c+c1}{\PYGZsh{}\PYGZsh{}\PYGZsh{}========== (c) change in Volume ================\PYGZsh{}\PYGZsh{}\PYGZsh{}}
\PYG{n}{delV} \PYG{o}{=} \PYG{n}{m} \PYG{o}{*} \PYG{p}{(}\PYG{n}{vol\PYGZus{}state2} \PYG{o}{\PYGZhy{}} \PYG{n}{vol\PYGZus{}state1}\PYG{p}{)}
\PYG{n+nb}{print}\PYG{p}{(}\PYG{l+s+s2}{\PYGZdq{}}\PYG{l+s+s2}{(c) Vol change: }\PYG{l+s+si}{\PYGZob{}\PYGZcb{}}\PYG{l+s+s2}{ m³}\PYG{l+s+s2}{\PYGZdq{}}\PYG{o}{.}\PYG{n}{format}\PYG{p}{(}\PYG{n+nb}{round}\PYG{p}{(}\PYG{n}{delV}\PYG{p}{,}\PYG{l+m+mi}{4}\PYG{p}{)}\PYG{p}{)}\PYG{p}{)}
 
\end{sphinxVerbatim}

\end{sphinxuseclass}\end{sphinxVerbatimInput}
\begin{sphinxVerbatimOutput}

\begin{sphinxuseclass}{cell_output}
\noindent\sphinxincludegraphics{{c857ed356bd21bc6b53a900af3415c3c24bd086f85873b8f00e333b36a8f4fb0}.png}

\noindent\sphinxincludegraphics{{fe9cec76ce5a77bc470bee78cfca0876e384f7e96eee54a79e59a994e3ca1077}.png}

\begin{sphinxVerbatim}[commandchars=\\\{\}]
(b) Final temperature : 170.41 °C
(c) Vol change: \PYGZhy{}0.219 m³
\end{sphinxVerbatim}

\end{sphinxuseclass}\end{sphinxVerbatimOutput}

\end{sphinxuseclass}
\sphinxstepscope


\section{Piston\sphinxhyphen{}cylinder: Isobaric\sphinxhyphen{} Isochoric process}
\label{\detokenize{notebooks/Chapter2/piston-cylinder-isobaric-isochoric:piston-cylinder-isobaric-isochoric-process}}\label{\detokenize{notebooks/Chapter2/piston-cylinder-isobaric-isochoric::doc}}

\subsection{Problem statement:}
\label{\detokenize{notebooks/Chapter2/piston-cylinder-isobaric-isochoric:problem-statement}}
\sphinxAtStartPar
A piston\sphinxhyphen{}cylinder device contains steam initially at 200 C and 200 kPa. The steam is first cooled isobarically to saturated liquid, then isochorically until its pressure reaches 25 kPa.
\begin{enumerate}
\sphinxsetlistlabels{\arabic}{enumi}{enumii}{}{.}%
\item {} 
\sphinxAtStartPar
Sketch the whole process on the P−vand T−v diagrams

\item {} 
\sphinxAtStartPar
Calculate the specific heat transfer in the whole process

\end{enumerate}


\subsection{Solution strategy}
\label{\detokenize{notebooks/Chapter2/piston-cylinder-isobaric-isochoric:solution-strategy}}
\sphinxAtStartPar
Identify the States:
State 1: Initial state \sphinxhyphen{} T1=200 C and P1=200kPa
State 2: After isobaric cooling to saturated liquid at P2=200kPa
State 3: After isochoric cooling with V3=V2 and P3=25kPa.

\sphinxAtStartPar
Sketch the Process on P\sphinxhyphen{}v and T\sphinxhyphen{}v Diagrams:
The process from State 1 to State 2 is a horizontal line on the P\sphinxhyphen{}v diagram and a downward slope on the T\sphinxhyphen{}v diagram.
The process from State 2 to State 3 is a vertical line downward on the P\sphinxhyphen{}v diagram and also a downward slope on the T\sphinxhyphen{}v diagram.

\sphinxAtStartPar
Calculate the Specific Heat Transfer:
Using the first law for a closed system:
Δu=q−w
Where:
Δu = Change in internal energy between states.
q = Heat transfer.
w = Work done.
For the isobaric process (State 1 to State 2), w=P(V2−V1).
For the isochoric process (State 2 to State 3), w=0.

\sphinxAtStartPar
Thus, the total heat transfer for the whole process is:
q=Δu+w

\sphinxAtStartPar
Let’s proceed with the code to sketch the diagrams and calculate the specific heat transfer:

\begin{sphinxuseclass}{cell}\begin{sphinxVerbatimInput}

\begin{sphinxuseclass}{cell_input}
\begin{sphinxVerbatim}[commandchars=\\\{\}]
\PYG{k+kn}{import} \PYG{n+nn}{numpy} \PYG{k}{as} \PYG{n+nn}{np}
\PYG{k+kn}{import} \PYG{n+nn}{matplotlib}\PYG{n+nn}{.}\PYG{n+nn}{pyplot} \PYG{k}{as} \PYG{n+nn}{plt}
\PYG{k+kn}{import} \PYG{n+nn}{CoolProp}\PYG{n+nn}{.}\PYG{n+nn}{CoolProp} \PYG{k}{as} \PYG{n+nn}{CP}

\PYG{c+c1}{\PYGZsh{} Given data}
\PYG{n}{T1} \PYG{o}{=} \PYG{l+m+mi}{350} \PYG{o}{+} \PYG{l+m+mf}{273.15}  \PYG{c+c1}{\PYGZsh{} Initial temperature in K}
\PYG{n}{P1} \PYG{o}{=} \PYG{l+m+mf}{1000e3}  \PYG{c+c1}{\PYGZsh{} Initial pressure in Pa}
\PYG{n}{P2} \PYG{o}{=} \PYG{l+m+mf}{1000e3}  \PYG{c+c1}{\PYGZsh{} Pressure after isobaric cooling in Pa}
\PYG{n}{P3} \PYG{o}{=} \PYG{l+m+mf}{25e3}  \PYG{c+c1}{\PYGZsh{} Final pressure after isochoric process in Pa}

\PYG{c+c1}{\PYGZsh{} State 1 properties}
\PYG{n}{v1} \PYG{o}{=} \PYG{l+m+mi}{1} \PYG{o}{/} \PYG{n}{CP}\PYG{o}{.}\PYG{n}{PropsSI}\PYG{p}{(}\PYG{l+s+s2}{\PYGZdq{}}\PYG{l+s+s2}{D}\PYG{l+s+s2}{\PYGZdq{}}\PYG{p}{,} \PYG{l+s+s2}{\PYGZdq{}}\PYG{l+s+s2}{T}\PYG{l+s+s2}{\PYGZdq{}}\PYG{p}{,} \PYG{n}{T1}\PYG{p}{,} \PYG{l+s+s2}{\PYGZdq{}}\PYG{l+s+s2}{P}\PYG{l+s+s2}{\PYGZdq{}}\PYG{p}{,} \PYG{n}{P1}\PYG{p}{,} \PYG{l+s+s2}{\PYGZdq{}}\PYG{l+s+s2}{Water}\PYG{l+s+s2}{\PYGZdq{}}\PYG{p}{)}
\PYG{n}{u1} \PYG{o}{=} \PYG{n}{CP}\PYG{o}{.}\PYG{n}{PropsSI}\PYG{p}{(}\PYG{l+s+s2}{\PYGZdq{}}\PYG{l+s+s2}{U}\PYG{l+s+s2}{\PYGZdq{}}\PYG{p}{,} \PYG{l+s+s2}{\PYGZdq{}}\PYG{l+s+s2}{T}\PYG{l+s+s2}{\PYGZdq{}}\PYG{p}{,} \PYG{n}{T1}\PYG{p}{,} \PYG{l+s+s2}{\PYGZdq{}}\PYG{l+s+s2}{P}\PYG{l+s+s2}{\PYGZdq{}}\PYG{p}{,} \PYG{n}{P1}\PYG{p}{,} \PYG{l+s+s2}{\PYGZdq{}}\PYG{l+s+s2}{Water}\PYG{l+s+s2}{\PYGZdq{}}\PYG{p}{)}

\PYG{c+c1}{\PYGZsh{} State 2 properties}
\PYG{n}{T2} \PYG{o}{=} \PYG{n}{CP}\PYG{o}{.}\PYG{n}{PropsSI}\PYG{p}{(}\PYG{l+s+s2}{\PYGZdq{}}\PYG{l+s+s2}{T}\PYG{l+s+s2}{\PYGZdq{}}\PYG{p}{,} \PYG{l+s+s2}{\PYGZdq{}}\PYG{l+s+s2}{P}\PYG{l+s+s2}{\PYGZdq{}}\PYG{p}{,} \PYG{n}{P2}\PYG{p}{,} \PYG{l+s+s2}{\PYGZdq{}}\PYG{l+s+s2}{Q}\PYG{l+s+s2}{\PYGZdq{}}\PYG{p}{,} \PYG{l+m+mi}{0}\PYG{p}{,} \PYG{l+s+s2}{\PYGZdq{}}\PYG{l+s+s2}{Water}\PYG{l+s+s2}{\PYGZdq{}}\PYG{p}{)}  \PYG{c+c1}{\PYGZsh{} Saturated liquid}
\PYG{n}{v2} \PYG{o}{=} \PYG{l+m+mi}{1} \PYG{o}{/} \PYG{n}{CP}\PYG{o}{.}\PYG{n}{PropsSI}\PYG{p}{(}\PYG{l+s+s2}{\PYGZdq{}}\PYG{l+s+s2}{D}\PYG{l+s+s2}{\PYGZdq{}}\PYG{p}{,} \PYG{l+s+s2}{\PYGZdq{}}\PYG{l+s+s2}{P}\PYG{l+s+s2}{\PYGZdq{}}\PYG{p}{,} \PYG{n}{P2}\PYG{p}{,} \PYG{l+s+s2}{\PYGZdq{}}\PYG{l+s+s2}{Q}\PYG{l+s+s2}{\PYGZdq{}}\PYG{p}{,} \PYG{l+m+mf}{0.5}\PYG{p}{,} \PYG{l+s+s2}{\PYGZdq{}}\PYG{l+s+s2}{Water}\PYG{l+s+s2}{\PYGZdq{}}\PYG{p}{)}
\PYG{n}{u2} \PYG{o}{=} \PYG{n}{CP}\PYG{o}{.}\PYG{n}{PropsSI}\PYG{p}{(}\PYG{l+s+s2}{\PYGZdq{}}\PYG{l+s+s2}{U}\PYG{l+s+s2}{\PYGZdq{}}\PYG{p}{,} \PYG{l+s+s2}{\PYGZdq{}}\PYG{l+s+s2}{P}\PYG{l+s+s2}{\PYGZdq{}}\PYG{p}{,} \PYG{n}{P2}\PYG{p}{,} \PYG{l+s+s2}{\PYGZdq{}}\PYG{l+s+s2}{Q}\PYG{l+s+s2}{\PYGZdq{}}\PYG{p}{,} \PYG{l+m+mi}{0}\PYG{p}{,} \PYG{l+s+s2}{\PYGZdq{}}\PYG{l+s+s2}{Water}\PYG{l+s+s2}{\PYGZdq{}}\PYG{p}{)}

\PYG{c+c1}{\PYGZsh{} State 3 properties (isochoric)}
\PYG{n}{v3} \PYG{o}{=} \PYG{n}{v2}
\PYG{n}{T3} \PYG{o}{=} \PYG{n}{CP}\PYG{o}{.}\PYG{n}{PropsSI}\PYG{p}{(}\PYG{l+s+s2}{\PYGZdq{}}\PYG{l+s+s2}{T}\PYG{l+s+s2}{\PYGZdq{}}\PYG{p}{,} \PYG{l+s+s2}{\PYGZdq{}}\PYG{l+s+s2}{D}\PYG{l+s+s2}{\PYGZdq{}}\PYG{p}{,} \PYG{l+m+mi}{1} \PYG{o}{/} \PYG{n}{v3}\PYG{p}{,} \PYG{l+s+s2}{\PYGZdq{}}\PYG{l+s+s2}{P}\PYG{l+s+s2}{\PYGZdq{}}\PYG{p}{,} \PYG{n}{P3}\PYG{p}{,} \PYG{l+s+s2}{\PYGZdq{}}\PYG{l+s+s2}{Water}\PYG{l+s+s2}{\PYGZdq{}}\PYG{p}{)}
\PYG{n}{u3} \PYG{o}{=} \PYG{n}{CP}\PYG{o}{.}\PYG{n}{PropsSI}\PYG{p}{(}\PYG{l+s+s2}{\PYGZdq{}}\PYG{l+s+s2}{U}\PYG{l+s+s2}{\PYGZdq{}}\PYG{p}{,} \PYG{l+s+s2}{\PYGZdq{}}\PYG{l+s+s2}{D}\PYG{l+s+s2}{\PYGZdq{}}\PYG{p}{,} \PYG{l+m+mi}{1} \PYG{o}{/} \PYG{n}{v3}\PYG{p}{,} \PYG{l+s+s2}{\PYGZdq{}}\PYG{l+s+s2}{P}\PYG{l+s+s2}{\PYGZdq{}}\PYG{p}{,} \PYG{n}{P3}\PYG{p}{,} \PYG{l+s+s2}{\PYGZdq{}}\PYG{l+s+s2}{Water}\PYG{l+s+s2}{\PYGZdq{}}\PYG{p}{)}

\PYG{c+c1}{\PYGZsh{} Calculate work and heat transfer}
\PYG{n}{w\PYGZus{}12} \PYG{o}{=} \PYG{n}{P1} \PYG{o}{*} \PYG{p}{(}\PYG{n}{v2} \PYG{o}{\PYGZhy{}} \PYG{n}{v1}\PYG{p}{)}  \PYG{c+c1}{\PYGZsh{} Isobaric process}
\PYG{n}{w\PYGZus{}23} \PYG{o}{=} \PYG{l+m+mi}{0}  \PYG{c+c1}{\PYGZsh{} Isochoric process}
\PYG{n}{q\PYGZus{}12} \PYG{o}{=} \PYG{p}{(}\PYG{n}{u2} \PYG{o}{\PYGZhy{}} \PYG{n}{u1}\PYG{p}{)} \PYG{o}{+} \PYG{n}{w\PYGZus{}12}
\PYG{n}{q\PYGZus{}23} \PYG{o}{=} \PYG{n}{u3} \PYG{o}{\PYGZhy{}} \PYG{n}{u2}  \PYG{c+c1}{\PYGZsh{} Isochoric process so no work}
\PYG{n}{q\PYGZus{}total} \PYG{o}{=} \PYG{p}{(}\PYG{n}{q\PYGZus{}12} \PYG{o}{+} \PYG{n}{q\PYGZus{}23}\PYG{p}{)}\PYG{o}{/}\PYG{l+m+mf}{1e3}

\PYG{n+nb}{print}\PYG{p}{(}\PYG{l+s+sa}{f}\PYG{l+s+s2}{\PYGZdq{}}\PYG{l+s+s2}{Specific heat transfer for the whole process: }\PYG{l+s+si}{\PYGZob{}}\PYG{n}{q\PYGZus{}total}\PYG{l+s+si}{:}\PYG{l+s+s2}{.2f}\PYG{l+s+si}{\PYGZcb{}}\PYG{l+s+s2}{ kJ/kg}\PYG{l+s+s2}{\PYGZdq{}}\PYG{p}{)}
\end{sphinxVerbatim}

\end{sphinxuseclass}\end{sphinxVerbatimInput}
\begin{sphinxVerbatimOutput}

\begin{sphinxuseclass}{cell_output}
\begin{sphinxVerbatim}[commandchars=\\\{\}]
Specific heat transfer for the whole process: \PYGZhy{}2754.35 kJ/kg
\end{sphinxVerbatim}

\end{sphinxuseclass}\end{sphinxVerbatimOutput}

\end{sphinxuseclass}

\subsection{Process diagrams:}
\label{\detokenize{notebooks/Chapter2/piston-cylinder-isobaric-isochoric:process-diagrams}}
\begin{sphinxuseclass}{cell}\begin{sphinxVerbatimInput}

\begin{sphinxuseclass}{cell_input}
\begin{sphinxVerbatim}[commandchars=\\\{\}]
\PYG{c+c1}{\PYGZsh{} Critical properties}
\PYG{n}{P\PYGZus{}critical} \PYG{o}{=} \PYG{n}{CP}\PYG{o}{.}\PYG{n}{PropsSI}\PYG{p}{(}\PYG{l+s+s2}{\PYGZdq{}}\PYG{l+s+s2}{Pcrit}\PYG{l+s+s2}{\PYGZdq{}}\PYG{p}{,} \PYG{l+s+s2}{\PYGZdq{}}\PYG{l+s+s2}{Water}\PYG{l+s+s2}{\PYGZdq{}}\PYG{p}{)}
\PYG{n}{T\PYGZus{}critical} \PYG{o}{=} \PYG{n}{CP}\PYG{o}{.}\PYG{n}{PropsSI}\PYG{p}{(}\PYG{l+s+s2}{\PYGZdq{}}\PYG{l+s+s2}{Tcrit}\PYG{l+s+s2}{\PYGZdq{}}\PYG{p}{,} \PYG{l+s+s2}{\PYGZdq{}}\PYG{l+s+s2}{Water}\PYG{l+s+s2}{\PYGZdq{}}\PYG{p}{)}
\PYG{n}{T\PYGZus{}triple} \PYG{o}{=} \PYG{n}{CP}\PYG{o}{.}\PYG{n}{PropsSI}\PYG{p}{(}\PYG{l+s+s2}{\PYGZdq{}}\PYG{l+s+s2}{Ttriple}\PYG{l+s+s2}{\PYGZdq{}}\PYG{p}{,}\PYG{l+s+s2}{\PYGZdq{}}\PYG{l+s+s2}{water}\PYG{l+s+s2}{\PYGZdq{}}\PYG{p}{)}
\PYG{c+c1}{\PYGZsh{} P\PYGZhy{}v diagram with dome}
\PYG{n}{plt}\PYG{o}{.}\PYG{n}{figure}\PYG{p}{(}\PYG{p}{)}
\PYG{n}{pressures} \PYG{o}{=} \PYG{n}{np}\PYG{o}{.}\PYG{n}{linspace}\PYG{p}{(}\PYG{n}{CP}\PYG{o}{.}\PYG{n}{PropsSI}\PYG{p}{(}\PYG{l+s+s2}{\PYGZdq{}}\PYG{l+s+s2}{P}\PYG{l+s+s2}{\PYGZdq{}}\PYG{p}{,} \PYG{l+s+s2}{\PYGZdq{}}\PYG{l+s+s2}{Q}\PYG{l+s+s2}{\PYGZdq{}}\PYG{p}{,} \PYG{l+m+mi}{0}\PYG{p}{,} \PYG{l+s+s2}{\PYGZdq{}}\PYG{l+s+s2}{T}\PYG{l+s+s2}{\PYGZdq{}}\PYG{p}{,} \PYG{n}{T3}\PYG{p}{,} \PYG{l+s+s2}{\PYGZdq{}}\PYG{l+s+s2}{Water}\PYG{l+s+s2}{\PYGZdq{}}\PYG{p}{)}\PYG{p}{,} \PYG{n}{P\PYGZus{}critical}\PYG{p}{,} \PYG{l+m+mi}{100}\PYG{p}{)}
\PYG{n}{v\PYGZus{}liquid} \PYG{o}{=} \PYG{p}{[}\PYG{l+m+mi}{1} \PYG{o}{/} \PYG{n}{CP}\PYG{o}{.}\PYG{n}{PropsSI}\PYG{p}{(}\PYG{l+s+s2}{\PYGZdq{}}\PYG{l+s+s2}{D}\PYG{l+s+s2}{\PYGZdq{}}\PYG{p}{,} \PYG{l+s+s2}{\PYGZdq{}}\PYG{l+s+s2}{Q}\PYG{l+s+s2}{\PYGZdq{}}\PYG{p}{,} \PYG{l+m+mi}{0}\PYG{p}{,} \PYG{l+s+s2}{\PYGZdq{}}\PYG{l+s+s2}{P}\PYG{l+s+s2}{\PYGZdq{}}\PYG{p}{,} \PYG{n}{P}\PYG{p}{,} \PYG{l+s+s2}{\PYGZdq{}}\PYG{l+s+s2}{Water}\PYG{l+s+s2}{\PYGZdq{}}\PYG{p}{)} \PYG{k}{for} \PYG{n}{P} \PYG{o+ow}{in} \PYG{n}{pressures}\PYG{p}{]}
\PYG{n}{v\PYGZus{}vapor} \PYG{o}{=} \PYG{p}{[}\PYG{l+m+mi}{1} \PYG{o}{/} \PYG{n}{CP}\PYG{o}{.}\PYG{n}{PropsSI}\PYG{p}{(}\PYG{l+s+s2}{\PYGZdq{}}\PYG{l+s+s2}{D}\PYG{l+s+s2}{\PYGZdq{}}\PYG{p}{,} \PYG{l+s+s2}{\PYGZdq{}}\PYG{l+s+s2}{Q}\PYG{l+s+s2}{\PYGZdq{}}\PYG{p}{,} \PYG{l+m+mi}{1}\PYG{p}{,} \PYG{l+s+s2}{\PYGZdq{}}\PYG{l+s+s2}{P}\PYG{l+s+s2}{\PYGZdq{}}\PYG{p}{,} \PYG{n}{P}\PYG{p}{,} \PYG{l+s+s2}{\PYGZdq{}}\PYG{l+s+s2}{Water}\PYG{l+s+s2}{\PYGZdq{}}\PYG{p}{)} \PYG{k}{for} \PYG{n}{P} \PYG{o+ow}{in} \PYG{n}{pressures}\PYG{p}{]}
\PYG{n}{plt}\PYG{o}{.}\PYG{n}{plot}\PYG{p}{(}\PYG{n}{v\PYGZus{}liquid}\PYG{p}{,} \PYG{n}{pressures} \PYG{o}{/} \PYG{l+m+mf}{1e3}\PYG{p}{,} \PYG{l+s+s2}{\PYGZdq{}}\PYG{l+s+s2}{b\PYGZhy{}}\PYG{l+s+s2}{\PYGZdq{}}\PYG{p}{,} \PYG{n}{label}\PYG{o}{=}\PYG{l+s+s2}{\PYGZdq{}}\PYG{l+s+s2}{Saturated Liquid}\PYG{l+s+s2}{\PYGZdq{}}\PYG{p}{)}
\PYG{n}{plt}\PYG{o}{.}\PYG{n}{plot}\PYG{p}{(}\PYG{n}{v\PYGZus{}vapor}\PYG{p}{,} \PYG{n}{pressures} \PYG{o}{/} \PYG{l+m+mf}{1e3}\PYG{p}{,} \PYG{l+s+s2}{\PYGZdq{}}\PYG{l+s+s2}{r\PYGZhy{}}\PYG{l+s+s2}{\PYGZdq{}}\PYG{p}{,} \PYG{n}{label}\PYG{o}{=}\PYG{l+s+s2}{\PYGZdq{}}\PYG{l+s+s2}{Saturated Vapor}\PYG{l+s+s2}{\PYGZdq{}}\PYG{p}{)}
\PYG{n}{plt}\PYG{o}{.}\PYG{n}{plot}\PYG{p}{(}
    \PYG{p}{[}\PYG{n}{v1}\PYG{p}{,} \PYG{n}{v2}\PYG{p}{,} \PYG{n}{v3}\PYG{p}{]}\PYG{p}{,}
    \PYG{n}{np}\PYG{o}{.}\PYG{n}{array}\PYG{p}{(}\PYG{p}{[}\PYG{n}{P1}\PYG{p}{,} \PYG{n}{P2}\PYG{p}{,} \PYG{n}{P3}\PYG{p}{]}\PYG{p}{)} \PYG{o}{/} \PYG{l+m+mf}{1e3}\PYG{p}{,}
    \PYG{l+s+s2}{\PYGZdq{}}\PYG{l+s+s2}{o\PYGZhy{}}\PYG{l+s+s2}{\PYGZdq{}}\PYG{p}{,}
    \PYG{n}{color}\PYG{o}{=}\PYG{l+s+s2}{\PYGZdq{}}\PYG{l+s+s2}{green}\PYG{l+s+s2}{\PYGZdq{}}\PYG{p}{,}
    \PYG{n}{markerfacecolor}\PYG{o}{=}\PYG{l+s+s2}{\PYGZdq{}}\PYG{l+s+s2}{yellow}\PYG{l+s+s2}{\PYGZdq{}}\PYG{p}{,}
\PYG{p}{)}
\PYG{n}{plt}\PYG{o}{.}\PYG{n}{arrow}\PYG{p}{(}
    \PYG{n}{v1}\PYG{p}{,} \PYG{n}{P1} \PYG{o}{/} \PYG{l+m+mf}{1e3}\PYG{p}{,} \PYG{n}{v2} \PYG{o}{\PYGZhy{}} \PYG{n}{v1}\PYG{p}{,} \PYG{l+m+mi}{0}\PYG{p}{,} \PYG{n}{head\PYGZus{}width}\PYG{o}{=}\PYG{l+m+mi}{5}\PYG{p}{,} \PYG{n}{head\PYGZus{}length}\PYG{o}{=}\PYG{l+m+mf}{0.01}\PYG{p}{,} \PYG{n}{fc}\PYG{o}{=}\PYG{l+s+s2}{\PYGZdq{}}\PYG{l+s+s2}{black}\PYG{l+s+s2}{\PYGZdq{}}\PYG{p}{,} \PYG{n}{ec}\PYG{o}{=}\PYG{l+s+s2}{\PYGZdq{}}\PYG{l+s+s2}{black}\PYG{l+s+s2}{\PYGZdq{}}
\PYG{p}{)}
\PYG{n}{plt}\PYG{o}{.}\PYG{n}{xscale}\PYG{p}{(}\PYG{l+s+s2}{\PYGZdq{}}\PYG{l+s+s2}{log}\PYG{l+s+s2}{\PYGZdq{}}\PYG{p}{)}
\PYG{n}{plt}\PYG{o}{.}\PYG{n}{xlabel}\PYG{p}{(}\PYG{l+s+s2}{\PYGZdq{}}\PYG{l+s+s2}{Specific Volume (m\PYGZca{}3/kg)}\PYG{l+s+s2}{\PYGZdq{}}\PYG{p}{)}
\PYG{n}{plt}\PYG{o}{.}\PYG{n}{ylabel}\PYG{p}{(}\PYG{l+s+s2}{\PYGZdq{}}\PYG{l+s+s2}{Pressure (kPa)}\PYG{l+s+s2}{\PYGZdq{}}\PYG{p}{)}
\PYG{n}{plt}\PYG{o}{.}\PYG{n}{title}\PYG{p}{(}\PYG{l+s+s2}{\PYGZdq{}}\PYG{l+s+s2}{P\PYGZhy{}v Diagram}\PYG{l+s+s2}{\PYGZdq{}}\PYG{p}{)}
\PYG{n}{plt}\PYG{o}{.}\PYG{n}{legend}\PYG{p}{(}\PYG{p}{)}
\PYG{n}{plt}\PYG{o}{.}\PYG{n}{grid}\PYG{p}{(}\PYG{k+kc}{True}\PYG{p}{,} \PYG{n}{which}\PYG{o}{=}\PYG{l+s+s2}{\PYGZdq{}}\PYG{l+s+s2}{both}\PYG{l+s+s2}{\PYGZdq{}}\PYG{p}{,} \PYG{n}{linestyle}\PYG{o}{=}\PYG{l+s+s2}{\PYGZdq{}}\PYG{l+s+s2}{\PYGZhy{}\PYGZhy{}}\PYG{l+s+s2}{\PYGZdq{}}\PYG{p}{,} \PYG{n}{linewidth}\PYG{o}{=}\PYG{l+m+mf}{0.5}\PYG{p}{)}

\PYG{c+c1}{\PYGZsh{} T\PYGZhy{}v diagram with dome}
\PYG{n}{plt}\PYG{o}{.}\PYG{n}{figure}\PYG{p}{(}\PYG{p}{)}
\PYG{n}{temperatures} \PYG{o}{=} \PYG{n}{np}\PYG{o}{.}\PYG{n}{linspace}\PYG{p}{(}\PYG{n}{T\PYGZus{}triple}\PYG{p}{,} \PYG{n}{T\PYGZus{}critical}\PYG{p}{,} \PYG{l+m+mi}{100}\PYG{p}{)}
\PYG{n}{v\PYGZus{}liquid\PYGZus{}T} \PYG{o}{=} \PYG{p}{[}\PYG{l+m+mi}{1} \PYG{o}{/} \PYG{n}{CP}\PYG{o}{.}\PYG{n}{PropsSI}\PYG{p}{(}\PYG{l+s+s2}{\PYGZdq{}}\PYG{l+s+s2}{D}\PYG{l+s+s2}{\PYGZdq{}}\PYG{p}{,} \PYG{l+s+s2}{\PYGZdq{}}\PYG{l+s+s2}{Q}\PYG{l+s+s2}{\PYGZdq{}}\PYG{p}{,} \PYG{l+m+mi}{0}\PYG{p}{,} \PYG{l+s+s2}{\PYGZdq{}}\PYG{l+s+s2}{T}\PYG{l+s+s2}{\PYGZdq{}}\PYG{p}{,} \PYG{n}{T}\PYG{p}{,} \PYG{l+s+s2}{\PYGZdq{}}\PYG{l+s+s2}{Water}\PYG{l+s+s2}{\PYGZdq{}}\PYG{p}{)} \PYG{k}{for} \PYG{n}{T} \PYG{o+ow}{in} \PYG{n}{temperatures}\PYG{p}{]}
\PYG{n}{v\PYGZus{}vapor\PYGZus{}T} \PYG{o}{=} \PYG{p}{[}\PYG{l+m+mi}{1} \PYG{o}{/} \PYG{n}{CP}\PYG{o}{.}\PYG{n}{PropsSI}\PYG{p}{(}\PYG{l+s+s2}{\PYGZdq{}}\PYG{l+s+s2}{D}\PYG{l+s+s2}{\PYGZdq{}}\PYG{p}{,} \PYG{l+s+s2}{\PYGZdq{}}\PYG{l+s+s2}{Q}\PYG{l+s+s2}{\PYGZdq{}}\PYG{p}{,} \PYG{l+m+mi}{1}\PYG{p}{,} \PYG{l+s+s2}{\PYGZdq{}}\PYG{l+s+s2}{T}\PYG{l+s+s2}{\PYGZdq{}}\PYG{p}{,} \PYG{n}{T}\PYG{p}{,} \PYG{l+s+s2}{\PYGZdq{}}\PYG{l+s+s2}{Water}\PYG{l+s+s2}{\PYGZdq{}}\PYG{p}{)} \PYG{k}{for} \PYG{n}{T} \PYG{o+ow}{in} \PYG{n}{temperatures}\PYG{p}{]}
\PYG{n}{plt}\PYG{o}{.}\PYG{n}{plot}\PYG{p}{(}\PYG{n}{v\PYGZus{}liquid\PYGZus{}T}\PYG{p}{,} \PYG{n}{temperatures}\PYG{p}{,} \PYG{l+s+s2}{\PYGZdq{}}\PYG{l+s+s2}{b\PYGZhy{}}\PYG{l+s+s2}{\PYGZdq{}}\PYG{p}{,} \PYG{n}{label}\PYG{o}{=}\PYG{l+s+s2}{\PYGZdq{}}\PYG{l+s+s2}{Saturated Liquid}\PYG{l+s+s2}{\PYGZdq{}}\PYG{p}{)}
\PYG{n}{plt}\PYG{o}{.}\PYG{n}{plot}\PYG{p}{(}\PYG{n}{v\PYGZus{}vapor\PYGZus{}T}\PYG{p}{,} \PYG{n}{temperatures}\PYG{p}{,} \PYG{l+s+s2}{\PYGZdq{}}\PYG{l+s+s2}{r\PYGZhy{}}\PYG{l+s+s2}{\PYGZdq{}}\PYG{p}{,} \PYG{n}{label}\PYG{o}{=}\PYG{l+s+s2}{\PYGZdq{}}\PYG{l+s+s2}{Saturated Vapor}\PYG{l+s+s2}{\PYGZdq{}}\PYG{p}{)}

\PYG{c+c1}{\PYGZsh{} Isobaric lines for T\PYGZhy{}v plot}
\PYG{n}{v\PYGZus{}range} \PYG{o}{=} \PYG{n}{np}\PYG{o}{.}\PYG{n}{logspace}\PYG{p}{(}\PYG{n}{np}\PYG{o}{.}\PYG{n}{log10}\PYG{p}{(}\PYG{n+nb}{min}\PYG{p}{(}\PYG{n}{v\PYGZus{}liquid\PYGZus{}T}\PYG{p}{)}\PYG{p}{)}\PYG{p}{,} \PYG{n}{np}\PYG{o}{.}\PYG{n}{log10}\PYG{p}{(}\PYG{n+nb}{max}\PYG{p}{(}\PYG{n}{v\PYGZus{}vapor\PYGZus{}T}\PYG{p}{)}\PYG{p}{)}\PYG{p}{,} \PYG{l+m+mi}{100}\PYG{p}{)}
\PYG{n}{T\PYGZus{}200kPa} \PYG{o}{=} \PYG{p}{[}\PYG{n}{CP}\PYG{o}{.}\PYG{n}{PropsSI}\PYG{p}{(}\PYG{l+s+s2}{\PYGZdq{}}\PYG{l+s+s2}{T}\PYG{l+s+s2}{\PYGZdq{}}\PYG{p}{,} \PYG{l+s+s2}{\PYGZdq{}}\PYG{l+s+s2}{D}\PYG{l+s+s2}{\PYGZdq{}}\PYG{p}{,} \PYG{l+m+mi}{1} \PYG{o}{/} \PYG{n}{v}\PYG{p}{,} \PYG{l+s+s2}{\PYGZdq{}}\PYG{l+s+s2}{P}\PYG{l+s+s2}{\PYGZdq{}}\PYG{p}{,} \PYG{n}{P1}\PYG{p}{,} \PYG{l+s+s2}{\PYGZdq{}}\PYG{l+s+s2}{Water}\PYG{l+s+s2}{\PYGZdq{}}\PYG{p}{)} \PYG{k}{for} \PYG{n}{v} \PYG{o+ow}{in} \PYG{n}{v\PYGZus{}range}\PYG{p}{]}
\PYG{n}{T\PYGZus{}25kPa} \PYG{o}{=} \PYG{p}{[}\PYG{n}{CP}\PYG{o}{.}\PYG{n}{PropsSI}\PYG{p}{(}\PYG{l+s+s2}{\PYGZdq{}}\PYG{l+s+s2}{T}\PYG{l+s+s2}{\PYGZdq{}}\PYG{p}{,} \PYG{l+s+s2}{\PYGZdq{}}\PYG{l+s+s2}{D}\PYG{l+s+s2}{\PYGZdq{}}\PYG{p}{,} \PYG{l+m+mi}{1} \PYG{o}{/} \PYG{n}{v}\PYG{p}{,} \PYG{l+s+s2}{\PYGZdq{}}\PYG{l+s+s2}{P}\PYG{l+s+s2}{\PYGZdq{}}\PYG{p}{,} \PYG{l+m+mf}{25e3}\PYG{p}{,} \PYG{l+s+s2}{\PYGZdq{}}\PYG{l+s+s2}{Water}\PYG{l+s+s2}{\PYGZdq{}}\PYG{p}{)} \PYG{k}{for} \PYG{n}{v} \PYG{o+ow}{in} \PYG{n}{v\PYGZus{}range}\PYG{p}{]}
\PYG{n}{plt}\PYG{o}{.}\PYG{n}{plot}\PYG{p}{(}\PYG{n}{v\PYGZus{}range}\PYG{p}{,} \PYG{n}{T\PYGZus{}200kPa}\PYG{p}{,} \PYG{l+s+s2}{\PYGZdq{}}\PYG{l+s+s2}{g\PYGZhy{}\PYGZhy{}}\PYG{l+s+s2}{\PYGZdq{}}\PYG{p}{,} \PYG{n}{label}\PYG{o}{=}\PYG{l+s+s2}{\PYGZdq{}}\PYG{l+s+s2}{p=200kPa}\PYG{l+s+s2}{\PYGZdq{}}\PYG{p}{)}
\PYG{n}{plt}\PYG{o}{.}\PYG{n}{plot}\PYG{p}{(}\PYG{n}{v\PYGZus{}range}\PYG{p}{,} \PYG{n}{T\PYGZus{}25kPa}\PYG{p}{,} \PYG{l+s+s2}{\PYGZdq{}}\PYG{l+s+s2}{m\PYGZhy{}\PYGZhy{}}\PYG{l+s+s2}{\PYGZdq{}}\PYG{p}{,} \PYG{n}{label}\PYG{o}{=}\PYG{l+s+s2}{\PYGZdq{}}\PYG{l+s+s2}{p=25kPa}\PYG{l+s+s2}{\PYGZdq{}}\PYG{p}{)}

\PYG{n}{plt}\PYG{o}{.}\PYG{n}{plot}\PYG{p}{(}\PYG{p}{[}\PYG{n}{v1}\PYG{p}{,} \PYG{n}{v2}\PYG{p}{,} \PYG{n}{v3}\PYG{p}{]}\PYG{p}{,} \PYG{p}{[}\PYG{n}{T1}\PYG{p}{,} \PYG{n}{T2}\PYG{p}{,} \PYG{n}{T3}\PYG{p}{]}\PYG{p}{,} \PYG{l+s+s2}{\PYGZdq{}}\PYG{l+s+s2}{o}\PYG{l+s+s2}{\PYGZdq{}}\PYG{p}{,} \PYG{n}{color}\PYG{o}{=}\PYG{l+s+s2}{\PYGZdq{}}\PYG{l+s+s2}{green}\PYG{l+s+s2}{\PYGZdq{}}\PYG{p}{,} \PYG{n}{markerfacecolor}\PYG{o}{=}\PYG{l+s+s2}{\PYGZdq{}}\PYG{l+s+s2}{yellow}\PYG{l+s+s2}{\PYGZdq{}}\PYG{p}{)}
\PYG{c+c1}{\PYGZsh{} plt.arrow(v1, T1, v2\PYGZhy{}v1, T2\PYGZhy{}T1, head\PYGZus{}width=0.01, head\PYGZus{}length=10, fc=\PYGZsq{}black\PYGZsq{}, ec=\PYGZsq{}black\PYGZsq{})}
\PYG{c+c1}{\PYGZsh{} plt.arrow(v2, T2, 0, T3\PYGZhy{}T2, head\PYGZus{}width=0.01, head\PYGZus{}length=10, fc=\PYGZsq{}black\PYGZsq{}, ec=\PYGZsq{}black\PYGZsq{})}
\PYG{n}{plt}\PYG{o}{.}\PYG{n}{xscale}\PYG{p}{(}\PYG{l+s+s2}{\PYGZdq{}}\PYG{l+s+s2}{log}\PYG{l+s+s2}{\PYGZdq{}}\PYG{p}{)}
\PYG{n}{plt}\PYG{o}{.}\PYG{n}{xlabel}\PYG{p}{(}\PYG{l+s+s2}{\PYGZdq{}}\PYG{l+s+s2}{Specific Volume (m\PYGZca{}3/kg)}\PYG{l+s+s2}{\PYGZdq{}}\PYG{p}{)}
\PYG{n}{plt}\PYG{o}{.}\PYG{n}{ylabel}\PYG{p}{(}\PYG{l+s+s2}{\PYGZdq{}}\PYG{l+s+s2}{Temperature (K)}\PYG{l+s+s2}{\PYGZdq{}}\PYG{p}{)}
\PYG{n}{plt}\PYG{o}{.}\PYG{n}{title}\PYG{p}{(}\PYG{l+s+s2}{\PYGZdq{}}\PYG{l+s+s2}{T\PYGZhy{}v Diagram}\PYG{l+s+s2}{\PYGZdq{}}\PYG{p}{)}
\PYG{n}{plt}\PYG{o}{.}\PYG{n}{ylim}\PYG{p}{(}\PYG{p}{[}\PYG{l+m+mi}{300}\PYG{p}{,} \PYG{l+m+mi}{660}\PYG{p}{]}\PYG{p}{)}
\PYG{n}{plt}\PYG{o}{.}\PYG{n}{legend}\PYG{p}{(}\PYG{p}{)}
\PYG{n}{plt}\PYG{o}{.}\PYG{n}{grid}\PYG{p}{(}\PYG{k+kc}{True}\PYG{p}{,} \PYG{n}{which}\PYG{o}{=}\PYG{l+s+s2}{\PYGZdq{}}\PYG{l+s+s2}{both}\PYG{l+s+s2}{\PYGZdq{}}\PYG{p}{,} \PYG{n}{linestyle}\PYG{o}{=}\PYG{l+s+s2}{\PYGZdq{}}\PYG{l+s+s2}{\PYGZhy{}\PYGZhy{}}\PYG{l+s+s2}{\PYGZdq{}}\PYG{p}{,} \PYG{n}{linewidth}\PYG{o}{=}\PYG{l+m+mf}{0.5}\PYG{p}{)}
\PYG{n}{plt}\PYG{o}{.}\PYG{n}{show}\PYG{p}{(}\PYG{p}{)}
\end{sphinxVerbatim}

\end{sphinxuseclass}\end{sphinxVerbatimInput}
\begin{sphinxVerbatimOutput}

\begin{sphinxuseclass}{cell_output}
\noindent\sphinxincludegraphics{{22e453b40418984a5197d144362c887672cfd72f3539b1435bc6bf5f67bc9408}.png}

\noindent\sphinxincludegraphics{{7d70f45afc5d7f919bbdb0834ba9181d7a9eff10d972474e306caf2a2598a234}.png}

\end{sphinxuseclass}\end{sphinxVerbatimOutput}

\end{sphinxuseclass}
\sphinxstepscope


\chapter{3. Ideal and Real Gases}
\label{\detokenize{notebooks/Chapter3/chapter3:ideal-and-real-gases}}\label{\detokenize{notebooks/Chapter3/chapter3::doc}}
\sphinxAtStartPar
This section covers concepts from \sphinxhref{https://pressbooks.bccampus.ca/thermo1/chapter/3-0-chapter-introduction-and-learning-objectives/}{Chapter 3 of “Introduction to Engineering Thermodynamics”} by Prof. Claire Yu Yan.

\sphinxstepscope


\section{Ideal gas law: Air}
\label{\detokenize{notebooks/Chapter3/CH3-Q1_v1:ideal-gas-law-air}}\label{\detokenize{notebooks/Chapter3/CH3-Q1_v1::doc}}
\sphinxAtStartPar
Imagine \(1\:kg\) of air confined in a space of \(1\:m^3\) volume. If the air is kept at room temperature, assuming ideal gas law applies to air in this case,

\sphinxAtStartPar
a) calculate air pressure in \(kPa\)?

\sphinxAtStartPar
b) what if the temperature increases to \(100 ^{\circ} C\)?

\sphinxAtStartPar
c) what if \(1\:kg\) air is added to the same compartment?

\sphinxAtStartPar
d) what if the space within which the air in kept is compressed to decrease the volume to \(0.8\:m^3\)


\subsection{Solution Approach for a)}
\label{\detokenize{notebooks/Chapter3/CH3-Q1_v1:solution-approach-for-a}}
\sphinxAtStartPar
the ideal gas law is an equation of state (EOS) which helps with predicting the state of a materiasl based on its properties Pressure \((P)\), specifiv volume \((v)\), and temperature \((T)\) as following:

\sphinxAtStartPar
\(Pv=RT\) ,which can also be written as 
\(PV=mRT\) where \(m\) is the mass of the material

\sphinxAtStartPar
to calculate pressure: 
\(P=mRT/V\)

\begin{sphinxuseclass}{cell}\begin{sphinxVerbatimInput}

\begin{sphinxuseclass}{cell_input}
\begin{sphinxVerbatim}[commandchars=\\\{\}]
\PYG{c+c1}{\PYGZsh{}defining variables}
\PYG{n}{m} \PYG{o}{=} \PYG{l+m+mi}{1} \PYG{c+c1}{\PYGZsh{}mass of air in kg}
\PYG{n}{R} \PYG{o}{=} \PYG{l+m+mf}{0.287} \PYG{c+c1}{\PYGZsh{}air gas constant in kJ/kg.K}
\PYG{n}{V} \PYG{o}{=} \PYG{l+m+mi}{1} \PYG{c+c1}{\PYGZsh{}volume of air in m3}
\PYG{n}{T} \PYG{o}{=} \PYG{l+m+mi}{25} \PYG{o}{+} \PYG{l+m+mf}{273.15} \PYG{c+c1}{\PYGZsh{}room temperature in K}

\PYG{c+c1}{\PYGZsh{} using ideal gas equation of state to calculate pressure}
\PYG{n}{P} \PYG{o}{=} \PYG{n}{m} \PYG{o}{*} \PYG{n}{R} \PYG{o}{*} \PYG{n}{T} \PYG{o}{/} \PYG{n}{V} \PYG{c+c1}{\PYGZsh{}air pressure in kPa}

\PYG{n+nb}{print}\PYG{p}{(}\PYG{l+s+s1}{\PYGZsq{}}\PYG{l+s+s1}{The air pressure is:}\PYG{l+s+s1}{\PYGZsq{}}\PYG{p}{,} \PYG{l+s+sa}{f}\PYG{l+s+s2}{\PYGZdq{}}\PYG{l+s+si}{\PYGZob{}}\PYG{n}{P}\PYG{l+s+si}{:}\PYG{l+s+s2}{.1f}\PYG{l+s+si}{\PYGZcb{}}\PYG{l+s+s2}{\PYGZdq{}}\PYG{p}{,} \PYG{l+s+s1}{\PYGZsq{}}\PYG{l+s+s1}{kPa}\PYG{l+s+s1}{\PYGZsq{}}\PYG{p}{)}
\end{sphinxVerbatim}

\end{sphinxuseclass}\end{sphinxVerbatimInput}
\begin{sphinxVerbatimOutput}

\begin{sphinxuseclass}{cell_output}
\begin{sphinxVerbatim}[commandchars=\\\{\}]
The air pressure is: 85.6 kPa
\end{sphinxVerbatim}

\end{sphinxuseclass}\end{sphinxVerbatimOutput}

\end{sphinxuseclass}

\subsection{Solution Approach for b)}
\label{\detokenize{notebooks/Chapter3/CH3-Q1_v1:solution-approach-for-b}}
\sphinxAtStartPar
for part b, the temperature increases to \(100 ^{\circ} C\). As a result, an increase in pressure is expected.

\begin{sphinxuseclass}{cell}\begin{sphinxVerbatimInput}

\begin{sphinxuseclass}{cell_input}
\begin{sphinxVerbatim}[commandchars=\\\{\}]
\PYG{n}{T} \PYG{o}{=} \PYG{l+m+mi}{100} \PYG{o}{+} \PYG{l+m+mf}{273.15} \PYG{c+c1}{\PYGZsh{}increased room temperature in K}
\PYG{n}{P} \PYG{o}{=} \PYG{n}{m} \PYG{o}{*} \PYG{n}{R} \PYG{o}{*} \PYG{n}{T} \PYG{o}{/} \PYG{n}{V} \PYG{c+c1}{\PYGZsh{}air pressure in kPa}
\PYG{n+nb}{print}\PYG{p}{(}\PYG{l+s+s1}{\PYGZsq{}}\PYG{l+s+s1}{The air pressure is:}\PYG{l+s+s1}{\PYGZsq{}}\PYG{p}{,} \PYG{l+s+sa}{f}\PYG{l+s+s2}{\PYGZdq{}}\PYG{l+s+si}{\PYGZob{}}\PYG{n}{P}\PYG{l+s+si}{:}\PYG{l+s+s2}{.1f}\PYG{l+s+si}{\PYGZcb{}}\PYG{l+s+s2}{\PYGZdq{}}\PYG{p}{,} \PYG{l+s+s1}{\PYGZsq{}}\PYG{l+s+s1}{kPa}\PYG{l+s+s1}{\PYGZsq{}}\PYG{p}{)}
\end{sphinxVerbatim}

\end{sphinxuseclass}\end{sphinxVerbatimInput}
\begin{sphinxVerbatimOutput}

\begin{sphinxuseclass}{cell_output}
\begin{sphinxVerbatim}[commandchars=\\\{\}]
The air pressure is: 107.1 kPa
\end{sphinxVerbatim}

\end{sphinxuseclass}\end{sphinxVerbatimOutput}

\end{sphinxuseclass}

\subsection{Solution Approach for c)}
\label{\detokenize{notebooks/Chapter3/CH3-Q1_v1:solution-approach-for-c}}
\sphinxAtStartPar
for part c, the number of air molecules increases. As a result, an increase in pressure is expected.

\begin{sphinxuseclass}{cell}\begin{sphinxVerbatimInput}

\begin{sphinxuseclass}{cell_input}
\begin{sphinxVerbatim}[commandchars=\\\{\}]
\PYG{n}{m} \PYG{o}{=} \PYG{l+m+mi}{2} \PYG{c+c1}{\PYGZsh{}increased mass of air in kg}
\PYG{n}{P} \PYG{o}{=} \PYG{n}{m} \PYG{o}{*} \PYG{n}{R} \PYG{o}{*} \PYG{n}{T} \PYG{o}{/} \PYG{n}{V} \PYG{c+c1}{\PYGZsh{}air pressure in kPa}
\PYG{n+nb}{print}\PYG{p}{(}\PYG{l+s+s1}{\PYGZsq{}}\PYG{l+s+s1}{The air pressure is:}\PYG{l+s+s1}{\PYGZsq{}}\PYG{p}{,} \PYG{l+s+sa}{f}\PYG{l+s+s2}{\PYGZdq{}}\PYG{l+s+si}{\PYGZob{}}\PYG{n}{P}\PYG{l+s+si}{:}\PYG{l+s+s2}{.1f}\PYG{l+s+si}{\PYGZcb{}}\PYG{l+s+s2}{\PYGZdq{}}\PYG{p}{,} \PYG{l+s+s1}{\PYGZsq{}}\PYG{l+s+s1}{kPa}\PYG{l+s+s1}{\PYGZsq{}}\PYG{p}{)}
\end{sphinxVerbatim}

\end{sphinxuseclass}\end{sphinxVerbatimInput}
\begin{sphinxVerbatimOutput}

\begin{sphinxuseclass}{cell_output}
\begin{sphinxVerbatim}[commandchars=\\\{\}]
The air pressure is: 214.2 kPa
\end{sphinxVerbatim}

\end{sphinxuseclass}\end{sphinxVerbatimOutput}

\end{sphinxuseclass}

\subsection{Solution Approach for d)}
\label{\detokenize{notebooks/Chapter3/CH3-Q1_v1:solution-approach-for-d}}
\sphinxAtStartPar
for part c, the volume of air decreases resulting in more air molecules per unit volume. As a result, an increase in pressure is expected.

\begin{sphinxuseclass}{cell}\begin{sphinxVerbatimInput}

\begin{sphinxuseclass}{cell_input}
\begin{sphinxVerbatim}[commandchars=\\\{\}]
\PYG{n}{V} \PYG{o}{=} \PYG{l+m+mf}{0.8} \PYG{c+c1}{\PYGZsh{}compressed volume of air in m3}
\PYG{n}{P} \PYG{o}{=} \PYG{n}{m} \PYG{o}{*} \PYG{n}{R} \PYG{o}{*} \PYG{n}{T} \PYG{o}{/} \PYG{n}{V} \PYG{c+c1}{\PYGZsh{}air pressure in kPa}
\PYG{n+nb}{print}\PYG{p}{(}\PYG{l+s+s1}{\PYGZsq{}}\PYG{l+s+s1}{The air pressure is:}\PYG{l+s+s1}{\PYGZsq{}}\PYG{p}{,} \PYG{l+s+sa}{f}\PYG{l+s+s2}{\PYGZdq{}}\PYG{l+s+si}{\PYGZob{}}\PYG{n}{P}\PYG{l+s+si}{:}\PYG{l+s+s2}{.1f}\PYG{l+s+si}{\PYGZcb{}}\PYG{l+s+s2}{\PYGZdq{}}\PYG{p}{,} \PYG{l+s+s1}{\PYGZsq{}}\PYG{l+s+s1}{kPa}\PYG{l+s+s1}{\PYGZsq{}}\PYG{p}{)}
\end{sphinxVerbatim}

\end{sphinxuseclass}\end{sphinxVerbatimInput}
\begin{sphinxVerbatimOutput}

\begin{sphinxuseclass}{cell_output}
\begin{sphinxVerbatim}[commandchars=\\\{\}]
The air pressure is: 267.7 kPa
\end{sphinxVerbatim}

\end{sphinxuseclass}\end{sphinxVerbatimOutput}

\end{sphinxuseclass}
\sphinxstepscope


\section{Thermo\sphinxhyphen{}mechanical Equilibrium: Partitions in a Box}
\label{\detokenize{notebooks/Chapter3/CH3-Q2_v1:thermo-mechanical-equilibrium-partitions-in-a-box}}\label{\detokenize{notebooks/Chapter3/CH3-Q2_v1::doc}}
\sphinxAtStartPar
consider a box of \(5\:m^3\) volume composed of three isolated compartments A, B and C containing air in three diffrent states. Part A contains \(0.5\:kg\) air at \(200\:kPa\). Part B measure \(25 ^{\circ} C\) in temperature, \(100\:kPa\) is pressure and occupies \(1\:m^3\) of space. Finally, part C contains \(2\:kg\) of the same materials at a temperature of \(15 ^{\circ} C\) and a pressure of \(50\:kPa\).

\sphinxAtStartPar
Assuming ideal gas law for air in these states, calculate:

\sphinxAtStartPar
a) volume of air in partition C

\sphinxAtStartPar
b) volume of partition A

\sphinxAtStartPar
c) mass of air in partition B

\sphinxAtStartPar
d) temperature of air in partition A in \(^{\circ}  C\)

\sphinxAtStartPar
e) now imagine all the seperators break and air in three compartments mix together. What would be the final pressure if the box is let to get in thermal stability with room temperature?


\section{Solution Approach for a)}
\label{\detokenize{notebooks/Chapter3/CH3-Q2_v1:solution-approach-for-a}}
\sphinxAtStartPar
ideal gas equation can be used as EOS since the assumption is valid as per the statement, so

\sphinxAtStartPar
\(PV=mRT\) 
\(V=mRT/P\)

\begin{sphinxuseclass}{cell}\begin{sphinxVerbatimInput}

\begin{sphinxuseclass}{cell_input}
\begin{sphinxVerbatim}[commandchars=\\\{\}]
\PYG{c+c1}{\PYGZsh{}defining variables and looking up tables}
\PYG{n}{R} \PYG{o}{=} \PYG{l+m+mf}{0.287} \PYG{c+c1}{\PYGZsh{}air gas constant in kJ/kg.K}
\PYG{n}{m\PYGZus{}C} \PYG{o}{=} \PYG{l+m+mi}{2} \PYG{c+c1}{\PYGZsh{}mass of air in partition c in kg}
\PYG{n}{T\PYGZus{}C} \PYG{o}{=} \PYG{l+m+mi}{15} \PYG{o}{+} \PYG{l+m+mf}{273.15} \PYG{c+c1}{\PYGZsh{}temperature of air in partc in K}
\PYG{n}{P\PYGZus{}C} \PYG{o}{=} \PYG{l+m+mi}{50} \PYG{c+c1}{\PYGZsh{}pressure of air in partition c in kPa}

\PYG{c+c1}{\PYGZsh{}using ideal gas law equation of state}
\PYG{n}{V\PYGZus{}C} \PYG{o}{=} \PYG{n}{m\PYGZus{}C} \PYG{o}{*} \PYG{n}{R} \PYG{o}{*} \PYG{n}{T\PYGZus{}C} \PYG{o}{/} \PYG{n}{P\PYGZus{}C} \PYG{c+c1}{\PYGZsh{}volume of partition c in m3}

\PYG{n+nb}{print}\PYG{p}{(}\PYG{l+s+s1}{\PYGZsq{}}\PYG{l+s+s1}{The air volume in partition C is:}\PYG{l+s+s1}{\PYGZsq{}}\PYG{p}{,} \PYG{l+s+sa}{f}\PYG{l+s+s2}{\PYGZdq{}}\PYG{l+s+si}{\PYGZob{}}\PYG{n}{V\PYGZus{}C}\PYG{l+s+si}{:}\PYG{l+s+s2}{.1f}\PYG{l+s+si}{\PYGZcb{}}\PYG{l+s+s2}{\PYGZdq{}}\PYG{p}{,} \PYG{l+s+s1}{\PYGZsq{}}\PYG{l+s+s1}{m3}\PYG{l+s+s1}{\PYGZsq{}}\PYG{p}{)}
\end{sphinxVerbatim}

\end{sphinxuseclass}\end{sphinxVerbatimInput}
\begin{sphinxVerbatimOutput}

\begin{sphinxuseclass}{cell_output}
\begin{sphinxVerbatim}[commandchars=\\\{\}]
The air volume in partition C is: 3.3 m3
\end{sphinxVerbatim}

\end{sphinxuseclass}\end{sphinxVerbatimOutput}

\end{sphinxuseclass}

\section{Solution Approach for b)}
\label{\detokenize{notebooks/Chapter3/CH3-Q2_v1:solution-approach-for-b}}
\sphinxAtStartPar
The box is composed of three partitions, therefore:

\sphinxAtStartPar
\(V_A+V_B+V_C=5\:m^3\) 
\(V_A=5-V_B-V_C\)

\begin{sphinxuseclass}{cell}\begin{sphinxVerbatimInput}

\begin{sphinxuseclass}{cell_input}
\begin{sphinxVerbatim}[commandchars=\\\{\}]
\PYG{n}{V\PYGZus{}B} \PYG{o}{=} \PYG{l+m+mi}{1} \PYG{c+c1}{\PYGZsh{}volume of partition B in m3}
\PYG{n}{V\PYGZus{}A} \PYG{o}{=} \PYG{l+m+mi}{5} \PYG{o}{\PYGZhy{}} \PYG{n}{V\PYGZus{}B} \PYG{o}{\PYGZhy{}} \PYG{n}{V\PYGZus{}C}
\PYG{n+nb}{print}\PYG{p}{(}\PYG{l+s+s1}{\PYGZsq{}}\PYG{l+s+s1}{The air volume in partition A is:}\PYG{l+s+s1}{\PYGZsq{}}\PYG{p}{,} \PYG{l+s+sa}{f}\PYG{l+s+s2}{\PYGZdq{}}\PYG{l+s+si}{\PYGZob{}}\PYG{n}{V\PYGZus{}A}\PYG{l+s+si}{:}\PYG{l+s+s2}{.3f}\PYG{l+s+si}{\PYGZcb{}}\PYG{l+s+s2}{\PYGZdq{}}\PYG{p}{,} \PYG{l+s+s1}{\PYGZsq{}}\PYG{l+s+s1}{m3}\PYG{l+s+s1}{\PYGZsq{}}\PYG{p}{)}
\end{sphinxVerbatim}

\end{sphinxuseclass}\end{sphinxVerbatimInput}
\begin{sphinxVerbatimOutput}

\begin{sphinxuseclass}{cell_output}
\begin{sphinxVerbatim}[commandchars=\\\{\}]
The air volume in partition A is: 0.692 m3
\end{sphinxVerbatim}

\end{sphinxuseclass}\end{sphinxVerbatimOutput}

\end{sphinxuseclass}

\section{Solution Approach for c)}
\label{\detokenize{notebooks/Chapter3/CH3-Q2_v1:solution-approach-for-c}}
\sphinxAtStartPar
from ideal gas law

\sphinxAtStartPar
\(m=PV/{RT}\)

\begin{sphinxuseclass}{cell}\begin{sphinxVerbatimInput}

\begin{sphinxuseclass}{cell_input}
\begin{sphinxVerbatim}[commandchars=\\\{\}]
\PYG{c+c1}{\PYGZsh{}defining state variables}
\PYG{n}{P\PYGZus{}B} \PYG{o}{=} \PYG{l+m+mi}{100} \PYG{c+c1}{\PYGZsh{}pressure of air in partition b in kPa}
\PYG{n}{V\PYGZus{}B} \PYG{o}{=} \PYG{l+m+mi}{1} \PYG{c+c1}{\PYGZsh{}volume of partition b in m3}
\PYG{n}{T\PYGZus{}B} \PYG{o}{=} \PYG{l+m+mi}{25} \PYG{o}{+} \PYG{l+m+mf}{273.15} \PYG{c+c1}{\PYGZsh{}temperature of air in partition b in K}

\PYG{c+c1}{\PYGZsh{}using ideal gas law equation of state}
\PYG{n}{m\PYGZus{}B} \PYG{o}{=} \PYG{n}{P\PYGZus{}B} \PYG{o}{*} \PYG{n}{V\PYGZus{}B} \PYG{o}{/} \PYG{p}{(}\PYG{n}{R} \PYG{o}{*} \PYG{n}{T\PYGZus{}B}\PYG{p}{)} \PYG{c+c1}{\PYGZsh{}mass of air in partiton b in kg}
\PYG{n+nb}{print}\PYG{p}{(}\PYG{l+s+s1}{\PYGZsq{}}\PYG{l+s+s1}{The mass of air in partition B is:}\PYG{l+s+s1}{\PYGZsq{}}\PYG{p}{,} \PYG{l+s+sa}{f}\PYG{l+s+s2}{\PYGZdq{}}\PYG{l+s+si}{\PYGZob{}}\PYG{n}{m\PYGZus{}B}\PYG{l+s+si}{:}\PYG{l+s+s2}{.1f}\PYG{l+s+si}{\PYGZcb{}}\PYG{l+s+s2}{\PYGZdq{}}\PYG{p}{,} \PYG{l+s+s1}{\PYGZsq{}}\PYG{l+s+s1}{kg}\PYG{l+s+s1}{\PYGZsq{}}\PYG{p}{)}
\end{sphinxVerbatim}

\end{sphinxuseclass}\end{sphinxVerbatimInput}
\begin{sphinxVerbatimOutput}

\begin{sphinxuseclass}{cell_output}
\begin{sphinxVerbatim}[commandchars=\\\{\}]
The mass of air in partition B is: 1.2 kg
\end{sphinxVerbatim}

\end{sphinxuseclass}\end{sphinxVerbatimOutput}

\end{sphinxuseclass}

\section{Solution Approach for d)}
\label{\detokenize{notebooks/Chapter3/CH3-Q2_v1:solution-approach-for-d}}
\sphinxAtStartPar
from ideal gas law

\sphinxAtStartPar
\(T=PV/{mR}\)

\begin{sphinxuseclass}{cell}\begin{sphinxVerbatimInput}

\begin{sphinxuseclass}{cell_input}
\begin{sphinxVerbatim}[commandchars=\\\{\}]
\PYG{c+c1}{\PYGZsh{}defining state variables}
\PYG{n}{P\PYGZus{}A} \PYG{o}{=} \PYG{l+m+mi}{200} \PYG{c+c1}{\PYGZsh{}pressure of air in partition a in kPa}
\PYG{n}{m\PYGZus{}A} \PYG{o}{=} \PYG{l+m+mf}{0.5} \PYG{c+c1}{\PYGZsh{}mass of air in partiton a in kg}

\PYG{c+c1}{\PYGZsh{}using ideal gas law equation of state}
\PYG{n}{T\PYGZus{}A} \PYG{o}{=} \PYG{n}{P\PYGZus{}A} \PYG{o}{*} \PYG{n}{V\PYGZus{}A} \PYG{o}{/} \PYG{p}{(}\PYG{n}{m\PYGZus{}A} \PYG{o}{*} \PYG{n}{R}\PYG{p}{)} \PYG{c+c1}{\PYGZsh{}temperature of air in partition a in K}

\PYG{c+c1}{\PYGZsh{}converting from Kelvins to Celsius}
\PYG{n}{T\PYGZus{}AC} \PYG{o}{=} \PYG{n}{T\PYGZus{}A} \PYG{o}{\PYGZhy{}} \PYG{l+m+mf}{273.15} \PYG{c+c1}{\PYGZsh{}temperature of air in partition a in C}
\PYG{n+nb}{print}\PYG{p}{(}\PYG{l+s+s1}{\PYGZsq{}}\PYG{l+s+s1}{The temperature of air in partition A is:}\PYG{l+s+s1}{\PYGZsq{}}\PYG{p}{,} \PYG{l+s+sa}{f}\PYG{l+s+s2}{\PYGZdq{}}\PYG{l+s+si}{\PYGZob{}}\PYG{n}{T\PYGZus{}AC}\PYG{l+s+si}{:}\PYG{l+s+s2}{.1f}\PYG{l+s+si}{\PYGZcb{}}\PYG{l+s+s2}{\PYGZdq{}}\PYG{p}{,} \PYG{l+s+s1}{\PYGZsq{}}\PYG{l+s+s1}{Degrees Celcius}\PYG{l+s+s1}{\PYGZsq{}}\PYG{p}{)}
\end{sphinxVerbatim}

\end{sphinxuseclass}\end{sphinxVerbatimInput}
\begin{sphinxVerbatimOutput}

\begin{sphinxuseclass}{cell_output}
\begin{sphinxVerbatim}[commandchars=\\\{\}]
The temperature of air in partition A is: 691.4 Degrees Celcius
\end{sphinxVerbatim}

\end{sphinxuseclass}\end{sphinxVerbatimOutput}

\end{sphinxuseclass}

\section{Solution Approach for e)}
\label{\detokenize{notebooks/Chapter3/CH3-Q2_v1:solution-approach-for-e}}
\sphinxAtStartPar
referring to ideal gas law equation of state,

\sphinxAtStartPar
\(P=mRT/V\)

\sphinxAtStartPar
The temperature would be in \(25 ^{\circ} C\) since the box is in thermal stability with room temperature.

\begin{sphinxuseclass}{cell}\begin{sphinxVerbatimInput}

\begin{sphinxuseclass}{cell_input}
\begin{sphinxVerbatim}[commandchars=\\\{\}]
\PYG{c+c1}{\PYGZsh{}defining/calculating state variables}
\PYG{n}{T} \PYG{o}{=} \PYG{l+m+mi}{25} \PYG{o}{+} \PYG{l+m+mf}{273.15} \PYG{c+c1}{\PYGZsh{}final temperature of air in K}
\PYG{n}{m} \PYG{o}{=} \PYG{n}{m\PYGZus{}A} \PYG{o}{+} \PYG{n}{m\PYGZus{}B} \PYG{o}{+} \PYG{n}{m\PYGZus{}C} \PYG{c+c1}{\PYGZsh{}final mass of the mixture}
\PYG{n}{V} \PYG{o}{=} \PYG{l+m+mi}{5} \PYG{c+c1}{\PYGZsh{}total volume of the box in m3}

\PYG{c+c1}{\PYGZsh{}using ideal gas law equation of state}
\PYG{n}{P} \PYG{o}{=} \PYG{n}{m} \PYG{o}{*} \PYG{n}{R} \PYG{o}{*} \PYG{n}{T} \PYG{o}{/} \PYG{n}{V} \PYG{c+c1}{\PYGZsh{}final pressure of the mixture in kPa}

\PYG{n+nb}{print}\PYG{p}{(}\PYG{l+s+s1}{\PYGZsq{}}\PYG{l+s+s1}{The final pressure of mixed air is:}\PYG{l+s+s1}{\PYGZsq{}}\PYG{p}{,} \PYG{l+s+sa}{f}\PYG{l+s+s2}{\PYGZdq{}}\PYG{l+s+si}{\PYGZob{}}\PYG{n}{P}\PYG{l+s+si}{:}\PYG{l+s+s2}{.1f}\PYG{l+s+si}{\PYGZcb{}}\PYG{l+s+s2}{\PYGZdq{}}\PYG{p}{,} \PYG{l+s+s1}{\PYGZsq{}}\PYG{l+s+s1}{kPa}\PYG{l+s+s1}{\PYGZsq{}}\PYG{p}{)}
\end{sphinxVerbatim}

\end{sphinxuseclass}\end{sphinxVerbatimInput}
\begin{sphinxVerbatimOutput}

\begin{sphinxuseclass}{cell_output}
\begin{sphinxVerbatim}[commandchars=\\\{\}]
The final pressure of mixed air is: 62.8 kPa
\end{sphinxVerbatim}

\end{sphinxuseclass}\end{sphinxVerbatimOutput}

\end{sphinxuseclass}
\sphinxstepscope


\section{Piston\sphinxhyphen{}cylinder system for Air}
\label{\detokenize{notebooks/Chapter3/CH3-Q3_edited_edited:piston-cylinder-system-for-air}}\label{\detokenize{notebooks/Chapter3/CH3-Q3_edited_edited::doc}}
\sphinxAtStartPar
Consider \(2\:kg\) air at \(200\:kPa\) and \(25 ^{\circ}  C\) stored in a cylinder\sphinxhyphen{}piston system which is in thermal equilibrium with its surrounding. Assuming air at this state as ideal gas,

\sphinxAtStartPar
a) calculate how much space does air occupy in this state

\sphinxAtStartPar
b) if the piston is moved to compress air to half its initial volume at constant temperature, calculate the final air pressurea

\sphinxAtStartPar
c) plot the process on P\sphinxhyphen{}V, P\sphinxhyphen{}T and T\sphinxhyphen{}V diagrams.


\subsection{Solution Approach for a)}
\label{\detokenize{notebooks/Chapter3/CH3-Q3_edited_edited:solution-approach-for-a}}
\sphinxAtStartPar
ideal gas equation can be used as EOS since the assumption is valid as per the statement, so

\sphinxAtStartPar
\(PV=mRT\) 
\(V=mRT/P\)

\begin{sphinxuseclass}{cell}\begin{sphinxVerbatimInput}

\begin{sphinxuseclass}{cell_input}
\begin{sphinxVerbatim}[commandchars=\\\{\}]
\PYG{c+c1}{\PYGZsh{}defining variables and looking up tables}
\PYG{n}{R} \PYG{o}{=} \PYG{l+m+mf}{0.287} \PYG{c+c1}{\PYGZsh{}air gas constant in kJ/kg.K}
\PYG{n}{m} \PYG{o}{=} \PYG{l+m+mi}{2} \PYG{c+c1}{\PYGZsh{}mass of air in kg}
\PYG{n}{T} \PYG{o}{=} \PYG{l+m+mi}{25} \PYG{o}{+} \PYG{l+m+mf}{273.15} \PYG{c+c1}{\PYGZsh{}temperature of air in K}
\PYG{n}{P} \PYG{o}{=} \PYG{l+m+mi}{200} \PYG{c+c1}{\PYGZsh{}pressure of airc in kPa}

\PYG{c+c1}{\PYGZsh{}using ideal gas law equation of state}
\PYG{n}{V\PYGZus{}1} \PYG{o}{=} \PYG{n}{m} \PYG{o}{*} \PYG{n}{R} \PYG{o}{*} \PYG{n}{T} \PYG{o}{/} \PYG{n}{P} \PYG{c+c1}{\PYGZsh{}volume of air in m3}

\PYG{n+nb}{print}\PYG{p}{(}\PYG{l+s+s1}{\PYGZsq{}}\PYG{l+s+s1}{The air volume is:}\PYG{l+s+s1}{\PYGZsq{}}\PYG{p}{,} \PYG{n+nb}{round}\PYG{p}{(}\PYG{n}{V\PYGZus{}1}\PYG{p}{,}\PYG{l+m+mi}{3}\PYG{p}{)}\PYG{p}{,} \PYG{l+s+s1}{\PYGZsq{}}\PYG{l+s+s1}{m3}\PYG{l+s+s1}{\PYGZsq{}}\PYG{p}{)}
\end{sphinxVerbatim}

\end{sphinxuseclass}\end{sphinxVerbatimInput}
\begin{sphinxVerbatimOutput}

\begin{sphinxuseclass}{cell_output}
\begin{sphinxVerbatim}[commandchars=\\\{\}]
The air volume is: 0.856 m3
\end{sphinxVerbatim}

\end{sphinxuseclass}\end{sphinxVerbatimOutput}

\end{sphinxuseclass}

\subsection{Solution Approach for b)}
\label{\detokenize{notebooks/Chapter3/CH3-Q3_edited_edited:solution-approach-for-b}}
\sphinxAtStartPar
The box is compressed to half its initial volume therefore \(V_2=V_1/2\) and temperature remains constant as per the question’s statement.

\sphinxAtStartPar
Ideal gas equation solved for pressure (P) is to be used to calculate pressure.

\sphinxAtStartPar
\(P=mRT/V\)

\begin{sphinxuseclass}{cell}\begin{sphinxVerbatimInput}

\begin{sphinxuseclass}{cell_input}
\begin{sphinxVerbatim}[commandchars=\\\{\}]
\PYG{n}{V\PYGZus{}2} \PYG{o}{=} \PYG{n}{V\PYGZus{}1} \PYG{o}{/} \PYG{l+m+mi}{2} \PYG{c+c1}{\PYGZsh{}volume of air after compression}

\PYG{c+c1}{\PYGZsh{}using ideal gas law equation of state}
\PYG{n}{P\PYGZus{}2} \PYG{o}{=} \PYG{n}{m} \PYG{o}{*} \PYG{n}{R} \PYG{o}{*} \PYG{n}{T} \PYG{o}{/} \PYG{n}{V\PYGZus{}2} \PYG{c+c1}{\PYGZsh{}final pressure of the mixture in kPa}
\PYG{n+nb}{print}\PYG{p}{(}\PYG{l+s+s1}{\PYGZsq{}}\PYG{l+s+s1}{Secondary air pressure is:}\PYG{l+s+s1}{\PYGZsq{}}\PYG{p}{,} \PYG{n}{P\PYGZus{}2}\PYG{p}{,} \PYG{l+s+s1}{\PYGZsq{}}\PYG{l+s+s1}{kPa}\PYG{l+s+s1}{\PYGZsq{}}\PYG{p}{)}
\end{sphinxVerbatim}

\end{sphinxuseclass}\end{sphinxVerbatimInput}
\begin{sphinxVerbatimOutput}

\begin{sphinxuseclass}{cell_output}
\begin{sphinxVerbatim}[commandchars=\\\{\}]
Secondary air pressure is: 400.0 kPa
\end{sphinxVerbatim}

\end{sphinxuseclass}\end{sphinxVerbatimOutput}

\end{sphinxuseclass}

\subsection{Solution Approach for c)}
\label{\detokenize{notebooks/Chapter3/CH3-Q3_edited_edited:solution-approach-for-c}}
\sphinxAtStartPar
for a P\sphinxhyphen{}v diagram, the ideal gas equation is to be used in a form in which pressure(\(P\)) and volume(\(v\)) are the variables calculated one based on the other one.

\sphinxAtStartPar
\(P=mRT(1/V)\)

\sphinxAtStartPar
where mRT would be a constant value

\begin{sphinxuseclass}{cell}\begin{sphinxVerbatimInput}

\begin{sphinxuseclass}{cell_input}
\begin{sphinxVerbatim}[commandchars=\\\{\}]
\PYG{c+c1}{\PYGZsh{}import the libraries we\PYGZsq{}ll need}
\PYG{k+kn}{import} \PYG{n+nn}{numpy} \PYG{k}{as} \PYG{n+nn}{np}
\PYG{k+kn}{import} \PYG{n+nn}{matplotlib}\PYG{n+nn}{.}\PYG{n+nn}{pyplot} \PYG{k}{as} \PYG{n+nn}{plt}

\PYG{c+c1}{\PYGZsh{}building a range for volume (v) so that pressure (P) is calculated based upon }
\PYG{n}{V\PYGZus{}max} \PYG{o}{=} \PYG{n}{V\PYGZus{}1} \PYG{c+c1}{\PYGZsh{}maximum amount for volume in the process}
\PYG{n}{V\PYGZus{}min} \PYG{o}{=} \PYG{n}{V\PYGZus{}2} \PYG{c+c1}{\PYGZsh{}minimum amount for volume in the process}
\PYG{n}{V\PYGZus{}vals} \PYG{o}{=} \PYG{n}{np}\PYG{o}{.}\PYG{n}{linspace}\PYG{p}{(}\PYG{n}{V\PYGZus{}min}\PYG{p}{,} \PYG{n}{V\PYGZus{}max}\PYG{p}{,} \PYG{l+m+mi}{1000}\PYG{p}{)}  \PYG{c+c1}{\PYGZsh{} define an array of values for volume (v)}

\PYG{c+c1}{\PYGZsh{}calculating pressure (P) for the array of volume values (V\PYGZus{}vals)}
\PYG{n}{P\PYGZus{}vals} \PYG{o}{=} \PYG{n}{m} \PYG{o}{*} \PYG{n}{R} \PYG{o}{*} \PYG{n}{T} \PYG{o}{/} \PYG{n}{V\PYGZus{}vals}

\PYG{n}{plt}\PYG{o}{.}\PYG{n}{plot}\PYG{p}{(}\PYG{n}{V\PYGZus{}vals}\PYG{p}{,} \PYG{n}{P\PYGZus{}vals}\PYG{p}{)}  \PYG{c+c1}{\PYGZsh{} plot pressure vs. volume}
\PYG{n}{plt}\PYG{o}{.}\PYG{n}{ylabel}\PYG{p}{(}\PYG{l+s+s2}{\PYGZdq{}}\PYG{l+s+s2}{Pressure [kPa]}\PYG{l+s+s2}{\PYGZdq{}}\PYG{p}{)}  \PYG{c+c1}{\PYGZsh{} give y axis a label}
\PYG{n}{plt}\PYG{o}{.}\PYG{n}{xlabel}\PYG{p}{(}\PYG{l+s+s2}{\PYGZdq{}}\PYG{l+s+s2}{Volume [m3]}\PYG{l+s+s2}{\PYGZdq{}}\PYG{p}{)}  \PYG{c+c1}{\PYGZsh{} give x axis a label}


\PYG{c+c1}{\PYGZsh{}add\PYGZhy{}ons to illustrate process path}
\PYG{n}{plt}\PYG{o}{.}\PYG{n}{xlim}\PYG{p}{(}\PYG{l+m+mf}{0.4}\PYG{p}{,} \PYG{l+m+mf}{0.9}\PYG{p}{)}
\PYG{n}{plt}\PYG{o}{.}\PYG{n}{ylim}\PYG{p}{(}\PYG{l+m+mi}{190}\PYG{p}{,} \PYG{l+m+mi}{425}\PYG{p}{)}
\PYG{n}{plt}\PYG{o}{.}\PYG{n}{text}\PYG{p}{(}\PYG{l+m+mf}{0.87}\PYG{p}{,} \PYG{l+m+mi}{198}\PYG{p}{,} \PYG{l+s+s1}{\PYGZsq{}}\PYG{l+s+s1}{1}\PYG{l+s+s1}{\PYGZsq{}}\PYG{p}{,} \PYG{n}{fontsize} \PYG{o}{=} \PYG{l+m+mi}{15}\PYG{p}{)}
\PYG{n}{plt}\PYG{o}{.}\PYG{n}{text}\PYG{p}{(}\PYG{l+m+mf}{0.42}\PYG{p}{,} \PYG{l+m+mi}{410}\PYG{p}{,} \PYG{l+s+s1}{\PYGZsq{}}\PYG{l+s+s1}{2}\PYG{l+s+s1}{\PYGZsq{}}\PYG{p}{,} \PYG{n}{fontsize} \PYG{o}{=} \PYG{l+m+mi}{15}\PYG{p}{)}
\PYG{n}{plt}\PYG{o}{.}\PYG{n}{plot}\PYG{p}{(}\PYG{n}{V\PYGZus{}vals}\PYG{p}{[}\PYG{l+m+mi}{0}\PYG{p}{]}\PYG{p}{,} \PYG{n}{P\PYGZus{}vals}\PYG{p}{[}\PYG{l+m+mi}{0}\PYG{p}{]}\PYG{p}{,} \PYG{l+s+s1}{\PYGZsq{}}\PYG{l+s+s1}{ro}\PYG{l+s+s1}{\PYGZsq{}}\PYG{p}{)}
\PYG{n}{plt}\PYG{o}{.}\PYG{n}{plot}\PYG{p}{(}\PYG{n}{V\PYGZus{}vals}\PYG{p}{[}\PYG{o}{\PYGZhy{}}\PYG{l+m+mi}{1}\PYG{p}{]}\PYG{p}{,} \PYG{n}{P\PYGZus{}vals}\PYG{p}{[}\PYG{o}{\PYGZhy{}}\PYG{l+m+mi}{1}\PYG{p}{]}\PYG{p}{,} \PYG{l+s+s1}{\PYGZsq{}}\PYG{l+s+s1}{ro}\PYG{l+s+s1}{\PYGZsq{}}\PYG{p}{)}
\end{sphinxVerbatim}

\end{sphinxuseclass}\end{sphinxVerbatimInput}
\begin{sphinxVerbatimOutput}

\begin{sphinxuseclass}{cell_output}
\begin{sphinxVerbatim}[commandchars=\\\{\}]
[\PYGZlt{}matplotlib.lines.Line2D at 0x7f6a345b4bb0\PYGZgt{}]
\end{sphinxVerbatim}

\noindent\sphinxincludegraphics{{9c591140b4937334eeb083f2b8250e11d0d5903120aedd73839d9766bc2b850d}.png}

\end{sphinxuseclass}\end{sphinxVerbatimOutput}

\end{sphinxuseclass}
\begin{sphinxuseclass}{cell}\begin{sphinxVerbatimInput}

\begin{sphinxuseclass}{cell_input}
\begin{sphinxVerbatim}[commandchars=\\\{\}]
\PYG{c+c1}{\PYGZsh{}building temperature range to plot P\PYGZhy{}T and V\PYGZus{}T diagrams}
\PYG{n}{T\PYGZus{}C} \PYG{o}{=} \PYG{n}{T} \PYG{o}{\PYGZhy{}} \PYG{l+m+mf}{273.15} \PYG{c+c1}{\PYGZsh{}process temperature in Celsius}
\PYG{n}{T\PYGZus{}vals} \PYG{o}{=} \PYG{n}{np}\PYG{o}{.}\PYG{n}{linspace}\PYG{p}{(}\PYG{n}{T\PYGZus{}C}\PYG{p}{,} \PYG{n}{T\PYGZus{}C}\PYG{p}{,} \PYG{l+m+mi}{1000}\PYG{p}{)}  \PYG{c+c1}{\PYGZsh{} define an array of values for temperature (T)}
\PYG{n}{plt}\PYG{o}{.}\PYG{n}{plot}\PYG{p}{(}\PYG{n}{T\PYGZus{}vals}\PYG{p}{,} \PYG{n}{P\PYGZus{}vals}\PYG{p}{)}  \PYG{c+c1}{\PYGZsh{} plot pressure vs. temperature}
\PYG{n}{plt}\PYG{o}{.}\PYG{n}{ylabel}\PYG{p}{(}\PYG{l+s+s2}{\PYGZdq{}}\PYG{l+s+s2}{Pressure [kPa]}\PYG{l+s+s2}{\PYGZdq{}}\PYG{p}{)}  \PYG{c+c1}{\PYGZsh{} give y axis a label}
\PYG{n}{plt}\PYG{o}{.}\PYG{n}{xlabel}\PYG{p}{(}\PYG{l+s+s2}{\PYGZdq{}}\PYG{l+s+s2}{Temperature [C]}\PYG{l+s+s2}{\PYGZdq{}}\PYG{p}{)}  \PYG{c+c1}{\PYGZsh{} give x axis a label}

\PYG{c+c1}{\PYGZsh{}add\PYGZhy{}ons to illustrate process path}
\PYG{n}{plt}\PYG{o}{.}\PYG{n}{xlim}\PYG{p}{(}\PYG{l+m+mf}{24.5}\PYG{p}{,} \PYG{l+m+mf}{25.5}\PYG{p}{)}
\PYG{n}{plt}\PYG{o}{.}\PYG{n}{ylim}\PYG{p}{(}\PYG{l+m+mi}{180}\PYG{p}{,} \PYG{l+m+mi}{420}\PYG{p}{)}
\PYG{n}{plt}\PYG{o}{.}\PYG{n}{text}\PYG{p}{(}\PYG{l+m+mf}{25.02}\PYG{p}{,} \PYG{l+m+mi}{190}\PYG{p}{,} \PYG{l+s+s1}{\PYGZsq{}}\PYG{l+s+s1}{1}\PYG{l+s+s1}{\PYGZsq{}}\PYG{p}{,} \PYG{n}{fontsize} \PYG{o}{=} \PYG{l+m+mi}{15}\PYG{p}{)}
\PYG{n}{plt}\PYG{o}{.}\PYG{n}{text}\PYG{p}{(}\PYG{l+m+mf}{25.02}\PYG{p}{,} \PYG{l+m+mi}{407}\PYG{p}{,} \PYG{l+s+s1}{\PYGZsq{}}\PYG{l+s+s1}{2}\PYG{l+s+s1}{\PYGZsq{}}\PYG{p}{,} \PYG{n}{fontsize} \PYG{o}{=} \PYG{l+m+mi}{15}\PYG{p}{)}
\PYG{n}{plt}\PYG{o}{.}\PYG{n}{plot}\PYG{p}{(}\PYG{n}{T\PYGZus{}vals}\PYG{p}{[}\PYG{l+m+mi}{0}\PYG{p}{]}\PYG{p}{,} \PYG{n}{P\PYGZus{}vals}\PYG{p}{[}\PYG{l+m+mi}{0}\PYG{p}{]}\PYG{p}{,} \PYG{l+s+s1}{\PYGZsq{}}\PYG{l+s+s1}{ro}\PYG{l+s+s1}{\PYGZsq{}}\PYG{p}{)}
\PYG{n}{plt}\PYG{o}{.}\PYG{n}{plot}\PYG{p}{(}\PYG{n}{T\PYGZus{}vals}\PYG{p}{[}\PYG{o}{\PYGZhy{}}\PYG{l+m+mi}{1}\PYG{p}{]}\PYG{p}{,} \PYG{n}{P\PYGZus{}vals}\PYG{p}{[}\PYG{o}{\PYGZhy{}}\PYG{l+m+mi}{1}\PYG{p}{]}\PYG{p}{,} \PYG{l+s+s1}{\PYGZsq{}}\PYG{l+s+s1}{ro}\PYG{l+s+s1}{\PYGZsq{}}\PYG{p}{)}
\end{sphinxVerbatim}

\end{sphinxuseclass}\end{sphinxVerbatimInput}
\begin{sphinxVerbatimOutput}

\begin{sphinxuseclass}{cell_output}
\begin{sphinxVerbatim}[commandchars=\\\{\}]
[\PYGZlt{}matplotlib.lines.Line2D at 0x7f6a33cac280\PYGZgt{}]
\end{sphinxVerbatim}

\noindent\sphinxincludegraphics{{22a8b5defd8698f356c6cf128a02fad1848e5c5da2bafa54ade1395108a51b0f}.png}

\end{sphinxuseclass}\end{sphinxVerbatimOutput}

\end{sphinxuseclass}
\begin{sphinxuseclass}{cell}\begin{sphinxVerbatimInput}

\begin{sphinxuseclass}{cell_input}
\begin{sphinxVerbatim}[commandchars=\\\{\}]
\PYG{n}{plt}\PYG{o}{.}\PYG{n}{plot}\PYG{p}{(}\PYG{n}{V\PYGZus{}vals}\PYG{p}{,}\PYG{n}{T\PYGZus{}vals}\PYG{p}{)}  \PYG{c+c1}{\PYGZsh{} plot volume vs. temperature}
\PYG{n}{plt}\PYG{o}{.}\PYG{n}{ylabel}\PYG{p}{(}\PYG{l+s+s2}{\PYGZdq{}}\PYG{l+s+s2}{Temperature [\PYGZdl{}\PYGZca{}}\PYG{l+s+s2}{\PYGZob{}}\PYG{l+s+s2}{\PYGZbs{}}\PYG{l+s+s2}{circ\PYGZcb{}\PYGZdl{} C]}\PYG{l+s+s2}{\PYGZdq{}}\PYG{p}{)}  \PYG{c+c1}{\PYGZsh{} give y axis a label}
\PYG{n}{plt}\PYG{o}{.}\PYG{n}{xlabel}\PYG{p}{(}\PYG{l+s+s2}{\PYGZdq{}}\PYG{l+s+s2}{Volume [m3]}\PYG{l+s+s2}{\PYGZdq{}}\PYG{p}{)}  \PYG{c+c1}{\PYGZsh{} give x axis a label}

\PYG{c+c1}{\PYGZsh{}add\PYGZhy{}ons to illustrate process path}
\PYG{n}{plt}\PYG{o}{.}\PYG{n}{xlim}\PYG{p}{(}\PYG{l+m+mf}{0.4}\PYG{p}{,} \PYG{l+m+mf}{0.9}\PYG{p}{)}
\PYG{n}{plt}\PYG{o}{.}\PYG{n}{ylim}\PYG{p}{(}\PYG{l+m+mi}{24}\PYG{p}{,} \PYG{l+m+mi}{26}\PYG{p}{)}
\PYG{n}{plt}\PYG{o}{.}\PYG{n}{text}\PYG{p}{(}\PYG{l+m+mf}{0.86}\PYG{p}{,} \PYG{l+m+mf}{25.05}\PYG{p}{,} \PYG{l+s+s1}{\PYGZsq{}}\PYG{l+s+s1}{1}\PYG{l+s+s1}{\PYGZsq{}}\PYG{p}{,} \PYG{n}{fontsize} \PYG{o}{=} \PYG{l+m+mi}{15}\PYG{p}{)}
\PYG{n}{plt}\PYG{o}{.}\PYG{n}{text}\PYG{p}{(}\PYG{l+m+mf}{0.41}\PYG{p}{,} \PYG{l+m+mf}{25.05}\PYG{p}{,} \PYG{l+s+s1}{\PYGZsq{}}\PYG{l+s+s1}{2}\PYG{l+s+s1}{\PYGZsq{}}\PYG{p}{,} \PYG{n}{fontsize} \PYG{o}{=} \PYG{l+m+mi}{15}\PYG{p}{)}
\PYG{n}{plt}\PYG{o}{.}\PYG{n}{plot}\PYG{p}{(}\PYG{n}{V\PYGZus{}vals}\PYG{p}{[}\PYG{l+m+mi}{0}\PYG{p}{]}\PYG{p}{,} \PYG{n}{T\PYGZus{}vals}\PYG{p}{[}\PYG{l+m+mi}{0}\PYG{p}{]}\PYG{p}{,} \PYG{l+s+s1}{\PYGZsq{}}\PYG{l+s+s1}{ro}\PYG{l+s+s1}{\PYGZsq{}}\PYG{p}{)}
\PYG{n}{plt}\PYG{o}{.}\PYG{n}{plot}\PYG{p}{(}\PYG{n}{V\PYGZus{}vals}\PYG{p}{[}\PYG{o}{\PYGZhy{}}\PYG{l+m+mi}{1}\PYG{p}{]}\PYG{p}{,} \PYG{n}{T\PYGZus{}vals}\PYG{p}{[}\PYG{o}{\PYGZhy{}}\PYG{l+m+mi}{1}\PYG{p}{]}\PYG{p}{,} \PYG{l+s+s1}{\PYGZsq{}}\PYG{l+s+s1}{ro}\PYG{l+s+s1}{\PYGZsq{}}\PYG{p}{)}
\end{sphinxVerbatim}

\end{sphinxuseclass}\end{sphinxVerbatimInput}
\begin{sphinxVerbatimOutput}

\begin{sphinxuseclass}{cell_output}
\begin{sphinxVerbatim}[commandchars=\\\{\}]
[\PYGZlt{}matplotlib.lines.Line2D at 0x7f6a33ca2a00\PYGZgt{}]
\end{sphinxVerbatim}

\noindent\sphinxincludegraphics{{5d49f1159a1c91ad11050883713a1fb086e19a1a1676dffd19d2b1bf5be1075a}.png}

\end{sphinxuseclass}\end{sphinxVerbatimOutput}

\end{sphinxuseclass}
\sphinxstepscope


\section{Isobaric\sphinxhyphen{}Isochoric process: Nitrogen as Ideal gas}
\label{\detokenize{notebooks/Chapter3/CH3-Q4_v1:isobaric-isochoric-process-nitrogen-as-ideal-gas}}\label{\detokenize{notebooks/Chapter3/CH3-Q4_v1::doc}}
\sphinxAtStartPar
Consider \(1\:kg\) of Nitrogen stored in a piston\sphinxhyphen{}cylinder system sitting at atmospheric pressure and temperature (\(101\:kPa\) and \(25 ^{\circ}  C\)). The system undergoes a thermodynamic cycle where its pressure is doubled through an isothermal process. Following the isothermal compression comes an isobaric expansion followed by an isochoric process to get Nitrogen to its initial state at \(101\:kPa\) and \(25 ^{\circ}  C\). Assuming Nitrogen to be ideal gas, plot P\sphinxhyphen{}V, P\sphinxhyphen{}T, and T\sphinxhyphen{}V diagrams for this process.


\subsection{Solution Approach}
\label{\detokenize{notebooks/Chapter3/CH3-Q4_v1:solution-approach}}
\sphinxAtStartPar
Let’s associate each state with a number to be able to progess in an organized way.

\sphinxAtStartPar
The initial state being (1) where

\sphinxAtStartPar
\(P_1=101\:kPa\) and \(T_1=25^{\circ}  C\)

\sphinxAtStartPar
Followed by an isothermal process where the pressure is doules to reach state (2) where

\sphinxAtStartPar
\(P_2=2P_1\) and \(T_2=T_1=25^{\circ}  C\)

\sphinxAtStartPar
Followed by an isobaric process to reach point (3) where

\sphinxAtStartPar
\(P_3=2P_2\)

\sphinxAtStartPar
and

\sphinxAtStartPar
\(V_3=V_1\)

\sphinxAtStartPar
because the following process from (3) back to (1) is isochoric where the volume keeps constant.

\sphinxAtStartPar
To Calculate \(V_1\) the ideal gas equation of state shall be used where

\sphinxAtStartPar
\(V=mRT/P\)

\begin{sphinxuseclass}{cell}\begin{sphinxVerbatimInput}

\begin{sphinxuseclass}{cell_input}
\begin{sphinxVerbatim}[commandchars=\\\{\}]
\PYG{c+c1}{\PYGZsh{}defining variables and looking up tables}
\PYG{n}{R} \PYG{o}{=} \PYG{l+m+mf}{0.2968} \PYG{c+c1}{\PYGZsh{}Nitrogen gas constant in kJ/kg.K}
\PYG{n}{m} \PYG{o}{=} \PYG{l+m+mi}{1} \PYG{c+c1}{\PYGZsh{}mass of Nitrogen in kg}
\PYG{n}{T\PYGZus{}1} \PYG{o}{=} \PYG{l+m+mi}{25} \PYG{o}{+} \PYG{l+m+mf}{273.15} \PYG{c+c1}{\PYGZsh{}temperature of Nitrogen in K}
\PYG{n}{P\PYGZus{}1} \PYG{o}{=} \PYG{l+m+mi}{101} \PYG{c+c1}{\PYGZsh{}pressure of Nitrogen in kPa}

\PYG{c+c1}{\PYGZsh{}using ideal gas law equation of state}
\PYG{n}{V\PYGZus{}1} \PYG{o}{=} \PYG{n}{m} \PYG{o}{*} \PYG{n}{R} \PYG{o}{*} \PYG{n}{T\PYGZus{}1} \PYG{o}{/} \PYG{n}{P\PYGZus{}1} \PYG{c+c1}{\PYGZsh{}volume of Nitrogen in m3}

\PYG{n+nb}{print}\PYG{p}{(}\PYG{l+s+s1}{\PYGZsq{}}\PYG{l+s+s1}{The volume of Nitrogen at its initial state is:}\PYG{l+s+s1}{\PYGZsq{}}\PYG{p}{,} \PYG{l+s+sa}{f}\PYG{l+s+s2}{\PYGZdq{}}\PYG{l+s+si}{\PYGZob{}}\PYG{n}{V\PYGZus{}1}\PYG{l+s+si}{:}\PYG{l+s+s2}{.3f}\PYG{l+s+si}{\PYGZcb{}}\PYG{l+s+s2}{\PYGZdq{}}\PYG{p}{,} \PYG{l+s+s1}{\PYGZsq{}}\PYG{l+s+s1}{m3}\PYG{l+s+s1}{\PYGZsq{}}\PYG{p}{)}
\end{sphinxVerbatim}

\end{sphinxuseclass}\end{sphinxVerbatimInput}
\begin{sphinxVerbatimOutput}

\begin{sphinxuseclass}{cell_output}
\begin{sphinxVerbatim}[commandchars=\\\{\}]
The volume of Nitrogen at its initial state is: 0.876 m3
\end{sphinxVerbatim}

\end{sphinxuseclass}\end{sphinxVerbatimOutput}

\end{sphinxuseclass}
\sphinxAtStartPar
Then Nitrogen goes through an isothermal compression to double its pressure.

\sphinxAtStartPar
given mass(\(m\)), gas constant (\(R\)) and temperature (\(T\)) are constant,

\sphinxAtStartPar
\(P_2V_2=P_1V_1=mRT_1(orT_2)\)

\sphinxAtStartPar
therefore

\sphinxAtStartPar
\(V_2=P_1V_1/P_2\)

\sphinxAtStartPar
where

\sphinxAtStartPar
\(P_2=2P_1\)

\sphinxAtStartPar
therfore

\sphinxAtStartPar
\(V_2=V_1/2\)

\begin{sphinxuseclass}{cell}\begin{sphinxVerbatimInput}

\begin{sphinxuseclass}{cell_input}
\begin{sphinxVerbatim}[commandchars=\\\{\}]
\PYG{c+c1}{\PYGZsh{}calculating variables at the second state}
\PYG{n}{P\PYGZus{}2} \PYG{o}{=} \PYG{l+m+mi}{2} \PYG{o}{*} \PYG{n}{P\PYGZus{}1} \PYG{c+c1}{\PYGZsh{}pressure of Nitrogen in kPa}
\PYG{n}{T\PYGZus{}2} \PYG{o}{=} \PYG{n}{T\PYGZus{}1} \PYG{c+c1}{\PYGZsh{}temperature of Nitrogen in K}
\PYG{n}{V\PYGZus{}2} \PYG{o}{=} \PYG{n}{V\PYGZus{}1} \PYG{o}{/} \PYG{l+m+mi}{2} \PYG{c+c1}{\PYGZsh{}volume of Nitrogen in m3 at the second state}
\PYG{n+nb}{print}\PYG{p}{(}\PYG{l+s+s1}{\PYGZsq{}}\PYG{l+s+s1}{The volume of Nitrogen at its second state is:}\PYG{l+s+s1}{\PYGZsq{}}\PYG{p}{,} \PYG{l+s+sa}{f}\PYG{l+s+s2}{\PYGZdq{}}\PYG{l+s+si}{\PYGZob{}}\PYG{n}{V\PYGZus{}2}\PYG{l+s+si}{:}\PYG{l+s+s2}{.3f}\PYG{l+s+si}{\PYGZcb{}}\PYG{l+s+s2}{\PYGZdq{}}\PYG{p}{,} \PYG{l+s+s1}{\PYGZsq{}}\PYG{l+s+s1}{m3}\PYG{l+s+s1}{\PYGZsq{}}\PYG{p}{)}
\end{sphinxVerbatim}

\end{sphinxuseclass}\end{sphinxVerbatimInput}
\begin{sphinxVerbatimOutput}

\begin{sphinxuseclass}{cell_output}
\begin{sphinxVerbatim}[commandchars=\\\{\}]
The volume of Nitrogen at its second state is: 0.438 m3
\end{sphinxVerbatim}

\end{sphinxuseclass}\end{sphinxVerbatimOutput}

\end{sphinxuseclass}
\sphinxAtStartPar
To calculate temperature at the third state, ideal gas equation of state is to be used as

\sphinxAtStartPar
\(T_3=P_3V_3/{mR}\)

\begin{sphinxuseclass}{cell}\begin{sphinxVerbatimInput}

\begin{sphinxuseclass}{cell_input}
\begin{sphinxVerbatim}[commandchars=\\\{\}]
\PYG{c+c1}{\PYGZsh{}\PYGZsh{}calculating variables at the third state}
\PYG{n}{P\PYGZus{}3} \PYG{o}{=} \PYG{n}{P\PYGZus{}2} \PYG{c+c1}{\PYGZsh{}pressure of Nitrogen in kPa}
\PYG{n}{V\PYGZus{}3} \PYG{o}{=} \PYG{n}{V\PYGZus{}1} \PYG{c+c1}{\PYGZsh{}volume of Nitrogen in m3 at the third state}

\PYG{c+c1}{\PYGZsh{}using ideal gas law equation of state}
\PYG{n}{T\PYGZus{}3} \PYG{o}{=} \PYG{n}{P\PYGZus{}3} \PYG{o}{*} \PYG{n}{V\PYGZus{}3} \PYG{o}{/} \PYG{p}{(}\PYG{n}{m} \PYG{o}{*} \PYG{n}{R}\PYG{p}{)} \PYG{c+c1}{\PYGZsh{}temperature of air in partition a in K}
\PYG{n+nb}{print}\PYG{p}{(}\PYG{l+s+s1}{\PYGZsq{}}\PYG{l+s+s1}{The temperature of Nitrogen at the third state is:}\PYG{l+s+s1}{\PYGZsq{}}\PYG{p}{,} \PYG{n}{T\PYGZus{}3}\PYG{p}{,} \PYG{l+s+s1}{\PYGZsq{}}\PYG{l+s+s1}{K}\PYG{l+s+s1}{\PYGZsq{}}\PYG{p}{)}
\end{sphinxVerbatim}

\end{sphinxuseclass}\end{sphinxVerbatimInput}
\begin{sphinxVerbatimOutput}

\begin{sphinxuseclass}{cell_output}
\begin{sphinxVerbatim}[commandchars=\\\{\}]
The temperature of Nitrogen at the third state is: 596.3 K
\end{sphinxVerbatim}

\end{sphinxuseclass}\end{sphinxVerbatimOutput}

\end{sphinxuseclass}
\begin{sphinxuseclass}{cell}\begin{sphinxVerbatimInput}

\begin{sphinxuseclass}{cell_input}
\begin{sphinxVerbatim}[commandchars=\\\{\}]
\PYG{c+c1}{\PYGZsh{}\PYGZsh{}plotting P\PYGZus{}V}
\PYG{c+c1}{\PYGZsh{}import the libraries we\PYGZsq{}ll need}
\PYG{k+kn}{import} \PYG{n+nn}{numpy} \PYG{k}{as} \PYG{n+nn}{np}
\PYG{k+kn}{import} \PYG{n+nn}{matplotlib}\PYG{n+nn}{.}\PYG{n+nn}{pyplot} \PYG{k}{as} \PYG{n+nn}{plt}

\PYG{c+c1}{\PYGZsh{}defining arrays of V values for 1\PYGZhy{}2}
\PYG{n}{V\PYGZus{}vals12} \PYG{o}{=} \PYG{n}{np}\PYG{o}{.}\PYG{n}{linspace}\PYG{p}{(}\PYG{n}{V\PYGZus{}1}\PYG{p}{,} \PYG{n}{V\PYGZus{}2}\PYG{p}{,} \PYG{l+m+mi}{1000}\PYG{p}{)}  \PYG{c+c1}{\PYGZsh{} define an array of values for volume (v) for the process 1 to 2}
\PYG{c+c1}{\PYGZsh{}calculating pressure (P) for the array of volume values (V\PYGZus{}vals12)}
\PYG{n}{P\PYGZus{}vals12} \PYG{o}{=} \PYG{n}{m} \PYG{o}{*} \PYG{n}{R} \PYG{o}{*} \PYG{n}{T\PYGZus{}1} \PYG{o}{/} \PYG{n}{V\PYGZus{}vals12}

\PYG{c+c1}{\PYGZsh{}defining arrays of V values for 2\PYGZhy{}3}
\PYG{n}{V\PYGZus{}vals23} \PYG{o}{=} \PYG{n}{np}\PYG{o}{.}\PYG{n}{linspace}\PYG{p}{(}\PYG{n}{V\PYGZus{}2}\PYG{p}{,} \PYG{n}{V\PYGZus{}3}\PYG{p}{,} \PYG{l+m+mi}{1000}\PYG{p}{)}  \PYG{c+c1}{\PYGZsh{} define an array of values for volume (v) for the process 2 to 3}
\PYG{c+c1}{\PYGZsh{}associated constant pressure for the process 2\PYGZhy{}3}
\PYG{n}{P\PYGZus{}vals23} \PYG{o}{=} \PYG{n}{np}\PYG{o}{.}\PYG{n}{linspace}\PYG{p}{(}\PYG{n}{P\PYGZus{}2}\PYG{p}{,} \PYG{n}{P\PYGZus{}3}\PYG{p}{,} \PYG{l+m+mi}{1000}\PYG{p}{)}

\PYG{c+c1}{\PYGZsh{}defining arrays of P values for 3\PYGZhy{}1}
\PYG{n}{P\PYGZus{}vals31} \PYG{o}{=} \PYG{n}{np}\PYG{o}{.}\PYG{n}{linspace}\PYG{p}{(}\PYG{n}{P\PYGZus{}3}\PYG{p}{,} \PYG{n}{P\PYGZus{}1}\PYG{p}{,} \PYG{l+m+mi}{1000}\PYG{p}{)}  \PYG{c+c1}{\PYGZsh{} define an array of values for pressure (P) for the process 2 to 3}
\PYG{c+c1}{\PYGZsh{}associated constant volume for the process 3\PYGZhy{}1}
\PYG{n}{V\PYGZus{}vals31} \PYG{o}{=} \PYG{n}{np}\PYG{o}{.}\PYG{n}{linspace}\PYG{p}{(}\PYG{n}{V\PYGZus{}3}\PYG{p}{,} \PYG{n}{V\PYGZus{}1}\PYG{p}{,} \PYG{l+m+mi}{1000}\PYG{p}{)}

\PYG{n}{plt}\PYG{o}{.}\PYG{n}{plot}\PYG{p}{(}\PYG{n}{V\PYGZus{}vals12}\PYG{p}{,} \PYG{n}{P\PYGZus{}vals12}\PYG{p}{,}\PYG{n}{label}\PYG{o}{=}\PYG{l+s+s1}{\PYGZsq{}}\PYG{l+s+s1}{isothermal}\PYG{l+s+s1}{\PYGZsq{}}\PYG{p}{)}  \PYG{c+c1}{\PYGZsh{} plot pressure vs. volume}
\PYG{n}{plt}\PYG{o}{.}\PYG{n}{plot}\PYG{p}{(}\PYG{n}{V\PYGZus{}vals23}\PYG{p}{,} \PYG{n}{P\PYGZus{}vals23}\PYG{p}{,}\PYG{n}{label}\PYG{o}{=}\PYG{l+s+s1}{\PYGZsq{}}\PYG{l+s+s1}{isobaric}\PYG{l+s+s1}{\PYGZsq{}}\PYG{p}{)}
\PYG{n}{plt}\PYG{o}{.}\PYG{n}{plot}\PYG{p}{(}\PYG{n}{V\PYGZus{}vals31}\PYG{p}{,} \PYG{n}{P\PYGZus{}vals31}\PYG{p}{,}\PYG{n}{label}\PYG{o}{=}\PYG{l+s+s1}{\PYGZsq{}}\PYG{l+s+s1}{isochoric}\PYG{l+s+s1}{\PYGZsq{}}\PYG{p}{)}
\PYG{n}{plt}\PYG{o}{.}\PYG{n}{legend}\PYG{p}{(}\PYG{p}{)}

\PYG{n}{plt}\PYG{o}{.}\PYG{n}{ylabel}\PYG{p}{(}\PYG{l+s+s2}{\PYGZdq{}}\PYG{l+s+s2}{Pressure [kPa]}\PYG{l+s+s2}{\PYGZdq{}}\PYG{p}{)}  \PYG{c+c1}{\PYGZsh{} give y axis a label}
\PYG{n}{plt}\PYG{o}{.}\PYG{n}{xlabel}\PYG{p}{(}\PYG{l+s+s2}{\PYGZdq{}}\PYG{l+s+s2}{Volume [m3]}\PYG{l+s+s2}{\PYGZdq{}}\PYG{p}{)}  \PYG{c+c1}{\PYGZsh{} give x axis a label}

\PYG{c+c1}{\PYGZsh{}add\PYGZhy{}ons to illustrate process path}
\PYG{n}{plt}\PYG{o}{.}\PYG{n}{xlim}\PYG{p}{(}\PYG{l+m+mf}{0.4}\PYG{p}{,} \PYG{l+m+mf}{0.9}\PYG{p}{)}
\PYG{n}{plt}\PYG{o}{.}\PYG{n}{ylim}\PYG{p}{(}\PYG{l+m+mi}{80}\PYG{p}{,} \PYG{l+m+mi}{220}\PYG{p}{)}
\PYG{n}{plt}\PYG{o}{.}\PYG{n}{text}\PYG{p}{(}\PYG{l+m+mf}{0.88}\PYG{p}{,} \PYG{l+m+mi}{87}\PYG{p}{,} \PYG{l+s+s1}{\PYGZsq{}}\PYG{l+s+s1}{1}\PYG{l+s+s1}{\PYGZsq{}}\PYG{p}{,} \PYG{n}{fontsize} \PYG{o}{=} \PYG{l+m+mi}{15}\PYG{p}{)}
\PYG{n}{plt}\PYG{o}{.}\PYG{n}{text}\PYG{p}{(}\PYG{l+m+mf}{0.42}\PYG{p}{,} \PYG{l+m+mi}{207}\PYG{p}{,} \PYG{l+s+s1}{\PYGZsq{}}\PYG{l+s+s1}{2}\PYG{l+s+s1}{\PYGZsq{}}\PYG{p}{,} \PYG{n}{fontsize} \PYG{o}{=} \PYG{l+m+mi}{15}\PYG{p}{)}
\PYG{n}{plt}\PYG{o}{.}\PYG{n}{text}\PYG{p}{(}\PYG{l+m+mf}{0.88}\PYG{p}{,} \PYG{l+m+mi}{207}\PYG{p}{,} \PYG{l+s+s1}{\PYGZsq{}}\PYG{l+s+s1}{3}\PYG{l+s+s1}{\PYGZsq{}}\PYG{p}{,} \PYG{n}{fontsize} \PYG{o}{=} \PYG{l+m+mi}{15}\PYG{p}{)}
\PYG{n}{plt}\PYG{o}{.}\PYG{n}{plot}\PYG{p}{(}\PYG{n}{V\PYGZus{}vals12}\PYG{p}{[}\PYG{l+m+mi}{0}\PYG{p}{]}\PYG{p}{,} \PYG{n}{P\PYGZus{}vals12}\PYG{p}{[}\PYG{l+m+mi}{0}\PYG{p}{]}\PYG{p}{,} \PYG{l+s+s1}{\PYGZsq{}}\PYG{l+s+s1}{ro}\PYG{l+s+s1}{\PYGZsq{}}\PYG{p}{)}
\PYG{n}{plt}\PYG{o}{.}\PYG{n}{plot}\PYG{p}{(}\PYG{n}{V\PYGZus{}vals23}\PYG{p}{[}\PYG{l+m+mi}{0}\PYG{p}{]}\PYG{p}{,} \PYG{n}{P\PYGZus{}vals23}\PYG{p}{[}\PYG{l+m+mi}{0}\PYG{p}{]}\PYG{p}{,} \PYG{l+s+s1}{\PYGZsq{}}\PYG{l+s+s1}{ro}\PYG{l+s+s1}{\PYGZsq{}}\PYG{p}{)}
\PYG{n}{plt}\PYG{o}{.}\PYG{n}{plot}\PYG{p}{(}\PYG{n}{V\PYGZus{}vals31}\PYG{p}{[}\PYG{l+m+mi}{0}\PYG{p}{]}\PYG{p}{,} \PYG{n}{P\PYGZus{}vals31}\PYG{p}{[}\PYG{l+m+mi}{0}\PYG{p}{]}\PYG{p}{,} \PYG{l+s+s1}{\PYGZsq{}}\PYG{l+s+s1}{ro}\PYG{l+s+s1}{\PYGZsq{}}\PYG{p}{)}
\end{sphinxVerbatim}

\end{sphinxuseclass}\end{sphinxVerbatimInput}
\begin{sphinxVerbatimOutput}

\begin{sphinxuseclass}{cell_output}
\begin{sphinxVerbatim}[commandchars=\\\{\}]
[\PYGZlt{}matplotlib.lines.Line2D at 0x7f28f6191c40\PYGZgt{}]
\end{sphinxVerbatim}

\noindent\sphinxincludegraphics{{e0606e4c50a16d709d0ac7060183832e701e891b366fb3418ae4b6a770792ced}.png}

\end{sphinxuseclass}\end{sphinxVerbatimOutput}

\end{sphinxuseclass}
\begin{sphinxuseclass}{cell}\begin{sphinxVerbatimInput}

\begin{sphinxuseclass}{cell_input}
\begin{sphinxVerbatim}[commandchars=\\\{\}]
\PYG{c+c1}{\PYGZsh{}\PYGZsh{}plotting P\PYGZus{}T}
\PYG{c+c1}{\PYGZsh{}defining arrays of T values for 1\PYGZhy{}2}
\PYG{n}{T\PYGZus{}vals12} \PYG{o}{=} \PYG{n}{np}\PYG{o}{.}\PYG{n}{linspace}\PYG{p}{(}\PYG{n}{T\PYGZus{}1}\PYG{p}{,} \PYG{n}{T\PYGZus{}2}\PYG{p}{,} \PYG{l+m+mi}{1000}\PYG{p}{)}  \PYG{c+c1}{\PYGZsh{} define an array of values for temperature (T) for the process 1 to 2}

\PYG{c+c1}{\PYGZsh{}defining arrays of T values for 2\PYGZhy{}3}
\PYG{n}{T\PYGZus{}vals23} \PYG{o}{=} \PYG{n}{np}\PYG{o}{.}\PYG{n}{linspace}\PYG{p}{(}\PYG{n}{T\PYGZus{}2}\PYG{p}{,} \PYG{n}{T\PYGZus{}3}\PYG{p}{,} \PYG{l+m+mi}{1000}\PYG{p}{)}  \PYG{c+c1}{\PYGZsh{} define an array of values for temperature (T) for the process 2 to 3}

\PYG{c+c1}{\PYGZsh{}defining arrays of T values for 3\PYGZhy{}1}
\PYG{n}{T\PYGZus{}vals31} \PYG{o}{=} \PYG{n}{np}\PYG{o}{.}\PYG{n}{linspace}\PYG{p}{(}\PYG{n}{T\PYGZus{}3}\PYG{p}{,} \PYG{n}{T\PYGZus{}1}\PYG{p}{,} \PYG{l+m+mi}{1000}\PYG{p}{)}  \PYG{c+c1}{\PYGZsh{} define an array of values for temperature (T) for the process 3 to 1}

\PYG{n}{plt}\PYG{o}{.}\PYG{n}{plot}\PYG{p}{(}\PYG{n}{T\PYGZus{}vals12}\PYG{p}{,} \PYG{n}{P\PYGZus{}vals12}\PYG{p}{,}\PYG{n}{label}\PYG{o}{=}\PYG{l+s+s1}{\PYGZsq{}}\PYG{l+s+s1}{isothermal}\PYG{l+s+s1}{\PYGZsq{}}\PYG{p}{)}  \PYG{c+c1}{\PYGZsh{} plot pressure vs. volume}
\PYG{n}{plt}\PYG{o}{.}\PYG{n}{plot}\PYG{p}{(}\PYG{n}{T\PYGZus{}vals23}\PYG{p}{,} \PYG{n}{P\PYGZus{}vals23}\PYG{p}{,}\PYG{n}{label}\PYG{o}{=}\PYG{l+s+s1}{\PYGZsq{}}\PYG{l+s+s1}{isobaric}\PYG{l+s+s1}{\PYGZsq{}}\PYG{p}{)}
\PYG{n}{plt}\PYG{o}{.}\PYG{n}{plot}\PYG{p}{(}\PYG{n}{T\PYGZus{}vals31}\PYG{p}{,} \PYG{n}{P\PYGZus{}vals31}\PYG{p}{,}\PYG{n}{label}\PYG{o}{=}\PYG{l+s+s1}{\PYGZsq{}}\PYG{l+s+s1}{isochoric}\PYG{l+s+s1}{\PYGZsq{}}\PYG{p}{)}
\PYG{n}{plt}\PYG{o}{.}\PYG{n}{legend}\PYG{p}{(}\PYG{p}{)}

\PYG{n}{plt}\PYG{o}{.}\PYG{n}{ylabel}\PYG{p}{(}\PYG{l+s+s2}{\PYGZdq{}}\PYG{l+s+s2}{Pressure [kPa]}\PYG{l+s+s2}{\PYGZdq{}}\PYG{p}{)}  \PYG{c+c1}{\PYGZsh{} give y axis a label}
\PYG{n}{plt}\PYG{o}{.}\PYG{n}{xlabel}\PYG{p}{(}\PYG{l+s+s2}{\PYGZdq{}}\PYG{l+s+s2}{Temperature [K]}\PYG{l+s+s2}{\PYGZdq{}}\PYG{p}{)}  \PYG{c+c1}{\PYGZsh{} give x axis a label}

\PYG{c+c1}{\PYGZsh{}add\PYGZhy{}ons to illustrate process path}
\PYG{n}{plt}\PYG{o}{.}\PYG{n}{xlim}\PYG{p}{(}\PYG{l+m+mi}{280}\PYG{p}{,} \PYG{l+m+mi}{615}\PYG{p}{)}
\PYG{n}{plt}\PYG{o}{.}\PYG{n}{ylim}\PYG{p}{(}\PYG{l+m+mi}{85}\PYG{p}{,} \PYG{l+m+mi}{215}\PYG{p}{)}
\PYG{n}{plt}\PYG{o}{.}\PYG{n}{text}\PYG{p}{(}\PYG{l+m+mi}{288}\PYG{p}{,} \PYG{l+m+mi}{93}\PYG{p}{,} \PYG{l+s+s1}{\PYGZsq{}}\PYG{l+s+s1}{1}\PYG{l+s+s1}{\PYGZsq{}}\PYG{p}{,} \PYG{n}{fontsize} \PYG{o}{=} \PYG{l+m+mi}{15}\PYG{p}{)}
\PYG{n}{plt}\PYG{o}{.}\PYG{n}{text}\PYG{p}{(}\PYG{l+m+mi}{288}\PYG{p}{,} \PYG{l+m+mi}{205}\PYG{p}{,} \PYG{l+s+s1}{\PYGZsq{}}\PYG{l+s+s1}{2}\PYG{l+s+s1}{\PYGZsq{}}\PYG{p}{,} \PYG{n}{fontsize} \PYG{o}{=} \PYG{l+m+mi}{15}\PYG{p}{)}
\PYG{n}{plt}\PYG{o}{.}\PYG{n}{text}\PYG{p}{(}\PYG{l+m+mi}{604}\PYG{p}{,} \PYG{l+m+mi}{200}\PYG{p}{,} \PYG{l+s+s1}{\PYGZsq{}}\PYG{l+s+s1}{3}\PYG{l+s+s1}{\PYGZsq{}}\PYG{p}{,} \PYG{n}{fontsize} \PYG{o}{=} \PYG{l+m+mi}{15}\PYG{p}{)}
\PYG{n}{plt}\PYG{o}{.}\PYG{n}{plot}\PYG{p}{(}\PYG{n}{T\PYGZus{}vals12}\PYG{p}{[}\PYG{l+m+mi}{0}\PYG{p}{]}\PYG{p}{,} \PYG{n}{P\PYGZus{}vals12}\PYG{p}{[}\PYG{l+m+mi}{0}\PYG{p}{]}\PYG{p}{,} \PYG{l+s+s1}{\PYGZsq{}}\PYG{l+s+s1}{ro}\PYG{l+s+s1}{\PYGZsq{}}\PYG{p}{)}
\PYG{n}{plt}\PYG{o}{.}\PYG{n}{plot}\PYG{p}{(}\PYG{n}{T\PYGZus{}vals23}\PYG{p}{[}\PYG{l+m+mi}{0}\PYG{p}{]}\PYG{p}{,} \PYG{n}{P\PYGZus{}vals23}\PYG{p}{[}\PYG{l+m+mi}{0}\PYG{p}{]}\PYG{p}{,} \PYG{l+s+s1}{\PYGZsq{}}\PYG{l+s+s1}{ro}\PYG{l+s+s1}{\PYGZsq{}}\PYG{p}{)}
\PYG{n}{plt}\PYG{o}{.}\PYG{n}{plot}\PYG{p}{(}\PYG{n}{T\PYGZus{}vals31}\PYG{p}{[}\PYG{l+m+mi}{0}\PYG{p}{]}\PYG{p}{,} \PYG{n}{P\PYGZus{}vals31}\PYG{p}{[}\PYG{l+m+mi}{0}\PYG{p}{]}\PYG{p}{,} \PYG{l+s+s1}{\PYGZsq{}}\PYG{l+s+s1}{ro}\PYG{l+s+s1}{\PYGZsq{}}\PYG{p}{)}
\end{sphinxVerbatim}

\end{sphinxuseclass}\end{sphinxVerbatimInput}
\begin{sphinxVerbatimOutput}

\begin{sphinxuseclass}{cell_output}
\begin{sphinxVerbatim}[commandchars=\\\{\}]
[\PYGZlt{}matplotlib.lines.Line2D at 0x7f28f58a2d30\PYGZgt{}]
\end{sphinxVerbatim}

\noindent\sphinxincludegraphics{{c8ab35ac96a298d52de1ec0d1cf0c1ebbb0df2361d6d3f9b76864775cde7b294}.png}

\end{sphinxuseclass}\end{sphinxVerbatimOutput}

\end{sphinxuseclass}
\begin{sphinxuseclass}{cell}\begin{sphinxVerbatimInput}

\begin{sphinxuseclass}{cell_input}
\begin{sphinxVerbatim}[commandchars=\\\{\}]
\PYG{c+c1}{\PYGZsh{}\PYGZsh{}plotting T\PYGZus{}V}

\PYG{n}{plt}\PYG{o}{.}\PYG{n}{plot}\PYG{p}{(}\PYG{n}{T\PYGZus{}vals12}\PYG{p}{,} \PYG{n}{V\PYGZus{}vals12}\PYG{p}{,}\PYG{n}{label}\PYG{o}{=}\PYG{l+s+s1}{\PYGZsq{}}\PYG{l+s+s1}{isothermal}\PYG{l+s+s1}{\PYGZsq{}}\PYG{p}{)}  \PYG{c+c1}{\PYGZsh{} plot pressure vs. volume}
\PYG{n}{plt}\PYG{o}{.}\PYG{n}{plot}\PYG{p}{(}\PYG{n}{T\PYGZus{}vals23}\PYG{p}{,} \PYG{n}{V\PYGZus{}vals23}\PYG{p}{,}\PYG{n}{label}\PYG{o}{=}\PYG{l+s+s1}{\PYGZsq{}}\PYG{l+s+s1}{isobaric}\PYG{l+s+s1}{\PYGZsq{}}\PYG{p}{)}
\PYG{n}{plt}\PYG{o}{.}\PYG{n}{plot}\PYG{p}{(}\PYG{n}{T\PYGZus{}vals31}\PYG{p}{,} \PYG{n}{V\PYGZus{}vals31}\PYG{p}{,}\PYG{n}{label}\PYG{o}{=}\PYG{l+s+s1}{\PYGZsq{}}\PYG{l+s+s1}{isochoric}\PYG{l+s+s1}{\PYGZsq{}}\PYG{p}{)}
\PYG{n}{plt}\PYG{o}{.}\PYG{n}{legend}\PYG{p}{(}\PYG{p}{)}

\PYG{n}{plt}\PYG{o}{.}\PYG{n}{ylabel}\PYG{p}{(}\PYG{l+s+s2}{\PYGZdq{}}\PYG{l+s+s2}{Volume [m3]}\PYG{l+s+s2}{\PYGZdq{}}\PYG{p}{)}  \PYG{c+c1}{\PYGZsh{} give y axis a label}
\PYG{n}{plt}\PYG{o}{.}\PYG{n}{xlabel}\PYG{p}{(}\PYG{l+s+s2}{\PYGZdq{}}\PYG{l+s+s2}{Temperature [K]}\PYG{l+s+s2}{\PYGZdq{}}\PYG{p}{)}  \PYG{c+c1}{\PYGZsh{} give x axis a label}

\PYG{c+c1}{\PYGZsh{}add\PYGZhy{}ons to illustrate process path}
\PYG{n}{plt}\PYG{o}{.}\PYG{n}{xlim}\PYG{p}{(}\PYG{l+m+mi}{280}\PYG{p}{,} \PYG{l+m+mi}{615}\PYG{p}{)}
\PYG{n}{plt}\PYG{o}{.}\PYG{n}{ylim}\PYG{p}{(}\PYG{l+m+mf}{0.38}\PYG{p}{,} \PYG{l+m+mf}{0.92}\PYG{p}{)}
\PYG{n}{plt}\PYG{o}{.}\PYG{n}{text}\PYG{p}{(}\PYG{l+m+mi}{288}\PYG{p}{,} \PYG{l+m+mf}{0.89}\PYG{p}{,} \PYG{l+s+s1}{\PYGZsq{}}\PYG{l+s+s1}{1}\PYG{l+s+s1}{\PYGZsq{}}\PYG{p}{,} \PYG{n}{fontsize} \PYG{o}{=} \PYG{l+m+mi}{15}\PYG{p}{)}
\PYG{n}{plt}\PYG{o}{.}\PYG{n}{text}\PYG{p}{(}\PYG{l+m+mi}{288}\PYG{p}{,} \PYG{l+m+mf}{0.40}\PYG{p}{,} \PYG{l+s+s1}{\PYGZsq{}}\PYG{l+s+s1}{2}\PYG{l+s+s1}{\PYGZsq{}}\PYG{p}{,} \PYG{n}{fontsize} \PYG{o}{=} \PYG{l+m+mi}{15}\PYG{p}{)}
\PYG{n}{plt}\PYG{o}{.}\PYG{n}{text}\PYG{p}{(}\PYG{l+m+mi}{602}\PYG{p}{,} \PYG{l+m+mf}{0.88}\PYG{p}{,} \PYG{l+s+s1}{\PYGZsq{}}\PYG{l+s+s1}{3}\PYG{l+s+s1}{\PYGZsq{}}\PYG{p}{,} \PYG{n}{fontsize} \PYG{o}{=} \PYG{l+m+mi}{15}\PYG{p}{)}
\PYG{n}{plt}\PYG{o}{.}\PYG{n}{plot}\PYG{p}{(}\PYG{n}{T\PYGZus{}vals12}\PYG{p}{[}\PYG{l+m+mi}{0}\PYG{p}{]}\PYG{p}{,} \PYG{n}{V\PYGZus{}vals12}\PYG{p}{[}\PYG{l+m+mi}{0}\PYG{p}{]}\PYG{p}{,} \PYG{l+s+s1}{\PYGZsq{}}\PYG{l+s+s1}{ro}\PYG{l+s+s1}{\PYGZsq{}}\PYG{p}{)}
\PYG{n}{plt}\PYG{o}{.}\PYG{n}{plot}\PYG{p}{(}\PYG{n}{T\PYGZus{}vals23}\PYG{p}{[}\PYG{l+m+mi}{0}\PYG{p}{]}\PYG{p}{,} \PYG{n}{V\PYGZus{}vals23}\PYG{p}{[}\PYG{l+m+mi}{0}\PYG{p}{]}\PYG{p}{,} \PYG{l+s+s1}{\PYGZsq{}}\PYG{l+s+s1}{ro}\PYG{l+s+s1}{\PYGZsq{}}\PYG{p}{)}
\PYG{n}{plt}\PYG{o}{.}\PYG{n}{plot}\PYG{p}{(}\PYG{n}{T\PYGZus{}vals31}\PYG{p}{[}\PYG{l+m+mi}{0}\PYG{p}{]}\PYG{p}{,} \PYG{n}{V\PYGZus{}vals31}\PYG{p}{[}\PYG{l+m+mi}{0}\PYG{p}{]}\PYG{p}{,} \PYG{l+s+s1}{\PYGZsq{}}\PYG{l+s+s1}{ro}\PYG{l+s+s1}{\PYGZsq{}}\PYG{p}{)}
\end{sphinxVerbatim}

\end{sphinxuseclass}\end{sphinxVerbatimInput}
\begin{sphinxVerbatimOutput}

\begin{sphinxuseclass}{cell_output}
\begin{sphinxVerbatim}[commandchars=\\\{\}]
[\PYGZlt{}matplotlib.lines.Line2D at 0x7f28f57c7a60\PYGZgt{}]
\end{sphinxVerbatim}

\noindent\sphinxincludegraphics{{cfb686cad76ddae592c52903070c4b8a774275dde86fb0b53483370bd456fc61}.png}

\end{sphinxuseclass}\end{sphinxVerbatimOutput}

\end{sphinxuseclass}
\sphinxstepscope


\section{Lee \sphinxhyphen{} Kesler Compressibility factor}
\label{\detokenize{notebooks/Chapter3/CH3-Q5_v1:lee-kesler-compressibility-factor}}\label{\detokenize{notebooks/Chapter3/CH3-Q5_v1::doc}}
\sphinxAtStartPar
The following reference is the paper upon which the Lee\sphinxhyphen{}Kesler compressibility factor is generated.

\sphinxAtStartPar
\sphinxhref{https://aiche.onlinelibrary.wiley.com/doi/abs/10.1002/aic.690210313}{Reference for Lee\sphinxhyphen{}Kesler compressibility factor}

\sphinxAtStartPar
a)Try to reproduce a python code to regenerate the compressibility factor curve based on reduced pressure (\(P_r\)) for reduced temperature (\(T_r\)) equal to \(1.5\).

\sphinxAtStartPar
b) Calculate the compressibility factor for \(T_r = 1.5\) and \(P_r = 3\)

\begin{sphinxuseclass}{cell}\begin{sphinxVerbatimInput}

\begin{sphinxuseclass}{cell_input}
\begin{sphinxVerbatim}[commandchars=\\\{\}]
\PYG{c+c1}{\PYGZsh{}importing required library}
\PYG{k+kn}{import} \PYG{n+nn}{numpy} \PYG{k}{as} \PYG{n+nn}{np}
\PYG{k+kn}{import} \PYG{n+nn}{matplotlib}\PYG{n+nn}{.}\PYG{n+nn}{pyplot} \PYG{k}{as} \PYG{n+nn}{plt}

\PYG{c+c1}{\PYGZsh{}introducing P\PYGZus{}r and T\PYGZus{}r}
\PYG{n}{T\PYGZus{}r} \PYG{o}{=} \PYG{l+m+mf}{1.5}

\PYG{c+c1}{\PYGZsh{}introducing constants}
\PYG{n}{b1} \PYG{o}{=} \PYG{l+m+mf}{0.1181193}
\PYG{n}{b2} \PYG{o}{=} \PYG{l+m+mf}{0.265728}
\PYG{n}{b3} \PYG{o}{=} \PYG{l+m+mf}{0.154790}
\PYG{n}{b4} \PYG{o}{=} \PYG{l+m+mf}{0.030323}
\PYG{n}{c1} \PYG{o}{=} \PYG{l+m+mf}{0.0236744}
\PYG{n}{c2} \PYG{o}{=} \PYG{l+m+mf}{0.0186984}
\PYG{n}{c3} \PYG{o}{=} \PYG{l+m+mi}{0}
\PYG{n}{c4} \PYG{o}{=} \PYG{l+m+mf}{0.042724}
\PYG{n}{d1} \PYG{o}{=} \PYG{l+m+mf}{0.155488E\PYGZhy{}4}
\PYG{n}{d2} \PYG{o}{=} \PYG{l+m+mf}{0.623689E\PYGZhy{}4}
\PYG{n}{betha} \PYG{o}{=} \PYG{l+m+mf}{0.65392}
\PYG{n}{gamma} \PYG{o}{=} \PYG{l+m+mf}{0.060167}

\PYG{n}{B} \PYG{o}{=} \PYG{n}{b1} \PYG{o}{\PYGZhy{}} \PYG{n}{b2}\PYG{o}{/}\PYG{n}{T\PYGZus{}r} \PYG{o}{\PYGZhy{}} \PYG{n}{b3}\PYG{o}{/}\PYG{n}{T\PYGZus{}r}\PYG{o}{*}\PYG{o}{*}\PYG{l+m+mi}{2} \PYG{o}{\PYGZhy{}} \PYG{n}{b4}\PYG{o}{/}\PYG{n}{T\PYGZus{}r}\PYG{o}{*}\PYG{o}{*}\PYG{l+m+mi}{3}
\PYG{n}{C} \PYG{o}{=} \PYG{n}{c1} \PYG{o}{\PYGZhy{}} \PYG{n}{c2}\PYG{o}{/}\PYG{n}{T\PYGZus{}r} \PYG{o}{+} \PYG{n}{c3}\PYG{o}{/}\PYG{n}{T\PYGZus{}r}\PYG{o}{*}\PYG{o}{*}\PYG{l+m+mi}{3}
\PYG{n}{D} \PYG{o}{=} \PYG{n}{d1} \PYG{o}{+} \PYG{n}{d2}\PYG{o}{/}\PYG{n}{T\PYGZus{}r}

\PYG{c+c1}{\PYGZsh{}V\PYGZus{}r in an array structure}
\PYG{c+c1}{\PYGZsh{} an array of V\PYGZus{}r is to be built so that Z is calculated based upon. Otherwise, the equation can\PYGZsq{}t be solved analytically for V\PYGZus{}r}
\PYG{n}{V\PYGZus{}r} \PYG{o}{=} \PYG{n}{np}\PYG{o}{.}\PYG{n}{logspace}\PYG{p}{(}\PYG{o}{\PYGZhy{}}\PYG{l+m+mf}{0.9}\PYG{p}{,} \PYG{l+m+mf}{1.65}\PYG{p}{,} \PYG{l+m+mi}{10000}\PYG{p}{)} 
\PYG{n}{Z\PYGZus{}array} \PYG{o}{=} \PYG{l+m+mi}{1} \PYG{o}{+} \PYG{n}{B}\PYG{o}{/}\PYG{n}{V\PYGZus{}r} \PYG{o}{+} \PYG{n}{C}\PYG{o}{/}\PYG{n}{V\PYGZus{}r}\PYG{o}{*}\PYG{o}{*}\PYG{l+m+mi}{2} \PYG{o}{+} \PYG{n}{D}\PYG{o}{/}\PYG{n}{V\PYGZus{}r}\PYG{o}{*}\PYG{o}{*}\PYG{l+m+mi}{5} \PYG{o}{+} \PYG{n}{c4}\PYG{o}{*}\PYG{p}{(}\PYG{n}{betha} \PYG{o}{+} \PYG{n}{gamma}\PYG{o}{/}\PYG{n}{V\PYGZus{}r}\PYG{o}{*}\PYG{o}{*}\PYG{l+m+mi}{2}\PYG{p}{)}\PYG{o}{*}\PYG{n}{np}\PYG{o}{.}\PYG{n}{exp}\PYG{p}{(}\PYG{o}{\PYGZhy{}}\PYG{n}{gamma}\PYG{o}{/}\PYG{n}{V\PYGZus{}r}\PYG{o}{*}\PYG{o}{*}\PYG{l+m+mi}{2}\PYG{p}{)}\PYG{o}{/}\PYG{p}{(}\PYG{n}{T\PYGZus{}r}\PYG{o}{*}\PYG{o}{*}\PYG{l+m+mi}{3} \PYG{o}{*} \PYG{n}{V\PYGZus{}r}\PYG{o}{*}\PYG{o}{*}\PYG{l+m+mi}{2}\PYG{p}{)} \PYG{c+c1}{\PYGZsh{}Lee\PYGZhy{}Kesler equation}
\PYG{n}{P\PYGZus{}r\PYGZus{}array} \PYG{o}{=} \PYG{n}{Z\PYGZus{}array} \PYG{o}{*} \PYG{n}{T\PYGZus{}r} \PYG{o}{/} \PYG{n}{V\PYGZus{}r} \PYG{c+c1}{\PYGZsh{}calculating P\PYGZus{}r based on the array built for V\PYGZus{}r}

\PYG{c+c1}{\PYGZsh{}Plotting}
\PYG{n}{plt}\PYG{o}{.}\PYG{n}{plot}\PYG{p}{(}\PYG{n}{P\PYGZus{}r\PYGZus{}array}\PYG{p}{,}\PYG{n}{Z\PYGZus{}array}\PYG{p}{)}
\PYG{n}{plt}\PYG{o}{.}\PYG{n}{xlim}\PYG{p}{(}\PYG{l+m+mf}{0.1}\PYG{p}{,} \PYG{l+m+mi}{10}\PYG{p}{)}
\PYG{n}{plt}\PYG{o}{.}\PYG{n}{ylim}\PYG{p}{(}\PYG{l+m+mf}{0.2}\PYG{p}{,} \PYG{l+m+mf}{1.2}\PYG{p}{)}
\PYG{n}{plt}\PYG{o}{.}\PYG{n}{xscale}\PYG{p}{(}\PYG{l+s+s1}{\PYGZsq{}}\PYG{l+s+s1}{log}\PYG{l+s+s1}{\PYGZsq{}}\PYG{p}{)}
\PYG{n}{plt}\PYG{o}{.}\PYG{n}{grid}\PYG{p}{(}\PYG{n}{ls}\PYG{o}{=}\PYG{l+s+s1}{\PYGZsq{}}\PYG{l+s+s1}{\PYGZhy{}\PYGZhy{}}\PYG{l+s+s1}{\PYGZsq{}}\PYG{p}{)}
\PYG{n}{plt}\PYG{o}{.}\PYG{n}{xlabel}\PYG{p}{(}\PYG{l+s+s1}{\PYGZsq{}}\PYG{l+s+s1}{Reduced Pressure Pr}\PYG{l+s+s1}{\PYGZsq{}}\PYG{p}{)}
\PYG{n}{plt}\PYG{o}{.}\PYG{n}{ylabel}\PYG{p}{(}\PYG{l+s+s1}{\PYGZsq{}}\PYG{l+s+s1}{Compressibility Factor Z}\PYG{l+s+s1}{\PYGZsq{}}\PYG{p}{)}
\end{sphinxVerbatim}

\end{sphinxuseclass}\end{sphinxVerbatimInput}
\begin{sphinxVerbatimOutput}

\begin{sphinxuseclass}{cell_output}
\begin{sphinxVerbatim}[commandchars=\\\{\}]
Text(0, 0.5, \PYGZsq{}Compressibility Factor Z\PYGZsq{})
\end{sphinxVerbatim}

\noindent\sphinxincludegraphics{{be4f818b6f7c69b7b475d2a2e75270bcd872197ca656296fca5c5b06ea8dc35f}.png}

\end{sphinxuseclass}\end{sphinxVerbatimOutput}

\end{sphinxuseclass}
\begin{sphinxuseclass}{cell}\begin{sphinxVerbatimInput}

\begin{sphinxuseclass}{cell_input}
\begin{sphinxVerbatim}[commandchars=\\\{\}]
\PYG{n}{P\PYGZus{}r} \PYG{o}{=} \PYG{l+m+mi}{3}

\PYG{c+c1}{\PYGZsh{}finding the index of the array element in P\PYGZus{}r\PYGZus{}array which is closest to the desired P\PYGZus{}r value}
\PYG{n}{difference\PYGZus{}array} \PYG{o}{=} \PYG{n}{np}\PYG{o}{.}\PYG{n}{absolute}\PYG{p}{(}\PYG{n}{P\PYGZus{}r\PYGZus{}array}\PYG{o}{\PYGZhy{}}\PYG{n}{P\PYGZus{}r}\PYG{p}{)}
\PYG{n}{index} \PYG{o}{=} \PYG{n}{difference\PYGZus{}array}\PYG{o}{.}\PYG{n}{argmin}\PYG{p}{(}\PYG{p}{)}
\PYG{n}{Z} \PYG{o}{=} \PYG{n}{Z\PYGZus{}array}\PYG{p}{[}\PYG{n}{index}\PYG{p}{]}
\PYG{n+nb}{print}\PYG{p}{(}\PYG{l+s+s1}{\PYGZsq{}}\PYG{l+s+s1}{The compressibility factor value for P\PYGZus{}r =}\PYG{l+s+s1}{\PYGZsq{}}\PYG{p}{,}\PYG{n}{P\PYGZus{}r}\PYG{p}{,} \PYG{l+s+s1}{\PYGZsq{}}\PYG{l+s+s1}{and T\PYGZus{}r =}\PYG{l+s+s1}{\PYGZsq{}}\PYG{p}{,} \PYG{n}{T\PYGZus{}r}\PYG{p}{,} \PYG{l+s+s1}{\PYGZsq{}}\PYG{l+s+s1}{is}\PYG{l+s+s1}{\PYGZsq{}}\PYG{p}{,} \PYG{l+s+sa}{f}\PYG{l+s+s2}{\PYGZdq{}}\PYG{l+s+si}{\PYGZob{}}\PYG{n}{Z}\PYG{l+s+si}{:}\PYG{l+s+s2}{.3f}\PYG{l+s+si}{\PYGZcb{}}\PYG{l+s+s2}{\PYGZdq{}}\PYG{p}{)}
\end{sphinxVerbatim}

\end{sphinxuseclass}\end{sphinxVerbatimInput}
\begin{sphinxVerbatimOutput}

\begin{sphinxuseclass}{cell_output}
\begin{sphinxVerbatim}[commandchars=\\\{\}]
The compressibility factor value for P\PYGZus{}r = 3 and T\PYGZus{}r = 1.5 is 0.789
\end{sphinxVerbatim}

\end{sphinxuseclass}\end{sphinxVerbatimOutput}

\end{sphinxuseclass}
\sphinxstepscope


\section{Ideal gas assumption for Water}
\label{\detokenize{notebooks/Chapter3/CH3-Q6_v1:ideal-gas-assumption-for-water}}\label{\detokenize{notebooks/Chapter3/CH3-Q6_v1::doc}}
\sphinxAtStartPar
Imagine \(1\:kg\) of water vapor at \(2\:MPa\) and \(400 ^{\circ}  C\). Calculate its volume based on the following. Calculate the error in percents.

\sphinxAtStartPar
a) thermodynamic tables using coolprop

\sphinxAtStartPar
b) ideal gas assumption

\sphinxAtStartPar
c) ideal gas equation of state coupled with compressibility factor (\(Z\))

\sphinxAtStartPar
d) pinpoint the water at this state on its phase diagram and monitor the ideal gas assumption error based on the distance from the triple point


\subsection{Solution Approach for a)}
\label{\detokenize{notebooks/Chapter3/CH3-Q6_v1:solution-approach-for-a}}
\begin{sphinxuseclass}{cell}\begin{sphinxVerbatimInput}

\begin{sphinxuseclass}{cell_input}
\begin{sphinxVerbatim}[commandchars=\\\{\}]
\PYG{c+c1}{\PYGZsh{}importing the required library}
\PYG{k+kn}{import} \PYG{n+nn}{numpy} \PYG{k}{as} \PYG{n+nn}{np}
\PYG{k+kn}{import} \PYG{n+nn}{CoolProp}\PYG{n+nn}{.}\PYG{n+nn}{CoolProp} \PYG{k}{as} \PYG{n+nn}{CP}
\PYG{n}{P} \PYG{o}{=} \PYG{l+m+mf}{2E+6} \PYG{c+c1}{\PYGZsh{}pressure of wator vapor in Pa}
\PYG{n}{T} \PYG{o}{=} \PYG{l+m+mi}{400} \PYG{o}{+} \PYG{l+m+mf}{273.15} \PYG{c+c1}{\PYGZsh{}water vapor temperature in K}
\PYG{n}{m} \PYG{o}{=} \PYG{l+m+mi}{1} \PYG{c+c1}{\PYGZsh{}mass of water vapor in kg}
\PYG{n}{D} \PYG{o}{=} \PYG{n}{CP}\PYG{o}{.}\PYG{n}{PropsSI}\PYG{p}{(}\PYG{l+s+s1}{\PYGZsq{}}\PYG{l+s+s1}{D}\PYG{l+s+s1}{\PYGZsq{}}\PYG{p}{,} \PYG{l+s+s1}{\PYGZsq{}}\PYG{l+s+s1}{P}\PYG{l+s+s1}{\PYGZsq{}}\PYG{p}{,} \PYG{n}{P}\PYG{p}{,} \PYG{l+s+s1}{\PYGZsq{}}\PYG{l+s+s1}{T}\PYG{l+s+s1}{\PYGZsq{}}\PYG{p}{,} \PYG{n}{T}\PYG{p}{,} \PYG{l+s+s1}{\PYGZsq{}}\PYG{l+s+s1}{Water}\PYG{l+s+s1}{\PYGZsq{}}\PYG{p}{)} \PYG{c+c1}{\PYGZsh{}calculating density using coolprop kg/m}
\PYG{n}{V} \PYG{o}{=} \PYG{n}{m} \PYG{o}{/} \PYG{n}{D} \PYG{c+c1}{\PYGZsh{}volume occupied m3}
\PYG{n+nb}{print}\PYG{p}{(}\PYG{l+s+s1}{\PYGZsq{}}\PYG{l+s+s1}{The volume of water wapor based on thermodynamic tables is}\PYG{l+s+s1}{\PYGZsq{}}\PYG{p}{,}\PYG{l+s+sa}{f}\PYG{l+s+s2}{\PYGZdq{}}\PYG{l+s+si}{\PYGZob{}}\PYG{n}{V}\PYG{l+s+si}{:}\PYG{l+s+s2}{.3f}\PYG{l+s+si}{\PYGZcb{}}\PYG{l+s+s2}{\PYGZdq{}}\PYG{p}{,}\PYG{l+s+s1}{\PYGZsq{}}\PYG{l+s+s1}{m3}\PYG{l+s+s1}{\PYGZsq{}}\PYG{p}{)}

\PYG{c+c1}{\PYGZsh{}\PYGZsh{} this value is treated as reference for error calculation since it\PYGZsq{}s based on experiments}
\PYG{n}{V\PYGZus{}ref} \PYG{o}{=} \PYG{n}{V}
\end{sphinxVerbatim}

\end{sphinxuseclass}\end{sphinxVerbatimInput}
\begin{sphinxVerbatimOutput}

\begin{sphinxuseclass}{cell_output}
\begin{sphinxVerbatim}[commandchars=\\\{\}]
The volume of water wapor based on thermodynamic tables is 0.151 m3
\end{sphinxVerbatim}

\end{sphinxuseclass}\end{sphinxVerbatimOutput}

\end{sphinxuseclass}

\subsection{Solution Approach for b)}
\label{\detokenize{notebooks/Chapter3/CH3-Q6_v1:solution-approach-for-b}}
\sphinxAtStartPar
To Calculate \(V\) the ideal gas equation of state shall be used where

\sphinxAtStartPar
\(V=mRT/P\)

\begin{sphinxuseclass}{cell}\begin{sphinxVerbatimInput}

\begin{sphinxuseclass}{cell_input}
\begin{sphinxVerbatim}[commandchars=\\\{\}]
\PYG{c+c1}{\PYGZsh{}introducing constant R}
\PYG{n}{R} \PYG{o}{=} \PYG{l+m+mf}{0.4615} \PYG{c+c1}{\PYGZsh{}Steam gas constant in kJ/kg.K}
\PYG{n}{P\PYGZus{}kpa} \PYG{o}{=} \PYG{n}{P} \PYG{o}{/} \PYG{l+m+mi}{1000} \PYG{c+c1}{\PYGZsh{}pressure need to be in KPa to be consistent in ideal gas equation}
\PYG{n}{V} \PYG{o}{=} \PYG{n}{m} \PYG{o}{*} \PYG{n}{R} \PYG{o}{*} \PYG{n}{T} \PYG{o}{/} \PYG{n}{P\PYGZus{}kpa}
\PYG{n+nb}{print}\PYG{p}{(}\PYG{l+s+s1}{\PYGZsq{}}\PYG{l+s+s1}{The volume of water wapor based on ideal gas correlation is}\PYG{l+s+s1}{\PYGZsq{}}\PYG{p}{,}\PYG{l+s+sa}{f}\PYG{l+s+s2}{\PYGZdq{}}\PYG{l+s+si}{\PYGZob{}}\PYG{n}{V}\PYG{l+s+si}{:}\PYG{l+s+s2}{.3f}\PYG{l+s+si}{\PYGZcb{}}\PYG{l+s+s2}{\PYGZdq{}}\PYG{p}{,}\PYG{l+s+s1}{\PYGZsq{}}\PYG{l+s+s1}{m3}\PYG{l+s+s1}{\PYGZsq{}}\PYG{p}{)}

\PYG{c+c1}{\PYGZsh{}calculating error}
\PYG{n}{E} \PYG{o}{=} \PYG{n}{np}\PYG{o}{.}\PYG{n}{abs}\PYG{p}{(}\PYG{n}{V}\PYG{o}{\PYGZhy{}}\PYG{n}{V\PYGZus{}ref}\PYG{p}{)}\PYG{o}{/}\PYG{n}{V\PYGZus{}ref} \PYG{o}{*} \PYG{l+m+mi}{100}
\PYG{n+nb}{print}\PYG{p}{(}\PYG{l+s+s1}{\PYGZsq{}}\PYG{l+s+s1}{The error based on ideal gas correlation is}\PYG{l+s+s1}{\PYGZsq{}}\PYG{p}{,}\PYG{l+s+sa}{f}\PYG{l+s+s2}{\PYGZdq{}}\PYG{l+s+si}{\PYGZob{}}\PYG{n}{E}\PYG{l+s+si}{:}\PYG{l+s+s2}{.3f}\PYG{l+s+si}{\PYGZcb{}}\PYG{l+s+s2}{\PYGZdq{}}\PYG{p}{,}\PYG{l+s+s1}{\PYGZsq{}}\PYG{l+s+s1}{\PYGZpc{}}\PYG{l+s+s1}{\PYGZsq{}}\PYG{p}{)}
\end{sphinxVerbatim}

\end{sphinxuseclass}\end{sphinxVerbatimInput}
\begin{sphinxVerbatimOutput}

\begin{sphinxuseclass}{cell_output}
\begin{sphinxVerbatim}[commandchars=\\\{\}]
The volume of water wapor based on ideal gas correlation is 0.155 m3
The error based on ideal gas correlation is 2.721 \PYGZpc{}
\end{sphinxVerbatim}

\end{sphinxuseclass}\end{sphinxVerbatimOutput}

\end{sphinxuseclass}

\subsection{Solution Approach for c)}
\label{\detokenize{notebooks/Chapter3/CH3-Q6_v1:solution-approach-for-c}}
\begin{sphinxuseclass}{cell}\begin{sphinxVerbatimInput}

\begin{sphinxuseclass}{cell_input}
\begin{sphinxVerbatim}[commandchars=\\\{\}]
\PYG{c+c1}{\PYGZsh{}Introducing critical values}
\PYG{n}{P\PYGZus{}crit} \PYG{o}{=} \PYG{l+m+mf}{22.06} \PYG{c+c1}{\PYGZsh{}critical pressure for water in MPa}
\PYG{n}{T\PYGZus{}crit} \PYG{o}{=} \PYG{l+m+mf}{647.1} \PYG{c+c1}{\PYGZsh{}critical temperature for water in k}

\PYG{c+c1}{\PYGZsh{}calculating reduced pressure and temperature}
\PYG{n}{P\PYGZus{}r} \PYG{o}{=} \PYG{n}{P} \PYG{o}{/} \PYG{l+m+mf}{22.06E+6} \PYG{c+c1}{\PYGZsh{}reduced pressure}
\PYG{n}{T\PYGZus{}r} \PYG{o}{=} \PYG{n}{T} \PYG{o}{/} \PYG{n}{T\PYGZus{}crit}  \PYG{c+c1}{\PYGZsh{}reduced temperature}

\PYG{c+c1}{\PYGZsh{}now the code for Question\PYGZsh{}5 in this chapter is used to calculate compressibility factor (Z)}
\PYG{c+c1}{\PYGZsh{}importing required library}
\PYG{k+kn}{import} \PYG{n+nn}{numpy} \PYG{k}{as} \PYG{n+nn}{np}
\PYG{k+kn}{import} \PYG{n+nn}{matplotlib}\PYG{n+nn}{.}\PYG{n+nn}{pyplot} \PYG{k}{as} \PYG{n+nn}{plt}

\PYG{c+c1}{\PYGZsh{}introducing constants}
\PYG{n}{b1} \PYG{o}{=} \PYG{l+m+mf}{0.1181193}
\PYG{n}{b2} \PYG{o}{=} \PYG{l+m+mf}{0.265728}
\PYG{n}{b3} \PYG{o}{=} \PYG{l+m+mf}{0.154790}
\PYG{n}{b4} \PYG{o}{=} \PYG{l+m+mf}{0.030323}
\PYG{n}{c1} \PYG{o}{=} \PYG{l+m+mf}{0.0236744}
\PYG{n}{c2} \PYG{o}{=} \PYG{l+m+mf}{0.0186984}
\PYG{n}{c3} \PYG{o}{=} \PYG{l+m+mi}{0}
\PYG{n}{c4} \PYG{o}{=} \PYG{l+m+mf}{0.042724}
\PYG{n}{d1} \PYG{o}{=} \PYG{l+m+mf}{0.155488E\PYGZhy{}4}
\PYG{n}{d2} \PYG{o}{=} \PYG{l+m+mf}{0.623689E\PYGZhy{}4}
\PYG{n}{betha} \PYG{o}{=} \PYG{l+m+mf}{0.65392}
\PYG{n}{gamma} \PYG{o}{=} \PYG{l+m+mf}{0.060167}

\PYG{n}{B} \PYG{o}{=} \PYG{n}{b1} \PYG{o}{\PYGZhy{}} \PYG{n}{b2}\PYG{o}{/}\PYG{n}{T\PYGZus{}r} \PYG{o}{\PYGZhy{}} \PYG{n}{b3}\PYG{o}{/}\PYG{n}{T\PYGZus{}r}\PYG{o}{*}\PYG{o}{*}\PYG{l+m+mi}{2} \PYG{o}{\PYGZhy{}} \PYG{n}{b4}\PYG{o}{/}\PYG{n}{T\PYGZus{}r}\PYG{o}{*}\PYG{o}{*}\PYG{l+m+mi}{3}
\PYG{n}{C} \PYG{o}{=} \PYG{n}{c1} \PYG{o}{\PYGZhy{}} \PYG{n}{c2}\PYG{o}{/}\PYG{n}{T\PYGZus{}r} \PYG{o}{+} \PYG{n}{c3}\PYG{o}{/}\PYG{n}{T\PYGZus{}r}\PYG{o}{*}\PYG{o}{*}\PYG{l+m+mi}{3}
\PYG{n}{D} \PYG{o}{=} \PYG{n}{d1} \PYG{o}{+} \PYG{n}{d2}\PYG{o}{/}\PYG{n}{T\PYGZus{}r}

\PYG{c+c1}{\PYGZsh{}V\PYGZus{}r in an array structure}
\PYG{c+c1}{\PYGZsh{} an array of V\PYGZus{}r is to be built so that Z is calculated based upon. Otherwise, the equation can\PYGZsq{}t be solved analytically for V\PYGZus{}r}
\PYG{n}{V\PYGZus{}r} \PYG{o}{=} \PYG{n}{np}\PYG{o}{.}\PYG{n}{logspace}\PYG{p}{(}\PYG{o}{\PYGZhy{}}\PYG{l+m+mf}{0.9}\PYG{p}{,} \PYG{l+m+mf}{1.65}\PYG{p}{,} \PYG{l+m+mi}{10000}\PYG{p}{)} 
\PYG{n}{Z\PYGZus{}array} \PYG{o}{=} \PYG{l+m+mi}{1} \PYG{o}{+} \PYG{n}{B}\PYG{o}{/}\PYG{n}{V\PYGZus{}r} \PYG{o}{+} \PYG{n}{C}\PYG{o}{/}\PYG{n}{V\PYGZus{}r}\PYG{o}{*}\PYG{o}{*}\PYG{l+m+mi}{2} \PYG{o}{+} \PYG{n}{D}\PYG{o}{/}\PYG{n}{V\PYGZus{}r}\PYG{o}{*}\PYG{o}{*}\PYG{l+m+mi}{5} \PYG{o}{+} \PYG{n}{c4}\PYG{o}{*}\PYG{p}{(}\PYG{n}{betha} \PYG{o}{+} \PYG{n}{gamma}\PYG{o}{/}\PYG{n}{V\PYGZus{}r}\PYG{o}{*}\PYG{o}{*}\PYG{l+m+mi}{2}\PYG{p}{)}\PYG{o}{*}\PYG{n}{np}\PYG{o}{.}\PYG{n}{exp}\PYG{p}{(}\PYG{o}{\PYGZhy{}}\PYG{n}{gamma}\PYG{o}{/}\PYG{n}{V\PYGZus{}r}\PYG{o}{*}\PYG{o}{*}\PYG{l+m+mi}{2}\PYG{p}{)}\PYG{o}{/}\PYG{p}{(}\PYG{n}{T\PYGZus{}r}\PYG{o}{*}\PYG{o}{*}\PYG{l+m+mi}{3} \PYG{o}{*} \PYG{n}{V\PYGZus{}r}\PYG{o}{*}\PYG{o}{*}\PYG{l+m+mi}{2}\PYG{p}{)} \PYG{c+c1}{\PYGZsh{}Lee\PYGZhy{}Kesler equation}
\PYG{n}{P\PYGZus{}r\PYGZus{}array} \PYG{o}{=} \PYG{n}{Z\PYGZus{}array} \PYG{o}{*} \PYG{n}{T\PYGZus{}r} \PYG{o}{/} \PYG{n}{V\PYGZus{}r} \PYG{c+c1}{\PYGZsh{}calculating P\PYGZus{}r based on the array built for V\PYGZus{}r}

\PYG{c+c1}{\PYGZsh{}finding the index of the array element in P\PYGZus{}r\PYGZus{}array which is closest to the desired P\PYGZus{}r value}
\PYG{n}{difference\PYGZus{}array} \PYG{o}{=} \PYG{n}{np}\PYG{o}{.}\PYG{n}{absolute}\PYG{p}{(}\PYG{n}{P\PYGZus{}r\PYGZus{}array}\PYG{o}{\PYGZhy{}}\PYG{n}{P\PYGZus{}r}\PYG{p}{)}
\PYG{n}{index} \PYG{o}{=} \PYG{n}{difference\PYGZus{}array}\PYG{o}{.}\PYG{n}{argmin}\PYG{p}{(}\PYG{p}{)}
\PYG{n}{Z} \PYG{o}{=} \PYG{n}{Z\PYGZus{}array}\PYG{p}{[}\PYG{n}{index}\PYG{p}{]}
\PYG{n+nb}{print}\PYG{p}{(}\PYG{l+s+s1}{\PYGZsq{}}\PYG{l+s+s1}{The compressibility factor value for P\PYGZus{}r =}\PYG{l+s+s1}{\PYGZsq{}}\PYG{p}{,}\PYG{l+s+sa}{f}\PYG{l+s+s2}{\PYGZdq{}}\PYG{l+s+si}{\PYGZob{}}\PYG{n}{P\PYGZus{}r}\PYG{l+s+si}{:}\PYG{l+s+s2}{.3f}\PYG{l+s+si}{\PYGZcb{}}\PYG{l+s+s2}{\PYGZdq{}}\PYG{p}{,} \PYG{l+s+s1}{\PYGZsq{}}\PYG{l+s+s1}{and T\PYGZus{}r =}\PYG{l+s+s1}{\PYGZsq{}}\PYG{p}{,} \PYG{l+s+sa}{f}\PYG{l+s+s2}{\PYGZdq{}}\PYG{l+s+si}{\PYGZob{}}\PYG{n}{T\PYGZus{}r}\PYG{l+s+si}{:}\PYG{l+s+s2}{.3f}\PYG{l+s+si}{\PYGZcb{}}\PYG{l+s+s2}{\PYGZdq{}}\PYG{p}{,} \PYG{l+s+s1}{\PYGZsq{}}\PYG{l+s+s1}{is}\PYG{l+s+s1}{\PYGZsq{}}\PYG{p}{,} \PYG{l+s+sa}{f}\PYG{l+s+s2}{\PYGZdq{}}\PYG{l+s+si}{\PYGZob{}}\PYG{n}{Z}\PYG{l+s+si}{:}\PYG{l+s+s2}{.3f}\PYG{l+s+si}{\PYGZcb{}}\PYG{l+s+s2}{\PYGZdq{}}\PYG{p}{)}
\end{sphinxVerbatim}

\end{sphinxuseclass}\end{sphinxVerbatimInput}
\begin{sphinxVerbatimOutput}

\begin{sphinxuseclass}{cell_output}
\begin{sphinxVerbatim}[commandchars=\\\{\}]
The compressibility factor value for P\PYGZus{}r = 0.091 and T\PYGZus{}r = 1.040 is 0.973
\end{sphinxVerbatim}

\end{sphinxuseclass}\end{sphinxVerbatimOutput}

\end{sphinxuseclass}
\sphinxAtStartPar
Looking at compressibility factor,

\sphinxAtStartPar
\(Z=Pv/RT=PV/mRT\)

\sphinxAtStartPar
rearranging for V,

\sphinxAtStartPar
\(V=ZmRT/P\)

\begin{sphinxuseclass}{cell}\begin{sphinxVerbatimInput}

\begin{sphinxuseclass}{cell_input}
\begin{sphinxVerbatim}[commandchars=\\\{\}]
\PYG{n}{V} \PYG{o}{=} \PYG{n}{Z} \PYG{o}{*} \PYG{n}{m} \PYG{o}{*} \PYG{n}{R} \PYG{o}{*} \PYG{n}{T} \PYG{o}{/} \PYG{n}{P\PYGZus{}kpa}
\PYG{n+nb}{print}\PYG{p}{(}\PYG{l+s+s1}{\PYGZsq{}}\PYG{l+s+s1}{The volume of water wapor based on ideal gas correlation coupled with compressibiity factor is}\PYG{l+s+s1}{\PYGZsq{}}\PYG{p}{,}\PYG{l+s+sa}{f}\PYG{l+s+s2}{\PYGZdq{}}\PYG{l+s+si}{\PYGZob{}}\PYG{n}{V}\PYG{l+s+si}{:}\PYG{l+s+s2}{.3f}\PYG{l+s+si}{\PYGZcb{}}\PYG{l+s+s2}{\PYGZdq{}}\PYG{p}{,}\PYG{l+s+s1}{\PYGZsq{}}\PYG{l+s+s1}{m3}\PYG{l+s+s1}{\PYGZsq{}}\PYG{p}{)}

\PYG{c+c1}{\PYGZsh{}calculating error}
\PYG{n}{E} \PYG{o}{=} \PYG{n}{np}\PYG{o}{.}\PYG{n}{abs}\PYG{p}{(}\PYG{n}{V}\PYG{o}{\PYGZhy{}}\PYG{n}{V\PYGZus{}ref}\PYG{p}{)}\PYG{o}{/}\PYG{n}{V\PYGZus{}ref} \PYG{o}{*} \PYG{l+m+mi}{100}
\PYG{n+nb}{print}\PYG{p}{(}\PYG{l+s+s1}{\PYGZsq{}}\PYG{l+s+s1}{The error based on ideal gas correlation coupled with compressibiity factor is}\PYG{l+s+s1}{\PYGZsq{}}\PYG{p}{,}\PYG{l+s+sa}{f}\PYG{l+s+s2}{\PYGZdq{}}\PYG{l+s+si}{\PYGZob{}}\PYG{n}{E}\PYG{l+s+si}{:}\PYG{l+s+s2}{.3f}\PYG{l+s+si}{\PYGZcb{}}\PYG{l+s+s2}{\PYGZdq{}}\PYG{p}{,}\PYG{l+s+s1}{\PYGZsq{}}\PYG{l+s+s1}{\PYGZpc{}}\PYG{l+s+s1}{\PYGZsq{}}\PYG{p}{)}
\end{sphinxVerbatim}

\end{sphinxuseclass}\end{sphinxVerbatimInput}
\begin{sphinxVerbatimOutput}

\begin{sphinxuseclass}{cell_output}
\begin{sphinxVerbatim}[commandchars=\\\{\}]
The volume of water wapor based on ideal gas correlation coupled with compressibiity factor is 0.151 m3
The error based on ideal gas correlation coupled with compressibiity factor is 0.082 \PYGZpc{}
\end{sphinxVerbatim}

\end{sphinxuseclass}\end{sphinxVerbatimOutput}

\end{sphinxuseclass}

\subsection{Solution Approach for d)}
\label{\detokenize{notebooks/Chapter3/CH3-Q6_v1:solution-approach-for-d}}
\begin{sphinxuseclass}{cell}\begin{sphinxVerbatimInput}

\begin{sphinxuseclass}{cell_input}
\begin{sphinxVerbatim}[commandchars=\\\{\}]
\PYG{c+c1}{\PYGZsh{} define variables}
\PYG{n}{Q} \PYG{o}{=} \PYG{l+m+mi}{1}  \PYG{c+c1}{\PYGZsh{} define the steam quality as 1, which is 100\PYGZpc{} vapor}
\PYG{n}{fluid} \PYG{o}{=} \PYG{l+s+s2}{\PYGZdq{}}\PYG{l+s+s2}{water}\PYG{l+s+s2}{\PYGZdq{}}  \PYG{c+c1}{\PYGZsh{} define the fluid or material of interest, for full list see CP.Fluidslist()}
\PYG{n}{T\PYGZus{}min} \PYG{o}{=} \PYG{n}{CP}\PYG{o}{.}\PYG{n}{PropsSI}\PYG{p}{(}\PYG{l+s+s2}{\PYGZdq{}}\PYG{l+s+s2}{Tmin}\PYG{l+s+s2}{\PYGZdq{}}\PYG{p}{,} \PYG{n}{fluid}\PYG{p}{)}  \PYG{c+c1}{\PYGZsh{} this is the triple\PYGZhy{}point temp we can get data for water}
\PYG{n}{P\PYGZus{}min} \PYG{o}{=} \PYG{n}{CP}\PYG{o}{.}\PYG{n}{PropsSI}\PYG{p}{(}\PYG{l+s+s2}{\PYGZdq{}}\PYG{l+s+s2}{P}\PYG{l+s+s2}{\PYGZdq{}}\PYG{p}{,} \PYG{l+s+s2}{\PYGZdq{}}\PYG{l+s+s2}{T}\PYG{l+s+s2}{\PYGZdq{}}\PYG{p}{,} \PYG{n}{T\PYGZus{}min}\PYG{p}{,} \PYG{l+s+s2}{\PYGZdq{}}\PYG{l+s+s2}{Q}\PYG{l+s+s2}{\PYGZdq{}}\PYG{p}{,} \PYG{n}{Q}\PYG{p}{,} \PYG{n}{fluid}\PYG{p}{)}  \PYG{c+c1}{\PYGZsh{} triple\PYGZhy{}point temp for pressure}
\PYG{n}{T\PYGZus{}max} \PYG{o}{=} \PYG{n}{CP}\PYG{o}{.}\PYG{n}{PropsSI}\PYG{p}{(}\PYG{l+s+s2}{\PYGZdq{}}\PYG{l+s+s2}{Tcrit}\PYG{l+s+s2}{\PYGZdq{}}\PYG{p}{,} \PYG{n}{fluid}\PYG{p}{)}  \PYG{c+c1}{\PYGZsh{} this is the max temp we can get data for water}
\PYG{n}{T\PYGZus{}vals} \PYG{o}{=} \PYG{n}{np}\PYG{o}{.}\PYG{n}{linspace}\PYG{p}{(}\PYG{n}{T\PYGZus{}min}\PYG{p}{,} \PYG{n}{T\PYGZus{}max}\PYG{p}{,} \PYG{l+m+mi}{1000}\PYG{p}{)}  \PYG{c+c1}{\PYGZsh{} define an array of values from T\PYGZus{}min to T\PYGZus{}max}

\PYG{n}{pressure} \PYG{o}{=} \PYG{p}{[}\PYG{n}{CP}\PYG{o}{.}\PYG{n}{PropsSI}\PYG{p}{(}\PYG{l+s+s2}{\PYGZdq{}}\PYG{l+s+s2}{P}\PYG{l+s+s2}{\PYGZdq{}}\PYG{p}{,} \PYG{l+s+s2}{\PYGZdq{}}\PYG{l+s+s2}{T}\PYG{l+s+s2}{\PYGZdq{}}\PYG{p}{,} \PYG{n}{T}\PYG{p}{,} \PYG{l+s+s2}{\PYGZdq{}}\PYG{l+s+s2}{Q}\PYG{l+s+s2}{\PYGZdq{}}\PYG{p}{,} \PYG{n}{Q}\PYG{p}{,} \PYG{n}{fluid}\PYG{p}{)} \PYG{k}{for} \PYG{n}{T} \PYG{o+ow}{in} \PYG{n}{T\PYGZus{}vals}\PYG{p}{]}  \PYG{c+c1}{\PYGZsh{} call for pressure values using CoolProp}

\PYG{n}{plt}\PYG{o}{.}\PYG{n}{plot}\PYG{p}{(}\PYG{n}{T\PYGZus{}vals}\PYG{o}{\PYGZhy{}}\PYG{l+m+mf}{273.15}\PYG{p}{,} \PYG{n}{pressure}\PYG{p}{,} \PYG{l+s+s2}{\PYGZdq{}}\PYG{l+s+s2}{\PYGZhy{}\PYGZhy{}b}\PYG{l+s+s2}{\PYGZdq{}}\PYG{p}{,} \PYG{n}{label}\PYG{o}{=}\PYG{l+s+s2}{\PYGZdq{}}\PYG{l+s+s2}{Saturation Line}\PYG{l+s+s2}{\PYGZdq{}}\PYG{p}{)}  \PYG{c+c1}{\PYGZsh{} plot temp vs specific vol}
\PYG{n}{plt}\PYG{o}{.}\PYG{n}{xscale}\PYG{p}{(}\PYG{l+s+s2}{\PYGZdq{}}\PYG{l+s+s2}{log}\PYG{l+s+s2}{\PYGZdq{}}\PYG{p}{)}  \PYG{c+c1}{\PYGZsh{} use log scale on x axis}

\PYG{c+c1}{\PYGZsh{}\PYGZsh{} something interesting happenning \PYGZhy{}\PYGZhy{} why does Saturated liq and vapor P\PYGZhy{}T curve fall into the same curve?}

\PYG{n}{plt}\PYG{o}{.}\PYG{n}{xlabel}\PYG{p}{(}\PYG{l+s+s2}{\PYGZdq{}}\PYG{l+s+s2}{Temperature [C]}\PYG{l+s+s2}{\PYGZdq{}}\PYG{p}{)}  \PYG{c+c1}{\PYGZsh{} give x axis a label}
\PYG{n}{plt}\PYG{o}{.}\PYG{n}{ylabel}\PYG{p}{(}\PYG{l+s+s2}{\PYGZdq{}}\PYG{l+s+s2}{Pressure [Pa]}\PYG{l+s+s2}{\PYGZdq{}}\PYG{p}{)}  \PYG{c+c1}{\PYGZsh{} give y axis a label}

\PYG{c+c1}{\PYGZsh{} plot various points on the T\PYGZhy{}v diagram:}
\PYG{n}{plt}\PYG{o}{.}\PYG{n}{plot}\PYG{p}{(}\PYG{n}{T}\PYG{p}{,}\PYG{n}{P}\PYG{p}{,}\PYG{l+s+s1}{\PYGZsq{}}\PYG{l+s+s1}{or}\PYG{l+s+s1}{\PYGZsq{}}\PYG{p}{,} \PYG{n}{label} \PYG{o}{=} \PYG{l+s+s1}{\PYGZsq{}}\PYG{l+s+s1}{State}\PYG{l+s+s1}{\PYGZsq{}}\PYG{p}{)}
\PYG{n}{plt}\PYG{o}{.}\PYG{n}{plot}\PYG{p}{(}\PYG{n}{T\PYGZus{}min}\PYG{o}{\PYGZhy{}}\PYG{l+m+mf}{273.15}\PYG{p}{,}\PYG{n}{P\PYGZus{}min}\PYG{p}{,}\PYG{l+s+s1}{\PYGZsq{}}\PYG{l+s+s1}{og}\PYG{l+s+s1}{\PYGZsq{}}\PYG{p}{,} \PYG{n}{label} \PYG{o}{=} \PYG{l+s+s1}{\PYGZsq{}}\PYG{l+s+s1}{Triple Point}\PYG{l+s+s1}{\PYGZsq{}}\PYG{p}{)}

\PYG{n}{plt}\PYG{o}{.}\PYG{n}{legend}\PYG{p}{(}\PYG{p}{)}
\end{sphinxVerbatim}

\end{sphinxuseclass}\end{sphinxVerbatimInput}
\begin{sphinxVerbatimOutput}

\begin{sphinxuseclass}{cell_output}
\begin{sphinxVerbatim}[commandchars=\\\{\}]
\PYGZlt{}matplotlib.legend.Legend at 0x7fec037114f0\PYGZgt{}
\end{sphinxVerbatim}

\noindent\sphinxincludegraphics{{e48c6614db95715554ed23053f2646510f9075f6cd8062fa0db691f5e2462564}.png}

\end{sphinxuseclass}\end{sphinxVerbatimOutput}

\end{sphinxuseclass}
\sphinxstepscope


\chapter{4. The First Law of Thermodynamics for Closed Systems}
\label{\detokenize{notebooks/Chapter4/first-law:the-first-law-of-thermodynamics-for-closed-systems}}\label{\detokenize{notebooks/Chapter4/first-law::doc}}
\sphinxAtStartPar
This Chapter goes through concepts from Chapter 4 from here: \sphinxurl{https://pressbooks.bccampus.ca/thermo1/chapter/4-0-chapter-introduction-and-learning-objectives/}
Using the first law, internal energy, work done and heat supplied/rejected are calculated for various senarios.

\sphinxstepscope


\section{Piston\sphinxhyphen{}cylinder: isothermal expansion}
\label{\detokenize{notebooks/Chapter4/Thermodynamics_Example_2:piston-cylinder-isothermal-expansion}}\label{\detokenize{notebooks/Chapter4/Thermodynamics_Example_2::doc}}

\subsection{Problem Statement:}
\label{\detokenize{notebooks/Chapter4/Thermodynamics_Example_2:problem-statement}}
\sphinxAtStartPar
A cylinder fitted with a frictionless piston contains 1 kg of oxygen gas at an initial temperature of 20°C
and a volume of 0.8 m³. The gas undergoes an isothermal expansion until its volume doubles.
During the process, the cylinder is in thermal contact with a heat reservoir at 20°C.

\sphinxAtStartPar
Calculate the following:
\begin{enumerate}
\sphinxsetlistlabels{\arabic}{enumi}{enumii}{}{.}%
\item {} 
\sphinxAtStartPar
The specific boundary work done by the oxygen gas during the expansion.

\item {} 
\sphinxAtStartPar
The amount of heat transferred to the gas during this process.

\item {} 
\sphinxAtStartPar
Assuming the specific internal energy of oxygen changes due to the process, calculate the change in specific internal energy.

\end{enumerate}

\sphinxAtStartPar
For the calculations, assume that oxygen behaves as an ideal gas and use its specific properties.
Also, consider that the process is isothermal, meaning the temperature remains constant throughout the process.


\subsection{Solution:}
\label{\detokenize{notebooks/Chapter4/Thermodynamics_Example_2:solution}}
\begin{sphinxuseclass}{cell}\begin{sphinxVerbatimInput}

\begin{sphinxuseclass}{cell_input}
\begin{sphinxVerbatim}[commandchars=\\\{\}]
\PYG{c+c1}{\PYGZsh{}\PYGZsh{} Solution:}

\PYG{k+kn}{import} \PYG{n+nn}{CoolProp}\PYG{n+nn}{.}\PYG{n+nn}{CoolProp} \PYG{k}{as} \PYG{n+nn}{CP}
\PYG{k+kn}{import} \PYG{n+nn}{math}

\PYG{c+c1}{\PYGZsh{} Gas choice}
\PYG{n}{gas} \PYG{o}{=} \PYG{l+s+s2}{\PYGZdq{}}\PYG{l+s+s2}{Oxygen}\PYG{l+s+s2}{\PYGZdq{}}

\PYG{c+c1}{\PYGZsh{} Given values}
\PYG{n}{m} \PYG{o}{=} \PYG{l+m+mf}{1.0}  \PYG{c+c1}{\PYGZsh{} Mass of the gas in kg}
\PYG{n}{T1} \PYG{o}{=} \PYG{l+m+mi}{20} \PYG{o}{+} \PYG{l+m+mf}{273.15}  \PYG{c+c1}{\PYGZsh{} Initial temperature in Kelvin (converted from 20°C)}
\PYG{n}{V1} \PYG{o}{=} \PYG{l+m+mf}{0.8}  \PYG{c+c1}{\PYGZsh{} Initial volume in m\PYGZca{}3}
\PYG{n}{V2} \PYG{o}{=} \PYG{l+m+mi}{2} \PYG{o}{*} \PYG{n}{V1}  \PYG{c+c1}{\PYGZsh{} Final volume (double the initial volume)}
\PYG{n}{R} \PYG{o}{=} \PYG{n}{CP}\PYG{o}{.}\PYG{n}{PropsSI}\PYG{p}{(}\PYG{l+s+s1}{\PYGZsq{}}\PYG{l+s+s1}{GAS\PYGZus{}CONSTANT}\PYG{l+s+s1}{\PYGZsq{}}\PYG{p}{,} \PYG{n}{gas}\PYG{p}{)} \PYG{o}{/} \PYG{n}{CP}\PYG{o}{.}\PYG{n}{PropsSI}\PYG{p}{(}\PYG{l+s+s1}{\PYGZsq{}}\PYG{l+s+s1}{MOLAR\PYGZus{}MASS}\PYG{l+s+s1}{\PYGZsq{}}\PYG{p}{,} \PYG{n}{gas}\PYG{p}{)}  \PYG{c+c1}{\PYGZsh{} Specific gas constant for Oxygen}

\PYG{c+c1}{\PYGZsh{} 1. Specific Boundary Work (w\PYGZus{}boundary)}
\PYG{c+c1}{\PYGZsh{} For an isothermal process, P1 * V1 = P2 * V2 (Ideal Gas Law), and P1 can be found from P1 = m * R * T1 / V1}
\PYG{n}{P1} \PYG{o}{=} \PYG{n}{m} \PYG{o}{*} \PYG{n}{R} \PYG{o}{*} \PYG{n}{T1} \PYG{o}{/} \PYG{n}{V1}
\PYG{c+c1}{\PYGZsh{} Calculating the boundary work}
\PYG{n}{w\PYGZus{}boundary} \PYG{o}{=} \PYG{n}{P1} \PYG{o}{*} \PYG{n}{V1} \PYG{o}{*} \PYG{n}{math}\PYG{o}{.}\PYG{n}{log}\PYG{p}{(}\PYG{n}{V2} \PYG{o}{/} \PYG{n}{V1}\PYG{p}{)}

\PYG{c+c1}{\PYGZsh{} 2. Heat Transfer (Q)}
\PYG{c+c1}{\PYGZsh{} For an isothermal process, the heat transfer is equal to the boundary work done}
\PYG{n}{Q} \PYG{o}{=} \PYG{n}{w\PYGZus{}boundary}

\PYG{c+c1}{\PYGZsh{} 3. Change in Specific Internal Energy (Δu)}
\PYG{c+c1}{\PYGZsh{} For an isothermal process of an ideal gas, Δu = 0}
\PYG{n}{delta\PYGZus{}u} \PYG{o}{=} \PYG{l+m+mi}{0}

\PYG{c+c1}{\PYGZsh{} Output the results}
\PYG{n+nb}{print}\PYG{p}{(}\PYG{l+s+sa}{f}\PYG{l+s+s2}{\PYGZdq{}}\PYG{l+s+s2}{Specific Boundary Work (w\PYGZus{}boundary): }\PYG{l+s+si}{\PYGZob{}}\PYG{n+nb}{round}\PYG{p}{(}\PYG{n}{w\PYGZus{}boundary}\PYG{o}{/}\PYG{l+m+mf}{1e3}\PYG{p}{,}\PYG{l+m+mi}{1}\PYG{p}{)}\PYG{l+s+si}{\PYGZcb{}}\PYG{l+s+s2}{ kJ}\PYG{l+s+s2}{\PYGZdq{}}\PYG{p}{)}
\PYG{n+nb}{print}\PYG{p}{(}\PYG{l+s+sa}{f}\PYG{l+s+s2}{\PYGZdq{}}\PYG{l+s+s2}{Heat Transfer (Q): }\PYG{l+s+si}{\PYGZob{}}\PYG{n+nb}{round}\PYG{p}{(}\PYG{n}{Q}\PYG{o}{/}\PYG{l+m+mf}{1e3}\PYG{p}{,}\PYG{l+m+mi}{1}\PYG{p}{)}\PYG{l+s+si}{\PYGZcb{}}\PYG{l+s+s2}{ kJ}\PYG{l+s+s2}{\PYGZdq{}}\PYG{p}{)}
\PYG{n+nb}{print}\PYG{p}{(}\PYG{l+s+sa}{f}\PYG{l+s+s2}{\PYGZdq{}}\PYG{l+s+s2}{Change in Specific Internal Energy (Δu): }\PYG{l+s+si}{\PYGZob{}}\PYG{n+nb}{round}\PYG{p}{(}\PYG{n}{delta\PYGZus{}u}\PYG{o}{/}\PYG{l+m+mf}{1e3}\PYG{p}{,}\PYG{l+m+mi}{1}\PYG{p}{)}\PYG{l+s+si}{\PYGZcb{}}\PYG{l+s+s2}{ kJ}\PYG{l+s+s2}{\PYGZdq{}}\PYG{p}{)}
\end{sphinxVerbatim}

\end{sphinxuseclass}\end{sphinxVerbatimInput}
\begin{sphinxVerbatimOutput}

\begin{sphinxuseclass}{cell_output}
\begin{sphinxVerbatim}[commandchars=\\\{\}]
Specific Boundary Work (w\PYGZus{}boundary): 52.8 kJ
Heat Transfer (Q): 52.8 kJ
Change in Specific Internal Energy (Δu): 0.0 kJ
\end{sphinxVerbatim}

\end{sphinxuseclass}\end{sphinxVerbatimOutput}

\end{sphinxuseclass}
\sphinxstepscope


\section{Isochoric Process for a Gas}
\label{\detokenize{notebooks/Chapter4/Isochoric_Process_Problem_3:isochoric-process-for-a-gas}}\label{\detokenize{notebooks/Chapter4/Isochoric_Process_Problem_3::doc}}

\subsection{Problem Statement:}
\label{\detokenize{notebooks/Chapter4/Isochoric_Process_Problem_3:problem-statement}}
\sphinxAtStartPar
Consider a container with a fixed volume of 0.3 m³ that initially contains helium gas at a temperature of 15°C
and a pressure of 100 kPa. The gas undergoes an isochoric (constant volume) heating process until its pressure triples.
Calculate:
\begin{enumerate}
\sphinxsetlistlabels{\arabic}{enumi}{enumii}{}{.}%
\item {} 
\sphinxAtStartPar
The final temperature of the helium gas.

\item {} 
\sphinxAtStartPar
The change in internal energy of the gas during this process, assuming the specific heat at constant volume (Cv) is known.

\item {} 
\sphinxAtStartPar
The total heat transfer to the helium gas.

\end{enumerate}

\sphinxAtStartPar
Use the ideal gas law and assume helium behaves as an ideal gas.


\subsection{Solution:}
\label{\detokenize{notebooks/Chapter4/Isochoric_Process_Problem_3:solution}}
\begin{sphinxuseclass}{cell}\begin{sphinxVerbatimInput}

\begin{sphinxuseclass}{cell_input}
\begin{sphinxVerbatim}[commandchars=\\\{\}]
\PYG{k+kn}{import} \PYG{n+nn}{CoolProp}\PYG{n+nn}{.}\PYG{n+nn}{CoolProp} \PYG{k}{as} \PYG{n+nn}{CP}
\PYG{k+kn}{import} \PYG{n+nn}{math}

\PYG{c+c1}{\PYGZsh{} Given values}
\PYG{n}{V} \PYG{o}{=} \PYG{l+m+mf}{0.3}  \PYG{c+c1}{\PYGZsh{} Volume in m\PYGZca{}3}
\PYG{n}{T1} \PYG{o}{=} \PYG{l+m+mi}{15} \PYG{o}{+} \PYG{l+m+mf}{273.15}  \PYG{c+c1}{\PYGZsh{} Initial temperature in Kelvin}
\PYG{n}{P1} \PYG{o}{=} \PYG{l+m+mi}{100000}  \PYG{c+c1}{\PYGZsh{} Initial pressure in Pa (100 kPa)}
\PYG{n}{P2} \PYG{o}{=} \PYG{l+m+mi}{3} \PYG{o}{*} \PYG{n}{P1}  \PYG{c+c1}{\PYGZsh{} Final pressure (tripled)}
\PYG{n}{gas} \PYG{o}{=} \PYG{l+s+s1}{\PYGZsq{}}\PYG{l+s+s1}{Helium}\PYG{l+s+s1}{\PYGZsq{}}

\PYG{c+c1}{\PYGZsh{} 1. Final Temperature (T2)}
\PYG{c+c1}{\PYGZsh{} Since the process is isochoric, P1/T1 = P2/T2 (Ideal Gas Law)}
\PYG{n}{T2} \PYG{o}{=} \PYG{n}{P2} \PYG{o}{*} \PYG{n}{T1} \PYG{o}{/} \PYG{n}{P1}

\PYG{c+c1}{\PYGZsh{} 2. Change in internal energy (Δu)}
\PYG{n}{Cv} \PYG{o}{=} \PYG{n}{CP}\PYG{o}{.}\PYG{n}{PropsSI}\PYG{p}{(}\PYG{l+s+s1}{\PYGZsq{}}\PYG{l+s+s1}{Cvmass}\PYG{l+s+s1}{\PYGZsq{}}\PYG{p}{,} \PYG{l+s+s1}{\PYGZsq{}}\PYG{l+s+s1}{T}\PYG{l+s+s1}{\PYGZsq{}}\PYG{p}{,} \PYG{n}{T1}\PYG{p}{,} \PYG{l+s+s1}{\PYGZsq{}}\PYG{l+s+s1}{P}\PYG{l+s+s1}{\PYGZsq{}}\PYG{p}{,} \PYG{n}{P1}\PYG{p}{,} \PYG{n}{gas}\PYG{p}{)}  \PYG{c+c1}{\PYGZsh{} Cv for helium}
\PYG{n}{delta\PYGZus{}u} \PYG{o}{=} \PYG{n}{Cv} \PYG{o}{*} \PYG{p}{(}\PYG{n}{T2} \PYG{o}{\PYGZhy{}} \PYG{n}{T1}\PYG{p}{)}

\PYG{c+c1}{\PYGZsh{} 3. Total heat transfer (Q)}
\PYG{c+c1}{\PYGZsh{} For an isochoric process, Q = Δu (First Law of Thermodynamics)}
\PYG{n}{Q} \PYG{o}{=} \PYG{n}{delta\PYGZus{}u}

\PYG{c+c1}{\PYGZsh{} Output the results}
\PYG{n+nb}{print}\PYG{p}{(}\PYG{l+s+sa}{f}\PYG{l+s+s2}{\PYGZdq{}}\PYG{l+s+s2}{Final Temperature (T2): }\PYG{l+s+si}{\PYGZob{}}\PYG{n+nb}{round}\PYG{p}{(}\PYG{n}{T2}\PYG{p}{,}\PYG{l+m+mi}{1}\PYG{p}{)}\PYG{l+s+si}{\PYGZcb{}}\PYG{l+s+s2}{ K}\PYG{l+s+s2}{\PYGZdq{}}\PYG{p}{)}
\PYG{n+nb}{print}\PYG{p}{(}\PYG{l+s+sa}{f}\PYG{l+s+s2}{\PYGZdq{}}\PYG{l+s+s2}{Change in internal energy (Δu): }\PYG{l+s+si}{\PYGZob{}}\PYG{n+nb}{round}\PYG{p}{(}\PYG{n}{delta\PYGZus{}u}\PYG{o}{/}\PYG{l+m+mf}{1e3}\PYG{p}{,}\PYG{l+m+mi}{1}\PYG{p}{)}\PYG{l+s+si}{\PYGZcb{}}\PYG{l+s+s2}{ kJ}\PYG{l+s+s2}{\PYGZdq{}}\PYG{p}{)}
\PYG{n+nb}{print}\PYG{p}{(}\PYG{l+s+sa}{f}\PYG{l+s+s2}{\PYGZdq{}}\PYG{l+s+s2}{Total heat transfer (Q): }\PYG{l+s+si}{\PYGZob{}}\PYG{n+nb}{round}\PYG{p}{(}\PYG{n}{Q}\PYG{o}{/}\PYG{l+m+mf}{1e3}\PYG{p}{,}\PYG{l+m+mi}{1}\PYG{p}{)}\PYG{l+s+si}{\PYGZcb{}}\PYG{l+s+s2}{ kJ}\PYG{l+s+s2}{\PYGZdq{}}\PYG{p}{)}
\end{sphinxVerbatim}

\end{sphinxuseclass}\end{sphinxVerbatimInput}
\begin{sphinxVerbatimOutput}

\begin{sphinxuseclass}{cell_output}
\begin{sphinxVerbatim}[commandchars=\\\{\}]
Final Temperature (T2): 864.5 K
Change in internal energy (Δu): 1795.8 kJ
Total heat transfer (Q): 1795.8 kJ
\end{sphinxVerbatim}

\end{sphinxuseclass}\end{sphinxVerbatimOutput}

\end{sphinxuseclass}
\sphinxstepscope


\section{Adiabatic Compression of an Ideal Gas}
\label{\detokenize{notebooks/Chapter4/Adiabatic_Compression_Problem_4:adiabatic-compression-of-an-ideal-gas}}\label{\detokenize{notebooks/Chapter4/Adiabatic_Compression_Problem_4::doc}}

\subsection{Problem Statement:}
\label{\detokenize{notebooks/Chapter4/Adiabatic_Compression_Problem_4:problem-statement}}
\sphinxAtStartPar
A piston\sphinxhyphen{}cylinder device initially contains 2 kg of air at 25°C and 1 atm.
The air is compressed adiabatically to one\sphinxhyphen{}eighth of its original volume.
Calculate:
\begin{enumerate}
\sphinxsetlistlabels{\arabic}{enumi}{enumii}{}{.}%
\item {} 
\sphinxAtStartPar
The final temperature and pressure of the air.

\item {} 
\sphinxAtStartPar
The work done during this adiabatic compression process.

\item {} 
\sphinxAtStartPar
Assuming air behaves as an ideal gas with a specific heat ratio (γ), determine the change in internal energy.

\end{enumerate}


\subsection{Solution:}
\label{\detokenize{notebooks/Chapter4/Adiabatic_Compression_Problem_4:solution}}
\begin{sphinxuseclass}{cell}\begin{sphinxVerbatimInput}

\begin{sphinxuseclass}{cell_input}
\begin{sphinxVerbatim}[commandchars=\\\{\}]
\PYG{k+kn}{import} \PYG{n+nn}{CoolProp}\PYG{n+nn}{.}\PYG{n+nn}{CoolProp} \PYG{k}{as} \PYG{n+nn}{CP}
\PYG{k+kn}{import} \PYG{n+nn}{math}

\PYG{c+c1}{\PYGZsh{} Given values}
\PYG{n}{m} \PYG{o}{=} \PYG{l+m+mf}{2.0}  \PYG{c+c1}{\PYGZsh{} Mass of air in kg}
\PYG{n}{T1} \PYG{o}{=} \PYG{l+m+mi}{25} \PYG{o}{+} \PYG{l+m+mf}{273.15}  \PYG{c+c1}{\PYGZsh{} Initial temperature in Kelvin}
\PYG{n}{P1} \PYG{o}{=} \PYG{l+m+mi}{101325}  \PYG{c+c1}{\PYGZsh{} Initial pressure in Pa (1 atm)}
\PYG{n}{V1} \PYG{o}{=} \PYG{l+m+mf}{1.0}  \PYG{c+c1}{\PYGZsh{} Initial volume (arbitrary value)}
\PYG{n}{V2} \PYG{o}{=} \PYG{n}{V1} \PYG{o}{/} \PYG{l+m+mi}{8}  \PYG{c+c1}{\PYGZsh{} Final volume (one\PYGZhy{}eighth of initial)}
\PYG{n}{gamma} \PYG{o}{=} \PYG{n}{CP}\PYG{o}{.}\PYG{n}{PropsSI}\PYG{p}{(}\PYG{l+s+s1}{\PYGZsq{}}\PYG{l+s+s1}{Cpmass}\PYG{l+s+s1}{\PYGZsq{}}\PYG{p}{,} \PYG{l+s+s1}{\PYGZsq{}}\PYG{l+s+s1}{T}\PYG{l+s+s1}{\PYGZsq{}}\PYG{p}{,} \PYG{n}{T1}\PYG{p}{,} \PYG{l+s+s1}{\PYGZsq{}}\PYG{l+s+s1}{P}\PYG{l+s+s1}{\PYGZsq{}}\PYG{p}{,} \PYG{n}{P1}\PYG{p}{,} \PYG{l+s+s1}{\PYGZsq{}}\PYG{l+s+s1}{Air}\PYG{l+s+s1}{\PYGZsq{}}\PYG{p}{)} \PYG{o}{/} \PYG{n}{CP}\PYG{o}{.}\PYG{n}{PropsSI}\PYG{p}{(}\PYG{l+s+s1}{\PYGZsq{}}\PYG{l+s+s1}{Cvmass}\PYG{l+s+s1}{\PYGZsq{}}\PYG{p}{,} \PYG{l+s+s1}{\PYGZsq{}}\PYG{l+s+s1}{T}\PYG{l+s+s1}{\PYGZsq{}}\PYG{p}{,} \PYG{n}{T1}\PYG{p}{,} \PYG{l+s+s1}{\PYGZsq{}}\PYG{l+s+s1}{P}\PYG{l+s+s1}{\PYGZsq{}}\PYG{p}{,} \PYG{n}{P1}\PYG{p}{,} \PYG{l+s+s1}{\PYGZsq{}}\PYG{l+s+s1}{Air}\PYG{l+s+s1}{\PYGZsq{}}\PYG{p}{)}  \PYG{c+c1}{\PYGZsh{} γ for air}

\PYG{c+c1}{\PYGZsh{} 1. Final Temperature and Pressure (T2, P2)}
\PYG{c+c1}{\PYGZsh{} For adiabatic process, T1 * V1\PYGZca{}(gamma \PYGZhy{} 1) = T2 * V2\PYGZca{}(gamma \PYGZhy{} 1)}
\PYG{n}{T2} \PYG{o}{=} \PYG{n}{T1} \PYG{o}{*} \PYG{p}{(}\PYG{n}{V1} \PYG{o}{/} \PYG{n}{V2}\PYG{p}{)} \PYG{o}{*}\PYG{o}{*} \PYG{p}{(}\PYG{n}{gamma} \PYG{o}{\PYGZhy{}} \PYG{l+m+mi}{1}\PYG{p}{)}
\PYG{c+c1}{\PYGZsh{} P2 using the adiabatic relation P2 * V2\PYGZca{}gamma = P1 * V1\PYGZca{}gamma}
\PYG{n}{P2} \PYG{o}{=} \PYG{n}{P1} \PYG{o}{*} \PYG{p}{(}\PYG{n}{V1} \PYG{o}{/} \PYG{n}{V2}\PYG{p}{)} \PYG{o}{*}\PYG{o}{*} \PYG{n}{gamma}

\PYG{c+c1}{\PYGZsh{} 2. Work done (W)}
\PYG{c+c1}{\PYGZsh{} For adiabatic process, W = (P1 * V1 \PYGZhy{} P2 * V2) / (gamma \PYGZhy{} 1)}
\PYG{n}{W} \PYG{o}{=} \PYG{p}{(}\PYG{n}{P1} \PYG{o}{*} \PYG{n}{V1} \PYG{o}{\PYGZhy{}} \PYG{n}{P2} \PYG{o}{*} \PYG{n}{V2}\PYG{p}{)} \PYG{o}{/} \PYG{p}{(}\PYG{n}{gamma} \PYG{o}{\PYGZhy{}} \PYG{l+m+mi}{1}\PYG{p}{)}

\PYG{c+c1}{\PYGZsh{} 3. Change in internal energy (Δu)}
\PYG{c+c1}{\PYGZsh{} Δu = Q \PYGZhy{} W, but for adiabatic process, Q = 0}
\PYG{n}{delta\PYGZus{}u} \PYG{o}{=} \PYG{o}{\PYGZhy{}}\PYG{n}{W}

\PYG{c+c1}{\PYGZsh{} Output the results}
\PYG{n+nb}{print}\PYG{p}{(}\PYG{l+s+sa}{f}\PYG{l+s+s2}{\PYGZdq{}}\PYG{l+s+s2}{Final Temperature (T2): }\PYG{l+s+si}{\PYGZob{}}\PYG{n+nb}{round}\PYG{p}{(}\PYG{n}{T2}\PYG{p}{,}\PYG{l+m+mi}{1}\PYG{p}{)}\PYG{l+s+si}{\PYGZcb{}}\PYG{l+s+s2}{ K}\PYG{l+s+s2}{\PYGZdq{}}\PYG{p}{)}
\PYG{n+nb}{print}\PYG{p}{(}\PYG{l+s+sa}{f}\PYG{l+s+s2}{\PYGZdq{}}\PYG{l+s+s2}{Final Pressure (P2): }\PYG{l+s+si}{\PYGZob{}}\PYG{n+nb}{round}\PYG{p}{(}\PYG{n}{P2}\PYG{o}{/}\PYG{l+m+mf}{1e3}\PYG{p}{,}\PYG{l+m+mi}{1}\PYG{p}{)}\PYG{l+s+si}{\PYGZcb{}}\PYG{l+s+s2}{ kPa}\PYG{l+s+s2}{\PYGZdq{}}\PYG{p}{)}
\PYG{n+nb}{print}\PYG{p}{(}\PYG{l+s+sa}{f}\PYG{l+s+s2}{\PYGZdq{}}\PYG{l+s+s2}{Work done (W): }\PYG{l+s+si}{\PYGZob{}}\PYG{n+nb}{round}\PYG{p}{(}\PYG{n}{W}\PYG{o}{/}\PYG{l+m+mf}{1e3}\PYG{p}{,}\PYG{l+m+mi}{1}\PYG{p}{)}\PYG{l+s+si}{\PYGZcb{}}\PYG{l+s+s2}{ kJ}\PYG{l+s+s2}{\PYGZdq{}}\PYG{p}{)}
\PYG{n+nb}{print}\PYG{p}{(}\PYG{l+s+sa}{f}\PYG{l+s+s2}{\PYGZdq{}}\PYG{l+s+s2}{Change in internal energy (Δu): }\PYG{l+s+si}{\PYGZob{}}\PYG{n+nb}{round}\PYG{p}{(}\PYG{n}{delta\PYGZus{}u}\PYG{o}{/}\PYG{l+m+mf}{1e3}\PYG{p}{,}\PYG{l+m+mi}{1}\PYG{p}{)}\PYG{l+s+si}{\PYGZcb{}}\PYG{l+s+s2}{ kJ}\PYG{l+s+s2}{\PYGZdq{}}\PYG{p}{)}
\end{sphinxVerbatim}

\end{sphinxuseclass}\end{sphinxVerbatimInput}
\begin{sphinxVerbatimOutput}

\begin{sphinxuseclass}{cell_output}
\begin{sphinxVerbatim}[commandchars=\\\{\}]
Final Temperature (T2): 687.5 K
Final Pressure (P2): 1869.1 kPa
Work done (W): \PYGZhy{}329.3 kJ
Change in internal energy (Δu): 329.3 kJ
\end{sphinxVerbatim}

\end{sphinxuseclass}\end{sphinxVerbatimOutput}

\end{sphinxuseclass}
\sphinxstepscope


\section{Heat Transfer in a Rigid Container}
\label{\detokenize{notebooks/Chapter4/Heat_Transfer_Rigid_Container_Problem_5:heat-transfer-in-a-rigid-container}}\label{\detokenize{notebooks/Chapter4/Heat_Transfer_Rigid_Container_Problem_5::doc}}

\subsection{Problem Statement:}
\label{\detokenize{notebooks/Chapter4/Heat_Transfer_Rigid_Container_Problem_5:problem-statement}}
\sphinxAtStartPar
A rigid container is initially filled with 0.5 kg of carbon dioxide at 20°C and 100 kPa.
The container is then heated until the temperature of the gas reaches 80°C.
Calculate the amount of heat transferred to the carbon dioxide and the change in its internal energy.


\subsection{Solution:}
\label{\detokenize{notebooks/Chapter4/Heat_Transfer_Rigid_Container_Problem_5:solution}}
\begin{sphinxuseclass}{cell}\begin{sphinxVerbatimInput}

\begin{sphinxuseclass}{cell_input}
\begin{sphinxVerbatim}[commandchars=\\\{\}]
\PYG{k+kn}{import} \PYG{n+nn}{CoolProp}\PYG{n+nn}{.}\PYG{n+nn}{CoolProp} \PYG{k}{as} \PYG{n+nn}{CP}

\PYG{c+c1}{\PYGZsh{} Given values}
\PYG{n}{m} \PYG{o}{=} \PYG{l+m+mf}{0.5}  \PYG{c+c1}{\PYGZsh{} Mass of carbon dioxide in kg}
\PYG{n}{T1} \PYG{o}{=} \PYG{l+m+mi}{20} \PYG{o}{+} \PYG{l+m+mf}{273.15}  \PYG{c+c1}{\PYGZsh{} Initial temperature in Kelvin}
\PYG{n}{T2} \PYG{o}{=} \PYG{l+m+mi}{80} \PYG{o}{+} \PYG{l+m+mf}{273.15}  \PYG{c+c1}{\PYGZsh{} Final temperature in Kelvin}
\PYG{n}{gas} \PYG{o}{=} \PYG{l+s+s1}{\PYGZsq{}}\PYG{l+s+s1}{CarbonDioxide}\PYG{l+s+s1}{\PYGZsq{}}

\PYG{c+c1}{\PYGZsh{} 1. Change in internal energy (Δu)}
\PYG{n}{Cv} \PYG{o}{=} \PYG{n}{CP}\PYG{o}{.}\PYG{n}{PropsSI}\PYG{p}{(}\PYG{l+s+s1}{\PYGZsq{}}\PYG{l+s+s1}{Cvmass}\PYG{l+s+s1}{\PYGZsq{}}\PYG{p}{,} \PYG{l+s+s1}{\PYGZsq{}}\PYG{l+s+s1}{T}\PYG{l+s+s1}{\PYGZsq{}}\PYG{p}{,} \PYG{n}{T1}\PYG{p}{,} \PYG{l+s+s1}{\PYGZsq{}}\PYG{l+s+s1}{P}\PYG{l+s+s1}{\PYGZsq{}}\PYG{p}{,} \PYG{l+m+mi}{100000}\PYG{p}{,} \PYG{n}{gas}\PYG{p}{)}  \PYG{c+c1}{\PYGZsh{} Cv for carbon dioxide}
\PYG{n}{delta\PYGZus{}u} \PYG{o}{=} \PYG{n}{Cv} \PYG{o}{*} \PYG{p}{(}\PYG{n}{T2} \PYG{o}{\PYGZhy{}} \PYG{n}{T1}\PYG{p}{)} \PYG{o}{*} \PYG{n}{m}

\PYG{c+c1}{\PYGZsh{} Convert delta\PYGZus{}u to kJ and round to one decimal point}
\PYG{n}{delta\PYGZus{}u\PYGZus{}kJ} \PYG{o}{=} \PYG{n+nb}{round}\PYG{p}{(}\PYG{n}{delta\PYGZus{}u} \PYG{o}{/} \PYG{l+m+mi}{1000}\PYG{p}{,} \PYG{l+m+mi}{1}\PYG{p}{)}

\PYG{c+c1}{\PYGZsh{} 2. Total heat transfer (Q)}
\PYG{c+c1}{\PYGZsh{} For a rigid container, Q = Δu (First Law of Thermodynamics)}
\PYG{n}{Q} \PYG{o}{=} \PYG{n}{delta\PYGZus{}u}

\PYG{c+c1}{\PYGZsh{} Convert Q to kJ and round to one decimal point}
\PYG{n}{Q\PYGZus{}kJ} \PYG{o}{=} \PYG{n+nb}{round}\PYG{p}{(}\PYG{n}{Q} \PYG{o}{/} \PYG{l+m+mi}{1000}\PYG{p}{,} \PYG{l+m+mi}{1}\PYG{p}{)}

\PYG{c+c1}{\PYGZsh{} Output the results}
\PYG{n+nb}{print}\PYG{p}{(}\PYG{l+s+sa}{f}\PYG{l+s+s2}{\PYGZdq{}}\PYG{l+s+s2}{Change in internal energy (Δu): }\PYG{l+s+si}{\PYGZob{}}\PYG{n}{delta\PYGZus{}u\PYGZus{}kJ}\PYG{l+s+si}{\PYGZcb{}}\PYG{l+s+s2}{ kJ}\PYG{l+s+s2}{\PYGZdq{}}\PYG{p}{)}
\PYG{n+nb}{print}\PYG{p}{(}\PYG{l+s+sa}{f}\PYG{l+s+s2}{\PYGZdq{}}\PYG{l+s+s2}{Total heat transfer (Q): }\PYG{l+s+si}{\PYGZob{}}\PYG{n}{Q\PYGZus{}kJ}\PYG{l+s+si}{\PYGZcb{}}\PYG{l+s+s2}{ kJ}\PYG{l+s+s2}{\PYGZdq{}}\PYG{p}{)}
\end{sphinxVerbatim}

\end{sphinxuseclass}\end{sphinxVerbatimInput}
\begin{sphinxVerbatimOutput}

\begin{sphinxuseclass}{cell_output}
\begin{sphinxVerbatim}[commandchars=\\\{\}]
Change in internal energy (Δu): 19.6 kJ
Total heat transfer (Q): 19.6 kJ
\end{sphinxVerbatim}

\end{sphinxuseclass}\end{sphinxVerbatimOutput}

\end{sphinxuseclass}
\sphinxstepscope


\section{Helium in a Spring\sphinxhyphen{}Loaded Cylinder}
\label{\detokenize{notebooks/Chapter4/Helium_Spring_Loaded_Cylinder_Problem_6:helium-in-a-spring-loaded-cylinder}}\label{\detokenize{notebooks/Chapter4/Helium_Spring_Loaded_Cylinder_Problem_6::doc}}

\subsection{Problem Statement:}
\label{\detokenize{notebooks/Chapter4/Helium_Spring_Loaded_Cylinder_Problem_6:problem-statement}}
\sphinxAtStartPar
A spring\sphinxhyphen{}loaded piston\sphinxhyphen{}cylinder device contains 0.3 kg of helium.
Initially, the helium is at 25°C and 150 kPa, and the spring is uncompressed.
The system is heated until the pressure doubles and the volume increases by 20\%.
Calculate the work done by the helium on the spring and the total heat added to the system.


\subsection{Solution:}
\label{\detokenize{notebooks/Chapter4/Helium_Spring_Loaded_Cylinder_Problem_6:solution}}
\begin{sphinxuseclass}{cell}\begin{sphinxVerbatimInput}

\begin{sphinxuseclass}{cell_input}
\begin{sphinxVerbatim}[commandchars=\\\{\}]
\PYG{k+kn}{import} \PYG{n+nn}{CoolProp}\PYG{n+nn}{.}\PYG{n+nn}{CoolProp} \PYG{k}{as} \PYG{n+nn}{CP}
\PYG{k+kn}{import} \PYG{n+nn}{math}

\PYG{c+c1}{\PYGZsh{} Given values}
\PYG{n}{m} \PYG{o}{=} \PYG{l+m+mf}{0.3}  \PYG{c+c1}{\PYGZsh{} Mass of helium in kg}
\PYG{n}{T1} \PYG{o}{=} \PYG{l+m+mi}{25} \PYG{o}{+} \PYG{l+m+mf}{273.15}  \PYG{c+c1}{\PYGZsh{} Initial temperature in Kelvin}
\PYG{n}{P1} \PYG{o}{=} \PYG{l+m+mi}{150000}  \PYG{c+c1}{\PYGZsh{} Initial pressure in Pa}

\PYG{c+c1}{\PYGZsh{} Calculate the initial volume using the density}
\PYG{n}{V1} \PYG{o}{=} \PYG{n}{m} \PYG{o}{/} \PYG{n}{CP}\PYG{o}{.}\PYG{n}{PropsSI}\PYG{p}{(}\PYG{l+s+s1}{\PYGZsq{}}\PYG{l+s+s1}{D}\PYG{l+s+s1}{\PYGZsq{}}\PYG{p}{,} \PYG{l+s+s1}{\PYGZsq{}}\PYG{l+s+s1}{T}\PYG{l+s+s1}{\PYGZsq{}}\PYG{p}{,} \PYG{n}{T1}\PYG{p}{,} \PYG{l+s+s1}{\PYGZsq{}}\PYG{l+s+s1}{P}\PYG{l+s+s1}{\PYGZsq{}}\PYG{p}{,} \PYG{n}{P1}\PYG{p}{,} \PYG{l+s+s1}{\PYGZsq{}}\PYG{l+s+s1}{Helium}\PYG{l+s+s1}{\PYGZsq{}}\PYG{p}{)}  
\PYG{n}{V2} \PYG{o}{=} \PYG{n}{V1} \PYG{o}{*} \PYG{l+m+mf}{1.2}  \PYG{c+c1}{\PYGZsh{} Final volume (20\PYGZpc{} increase)}
\PYG{n}{P2} \PYG{o}{=} \PYG{l+m+mi}{2} \PYG{o}{*} \PYG{n}{P1}  \PYG{c+c1}{\PYGZsh{} Final pressure (doubled)}

\PYG{c+c1}{\PYGZsh{} Polytropic exponent for helium}
\PYG{n}{n} \PYG{o}{=} \PYG{n}{CP}\PYG{o}{.}\PYG{n}{PropsSI}\PYG{p}{(}\PYG{l+s+s1}{\PYGZsq{}}\PYG{l+s+s1}{Cpmass}\PYG{l+s+s1}{\PYGZsq{}}\PYG{p}{,} \PYG{l+s+s1}{\PYGZsq{}}\PYG{l+s+s1}{T}\PYG{l+s+s1}{\PYGZsq{}}\PYG{p}{,} \PYG{n}{T1}\PYG{p}{,} \PYG{l+s+s1}{\PYGZsq{}}\PYG{l+s+s1}{P}\PYG{l+s+s1}{\PYGZsq{}}\PYG{p}{,} \PYG{n}{P1}\PYG{p}{,} \PYG{l+s+s1}{\PYGZsq{}}\PYG{l+s+s1}{Helium}\PYG{l+s+s1}{\PYGZsq{}}\PYG{p}{)} \PYG{o}{/} \PYG{n}{CP}\PYG{o}{.}\PYG{n}{PropsSI}\PYG{p}{(}\PYG{l+s+s1}{\PYGZsq{}}\PYG{l+s+s1}{Cvmass}\PYG{l+s+s1}{\PYGZsq{}}\PYG{p}{,} \PYG{l+s+s1}{\PYGZsq{}}\PYG{l+s+s1}{T}\PYG{l+s+s1}{\PYGZsq{}}\PYG{p}{,} \PYG{n}{T1}\PYG{p}{,} \PYG{l+s+s1}{\PYGZsq{}}\PYG{l+s+s1}{P}\PYG{l+s+s1}{\PYGZsq{}}\PYG{p}{,} \PYG{n}{P1}\PYG{p}{,} \PYG{l+s+s1}{\PYGZsq{}}\PYG{l+s+s1}{Helium}\PYG{l+s+s1}{\PYGZsq{}}\PYG{p}{)}

\PYG{c+c1}{\PYGZsh{} Calculate final temperature T2 using the polytropic process relation}
\PYG{n}{T2} \PYG{o}{=} \PYG{n}{T1} \PYG{o}{*} \PYG{p}{(}\PYG{n}{P2} \PYG{o}{/} \PYG{n}{P1}\PYG{p}{)} \PYG{o}{*} \PYG{p}{(}\PYG{n}{V1} \PYG{o}{/} \PYG{n}{V2}\PYG{p}{)}\PYG{o}{*}\PYG{o}{*}\PYG{n}{n}

\PYG{c+c1}{\PYGZsh{} Calculate work done by the helium (W)}
\PYG{k}{if} \PYG{n}{n} \PYG{o}{!=} \PYG{l+m+mi}{1}\PYG{p}{:}
    \PYG{n}{W} \PYG{o}{=} \PYG{p}{(}\PYG{n}{P2} \PYG{o}{*} \PYG{n}{V2} \PYG{o}{\PYGZhy{}} \PYG{n}{P1} \PYG{o}{*} \PYG{n}{V1}\PYG{p}{)} \PYG{o}{/} \PYG{p}{(}\PYG{l+m+mi}{1} \PYG{o}{\PYGZhy{}} \PYG{n}{n}\PYG{p}{)}
\PYG{k}{else}\PYG{p}{:}
    \PYG{n}{W} \PYG{o}{=} \PYG{n}{P1} \PYG{o}{*} \PYG{n}{V1} \PYG{o}{*} \PYG{n}{math}\PYG{o}{.}\PYG{n}{log}\PYG{p}{(}\PYG{n}{V2} \PYG{o}{/} \PYG{n}{V1}\PYG{p}{)}

\PYG{c+c1}{\PYGZsh{} Convert W to kJ and round to 1 decimal place}
\PYG{n}{W\PYGZus{}kJ} \PYG{o}{=} \PYG{n+nb}{round}\PYG{p}{(}\PYG{n}{W} \PYG{o}{/} \PYG{l+m+mi}{1000}\PYG{p}{,} \PYG{l+m+mi}{1}\PYG{p}{)}

\PYG{c+c1}{\PYGZsh{} Calculate the total heat added (Q)}
\PYG{n}{Cv} \PYG{o}{=} \PYG{n}{CP}\PYG{o}{.}\PYG{n}{PropsSI}\PYG{p}{(}\PYG{l+s+s1}{\PYGZsq{}}\PYG{l+s+s1}{Cvmass}\PYG{l+s+s1}{\PYGZsq{}}\PYG{p}{,} \PYG{l+s+s1}{\PYGZsq{}}\PYG{l+s+s1}{T}\PYG{l+s+s1}{\PYGZsq{}}\PYG{p}{,} \PYG{n}{T1}\PYG{p}{,} \PYG{l+s+s1}{\PYGZsq{}}\PYG{l+s+s1}{P}\PYG{l+s+s1}{\PYGZsq{}}\PYG{p}{,} \PYG{n}{P1}\PYG{p}{,} \PYG{l+s+s1}{\PYGZsq{}}\PYG{l+s+s1}{Helium}\PYG{l+s+s1}{\PYGZsq{}}\PYG{p}{)}
\PYG{n}{delta\PYGZus{}U} \PYG{o}{=} \PYG{n}{Cv} \PYG{o}{*} \PYG{p}{(}\PYG{n}{T2} \PYG{o}{\PYGZhy{}} \PYG{n}{T1}\PYG{p}{)} \PYG{o}{*} \PYG{n}{m}

\PYG{c+c1}{\PYGZsh{} Convert delta\PYGZus{}U to kJ and round to 1 decimal place}
\PYG{n}{delta\PYGZus{}U\PYGZus{}kJ} \PYG{o}{=} \PYG{n+nb}{round}\PYG{p}{(}\PYG{n}{delta\PYGZus{}U} \PYG{o}{/} \PYG{l+m+mi}{1000}\PYG{p}{,} \PYG{l+m+mi}{1}\PYG{p}{)}

\PYG{c+c1}{\PYGZsh{} Q is the sum of delta\PYGZus{}U and W, converted to kJ and rounded}
\PYG{n}{Q\PYGZus{}kJ} \PYG{o}{=} \PYG{n+nb}{round}\PYG{p}{(}\PYG{p}{(}\PYG{n}{delta\PYGZus{}U} \PYG{o}{+} \PYG{n}{W}\PYG{p}{)} \PYG{o}{/} \PYG{l+m+mi}{1000}\PYG{p}{,} \PYG{l+m+mi}{1}\PYG{p}{)}

\PYG{c+c1}{\PYGZsh{} Output the results}
\PYG{n+nb}{print}\PYG{p}{(}\PYG{l+s+sa}{f}\PYG{l+s+s2}{\PYGZdq{}}\PYG{l+s+s2}{Final Temperature (T2): }\PYG{l+s+si}{\PYGZob{}}\PYG{n+nb}{round}\PYG{p}{(}\PYG{n}{T2}\PYG{p}{,}\PYG{+w}{ }\PYG{l+m+mi}{1}\PYG{p}{)}\PYG{l+s+si}{\PYGZcb{}}\PYG{l+s+s2}{ K}\PYG{l+s+s2}{\PYGZdq{}}\PYG{p}{)}
\PYG{n+nb}{print}\PYG{p}{(}\PYG{l+s+sa}{f}\PYG{l+s+s2}{\PYGZdq{}}\PYG{l+s+s2}{Work done by the helium (W): }\PYG{l+s+si}{\PYGZob{}}\PYG{n}{W\PYGZus{}kJ}\PYG{l+s+si}{\PYGZcb{}}\PYG{l+s+s2}{ kJ}\PYG{l+s+s2}{\PYGZdq{}}\PYG{p}{)}
\PYG{n+nb}{print}\PYG{p}{(}\PYG{l+s+sa}{f}\PYG{l+s+s2}{\PYGZdq{}}\PYG{l+s+s2}{Total heat added (Q): }\PYG{l+s+si}{\PYGZob{}}\PYG{n}{Q\PYGZus{}kJ}\PYG{l+s+si}{\PYGZcb{}}\PYG{l+s+s2}{ kJ}\PYG{l+s+s2}{\PYGZdq{}}\PYG{p}{)}
\PYG{n+nb}{print}\PYG{p}{(}\PYG{l+s+sa}{f}\PYG{l+s+s2}{\PYGZdq{}}\PYG{l+s+s2}{Change in Internal Energy (ΔU): }\PYG{l+s+si}{\PYGZob{}}\PYG{n}{delta\PYGZus{}U\PYGZus{}kJ}\PYG{l+s+si}{\PYGZcb{}}\PYG{l+s+s2}{ kJ}\PYG{l+s+s2}{\PYGZdq{}}\PYG{p}{)}
\end{sphinxVerbatim}

\end{sphinxuseclass}\end{sphinxVerbatimInput}
\begin{sphinxVerbatimOutput}

\begin{sphinxuseclass}{cell_output}
\begin{sphinxVerbatim}[commandchars=\\\{\}]
Final Temperature (T2): 440.1 K
Work done by the helium (W): \PYGZhy{}390.6 kJ
Total heat added (Q): \PYGZhy{}257.9 kJ
Change in Internal Energy (ΔU): 132.7 kJ
\end{sphinxVerbatim}

\end{sphinxuseclass}\end{sphinxVerbatimOutput}

\end{sphinxuseclass}
\sphinxstepscope


\section{Isothermal Expansion of Nitrogen}
\label{\detokenize{notebooks/Chapter4/Isothermal_Expansion_Nitrogen_Problem_7:isothermal-expansion-of-nitrogen}}\label{\detokenize{notebooks/Chapter4/Isothermal_Expansion_Nitrogen_Problem_7::doc}}

\subsection{Problem Statement:}
\label{\detokenize{notebooks/Chapter4/Isothermal_Expansion_Nitrogen_Problem_7:problem-statement}}
\sphinxAtStartPar
A cylinder with a movable piston contains 1 kg of nitrogen at 100 kPa and 300 K.
It undergoes an isothermal expansion until the volume triples.
Calculate the boundary work done during this process and the heat transfer involved.


\subsection{Solution:}
\label{\detokenize{notebooks/Chapter4/Isothermal_Expansion_Nitrogen_Problem_7:solution}}
\begin{sphinxuseclass}{cell}\begin{sphinxVerbatimInput}

\begin{sphinxuseclass}{cell_input}
\begin{sphinxVerbatim}[commandchars=\\\{\}]
\PYG{k+kn}{import} \PYG{n+nn}{CoolProp}\PYG{n+nn}{.}\PYG{n+nn}{CoolProp} \PYG{k}{as} \PYG{n+nn}{CP}
\PYG{k+kn}{import} \PYG{n+nn}{math}

\PYG{c+c1}{\PYGZsh{} Given values}
\PYG{n}{m} \PYG{o}{=} \PYG{l+m+mf}{1.0}  \PYG{c+c1}{\PYGZsh{} Mass of nitrogen in kg}
\PYG{n}{T} \PYG{o}{=} \PYG{l+m+mi}{300}  \PYG{c+c1}{\PYGZsh{} Temperature in Kelvin (constant)}
\PYG{n}{P1} \PYG{o}{=} \PYG{l+m+mi}{100000}  \PYG{c+c1}{\PYGZsh{} Initial pressure in Pa}
\PYG{n}{V1} \PYG{o}{=} \PYG{n}{m} \PYG{o}{*} \PYG{n}{CP}\PYG{o}{.}\PYG{n}{PropsSI}\PYG{p}{(}\PYG{l+s+s1}{\PYGZsq{}}\PYG{l+s+s1}{Dmolar}\PYG{l+s+s1}{\PYGZsq{}}\PYG{p}{,} \PYG{l+s+s1}{\PYGZsq{}}\PYG{l+s+s1}{T}\PYG{l+s+s1}{\PYGZsq{}}\PYG{p}{,} \PYG{n}{T}\PYG{p}{,} \PYG{l+s+s1}{\PYGZsq{}}\PYG{l+s+s1}{P}\PYG{l+s+s1}{\PYGZsq{}}\PYG{p}{,} \PYG{n}{P1}\PYG{p}{,} \PYG{l+s+s1}{\PYGZsq{}}\PYG{l+s+s1}{Nitrogen}\PYG{l+s+s1}{\PYGZsq{}}\PYG{p}{)}  \PYG{c+c1}{\PYGZsh{} Initial volume using density}
\PYG{n}{V2} \PYG{o}{=} \PYG{l+m+mi}{3} \PYG{o}{*} \PYG{n}{V1}  \PYG{c+c1}{\PYGZsh{} Final volume (tripled)}

\PYG{c+c1}{\PYGZsh{} 1. Boundary work (W)}
\PYG{c+c1}{\PYGZsh{} For isothermal process, W = nRT ln(V2/V1)}
\PYG{n}{R} \PYG{o}{=} \PYG{n}{CP}\PYG{o}{.}\PYG{n}{PropsSI}\PYG{p}{(}\PYG{l+s+s1}{\PYGZsq{}}\PYG{l+s+s1}{GAS\PYGZus{}CONSTANT}\PYG{l+s+s1}{\PYGZsq{}}\PYG{p}{,} \PYG{l+s+s1}{\PYGZsq{}}\PYG{l+s+s1}{Nitrogen}\PYG{l+s+s1}{\PYGZsq{}}\PYG{p}{)} \PYG{o}{/} \PYG{n}{CP}\PYG{o}{.}\PYG{n}{PropsSI}\PYG{p}{(}\PYG{l+s+s1}{\PYGZsq{}}\PYG{l+s+s1}{MOLAR\PYGZus{}MASS}\PYG{l+s+s1}{\PYGZsq{}}\PYG{p}{,} \PYG{l+s+s1}{\PYGZsq{}}\PYG{l+s+s1}{Nitrogen}\PYG{l+s+s1}{\PYGZsq{}}\PYG{p}{)}
\PYG{n}{W} \PYG{o}{=} \PYG{n}{m} \PYG{o}{*} \PYG{n}{R} \PYG{o}{*} \PYG{n}{T} \PYG{o}{*} \PYG{n}{math}\PYG{o}{.}\PYG{n}{log}\PYG{p}{(}\PYG{n}{V2} \PYG{o}{/} \PYG{n}{V1}\PYG{p}{)}

\PYG{c+c1}{\PYGZsh{} 2. Heat transfer (Q)}
\PYG{c+c1}{\PYGZsh{} For an isothermal process, Q = W}
\PYG{n}{Q} \PYG{o}{=} \PYG{n}{W}

\PYG{c+c1}{\PYGZsh{} Output the results}
\PYG{n+nb}{print}\PYG{p}{(}\PYG{l+s+sa}{f}\PYG{l+s+s2}{\PYGZdq{}}\PYG{l+s+s2}{Boundary work done (W): }\PYG{l+s+si}{\PYGZob{}}\PYG{n+nb}{round}\PYG{p}{(}\PYG{n}{W}\PYG{o}{/}\PYG{l+m+mf}{1e3}\PYG{p}{,}\PYG{l+m+mi}{1}\PYG{p}{)}\PYG{l+s+si}{\PYGZcb{}}\PYG{l+s+s2}{ kJ}\PYG{l+s+s2}{\PYGZdq{}}\PYG{p}{)}
\PYG{n+nb}{print}\PYG{p}{(}\PYG{l+s+sa}{f}\PYG{l+s+s2}{\PYGZdq{}}\PYG{l+s+s2}{Heat transfer (Q): }\PYG{l+s+si}{\PYGZob{}}\PYG{n+nb}{round}\PYG{p}{(}\PYG{n}{Q}\PYG{o}{/}\PYG{l+m+mf}{1e3}\PYG{p}{,}\PYG{l+m+mi}{1}\PYG{p}{)}\PYG{l+s+si}{\PYGZcb{}}\PYG{l+s+s2}{ kJ}\PYG{l+s+s2}{\PYGZdq{}}\PYG{p}{)}
\end{sphinxVerbatim}

\end{sphinxuseclass}\end{sphinxVerbatimInput}
\begin{sphinxVerbatimOutput}

\begin{sphinxuseclass}{cell_output}
\begin{sphinxVerbatim}[commandchars=\\\{\}]
Boundary work done (W): 97.8 kJ
Heat transfer (Q): 97.8 kJ
\end{sphinxVerbatim}

\end{sphinxuseclass}\end{sphinxVerbatimOutput}

\end{sphinxuseclass}
\sphinxstepscope


\section{Adiabatic Compression in a Rigid Container}
\label{\detokenize{notebooks/Chapter4/Adiabatic_Compression_Rigid_Container_Problem_8:adiabatic-compression-in-a-rigid-container}}\label{\detokenize{notebooks/Chapter4/Adiabatic_Compression_Rigid_Container_Problem_8::doc}}

\subsection{Problem Statement:}
\label{\detokenize{notebooks/Chapter4/Adiabatic_Compression_Rigid_Container_Problem_8:problem-statement}}
\sphinxAtStartPar
A rigid container with 2 kg of oxygen is initially at 1 atm and 25°C.
The oxygen is compressed adiabatically until the pressure increases to 5 atm.
Calculate the final temperature, the work done on the gas, and the change in internal energy.

\begin{sphinxuseclass}{cell}\begin{sphinxVerbatimInput}

\begin{sphinxuseclass}{cell_input}
\begin{sphinxVerbatim}[commandchars=\\\{\}]
\PYG{k+kn}{import} \PYG{n+nn}{CoolProp}\PYG{n+nn}{.}\PYG{n+nn}{CoolProp} \PYG{k}{as} \PYG{n+nn}{CP}
\PYG{k+kn}{import} \PYG{n+nn}{math}

\PYG{c+c1}{\PYGZsh{} Given values}
\PYG{n}{m} \PYG{o}{=} \PYG{l+m+mf}{2.0}  \PYG{c+c1}{\PYGZsh{} Mass of oxygen in kg}
\PYG{n}{T1} \PYG{o}{=} \PYG{l+m+mi}{25} \PYG{o}{+} \PYG{l+m+mf}{273.15}  \PYG{c+c1}{\PYGZsh{} Initial temperature in Kelvin}
\PYG{n}{P1} \PYG{o}{=} \PYG{l+m+mi}{101325}  \PYG{c+c1}{\PYGZsh{} Initial pressure in Pa (1 atm)}
\PYG{n}{P2} \PYG{o}{=} \PYG{l+m+mi}{5} \PYG{o}{*} \PYG{n}{P1}  \PYG{c+c1}{\PYGZsh{} Final pressure (5 atm)}
\PYG{n}{gamma} \PYG{o}{=} \PYG{n}{CP}\PYG{o}{.}\PYG{n}{PropsSI}\PYG{p}{(}\PYG{l+s+s1}{\PYGZsq{}}\PYG{l+s+s1}{Cpmass}\PYG{l+s+s1}{\PYGZsq{}}\PYG{p}{,} \PYG{l+s+s1}{\PYGZsq{}}\PYG{l+s+s1}{T}\PYG{l+s+s1}{\PYGZsq{}}\PYG{p}{,} \PYG{n}{T1}\PYG{p}{,} \PYG{l+s+s1}{\PYGZsq{}}\PYG{l+s+s1}{P}\PYG{l+s+s1}{\PYGZsq{}}\PYG{p}{,} \PYG{n}{P1}\PYG{p}{,} \PYG{l+s+s1}{\PYGZsq{}}\PYG{l+s+s1}{Oxygen}\PYG{l+s+s1}{\PYGZsq{}}\PYG{p}{)} \PYG{o}{/} \PYG{n}{CP}\PYG{o}{.}\PYG{n}{PropsSI}\PYG{p}{(}\PYG{l+s+s1}{\PYGZsq{}}\PYG{l+s+s1}{Cvmass}\PYG{l+s+s1}{\PYGZsq{}}\PYG{p}{,} \PYG{l+s+s1}{\PYGZsq{}}\PYG{l+s+s1}{T}\PYG{l+s+s1}{\PYGZsq{}}\PYG{p}{,} \PYG{n}{T1}\PYG{p}{,} \PYG{l+s+s1}{\PYGZsq{}}\PYG{l+s+s1}{P}\PYG{l+s+s1}{\PYGZsq{}}\PYG{p}{,} \PYG{n}{P1}\PYG{p}{,} \PYG{l+s+s1}{\PYGZsq{}}\PYG{l+s+s1}{Oxygen}\PYG{l+s+s1}{\PYGZsq{}}\PYG{p}{)}  \PYG{c+c1}{\PYGZsh{} γ for oxygen}

\PYG{c+c1}{\PYGZsh{} Final Temperature (T2) for adiabatic process}
\PYG{n}{T2} \PYG{o}{=} \PYG{n}{T1} \PYG{o}{*} \PYG{p}{(}\PYG{n}{P2} \PYG{o}{/} \PYG{n}{P1}\PYG{p}{)} \PYG{o}{*}\PYG{o}{*} \PYG{p}{(}\PYG{p}{(}\PYG{n}{gamma} \PYG{o}{\PYGZhy{}} \PYG{l+m+mi}{1}\PYG{p}{)} \PYG{o}{/} \PYG{n}{gamma}\PYG{p}{)}

\PYG{c+c1}{\PYGZsh{} Calculate specific volumes V1 and V2 using ideal gas law}
\PYG{n}{R} \PYG{o}{=} \PYG{n}{CP}\PYG{o}{.}\PYG{n}{PropsSI}\PYG{p}{(}\PYG{l+s+s1}{\PYGZsq{}}\PYG{l+s+s1}{GAS\PYGZus{}CONSTANT}\PYG{l+s+s1}{\PYGZsq{}}\PYG{p}{,} \PYG{l+s+s1}{\PYGZsq{}}\PYG{l+s+s1}{Oxygen}\PYG{l+s+s1}{\PYGZsq{}}\PYG{p}{)} \PYG{o}{/} \PYG{n}{CP}\PYG{o}{.}\PYG{n}{PropsSI}\PYG{p}{(}\PYG{l+s+s1}{\PYGZsq{}}\PYG{l+s+s1}{MOLAR\PYGZus{}MASS}\PYG{l+s+s1}{\PYGZsq{}}\PYG{p}{,} \PYG{l+s+s1}{\PYGZsq{}}\PYG{l+s+s1}{Oxygen}\PYG{l+s+s1}{\PYGZsq{}}\PYG{p}{)}  \PYG{c+c1}{\PYGZsh{} Specific gas constant for Oxygen}
\PYG{n}{V1} \PYG{o}{=} \PYG{n}{m} \PYG{o}{*} \PYG{n}{R} \PYG{o}{*} \PYG{n}{T1} \PYG{o}{/} \PYG{n}{P1}
\PYG{n}{V2} \PYG{o}{=} \PYG{n}{m} \PYG{o}{*} \PYG{n}{R} \PYG{o}{*} \PYG{n}{T2} \PYG{o}{/} \PYG{n}{P2}

\PYG{c+c1}{\PYGZsh{} Calculate work done (W) during adiabatic process}
\PYG{n}{W} \PYG{o}{=} \PYG{p}{(}\PYG{n}{P1} \PYG{o}{*} \PYG{n}{V1} \PYG{o}{\PYGZhy{}} \PYG{n}{P2} \PYG{o}{*} \PYG{n}{V2}\PYG{p}{)} \PYG{o}{/} \PYG{p}{(}\PYG{n}{gamma} \PYG{o}{\PYGZhy{}} \PYG{l+m+mi}{1}\PYG{p}{)}

\PYG{c+c1}{\PYGZsh{} Change in internal energy (ΔU) for adiabatic process}
\PYG{n}{delta\PYGZus{}U} \PYG{o}{=} \PYG{o}{\PYGZhy{}}\PYG{n}{W}

\PYG{c+c1}{\PYGZsh{} Output the results (W and ΔU in kJ)}
\PYG{n+nb}{print}\PYG{p}{(}\PYG{l+s+sa}{f}\PYG{l+s+s2}{\PYGZdq{}}\PYG{l+s+s2}{Final Temperature (T2): }\PYG{l+s+si}{\PYGZob{}}\PYG{n+nb}{round}\PYG{p}{(}\PYG{n}{T2}\PYG{p}{)}\PYG{l+s+si}{\PYGZcb{}}\PYG{l+s+s2}{ K}\PYG{l+s+s2}{\PYGZdq{}}\PYG{p}{)}
\PYG{n+nb}{print}\PYG{p}{(}\PYG{l+s+sa}{f}\PYG{l+s+s2}{\PYGZdq{}}\PYG{l+s+s2}{Work done (W): }\PYG{l+s+si}{\PYGZob{}}\PYG{n+nb}{round}\PYG{p}{(}\PYG{n}{W}\PYG{o}{/}\PYG{l+m+mi}{1000}\PYG{p}{,}\PYG{l+m+mi}{1}\PYG{p}{)}\PYG{l+s+si}{\PYGZcb{}}\PYG{l+s+s2}{ kJ}\PYG{l+s+s2}{\PYGZdq{}}\PYG{p}{)}
\PYG{n+nb}{print}\PYG{p}{(}\PYG{l+s+sa}{f}\PYG{l+s+s2}{\PYGZdq{}}\PYG{l+s+s2}{Change in internal energy (ΔU): }\PYG{l+s+si}{\PYGZob{}}\PYG{n+nb}{round}\PYG{p}{(}\PYG{n}{delta\PYGZus{}U}\PYG{o}{/}\PYG{l+m+mf}{1e3}\PYG{p}{,}\PYG{l+m+mi}{1}\PYG{p}{)}\PYG{l+s+si}{\PYGZcb{}}\PYG{l+s+s2}{ kJ}\PYG{l+s+s2}{\PYGZdq{}}\PYG{p}{)}
\end{sphinxVerbatim}

\end{sphinxuseclass}\end{sphinxVerbatimInput}
\begin{sphinxVerbatimOutput}

\begin{sphinxuseclass}{cell_output}
\begin{sphinxVerbatim}[commandchars=\\\{\}]
Final Temperature (T2): 471 K
Work done (W): \PYGZhy{}226.3 kJ
Change in internal energy (ΔU): 226.3 kJ
\end{sphinxVerbatim}

\end{sphinxuseclass}\end{sphinxVerbatimOutput}

\end{sphinxuseclass}
\begin{sphinxuseclass}{cell}\begin{sphinxVerbatimInput}

\begin{sphinxuseclass}{cell_input}
\begin{sphinxVerbatim}[commandchars=\\\{\}]
\PYG{n}{fluid} \PYG{o}{=} \PYG{l+s+s1}{\PYGZsq{}}\PYG{l+s+s1}{Oxygen}\PYG{l+s+s1}{\PYGZsq{}}
\PYG{k+kn}{import} \PYG{n+nn}{matplotlib}\PYG{n+nn}{.}\PYG{n+nn}{pyplot} \PYG{k}{as} \PYG{n+nn}{plt}
\PYG{c+c1}{\PYGZsh{} Calculate entropy at initial and final states}
\PYG{n}{S1} \PYG{o}{=} \PYG{n}{CP}\PYG{o}{.}\PYG{n}{PropsSI}\PYG{p}{(}\PYG{l+s+s1}{\PYGZsq{}}\PYG{l+s+s1}{S}\PYG{l+s+s1}{\PYGZsq{}}\PYG{p}{,} \PYG{l+s+s1}{\PYGZsq{}}\PYG{l+s+s1}{T}\PYG{l+s+s1}{\PYGZsq{}}\PYG{p}{,} \PYG{n}{T1}\PYG{p}{,} \PYG{l+s+s1}{\PYGZsq{}}\PYG{l+s+s1}{P}\PYG{l+s+s1}{\PYGZsq{}}\PYG{p}{,} \PYG{n}{P1}\PYG{p}{,} \PYG{n}{fluid}\PYG{p}{)}
\PYG{n}{S2} \PYG{o}{=} \PYG{n}{CP}\PYG{o}{.}\PYG{n}{PropsSI}\PYG{p}{(}\PYG{l+s+s1}{\PYGZsq{}}\PYG{l+s+s1}{S}\PYG{l+s+s1}{\PYGZsq{}}\PYG{p}{,} \PYG{l+s+s1}{\PYGZsq{}}\PYG{l+s+s1}{T}\PYG{l+s+s1}{\PYGZsq{}}\PYG{p}{,} \PYG{n}{T2}\PYG{p}{,} \PYG{l+s+s1}{\PYGZsq{}}\PYG{l+s+s1}{P}\PYG{l+s+s1}{\PYGZsq{}}\PYG{p}{,} \PYG{n}{P2}\PYG{p}{,} \PYG{n}{fluid}\PYG{p}{)}
\PYG{n}{plt}\PYG{o}{.}\PYG{n}{plot}\PYG{p}{(}\PYG{p}{[}\PYG{n}{S1} \PYG{o}{/} \PYG{l+m+mi}{1000}\PYG{p}{,} \PYG{n}{S2} \PYG{o}{/} \PYG{l+m+mi}{1000}\PYG{p}{]}\PYG{p}{,} \PYG{p}{[}\PYG{n}{T1}\PYG{p}{,} \PYG{n}{T2}\PYG{p}{]}\PYG{p}{,} \PYG{l+s+s1}{\PYGZsq{}}\PYG{l+s+s1}{ro\PYGZhy{}}\PYG{l+s+s1}{\PYGZsq{}}\PYG{p}{)}  \PYG{c+c1}{\PYGZsh{} Adiabatic process}
\PYG{n}{plt}\PYG{o}{.}\PYG{n}{xlabel}\PYG{p}{(}\PYG{l+s+s1}{\PYGZsq{}}\PYG{l+s+s1}{Entropy (kJ/kg.K)}\PYG{l+s+s1}{\PYGZsq{}}\PYG{p}{)}
\PYG{n}{plt}\PYG{o}{.}\PYG{n}{ylabel}\PYG{p}{(}\PYG{l+s+s1}{\PYGZsq{}}\PYG{l+s+s1}{Temperature (K)}\PYG{l+s+s1}{\PYGZsq{}}\PYG{p}{)}
\PYG{n}{plt}\PYG{o}{.}\PYG{n}{title}\PYG{p}{(}\PYG{l+s+s1}{\PYGZsq{}}\PYG{l+s+s1}{T\PYGZhy{}s Diagram for Adiabatic Compression of Oxygen}\PYG{l+s+s1}{\PYGZsq{}}\PYG{p}{)}
\PYG{n}{plt}\PYG{o}{.}\PYG{n}{grid}\PYG{p}{(}\PYG{k+kc}{True}\PYG{p}{)}
\PYG{n}{plt}\PYG{o}{.}\PYG{n}{show}\PYG{p}{(}\PYG{p}{)}
\end{sphinxVerbatim}

\end{sphinxuseclass}\end{sphinxVerbatimInput}
\begin{sphinxVerbatimOutput}

\begin{sphinxuseclass}{cell_output}
\noindent\sphinxincludegraphics{{fa3eb7720e16c3b8f97c881e59b6c479bfc356023478a3a4456fd8b52ef49632}.png}

\end{sphinxuseclass}\end{sphinxVerbatimOutput}

\end{sphinxuseclass}
\sphinxstepscope


\section{Linear interpolation for Internal energy of Superheated water}
\label{\detokenize{notebooks/Chapter4/linear-interpolation-nternal-energy:linear-interpolation-for-internal-energy-of-superheated-water}}\label{\detokenize{notebooks/Chapter4/linear-interpolation-nternal-energy::doc}}
\sphinxAtStartPar
A function named “linear\_interpolation” is defined, arguments of the same are T1, T2 (the two ends of the temperatures), T (the temperature at which a property needs to be interpolated) and Prop1, Prop2 are the proeprty values at T1 and T2.

\begin{sphinxuseclass}{cell}\begin{sphinxVerbatimInput}

\begin{sphinxuseclass}{cell_input}
\begin{sphinxVerbatim}[commandchars=\\\{\}]
\PYG{k}{def} \PYG{n+nf}{linear\PYGZus{}interpolation}\PYG{p}{(}\PYG{n}{x}\PYG{p}{,} \PYG{n}{x1}\PYG{p}{,} \PYG{n}{x2}\PYG{p}{,} \PYG{n}{y1}\PYG{p}{,} \PYG{n}{y2}\PYG{p}{)}\PYG{p}{:}
    \PYG{c+c1}{\PYGZsh{} Function to interpolate between two known points}
    \PYG{k}{return} \PYG{n}{y1} \PYG{o}{+} \PYG{p}{(}\PYG{n}{x} \PYG{o}{\PYGZhy{}} \PYG{n}{x1}\PYG{p}{)} \PYG{o}{/} \PYG{p}{(}\PYG{n}{x2} \PYG{o}{\PYGZhy{}} \PYG{n}{x1}\PYG{p}{)} \PYG{o}{*} \PYG{p}{(}\PYG{n}{y2} \PYG{o}{\PYGZhy{}} \PYG{n}{y1}\PYG{p}{)}
\end{sphinxVerbatim}

\end{sphinxuseclass}\end{sphinxVerbatimInput}

\end{sphinxuseclass}
\sphinxAtStartPar
A function named “calculate\_relative\_error” is defined, arguments of the same are x1, x2 (the two ends of the input variable), x (the x\sphinxhyphen{}value at which a property needs to be interpolated) and y1, y2 are the property values at x1 and x2.

\begin{sphinxuseclass}{cell}\begin{sphinxVerbatimInput}

\begin{sphinxuseclass}{cell_input}
\begin{sphinxVerbatim}[commandchars=\\\{\}]
\PYG{k}{def} \PYG{n+nf}{calculate\PYGZus{}relative\PYGZus{}error}\PYG{p}{(}\PYG{n}{x}\PYG{p}{,} \PYG{n}{x1}\PYG{p}{,} \PYG{n}{x2}\PYG{p}{,} \PYG{n}{y1}\PYG{p}{,} \PYG{n}{y2}\PYG{p}{,} \PYG{n}{fluid}\PYG{p}{)}\PYG{p}{:}
    \PYG{c+c1}{\PYGZsh{} Calculate the interpolated value}
    \PYG{n}{y\PYGZus{}interpolated} \PYG{o}{=} \PYG{n}{linear\PYGZus{}interpolation}\PYG{p}{(}\PYG{n}{x}\PYG{p}{,} \PYG{n}{x1}\PYG{p}{,} \PYG{n}{x2}\PYG{p}{,} \PYG{n}{y1}\PYG{p}{,} \PYG{n}{y2}\PYG{p}{)}
    
    \PYG{c+c1}{\PYGZsh{} Get the value from CoolProp}
    \PYG{n}{y\PYGZus{}coolprop} \PYG{o}{=} \PYG{n}{CP}\PYG{o}{.}\PYG{n}{PropsSI}\PYG{p}{(}\PYG{l+s+s2}{\PYGZdq{}}\PYG{l+s+s2}{U}\PYG{l+s+s2}{\PYGZdq{}}\PYG{p}{,} \PYG{l+s+s2}{\PYGZdq{}}\PYG{l+s+s2}{P}\PYG{l+s+s2}{\PYGZdq{}}\PYG{p}{,} \PYG{n}{P}\PYG{p}{,} \PYG{l+s+s2}{\PYGZdq{}}\PYG{l+s+s2}{T}\PYG{l+s+s2}{\PYGZdq{}}\PYG{p}{,} \PYG{n}{x}\PYG{p}{,} \PYG{n}{fluid}\PYG{p}{)} \PYG{o}{/} \PYG{l+m+mf}{1e3}  \PYG{c+c1}{\PYGZsh{} Convert from J/kg to kJ/kg}
    
    \PYG{c+c1}{\PYGZsh{} Calculate absolute and relative errors}
    \PYG{n}{absolute\PYGZus{}error} \PYG{o}{=} \PYG{n+nb}{abs}\PYG{p}{(}\PYG{n}{y\PYGZus{}coolprop} \PYG{o}{\PYGZhy{}} \PYG{n}{y\PYGZus{}interpolated}\PYG{p}{)}
    \PYG{n}{relative\PYGZus{}error} \PYG{o}{=} \PYG{p}{(}\PYG{n}{absolute\PYGZus{}error} \PYG{o}{/} \PYG{n}{y\PYGZus{}coolprop}\PYG{p}{)} \PYG{o}{*} \PYG{l+m+mi}{100}
    
    \PYG{k}{return} \PYG{n}{relative\PYGZus{}error}
\end{sphinxVerbatim}

\end{sphinxuseclass}\end{sphinxVerbatimInput}

\end{sphinxuseclass}
\begin{sphinxuseclass}{cell}\begin{sphinxVerbatimInput}

\begin{sphinxuseclass}{cell_input}
\begin{sphinxVerbatim}[commandchars=\\\{\}]
\PYG{k+kn}{import} \PYG{n+nn}{CoolProp}\PYG{n+nn}{.}\PYG{n+nn}{CoolProp} \PYG{k}{as} \PYG{n+nn}{CP}
\PYG{c+c1}{\PYGZsh{} Example usage from Superheated water:}
\PYG{c+c1}{\PYGZsh{} https://pressbooks.bccampus.ca/thermo1/back\PYGZhy{}matter/thermodynamic\PYGZhy{}properties\PYGZhy{}of\PYGZhy{}water/\PYGZsh{}TA2}
\PYG{n}{T1}\PYG{p}{,} \PYG{n}{T2} \PYG{o}{=} \PYG{l+m+mf}{273.15} \PYG{o}{+} \PYG{l+m+mi}{100}\PYG{p}{,} \PYG{l+m+mf}{273.15} \PYG{o}{+} \PYG{l+m+mi}{150}  \PYG{c+c1}{\PYGZsh{} Temperatures in K}
\PYG{n}{P} \PYG{o}{=} \PYG{l+m+mf}{10e3}  \PYG{c+c1}{\PYGZsh{} in Pa}
\PYG{n}{U1}\PYG{p}{,} \PYG{n}{U2} \PYG{o}{=} \PYG{l+m+mf}{2515.49}\PYG{p}{,} \PYG{l+m+mf}{2587.91}  \PYG{c+c1}{\PYGZsh{} Properties in SI units}
\PYG{n}{fluid} \PYG{o}{=} \PYG{l+s+s2}{\PYGZdq{}}\PYG{l+s+s2}{water}\PYG{l+s+s2}{\PYGZdq{}}
\PYG{n}{T} \PYG{o}{=} \PYG{l+m+mf}{273.15} \PYG{o}{+} \PYG{l+m+mi}{133}  \PYG{c+c1}{\PYGZsh{} Temperature at which we want the interpolated property}
\PYG{n}{Prop\PYGZus{}interpolated} \PYG{o}{=} \PYG{n}{linear\PYGZus{}interpolation}\PYG{p}{(}\PYG{n}{T}\PYG{p}{,} \PYG{n}{T1}\PYG{p}{,} \PYG{n}{T2}\PYG{p}{,} \PYG{n}{U1}\PYG{p}{,} \PYG{n}{U2}\PYG{p}{)}
\PYG{n+nb}{print}\PYG{p}{(}\PYG{l+s+s2}{\PYGZdq{}}\PYG{l+s+s2}{Interpolated property at }\PYG{l+s+si}{\PYGZob{}\PYGZcb{}}\PYG{l+s+s2}{ K: }\PYG{l+s+si}{\PYGZob{}\PYGZcb{}}\PYG{l+s+s2}{ kJ/kg}\PYG{l+s+s2}{\PYGZdq{}}\PYG{o}{.}\PYG{n}{format}\PYG{p}{(}\PYG{n}{T}\PYG{p}{,} \PYG{n+nb}{round}\PYG{p}{(}\PYG{n}{Prop\PYGZus{}interpolated}\PYG{p}{,} \PYG{l+m+mi}{2}\PYG{p}{)}\PYG{p}{)}\PYG{p}{)}


\PYG{n}{cool\PYGZus{}prop} \PYG{o}{=} \PYG{n}{CP}\PYG{o}{.}\PYG{n}{PropsSI}\PYG{p}{(}\PYG{l+s+s2}{\PYGZdq{}}\PYG{l+s+s2}{U}\PYG{l+s+s2}{\PYGZdq{}}\PYG{p}{,} \PYG{l+s+s2}{\PYGZdq{}}\PYG{l+s+s2}{P}\PYG{l+s+s2}{\PYGZdq{}}\PYG{p}{,} \PYG{n}{P}\PYG{p}{,} \PYG{l+s+s2}{\PYGZdq{}}\PYG{l+s+s2}{T}\PYG{l+s+s2}{\PYGZdq{}}\PYG{p}{,} \PYG{n}{T}\PYG{p}{,} \PYG{n}{fluid}\PYG{p}{)} \PYG{o}{/} \PYG{l+m+mf}{1e3}  \PYG{c+c1}{\PYGZsh{}\PYGZsh{} in kJ/kg}
\PYG{n+nb}{print}\PYG{p}{(}\PYG{l+s+s2}{\PYGZdq{}}\PYG{l+s+s2}{Property from CoolProp at }\PYG{l+s+si}{\PYGZob{}\PYGZcb{}}\PYG{l+s+s2}{ K: }\PYG{l+s+si}{\PYGZob{}\PYGZcb{}}\PYG{l+s+s2}{ kJ/kg}\PYG{l+s+s2}{\PYGZdq{}}\PYG{o}{.}\PYG{n}{format}\PYG{p}{(}\PYG{n}{T}\PYG{p}{,} \PYG{n+nb}{round}\PYG{p}{(}\PYG{n}{cool\PYGZus{}prop}\PYG{p}{,} \PYG{l+m+mi}{2}\PYG{p}{)}\PYG{p}{)}\PYG{p}{)}

\PYG{n}{absolute\PYGZus{}difference} \PYG{o}{=} \PYG{n+nb}{abs}\PYG{p}{(}\PYG{n}{cool\PYGZus{}prop} \PYG{o}{\PYGZhy{}} \PYG{n}{Prop\PYGZus{}interpolated}\PYG{p}{)}
\PYG{n}{percentage\PYGZus{}difference} \PYG{o}{=} \PYG{p}{(}\PYG{n}{absolute\PYGZus{}difference} \PYG{o}{/} \PYG{n}{Prop\PYGZus{}interpolated}\PYG{p}{)} \PYG{o}{*} \PYG{l+m+mi}{100}
\PYG{n+nb}{print}\PYG{p}{(}\PYG{l+s+s2}{\PYGZdq{}}\PYG{l+s+s2}{Relative difference :}\PYG{l+s+si}{\PYGZob{}\PYGZcb{}}\PYG{l+s+s2}{ }\PYG{l+s+s2}{\PYGZpc{}}\PYG{l+s+s2}{\PYGZdq{}}\PYG{o}{.}\PYG{n}{format}\PYG{p}{(}\PYG{n+nb}{round}\PYG{p}{(}\PYG{n}{percentage\PYGZus{}difference}\PYG{p}{,} \PYG{l+m+mi}{4}\PYG{p}{)}\PYG{p}{)}\PYG{p}{)}
\end{sphinxVerbatim}

\end{sphinxuseclass}\end{sphinxVerbatimInput}
\begin{sphinxVerbatimOutput}

\begin{sphinxuseclass}{cell_output}
\begin{sphinxVerbatim}[commandchars=\\\{\}]
Interpolated property at 406.15 K: 2563.29 kJ/kg
Property from CoolProp at 406.15 K: 2563.19 kJ/kg
Relative difference :0.0036 \PYGZpc{}
\end{sphinxVerbatim}

\end{sphinxuseclass}\end{sphinxVerbatimOutput}

\end{sphinxuseclass}

\section{Linear interpolation for Internal energy of R\sphinxhyphen{}134a refrigerant}
\label{\detokenize{notebooks/Chapter4/linear-interpolation-nternal-energy:linear-interpolation-for-internal-energy-of-r-134a-refrigerant}}
\begin{sphinxuseclass}{cell}\begin{sphinxVerbatimInput}

\begin{sphinxuseclass}{cell_input}
\begin{sphinxVerbatim}[commandchars=\\\{\}]
\PYG{k+kn}{import} \PYG{n+nn}{CoolProp}\PYG{n+nn}{.}\PYG{n+nn}{CoolProp} \PYG{k}{as} \PYG{n+nn}{CP}


\PYG{c+c1}{\PYGZsh{} Example usage from Superheated R134a:}
\PYG{c+c1}{\PYGZsh{} https://pressbooks.bccampus.ca/thermo1/back\PYGZhy{}matter/thermodynamic\PYGZhy{}properties\PYGZhy{}of\PYGZhy{}r134a/\PYGZsh{}TC2}
\PYG{n}{T1}\PYG{p}{,} \PYG{n}{T2} \PYG{o}{=} \PYG{l+m+mf}{273.15} \PYG{o}{+} \PYG{l+m+mi}{40}\PYG{p}{,} \PYG{l+m+mf}{273.15} \PYG{o}{+} \PYG{l+m+mi}{50}  \PYG{c+c1}{\PYGZsh{} Temperatures in K}
\PYG{n}{P} \PYG{o}{=} \PYG{l+m+mf}{100e3}  \PYG{c+c1}{\PYGZsh{} in Pa}
\PYG{n}{U1}\PYG{p}{,} \PYG{n}{U2} \PYG{o}{=} \PYG{l+m+mf}{412.4}\PYG{p}{,} \PYG{l+m+mf}{420.37}  \PYG{c+c1}{\PYGZsh{} Properties in SI units}
\PYG{n}{fluid} \PYG{o}{=} \PYG{l+s+s2}{\PYGZdq{}}\PYG{l+s+s2}{R134a}\PYG{l+s+s2}{\PYGZdq{}}
\PYG{n}{T} \PYG{o}{=} \PYG{l+m+mf}{273.15} \PYG{o}{+} \PYG{l+m+mi}{43}  \PYG{c+c1}{\PYGZsh{} Temperature at which we want the interpolated property}
\PYG{n}{Prop\PYGZus{}interpolated} \PYG{o}{=} \PYG{n}{linear\PYGZus{}interpolation}\PYG{p}{(}\PYG{n}{T}\PYG{p}{,} \PYG{n}{T1}\PYG{p}{,} \PYG{n}{T2}\PYG{p}{,} \PYG{n}{U1}\PYG{p}{,} \PYG{n}{U2}\PYG{p}{)}
\PYG{n+nb}{print}\PYG{p}{(}\PYG{l+s+s2}{\PYGZdq{}}\PYG{l+s+s2}{Interpolated property at }\PYG{l+s+si}{\PYGZob{}\PYGZcb{}}\PYG{l+s+s2}{ K: }\PYG{l+s+si}{\PYGZob{}\PYGZcb{}}\PYG{l+s+s2}{ kJ/kg}\PYG{l+s+s2}{\PYGZdq{}}\PYG{o}{.}\PYG{n}{format}\PYG{p}{(}\PYG{n}{T}\PYG{p}{,} \PYG{n+nb}{round}\PYG{p}{(}\PYG{n}{Prop\PYGZus{}interpolated}\PYG{p}{,}\PYG{l+m+mi}{2}\PYG{p}{)}\PYG{p}{)}\PYG{p}{)}


\PYG{n}{cool\PYGZus{}prop} \PYG{o}{=} \PYG{n}{CP}\PYG{o}{.}\PYG{n}{PropsSI}\PYG{p}{(}\PYG{l+s+s2}{\PYGZdq{}}\PYG{l+s+s2}{U}\PYG{l+s+s2}{\PYGZdq{}}\PYG{p}{,} \PYG{l+s+s2}{\PYGZdq{}}\PYG{l+s+s2}{P}\PYG{l+s+s2}{\PYGZdq{}}\PYG{p}{,} \PYG{n}{P}\PYG{p}{,} \PYG{l+s+s2}{\PYGZdq{}}\PYG{l+s+s2}{T}\PYG{l+s+s2}{\PYGZdq{}}\PYG{p}{,} \PYG{n}{T}\PYG{p}{,} \PYG{n}{fluid}\PYG{p}{)} \PYG{o}{/} \PYG{l+m+mf}{1e3}  \PYG{c+c1}{\PYGZsh{}\PYGZsh{} in kJ/kg}
\PYG{n+nb}{print}\PYG{p}{(}\PYG{l+s+s2}{\PYGZdq{}}\PYG{l+s+s2}{Property from CoolProp at }\PYG{l+s+si}{\PYGZob{}\PYGZcb{}}\PYG{l+s+s2}{ K: }\PYG{l+s+si}{\PYGZob{}\PYGZcb{}}\PYG{l+s+s2}{ kJ/kg}\PYG{l+s+s2}{\PYGZdq{}}\PYG{o}{.}\PYG{n}{format}\PYG{p}{(}\PYG{n}{T}\PYG{p}{,} \PYG{n+nb}{round}\PYG{p}{(}\PYG{n}{cool\PYGZus{}prop}\PYG{p}{,}\PYG{l+m+mi}{2}\PYG{p}{)}\PYG{p}{)}\PYG{p}{)}

\PYG{n}{absolute\PYGZus{}difference} \PYG{o}{=} \PYG{n+nb}{abs}\PYG{p}{(}\PYG{n}{cool\PYGZus{}prop} \PYG{o}{\PYGZhy{}} \PYG{n}{Prop\PYGZus{}interpolated}\PYG{p}{)}
\PYG{n}{percentage\PYGZus{}difference} \PYG{o}{=} \PYG{p}{(}\PYG{n}{absolute\PYGZus{}difference} \PYG{o}{/} \PYG{n}{Prop\PYGZus{}interpolated}\PYG{p}{)} \PYG{o}{*} \PYG{l+m+mi}{100}
\PYG{n+nb}{print}\PYG{p}{(}\PYG{l+s+s2}{\PYGZdq{}}\PYG{l+s+s2}{Relative difference :}\PYG{l+s+si}{\PYGZob{}\PYGZcb{}}\PYG{l+s+s2}{ }\PYG{l+s+s2}{\PYGZpc{}}\PYG{l+s+s2}{\PYGZdq{}}\PYG{o}{.}\PYG{n}{format}\PYG{p}{(}\PYG{n+nb}{round}\PYG{p}{(}\PYG{n}{percentage\PYGZus{}difference}\PYG{p}{,}\PYG{l+m+mi}{4}\PYG{p}{)}\PYG{p}{)}\PYG{p}{)}
\end{sphinxVerbatim}

\end{sphinxuseclass}\end{sphinxVerbatimInput}
\begin{sphinxVerbatimOutput}

\begin{sphinxuseclass}{cell_output}
\begin{sphinxVerbatim}[commandchars=\\\{\}]
Interpolated property at 316.15 K: 414.79 kJ/kg
Property from CoolProp at 316.15 K: 414.77 kJ/kg
Relative difference :0.0046 \PYGZpc{}
\end{sphinxVerbatim}

\end{sphinxuseclass}\end{sphinxVerbatimOutput}

\end{sphinxuseclass}
\begin{sphinxuseclass}{cell}\begin{sphinxVerbatimInput}

\begin{sphinxuseclass}{cell_input}
\begin{sphinxVerbatim}[commandchars=\\\{\}]
\PYG{k+kn}{import} \PYG{n+nn}{CoolProp}\PYG{n+nn}{.}\PYG{n+nn}{CoolProp} \PYG{k}{as} \PYG{n+nn}{CP}
\PYG{k+kn}{import} \PYG{n+nn}{numpy} \PYG{k}{as} \PYG{n+nn}{np}
\PYG{k+kn}{import} \PYG{n+nn}{matplotlib}\PYG{n+nn}{.}\PYG{n+nn}{pyplot} \PYG{k}{as} \PYG{n+nn}{plt}

\PYG{c+c1}{\PYGZsh{} Constants}
\PYG{n}{P} \PYG{o}{=} \PYG{l+m+mf}{10e3}  \PYG{c+c1}{\PYGZsh{} Pressure in Pa}
\PYG{n}{fluid} \PYG{o}{=} \PYG{l+s+s2}{\PYGZdq{}}\PYG{l+s+s2}{R134a}\PYG{l+s+s2}{\PYGZdq{}}
\PYG{n}{T\PYGZus{}target} \PYG{o}{=} \PYG{l+m+mf}{273.15} \PYG{o}{+} \PYG{l+m+mi}{135}  \PYG{c+c1}{\PYGZsh{} Target temperature for property evaluation}

\PYG{c+c1}{\PYGZsh{} Range of interval sizes}
\PYG{n}{interval\PYGZus{}sizes} \PYG{o}{=} \PYG{n}{np}\PYG{o}{.}\PYG{n}{linspace}\PYG{p}{(}\PYG{l+m+mi}{10}\PYG{p}{,}\PYG{l+m+mi}{200}\PYG{p}{,}\PYG{l+m+mi}{20}\PYG{p}{)}
\PYG{n}{relative\PYGZus{}errors} \PYG{o}{=} \PYG{p}{[}\PYG{p}{]}

\PYG{c+c1}{\PYGZsh{} Loop over interval sizes and calculate relative errors}
\PYG{k}{for} \PYG{n}{interval} \PYG{o+ow}{in} \PYG{n}{interval\PYGZus{}sizes}\PYG{p}{:}
    \PYG{n}{T1} \PYG{o}{=} \PYG{n}{T\PYGZus{}target} \PYG{o}{\PYGZhy{}} \PYG{n}{interval} \PYG{o}{/} \PYG{l+m+mi}{2}
    \PYG{n}{T2} \PYG{o}{=} \PYG{n}{T\PYGZus{}target} \PYG{o}{+} \PYG{n}{interval} \PYG{o}{/} \PYG{l+m+mi}{2}
    
    \PYG{c+c1}{\PYGZsh{} Get properties from CoolProp for the interval boundaries}
    \PYG{n}{U1} \PYG{o}{=} \PYG{n}{CP}\PYG{o}{.}\PYG{n}{PropsSI}\PYG{p}{(}\PYG{l+s+s2}{\PYGZdq{}}\PYG{l+s+s2}{U}\PYG{l+s+s2}{\PYGZdq{}}\PYG{p}{,} \PYG{l+s+s2}{\PYGZdq{}}\PYG{l+s+s2}{P}\PYG{l+s+s2}{\PYGZdq{}}\PYG{p}{,} \PYG{n}{P}\PYG{p}{,} \PYG{l+s+s2}{\PYGZdq{}}\PYG{l+s+s2}{T}\PYG{l+s+s2}{\PYGZdq{}}\PYG{p}{,} \PYG{n}{T1}\PYG{p}{,} \PYG{n}{fluid}\PYG{p}{)} \PYG{o}{/} \PYG{l+m+mf}{1e3}
    \PYG{n}{U2} \PYG{o}{=} \PYG{n}{CP}\PYG{o}{.}\PYG{n}{PropsSI}\PYG{p}{(}\PYG{l+s+s2}{\PYGZdq{}}\PYG{l+s+s2}{U}\PYG{l+s+s2}{\PYGZdq{}}\PYG{p}{,} \PYG{l+s+s2}{\PYGZdq{}}\PYG{l+s+s2}{P}\PYG{l+s+s2}{\PYGZdq{}}\PYG{p}{,} \PYG{n}{P}\PYG{p}{,} \PYG{l+s+s2}{\PYGZdq{}}\PYG{l+s+s2}{T}\PYG{l+s+s2}{\PYGZdq{}}\PYG{p}{,} \PYG{n}{T2}\PYG{p}{,} \PYG{n}{fluid}\PYG{p}{)} \PYG{o}{/} \PYG{l+m+mf}{1e3}
    
    \PYG{c+c1}{\PYGZsh{} Calculate relative error}
    \PYG{n}{error} \PYG{o}{=} \PYG{n}{calculate\PYGZus{}relative\PYGZus{}error}\PYG{p}{(}\PYG{n}{T\PYGZus{}target}\PYG{p}{,} \PYG{n}{T1}\PYG{p}{,} \PYG{n}{T2}\PYG{p}{,} \PYG{n}{U1}\PYG{p}{,} \PYG{n}{U2}\PYG{p}{,} \PYG{n}{fluid}\PYG{p}{)}
    \PYG{n}{relative\PYGZus{}errors}\PYG{o}{.}\PYG{n}{append}\PYG{p}{(}\PYG{n}{error}\PYG{p}{)}
\end{sphinxVerbatim}

\end{sphinxuseclass}\end{sphinxVerbatimInput}

\end{sphinxuseclass}

\section{A plot to illustrate the relative error as a function of size of interval of input}
\label{\detokenize{notebooks/Chapter4/linear-interpolation-nternal-energy:a-plot-to-illustrate-the-relative-error-as-a-function-of-size-of-interval-of-input}}
\begin{sphinxuseclass}{cell}\begin{sphinxVerbatimInput}

\begin{sphinxuseclass}{cell_input}
\begin{sphinxVerbatim}[commandchars=\\\{\}]
\PYG{c+c1}{\PYGZsh{} Plotting}
\PYG{c+c1}{\PYGZsh{} Plotting with logarithmic scale and improved aesthetics}
\PYG{n}{plt}\PYG{o}{.}\PYG{n}{figure}\PYG{p}{(}\PYG{n}{figsize}\PYG{o}{=}\PYG{p}{(}\PYG{l+m+mi}{10}\PYG{p}{,} \PYG{l+m+mi}{6}\PYG{p}{)}\PYG{p}{)}  \PYG{c+c1}{\PYGZsh{} Sets the figure size}
\PYG{n}{plt}\PYG{o}{.}\PYG{n}{plot}\PYG{p}{(}\PYG{n}{interval\PYGZus{}sizes}\PYG{p}{,} \PYG{n}{relative\PYGZus{}errors}\PYG{p}{,} \PYG{n}{marker}\PYG{o}{=}\PYG{l+s+s1}{\PYGZsq{}}\PYG{l+s+s1}{o}\PYG{l+s+s1}{\PYGZsq{}}\PYG{p}{,} \PYG{n}{linestyle}\PYG{o}{=}\PYG{l+s+s1}{\PYGZsq{}}\PYG{l+s+s1}{\PYGZhy{}}\PYG{l+s+s1}{\PYGZsq{}}\PYG{p}{,} \PYG{n}{color}\PYG{o}{=}\PYG{l+s+s1}{\PYGZsq{}}\PYG{l+s+s1}{blue}\PYG{l+s+s1}{\PYGZsq{}}\PYG{p}{,} \PYG{n}{label}\PYG{o}{=}\PYG{l+s+s1}{\PYGZsq{}}\PYG{l+s+s1}{Relative Error}\PYG{l+s+s1}{\PYGZsq{}}\PYG{p}{)}  \PYG{c+c1}{\PYGZsh{} Adds color, line style, and markers}

\PYG{n}{plt}\PYG{o}{.}\PYG{n}{xlabel}\PYG{p}{(}\PYG{l+s+s1}{\PYGZsq{}}\PYG{l+s+s1}{Size of Temperature Interval (K)}\PYG{l+s+s1}{\PYGZsq{}}\PYG{p}{,} \PYG{n}{fontsize}\PYG{o}{=}\PYG{l+m+mi}{14}\PYG{p}{,} \PYG{n}{labelpad}\PYG{o}{=}\PYG{l+m+mi}{12}\PYG{p}{)}
\PYG{n}{plt}\PYG{o}{.}\PYG{n}{ylabel}\PYG{p}{(}\PYG{l+s+s1}{\PYGZsq{}}\PYG{l+s+s1}{Relative Error (}\PYG{l+s+s1}{\PYGZpc{}}\PYG{l+s+s1}{)}\PYG{l+s+s1}{\PYGZsq{}}\PYG{p}{,} \PYG{n}{fontsize}\PYG{o}{=}\PYG{l+m+mi}{14}\PYG{p}{,} \PYG{n}{labelpad}\PYG{o}{=}\PYG{l+m+mi}{12}\PYG{p}{)}
\PYG{n}{plt}\PYG{o}{.}\PYG{n}{title}\PYG{p}{(}\PYG{l+s+s1}{\PYGZsq{}}\PYG{l+s+s1}{Relative Error vs. Temperature Interval Size}\PYG{l+s+s1}{\PYGZsq{}}\PYG{p}{,} \PYG{n}{fontsize}\PYG{o}{=}\PYG{l+m+mi}{16}\PYG{p}{,} \PYG{n}{pad}\PYG{o}{=}\PYG{l+m+mi}{20}\PYG{p}{)}

\PYG{c+c1}{\PYGZsh{}plt.xscale(\PYGZsq{}log\PYGZsq{})  \PYGZsh{} Logarithmic scale for the x\PYGZhy{}axis}
\PYG{c+c1}{\PYGZsh{}plt.yscale(\PYGZsq{}log\PYGZsq{})  \PYGZsh{} Logarithmic scale for the y\PYGZhy{}axis}

\PYG{n}{plt}\PYG{o}{.}\PYG{n}{legend}\PYG{p}{(}\PYG{n}{fontsize}\PYG{o}{=}\PYG{l+m+mi}{12}\PYG{p}{)}
\PYG{n}{plt}\PYG{o}{.}\PYG{n}{grid}\PYG{p}{(}\PYG{k+kc}{True}\PYG{p}{,} \PYG{n}{which}\PYG{o}{=}\PYG{l+s+s2}{\PYGZdq{}}\PYG{l+s+s2}{both}\PYG{l+s+s2}{\PYGZdq{}}\PYG{p}{,} \PYG{n}{linestyle}\PYG{o}{=}\PYG{l+s+s1}{\PYGZsq{}}\PYG{l+s+s1}{\PYGZhy{}\PYGZhy{}}\PYG{l+s+s1}{\PYGZsq{}}\PYG{p}{,} \PYG{n}{linewidth}\PYG{o}{=}\PYG{l+m+mf}{0.5}\PYG{p}{)}  \PYG{c+c1}{\PYGZsh{} Adds gridlines for both major and minor ticks and customizes their style}
\PYG{n}{plt}\PYG{o}{.}\PYG{n}{tick\PYGZus{}params}\PYG{p}{(}\PYG{n}{labelsize}\PYG{o}{=}\PYG{l+m+mi}{12}\PYG{p}{)}  \PYG{c+c1}{\PYGZsh{} Adjust the size of the axis ticks labels}

\PYG{n}{plt}\PYG{o}{.}\PYG{n}{tight\PYGZus{}layout}\PYG{p}{(}\PYG{p}{)}  \PYG{c+c1}{\PYGZsh{} Adjusts subplot params so that the subplot(s) fits in to the figure area}

\PYG{n}{plt}\PYG{o}{.}\PYG{n}{show}\PYG{p}{(}\PYG{p}{)}
\end{sphinxVerbatim}

\end{sphinxuseclass}\end{sphinxVerbatimInput}
\begin{sphinxVerbatimOutput}

\begin{sphinxuseclass}{cell_output}
\noindent\sphinxincludegraphics{{51c27da274678323ba2837a5bc01999272f6c36550bafd87be7751b5d05a157b}.png}

\end{sphinxuseclass}\end{sphinxVerbatimOutput}

\end{sphinxuseclass}
\sphinxstepscope


\section{Reference States in Thermodynamics}
\label{\detokenize{notebooks/Chapter4/use_different_references_in_CoolProp:reference-states-in-thermodynamics}}\label{\detokenize{notebooks/Chapter4/use_different_references_in_CoolProp::doc}}
\sphinxAtStartPar
In thermodynamics, a reference state serves as a baseline or zero point for measuring thermodynamic properties such as enthalpy (H) and entropy (S). Choosing different reference states doesn’t affect the differences in properties calculated using different states of a substance.

\sphinxAtStartPar
For substances such as refrigerants (e.g., R\sphinxhyphen{}134a), commonly used reference states include the Normal Boiling Point (NBP) and the International Institute of Refrigeration (IIR) standard:

\begin{sphinxVerbatim}[commandchars=\\\{\}]
NBP: It sets the enthalpy and entropy at the boiling point of the substance under 1 atmosphere to zero. This is a common reference for refrigerants.

IIR: This reference state, set by the International Institute of Refrigeration, defines the enthalpy and entropy at 0°C and a pressure where the liquid is saturated to be zero for refrigerants.

For refrigerants, the ASHRAE reference state typically sets a specific point where the enthalpy and entropy are defined to be zero. This point is the saturated liquid state at a specified temperature, often associated with a standard condition such as under a temperature of \PYGZhy{}40°C or \PYGZhy{}40°F. (This is what is being followed in the textbook!)
\end{sphinxVerbatim}

\sphinxAtStartPar
Despite the reference state chosen, the differences in enthalpy (ΔH) and entropy (ΔS) between two defined states of R\sphinxhyphen{}134a remain the same. These differences are intrinsic and are not influenced by the reference points, as you’ll observe in the CoolProp calculations that follow.


\subsection{Problem statement:}
\label{\detokenize{notebooks/Chapter4/use_different_references_in_CoolProp:problem-statement}}
\sphinxAtStartPar
Use threee different references and calculate absolute and relative differences of S, H for R134a refigerant fluid.

\begin{sphinxuseclass}{cell}\begin{sphinxVerbatimInput}

\begin{sphinxuseclass}{cell_input}
\begin{sphinxVerbatim}[commandchars=\\\{\}]
\PYG{c+c1}{\PYGZsh{}\PYGZsh{} add internal energy as another property}
\PYG{c+c1}{\PYGZsh{}\PYGZsh{} make it clear what the textbook uses, in this case it is ASHRAE. A note at the end would be great.}
\end{sphinxVerbatim}

\end{sphinxuseclass}\end{sphinxVerbatimInput}

\end{sphinxuseclass}

\subsection{Solution}
\label{\detokenize{notebooks/Chapter4/use_different_references_in_CoolProp:solution}}
\begin{sphinxuseclass}{cell}\begin{sphinxVerbatimInput}

\begin{sphinxuseclass}{cell_input}
\begin{sphinxVerbatim}[commandchars=\\\{\}]
\PYG{k+kn}{import} \PYG{n+nn}{CoolProp}\PYG{n+nn}{.}\PYG{n+nn}{CoolProp} \PYG{k}{as} \PYG{n+nn}{CP}

\PYG{c+c1}{\PYGZsh{} Define two states for R\PYGZhy{}134a}
\PYG{n}{T1} \PYG{o}{=} \PYG{l+m+mi}{280}  \PYG{c+c1}{\PYGZsh{} Temperature in K for state 1}
\PYG{n}{P1} \PYG{o}{=} \PYG{l+m+mi}{101325}  \PYG{c+c1}{\PYGZsh{} Pressure in Pa for state 1 (1 atm)}
\PYG{n}{T2} \PYG{o}{=} \PYG{l+m+mi}{300}  \PYG{c+c1}{\PYGZsh{} Temperature in K for state 2}
\PYG{n}{P2} \PYG{o}{=} \PYG{l+m+mi}{200000}  \PYG{c+c1}{\PYGZsh{} Pressure in Pa for state 2}

\PYG{c+c1}{\PYGZsh{} Set the IIR reference state}
\PYG{n}{CP}\PYG{o}{.}\PYG{n}{set\PYGZus{}reference\PYGZus{}state}\PYG{p}{(}\PYG{l+s+s1}{\PYGZsq{}}\PYG{l+s+s1}{R134a}\PYG{l+s+s1}{\PYGZsq{}}\PYG{p}{,} \PYG{l+s+s1}{\PYGZsq{}}\PYG{l+s+s1}{IIR}\PYG{l+s+s1}{\PYGZsq{}}\PYG{p}{)}

\PYG{c+c1}{\PYGZsh{} Calculate enthalpy and entropy for both states with the IIR reference}
\PYG{n}{h1\PYGZus{}iir} \PYG{o}{=} \PYG{n}{CP}\PYG{o}{.}\PYG{n}{PropsSI}\PYG{p}{(}\PYG{l+s+s1}{\PYGZsq{}}\PYG{l+s+s1}{H}\PYG{l+s+s1}{\PYGZsq{}}\PYG{p}{,} \PYG{l+s+s1}{\PYGZsq{}}\PYG{l+s+s1}{T}\PYG{l+s+s1}{\PYGZsq{}}\PYG{p}{,} \PYG{n}{T1}\PYG{p}{,} \PYG{l+s+s1}{\PYGZsq{}}\PYG{l+s+s1}{P}\PYG{l+s+s1}{\PYGZsq{}}\PYG{p}{,} \PYG{n}{P1}\PYG{p}{,} \PYG{l+s+s1}{\PYGZsq{}}\PYG{l+s+s1}{R134a}\PYG{l+s+s1}{\PYGZsq{}}\PYG{p}{)} 
\PYG{n}{s1\PYGZus{}iir} \PYG{o}{=} \PYG{n}{CP}\PYG{o}{.}\PYG{n}{PropsSI}\PYG{p}{(}\PYG{l+s+s1}{\PYGZsq{}}\PYG{l+s+s1}{S}\PYG{l+s+s1}{\PYGZsq{}}\PYG{p}{,} \PYG{l+s+s1}{\PYGZsq{}}\PYG{l+s+s1}{T}\PYG{l+s+s1}{\PYGZsq{}}\PYG{p}{,} \PYG{n}{T1}\PYG{p}{,} \PYG{l+s+s1}{\PYGZsq{}}\PYG{l+s+s1}{P}\PYG{l+s+s1}{\PYGZsq{}}\PYG{p}{,} \PYG{n}{P1}\PYG{p}{,} \PYG{l+s+s1}{\PYGZsq{}}\PYG{l+s+s1}{R134a}\PYG{l+s+s1}{\PYGZsq{}}\PYG{p}{)} 
\PYG{n}{h2\PYGZus{}iir} \PYG{o}{=} \PYG{n}{CP}\PYG{o}{.}\PYG{n}{PropsSI}\PYG{p}{(}\PYG{l+s+s1}{\PYGZsq{}}\PYG{l+s+s1}{H}\PYG{l+s+s1}{\PYGZsq{}}\PYG{p}{,} \PYG{l+s+s1}{\PYGZsq{}}\PYG{l+s+s1}{T}\PYG{l+s+s1}{\PYGZsq{}}\PYG{p}{,} \PYG{n}{T2}\PYG{p}{,} \PYG{l+s+s1}{\PYGZsq{}}\PYG{l+s+s1}{P}\PYG{l+s+s1}{\PYGZsq{}}\PYG{p}{,} \PYG{n}{P2}\PYG{p}{,} \PYG{l+s+s1}{\PYGZsq{}}\PYG{l+s+s1}{R134a}\PYG{l+s+s1}{\PYGZsq{}}\PYG{p}{)} 
\PYG{n}{s2\PYGZus{}iir} \PYG{o}{=} \PYG{n}{CP}\PYG{o}{.}\PYG{n}{PropsSI}\PYG{p}{(}\PYG{l+s+s1}{\PYGZsq{}}\PYG{l+s+s1}{S}\PYG{l+s+s1}{\PYGZsq{}}\PYG{p}{,} \PYG{l+s+s1}{\PYGZsq{}}\PYG{l+s+s1}{T}\PYG{l+s+s1}{\PYGZsq{}}\PYG{p}{,} \PYG{n}{T2}\PYG{p}{,} \PYG{l+s+s1}{\PYGZsq{}}\PYG{l+s+s1}{P}\PYG{l+s+s1}{\PYGZsq{}}\PYG{p}{,} \PYG{n}{P2}\PYG{p}{,} \PYG{l+s+s1}{\PYGZsq{}}\PYG{l+s+s1}{R134a}\PYG{l+s+s1}{\PYGZsq{}}\PYG{p}{)} 

\PYG{c+c1}{\PYGZsh{} Calculate differences with IIR reference}
\PYG{n}{dh\PYGZus{}iir} \PYG{o}{=} \PYG{p}{(}\PYG{n}{h2\PYGZus{}iir} \PYG{o}{\PYGZhy{}} \PYG{n}{h1\PYGZus{}iir}\PYG{p}{)} \PYG{o}{/} \PYG{l+m+mi}{1000}
\PYG{n}{ds\PYGZus{}iir} \PYG{o}{=} \PYG{p}{(}\PYG{n}{s2\PYGZus{}iir} \PYG{o}{\PYGZhy{}} \PYG{n}{s1\PYGZus{}iir}\PYG{p}{)} \PYG{o}{/} \PYG{l+m+mi}{1000}

\PYG{c+c1}{\PYGZsh{} Set the NBP reference state}
\PYG{n}{CP}\PYG{o}{.}\PYG{n}{set\PYGZus{}reference\PYGZus{}state}\PYG{p}{(}\PYG{l+s+s1}{\PYGZsq{}}\PYG{l+s+s1}{R134a}\PYG{l+s+s1}{\PYGZsq{}}\PYG{p}{,} \PYG{l+s+s1}{\PYGZsq{}}\PYG{l+s+s1}{NBP}\PYG{l+s+s1}{\PYGZsq{}}\PYG{p}{)}

\PYG{c+c1}{\PYGZsh{} Calculate enthalpy and entropy for both states with the NBP reference}
\PYG{n}{h1\PYGZus{}nbp} \PYG{o}{=} \PYG{n}{CP}\PYG{o}{.}\PYG{n}{PropsSI}\PYG{p}{(}\PYG{l+s+s1}{\PYGZsq{}}\PYG{l+s+s1}{H}\PYG{l+s+s1}{\PYGZsq{}}\PYG{p}{,} \PYG{l+s+s1}{\PYGZsq{}}\PYG{l+s+s1}{T}\PYG{l+s+s1}{\PYGZsq{}}\PYG{p}{,} \PYG{n}{T1}\PYG{p}{,} \PYG{l+s+s1}{\PYGZsq{}}\PYG{l+s+s1}{P}\PYG{l+s+s1}{\PYGZsq{}}\PYG{p}{,} \PYG{n}{P1}\PYG{p}{,} \PYG{l+s+s1}{\PYGZsq{}}\PYG{l+s+s1}{R134a}\PYG{l+s+s1}{\PYGZsq{}}\PYG{p}{)} 
\PYG{n}{s1\PYGZus{}nbp} \PYG{o}{=} \PYG{n}{CP}\PYG{o}{.}\PYG{n}{PropsSI}\PYG{p}{(}\PYG{l+s+s1}{\PYGZsq{}}\PYG{l+s+s1}{S}\PYG{l+s+s1}{\PYGZsq{}}\PYG{p}{,} \PYG{l+s+s1}{\PYGZsq{}}\PYG{l+s+s1}{T}\PYG{l+s+s1}{\PYGZsq{}}\PYG{p}{,} \PYG{n}{T1}\PYG{p}{,} \PYG{l+s+s1}{\PYGZsq{}}\PYG{l+s+s1}{P}\PYG{l+s+s1}{\PYGZsq{}}\PYG{p}{,} \PYG{n}{P1}\PYG{p}{,} \PYG{l+s+s1}{\PYGZsq{}}\PYG{l+s+s1}{R134a}\PYG{l+s+s1}{\PYGZsq{}}\PYG{p}{)} 
\PYG{n}{h2\PYGZus{}nbp} \PYG{o}{=} \PYG{n}{CP}\PYG{o}{.}\PYG{n}{PropsSI}\PYG{p}{(}\PYG{l+s+s1}{\PYGZsq{}}\PYG{l+s+s1}{H}\PYG{l+s+s1}{\PYGZsq{}}\PYG{p}{,} \PYG{l+s+s1}{\PYGZsq{}}\PYG{l+s+s1}{T}\PYG{l+s+s1}{\PYGZsq{}}\PYG{p}{,} \PYG{n}{T2}\PYG{p}{,} \PYG{l+s+s1}{\PYGZsq{}}\PYG{l+s+s1}{P}\PYG{l+s+s1}{\PYGZsq{}}\PYG{p}{,} \PYG{n}{P2}\PYG{p}{,} \PYG{l+s+s1}{\PYGZsq{}}\PYG{l+s+s1}{R134a}\PYG{l+s+s1}{\PYGZsq{}}\PYG{p}{)} 
\PYG{n}{s2\PYGZus{}nbp} \PYG{o}{=} \PYG{n}{CP}\PYG{o}{.}\PYG{n}{PropsSI}\PYG{p}{(}\PYG{l+s+s1}{\PYGZsq{}}\PYG{l+s+s1}{S}\PYG{l+s+s1}{\PYGZsq{}}\PYG{p}{,} \PYG{l+s+s1}{\PYGZsq{}}\PYG{l+s+s1}{T}\PYG{l+s+s1}{\PYGZsq{}}\PYG{p}{,} \PYG{n}{T2}\PYG{p}{,} \PYG{l+s+s1}{\PYGZsq{}}\PYG{l+s+s1}{P}\PYG{l+s+s1}{\PYGZsq{}}\PYG{p}{,} \PYG{n}{P2}\PYG{p}{,} \PYG{l+s+s1}{\PYGZsq{}}\PYG{l+s+s1}{R134a}\PYG{l+s+s1}{\PYGZsq{}}\PYG{p}{)} 

\PYG{c+c1}{\PYGZsh{} Calculate differences with NBP reference}
\PYG{n}{dh\PYGZus{}nbp} \PYG{o}{=} \PYG{p}{(}\PYG{n}{h2\PYGZus{}nbp} \PYG{o}{\PYGZhy{}} \PYG{n}{h1\PYGZus{}nbp}\PYG{p}{)} \PYG{o}{/} \PYG{l+m+mi}{1000}
\PYG{n}{ds\PYGZus{}nbp} \PYG{o}{=} \PYG{p}{(}\PYG{n}{s2\PYGZus{}nbp} \PYG{o}{\PYGZhy{}} \PYG{n}{s1\PYGZus{}nbp}\PYG{p}{)} \PYG{o}{/} \PYG{l+m+mi}{1000}

\PYG{c+c1}{\PYGZsh{} Set the ASHRAE reference state}
\PYG{n}{CP}\PYG{o}{.}\PYG{n}{set\PYGZus{}reference\PYGZus{}state}\PYG{p}{(}\PYG{l+s+s1}{\PYGZsq{}}\PYG{l+s+s1}{R134a}\PYG{l+s+s1}{\PYGZsq{}}\PYG{p}{,} \PYG{l+s+s1}{\PYGZsq{}}\PYG{l+s+s1}{ASHRAE}\PYG{l+s+s1}{\PYGZsq{}}\PYG{p}{)}

\PYG{c+c1}{\PYGZsh{} Calculate enthalpy and entropy for both states with the ASHRAE reference}
\PYG{n}{h1\PYGZus{}ashrae} \PYG{o}{=} \PYG{n}{CP}\PYG{o}{.}\PYG{n}{PropsSI}\PYG{p}{(}\PYG{l+s+s1}{\PYGZsq{}}\PYG{l+s+s1}{H}\PYG{l+s+s1}{\PYGZsq{}}\PYG{p}{,} \PYG{l+s+s1}{\PYGZsq{}}\PYG{l+s+s1}{T}\PYG{l+s+s1}{\PYGZsq{}}\PYG{p}{,} \PYG{n}{T1}\PYG{p}{,} \PYG{l+s+s1}{\PYGZsq{}}\PYG{l+s+s1}{P}\PYG{l+s+s1}{\PYGZsq{}}\PYG{p}{,} \PYG{n}{P1}\PYG{p}{,} \PYG{l+s+s1}{\PYGZsq{}}\PYG{l+s+s1}{R134a}\PYG{l+s+s1}{\PYGZsq{}}\PYG{p}{)} 
\PYG{n}{s1\PYGZus{}ashrae} \PYG{o}{=} \PYG{n}{CP}\PYG{o}{.}\PYG{n}{PropsSI}\PYG{p}{(}\PYG{l+s+s1}{\PYGZsq{}}\PYG{l+s+s1}{S}\PYG{l+s+s1}{\PYGZsq{}}\PYG{p}{,} \PYG{l+s+s1}{\PYGZsq{}}\PYG{l+s+s1}{T}\PYG{l+s+s1}{\PYGZsq{}}\PYG{p}{,} \PYG{n}{T1}\PYG{p}{,} \PYG{l+s+s1}{\PYGZsq{}}\PYG{l+s+s1}{P}\PYG{l+s+s1}{\PYGZsq{}}\PYG{p}{,} \PYG{n}{P1}\PYG{p}{,} \PYG{l+s+s1}{\PYGZsq{}}\PYG{l+s+s1}{R134a}\PYG{l+s+s1}{\PYGZsq{}}\PYG{p}{)} 
\PYG{n}{h2\PYGZus{}ashrae} \PYG{o}{=} \PYG{n}{CP}\PYG{o}{.}\PYG{n}{PropsSI}\PYG{p}{(}\PYG{l+s+s1}{\PYGZsq{}}\PYG{l+s+s1}{H}\PYG{l+s+s1}{\PYGZsq{}}\PYG{p}{,} \PYG{l+s+s1}{\PYGZsq{}}\PYG{l+s+s1}{T}\PYG{l+s+s1}{\PYGZsq{}}\PYG{p}{,} \PYG{n}{T2}\PYG{p}{,} \PYG{l+s+s1}{\PYGZsq{}}\PYG{l+s+s1}{P}\PYG{l+s+s1}{\PYGZsq{}}\PYG{p}{,} \PYG{n}{P2}\PYG{p}{,} \PYG{l+s+s1}{\PYGZsq{}}\PYG{l+s+s1}{R134a}\PYG{l+s+s1}{\PYGZsq{}}\PYG{p}{)} 
\PYG{n}{s2\PYGZus{}ashrae} \PYG{o}{=} \PYG{n}{CP}\PYG{o}{.}\PYG{n}{PropsSI}\PYG{p}{(}\PYG{l+s+s1}{\PYGZsq{}}\PYG{l+s+s1}{S}\PYG{l+s+s1}{\PYGZsq{}}\PYG{p}{,} \PYG{l+s+s1}{\PYGZsq{}}\PYG{l+s+s1}{T}\PYG{l+s+s1}{\PYGZsq{}}\PYG{p}{,} \PYG{n}{T2}\PYG{p}{,} \PYG{l+s+s1}{\PYGZsq{}}\PYG{l+s+s1}{P}\PYG{l+s+s1}{\PYGZsq{}}\PYG{p}{,} \PYG{n}{P2}\PYG{p}{,} \PYG{l+s+s1}{\PYGZsq{}}\PYG{l+s+s1}{R134a}\PYG{l+s+s1}{\PYGZsq{}}\PYG{p}{)} 

\PYG{c+c1}{\PYGZsh{} Calculate differences with ASHRAE reference}
\PYG{n}{dh\PYGZus{}ashrae} \PYG{o}{=} \PYG{p}{(}\PYG{n}{h2\PYGZus{}ashrae} \PYG{o}{\PYGZhy{}} \PYG{n}{h1\PYGZus{}ashrae}\PYG{p}{)} \PYG{o}{/} \PYG{l+m+mi}{1000}
\PYG{n}{ds\PYGZus{}ashrae} \PYG{o}{=} \PYG{p}{(}\PYG{n}{s2\PYGZus{}ashrae} \PYG{o}{\PYGZhy{}} \PYG{n}{s1\PYGZus{}ashrae}\PYG{p}{)} \PYG{o}{/} \PYG{l+m+mi}{1000}

\PYG{n+nb}{print}\PYG{p}{(}\PYG{l+s+s2}{\PYGZdq{}}\PYG{l+s+s2}{Properties at T = 280 K and P = 1 bar}\PYG{l+s+s2}{\PYGZdq{}}\PYG{p}{)}
\PYG{c+c1}{\PYGZsh{} Print statements for each state}
\PYG{n+nb}{print}\PYG{p}{(}\PYG{l+s+s2}{\PYGZdq{}}\PYG{l+s+s2}{IIR Reference State:}\PYG{l+s+s2}{\PYGZdq{}}\PYG{p}{)}
\PYG{n+nb}{print}\PYG{p}{(}\PYG{l+s+sa}{f}\PYG{l+s+s2}{\PYGZdq{}}\PYG{l+s+s2}{State 1 Enthalpy (kJ/kg): }\PYG{l+s+si}{\PYGZob{}}\PYG{n}{h1\PYGZus{}iir}\PYG{+w}{ }\PYG{o}{/}\PYG{+w}{ }\PYG{l+m+mi}{1000}\PYG{l+s+si}{:}\PYG{l+s+s2}{.3f}\PYG{l+s+si}{\PYGZcb{}}\PYG{l+s+s2}{\PYGZdq{}}\PYG{p}{)}
\PYG{n+nb}{print}\PYG{p}{(}\PYG{l+s+sa}{f}\PYG{l+s+s2}{\PYGZdq{}}\PYG{l+s+s2}{State 1 Entropy (kJ/kgK): }\PYG{l+s+si}{\PYGZob{}}\PYG{n}{s1\PYGZus{}iir}\PYG{+w}{ }\PYG{o}{/}\PYG{+w}{ }\PYG{l+m+mi}{1000}\PYG{l+s+si}{:}\PYG{l+s+s2}{.3f}\PYG{l+s+si}{\PYGZcb{}}\PYG{l+s+s2}{\PYGZdq{}}\PYG{p}{)}
\PYG{n+nb}{print}\PYG{p}{(}\PYG{l+s+sa}{f}\PYG{l+s+s2}{\PYGZdq{}}\PYG{l+s+s2}{State 2 Enthalpy (kJ/kg): }\PYG{l+s+si}{\PYGZob{}}\PYG{n}{h2\PYGZus{}iir}\PYG{+w}{ }\PYG{o}{/}\PYG{+w}{ }\PYG{l+m+mi}{1000}\PYG{l+s+si}{:}\PYG{l+s+s2}{.3f}\PYG{l+s+si}{\PYGZcb{}}\PYG{l+s+s2}{\PYGZdq{}}\PYG{p}{)}
\PYG{n+nb}{print}\PYG{p}{(}\PYG{l+s+sa}{f}\PYG{l+s+s2}{\PYGZdq{}}\PYG{l+s+s2}{State 2 Entropy (kJ/kgK): }\PYG{l+s+si}{\PYGZob{}}\PYG{n}{s2\PYGZus{}iir}\PYG{+w}{ }\PYG{o}{/}\PYG{+w}{ }\PYG{l+m+mi}{1000}\PYG{l+s+si}{:}\PYG{l+s+s2}{.3f}\PYG{l+s+si}{\PYGZcb{}}\PYG{l+s+s2}{\PYGZdq{}}\PYG{p}{)}
\PYG{n+nb}{print}\PYG{p}{(}\PYG{l+s+sa}{f}\PYG{l+s+s2}{\PYGZdq{}}\PYG{l+s+s2}{Enthalpy Difference (kJ/kg): }\PYG{l+s+si}{\PYGZob{}}\PYG{n}{dh\PYGZus{}iir}\PYG{l+s+si}{:}\PYG{l+s+s2}{.3f}\PYG{l+s+si}{\PYGZcb{}}\PYG{l+s+s2}{\PYGZdq{}}\PYG{p}{)}
\PYG{n+nb}{print}\PYG{p}{(}\PYG{l+s+sa}{f}\PYG{l+s+s2}{\PYGZdq{}}\PYG{l+s+s2}{Entropy Difference (kJ/kgK): }\PYG{l+s+si}{\PYGZob{}}\PYG{n}{ds\PYGZus{}iir}\PYG{l+s+si}{:}\PYG{l+s+s2}{.3f}\PYG{l+s+si}{\PYGZcb{}}\PYG{l+s+se}{\PYGZbs{}n}\PYG{l+s+s2}{\PYGZdq{}}\PYG{p}{)}

\PYG{n+nb}{print}\PYG{p}{(}\PYG{l+s+s2}{\PYGZdq{}}\PYG{l+s+s2}{NBP Reference State:}\PYG{l+s+s2}{\PYGZdq{}}\PYG{p}{)}
\PYG{n+nb}{print}\PYG{p}{(}\PYG{l+s+sa}{f}\PYG{l+s+s2}{\PYGZdq{}}\PYG{l+s+s2}{State 1 Enthalpy (kJ/kg): }\PYG{l+s+si}{\PYGZob{}}\PYG{n}{h1\PYGZus{}nbp}\PYG{+w}{ }\PYG{o}{/}\PYG{+w}{ }\PYG{l+m+mi}{1000}\PYG{l+s+si}{:}\PYG{l+s+s2}{.3f}\PYG{l+s+si}{\PYGZcb{}}\PYG{l+s+s2}{\PYGZdq{}}\PYG{p}{)}
\PYG{n+nb}{print}\PYG{p}{(}\PYG{l+s+sa}{f}\PYG{l+s+s2}{\PYGZdq{}}\PYG{l+s+s2}{State 1 Entropy (kJ/kgK): }\PYG{l+s+si}{\PYGZob{}}\PYG{n}{s1\PYGZus{}nbp}\PYG{+w}{ }\PYG{o}{/}\PYG{+w}{ }\PYG{l+m+mi}{1000}\PYG{l+s+si}{:}\PYG{l+s+s2}{.3f}\PYG{l+s+si}{\PYGZcb{}}\PYG{l+s+s2}{\PYGZdq{}}\PYG{p}{)}
\PYG{n+nb}{print}\PYG{p}{(}\PYG{l+s+sa}{f}\PYG{l+s+s2}{\PYGZdq{}}\PYG{l+s+s2}{State 2 Enthalpy (kJ/kg): }\PYG{l+s+si}{\PYGZob{}}\PYG{n}{h2\PYGZus{}nbp}\PYG{+w}{ }\PYG{o}{/}\PYG{+w}{ }\PYG{l+m+mi}{1000}\PYG{l+s+si}{:}\PYG{l+s+s2}{.3f}\PYG{l+s+si}{\PYGZcb{}}\PYG{l+s+s2}{\PYGZdq{}}\PYG{p}{)}
\PYG{n+nb}{print}\PYG{p}{(}\PYG{l+s+sa}{f}\PYG{l+s+s2}{\PYGZdq{}}\PYG{l+s+s2}{State 2 Entropy (kJ/kgK): }\PYG{l+s+si}{\PYGZob{}}\PYG{n}{s2\PYGZus{}nbp}\PYG{+w}{ }\PYG{o}{/}\PYG{+w}{ }\PYG{l+m+mi}{1000}\PYG{l+s+si}{:}\PYG{l+s+s2}{.3f}\PYG{l+s+si}{\PYGZcb{}}\PYG{l+s+s2}{\PYGZdq{}}\PYG{p}{)}
\PYG{n+nb}{print}\PYG{p}{(}\PYG{l+s+sa}{f}\PYG{l+s+s2}{\PYGZdq{}}\PYG{l+s+s2}{Enthalpy Difference (kJ/kg): }\PYG{l+s+si}{\PYGZob{}}\PYG{n}{dh\PYGZus{}nbp}\PYG{l+s+si}{:}\PYG{l+s+s2}{.3f}\PYG{l+s+si}{\PYGZcb{}}\PYG{l+s+s2}{\PYGZdq{}}\PYG{p}{)}
\PYG{n+nb}{print}\PYG{p}{(}\PYG{l+s+sa}{f}\PYG{l+s+s2}{\PYGZdq{}}\PYG{l+s+s2}{Entropy Difference (kJ/kgK): }\PYG{l+s+si}{\PYGZob{}}\PYG{n}{ds\PYGZus{}nbp}\PYG{l+s+si}{:}\PYG{l+s+s2}{.3f}\PYG{l+s+si}{\PYGZcb{}}\PYG{l+s+se}{\PYGZbs{}n}\PYG{l+s+s2}{\PYGZdq{}}\PYG{p}{)}

\PYG{n+nb}{print}\PYG{p}{(}\PYG{l+s+s2}{\PYGZdq{}}\PYG{l+s+s2}{ASHRAE Reference State:}\PYG{l+s+s2}{\PYGZdq{}}\PYG{p}{)}
\PYG{n+nb}{print}\PYG{p}{(}\PYG{l+s+sa}{f}\PYG{l+s+s2}{\PYGZdq{}}\PYG{l+s+s2}{State 1 Enthalpy (kJ/kg): }\PYG{l+s+si}{\PYGZob{}}\PYG{n}{h1\PYGZus{}ashrae}\PYG{+w}{ }\PYG{o}{/}\PYG{+w}{ }\PYG{l+m+mi}{1000}\PYG{l+s+si}{:}\PYG{l+s+s2}{.3f}\PYG{l+s+si}{\PYGZcb{}}\PYG{l+s+s2}{\PYGZdq{}}\PYG{p}{)}
\PYG{n+nb}{print}\PYG{p}{(}\PYG{l+s+sa}{f}\PYG{l+s+s2}{\PYGZdq{}}\PYG{l+s+s2}{State 1 Entropy (kJ/kgK): }\PYG{l+s+si}{\PYGZob{}}\PYG{n}{s1\PYGZus{}ashrae}\PYG{+w}{ }\PYG{o}{/}\PYG{+w}{ }\PYG{l+m+mi}{1000}\PYG{l+s+si}{:}\PYG{l+s+s2}{.3f}\PYG{l+s+si}{\PYGZcb{}}\PYG{l+s+s2}{\PYGZdq{}}\PYG{p}{)}
\PYG{n+nb}{print}\PYG{p}{(}\PYG{l+s+sa}{f}\PYG{l+s+s2}{\PYGZdq{}}\PYG{l+s+s2}{State 2 Enthalpy (kJ/kg): }\PYG{l+s+si}{\PYGZob{}}\PYG{n}{h2\PYGZus{}ashrae}\PYG{+w}{ }\PYG{o}{/}\PYG{+w}{ }\PYG{l+m+mi}{1000}\PYG{l+s+si}{:}\PYG{l+s+s2}{.3f}\PYG{l+s+si}{\PYGZcb{}}\PYG{l+s+s2}{\PYGZdq{}}\PYG{p}{)}
\PYG{n+nb}{print}\PYG{p}{(}\PYG{l+s+sa}{f}\PYG{l+s+s2}{\PYGZdq{}}\PYG{l+s+s2}{State 2 Entropy (kJ/kgK): }\PYG{l+s+si}{\PYGZob{}}\PYG{n}{s2\PYGZus{}ashrae}\PYG{+w}{ }\PYG{o}{/}\PYG{+w}{ }\PYG{l+m+mi}{1000}\PYG{l+s+si}{:}\PYG{l+s+s2}{.3f}\PYG{l+s+si}{\PYGZcb{}}\PYG{l+s+s2}{\PYGZdq{}}\PYG{p}{)}
\PYG{n+nb}{print}\PYG{p}{(}\PYG{l+s+sa}{f}\PYG{l+s+s2}{\PYGZdq{}}\PYG{l+s+s2}{Enthalpy Difference (kJ/kg): }\PYG{l+s+si}{\PYGZob{}}\PYG{n}{dh\PYGZus{}ashrae}\PYG{l+s+si}{:}\PYG{l+s+s2}{.3f}\PYG{l+s+si}{\PYGZcb{}}\PYG{l+s+s2}{\PYGZdq{}}\PYG{p}{)}
\PYG{n+nb}{print}\PYG{p}{(}\PYG{l+s+sa}{f}\PYG{l+s+s2}{\PYGZdq{}}\PYG{l+s+s2}{Entropy Difference (kJ/kgK): }\PYG{l+s+si}{\PYGZob{}}\PYG{n}{ds\PYGZus{}ashrae}\PYG{l+s+si}{:}\PYG{l+s+s2}{.3f}\PYG{l+s+si}{\PYGZcb{}}\PYG{l+s+se}{\PYGZbs{}n}\PYG{l+s+s2}{\PYGZdq{}}\PYG{p}{)}
\end{sphinxVerbatim}

\end{sphinxuseclass}\end{sphinxVerbatimInput}
\begin{sphinxVerbatimOutput}

\begin{sphinxuseclass}{cell_output}
\begin{sphinxVerbatim}[commandchars=\\\{\}]
Properties at T = 280 K and P = 1 bar
IIR Reference State:
State 1 Enthalpy (kJ/kg): 409.317
State 1 Entropy (kJ/kgK): 1.848
State 2 Enthalpy (kJ/kg): 424.282
State 2 Entropy (kJ/kgK): 1.846
Enthalpy Difference (kJ/kg): 14.965
Entropy Difference (kJ/kgK): \PYGZhy{}0.002

NBP Reference State:
State 1 Enthalpy (kJ/kg): 243.507
State 1 Entropy (kJ/kgK): 0.979
State 2 Enthalpy (kJ/kg): 258.472
State 2 Entropy (kJ/kgK): 0.977
Enthalpy Difference (kJ/kg): 14.965
Entropy Difference (kJ/kgK): \PYGZhy{}0.002

ASHRAE Reference State:
State 1 Enthalpy (kJ/kg): 261.173
State 1 Entropy (kJ/kgK): 1.052
State 2 Enthalpy (kJ/kg): 276.138
State 2 Entropy (kJ/kgK): 1.050
Enthalpy Difference (kJ/kg): 14.965
Entropy Difference (kJ/kgK): \PYGZhy{}0.002
\end{sphinxVerbatim}

\end{sphinxuseclass}\end{sphinxVerbatimOutput}

\end{sphinxuseclass}

\begin{savenotes}\sphinxattablestart
\centering
\begin{tabulary}{\linewidth}[t]{|T|T|T|T|T|T|}
\hline
\sphinxstyletheadfamily 
\sphinxAtStartPar
Reference
&\sphinxstyletheadfamily 
\sphinxAtStartPar
State
&\sphinxstyletheadfamily 
\sphinxAtStartPar
Enthalpy (kJ/kg)
&\sphinxstyletheadfamily 
\sphinxAtStartPar
Entropy (kJ/(kg·K))
&\sphinxstyletheadfamily 
\sphinxAtStartPar
ΔH (kJ/kg)
&\sphinxstyletheadfamily 
\sphinxAtStartPar
ΔS (kJ/(kg·K))
\\
\hline
\sphinxAtStartPar
IIR
&
\sphinxAtStartPar
1(280 K, 1bar)
&
\sphinxAtStartPar
409.317
&
\sphinxAtStartPar
1.848
&
\sphinxAtStartPar

&
\sphinxAtStartPar

\\
\hline
\sphinxAtStartPar
IIR
&
\sphinxAtStartPar
2
&
\sphinxAtStartPar
424.282
&
\sphinxAtStartPar
1.846
&
\sphinxAtStartPar
14.965
&
\sphinxAtStartPar
\sphinxhyphen{}0.002
\\
\hline
\sphinxAtStartPar
NBP
&
\sphinxAtStartPar
1
&
\sphinxAtStartPar
243.507
&
\sphinxAtStartPar
0.979
&
\sphinxAtStartPar

&
\sphinxAtStartPar

\\
\hline
\sphinxAtStartPar
NBP
&
\sphinxAtStartPar
2
&
\sphinxAtStartPar
258.472
&
\sphinxAtStartPar
0.977
&
\sphinxAtStartPar
14.965
&
\sphinxAtStartPar
\sphinxhyphen{}0.002
\\
\hline
\sphinxAtStartPar
ASHRAE
&
\sphinxAtStartPar
1
&
\sphinxAtStartPar
261.173
&
\sphinxAtStartPar
1.052
&
\sphinxAtStartPar

&
\sphinxAtStartPar

\\
\hline
\sphinxAtStartPar
ASHRAE
&
\sphinxAtStartPar
2
&
\sphinxAtStartPar
276.138
&
\sphinxAtStartPar
1.050
&
\sphinxAtStartPar
14.965
&
\sphinxAtStartPar
\sphinxhyphen{}0.002
\\
\hline
\end{tabulary}
\par
\sphinxattableend\end{savenotes}

\begin{sphinxuseclass}{cell}\begin{sphinxVerbatimInput}

\begin{sphinxuseclass}{cell_input}
\begin{sphinxVerbatim}[commandchars=\\\{\}]
\PYG{k+kn}{import} \PYG{n+nn}{numpy} \PYG{k}{as} \PYG{n+nn}{np}

\PYG{c+c1}{\PYGZsh{} Arrays of known points}
\PYG{n}{x\PYGZus{}points} \PYG{o}{=} \PYG{n}{np}\PYG{o}{.}\PYG{n}{array}\PYG{p}{(}\PYG{p}{[}\PYG{l+m+mi}{0}\PYG{p}{,} \PYG{l+m+mi}{10}\PYG{p}{]}\PYG{p}{)}
\PYG{n}{y\PYGZus{}points} \PYG{o}{=} \PYG{n}{np}\PYG{o}{.}\PYG{n}{array}\PYG{p}{(}\PYG{p}{[}\PYG{l+m+mf}{1.0333}\PYG{p}{,} \PYG{l+m+mf}{1.0628}\PYG{p}{]}\PYG{p}{)}

\PYG{c+c1}{\PYGZsh{} Point to interpolate}
\PYG{n}{x}\PYG{o}{=} \PYG{l+m+mi}{280} \PYG{o}{\PYGZhy{}} \PYG{l+m+mf}{273.15}
\PYG{c+c1}{\PYGZsh{} Perform interpolation}
\PYG{n}{y} \PYG{o}{=} \PYG{n}{np}\PYG{o}{.}\PYG{n}{interp}\PYG{p}{(}\PYG{n}{x}\PYG{p}{,} \PYG{n}{x\PYGZus{}points}\PYG{p}{,} \PYG{n}{y\PYGZus{}points}\PYG{p}{)}

\PYG{n+nb}{print}\PYG{p}{(}\PYG{l+s+s2}{\PYGZdq{}}\PYG{l+s+s2}{Interpolated value at x = }\PYG{l+s+si}{\PYGZob{}\PYGZcb{}}\PYG{l+s+s2}{: y = }\PYG{l+s+si}{\PYGZob{}\PYGZcb{}}\PYG{l+s+s2}{\PYGZdq{}}\PYG{o}{.}\PYG{n}{format}\PYG{p}{(}\PYG{n+nb}{round}\PYG{p}{(}\PYG{n}{x}\PYG{p}{,}\PYG{l+m+mi}{3}\PYG{p}{)}\PYG{p}{,} \PYG{n+nb}{round}\PYG{p}{(}\PYG{n}{y}\PYG{p}{,}\PYG{l+m+mi}{3}\PYG{p}{)}\PYG{p}{)}\PYG{p}{)}
\end{sphinxVerbatim}

\end{sphinxuseclass}\end{sphinxVerbatimInput}
\begin{sphinxVerbatimOutput}

\begin{sphinxuseclass}{cell_output}
\begin{sphinxVerbatim}[commandchars=\\\{\}]
Interpolated value at x = 6.85: y = 1.054
\end{sphinxVerbatim}

\end{sphinxuseclass}\end{sphinxVerbatimOutput}

\end{sphinxuseclass}
\begin{sphinxuseclass}{cell}\begin{sphinxVerbatimInput}

\begin{sphinxuseclass}{cell_input}
\begin{sphinxVerbatim}[commandchars=\\\{\}]
\PYG{k+kn}{import} \PYG{n+nn}{numpy} \PYG{k}{as} \PYG{n+nn}{np}

\PYG{c+c1}{\PYGZsh{} Arrays of known points}
\PYG{n}{x\PYGZus{}points} \PYG{o}{=} \PYG{n}{np}\PYG{o}{.}\PYG{n}{array}\PYG{p}{(}\PYG{p}{[}\PYG{l+m+mi}{20}\PYG{p}{,} \PYG{l+m+mi}{30}\PYG{p}{]}\PYG{p}{)}
\PYG{n}{y\PYGZus{}points} \PYG{o}{=} \PYG{n}{np}\PYG{o}{.}\PYG{n}{array}\PYG{p}{(}\PYG{p}{[}\PYG{l+m+mf}{270.20}\PYG{p}{,} \PYG{l+m+mf}{278.91}\PYG{p}{]}\PYG{p}{)}

\PYG{c+c1}{\PYGZsh{} Point to interpolate}
\PYG{n}{x}\PYG{o}{=} \PYG{l+m+mi}{300} \PYG{o}{\PYGZhy{}} \PYG{l+m+mf}{273.15}
\PYG{c+c1}{\PYGZsh{} Perform interpolation}
\PYG{n}{y} \PYG{o}{=} \PYG{n}{np}\PYG{o}{.}\PYG{n}{interp}\PYG{p}{(}\PYG{n}{x}\PYG{p}{,} \PYG{n}{x\PYGZus{}points}\PYG{p}{,} \PYG{n}{y\PYGZus{}points}\PYG{p}{)}

\PYG{n+nb}{print}\PYG{p}{(}\PYG{l+s+s2}{\PYGZdq{}}\PYG{l+s+s2}{Interpolated value at x = }\PYG{l+s+si}{\PYGZob{}\PYGZcb{}}\PYG{l+s+s2}{: y = }\PYG{l+s+si}{\PYGZob{}\PYGZcb{}}\PYG{l+s+s2}{\PYGZdq{}}\PYG{o}{.}\PYG{n}{format}\PYG{p}{(}\PYG{n+nb}{round}\PYG{p}{(}\PYG{n}{x}\PYG{p}{,}\PYG{l+m+mi}{2}\PYG{p}{)}\PYG{p}{,} \PYG{n+nb}{round}\PYG{p}{(}\PYG{n}{y}\PYG{p}{,}\PYG{l+m+mi}{2}\PYG{p}{)}\PYG{p}{)}\PYG{p}{)}
\end{sphinxVerbatim}

\end{sphinxuseclass}\end{sphinxVerbatimInput}
\begin{sphinxVerbatimOutput}

\begin{sphinxuseclass}{cell_output}
\begin{sphinxVerbatim}[commandchars=\\\{\}]
Interpolated value at x = 26.85: y = 276.17
\end{sphinxVerbatim}

\end{sphinxuseclass}\end{sphinxVerbatimOutput}

\end{sphinxuseclass}
\sphinxstepscope


\section{Comparison of work done in different processes}
\label{\detokenize{notebooks/Chapter4/Polytropic_Processes_Problem_Extended:comparison-of-work-done-in-different-processes}}\label{\detokenize{notebooks/Chapter4/Polytropic_Processes_Problem_Extended::doc}}

\subsection{Problem Statement}
\label{\detokenize{notebooks/Chapter4/Polytropic_Processes_Problem_Extended:problem-statement}}
\sphinxAtStartPar
Consider an ideal gas undergoing a polytropic process. At the initial state, the pressure (P\_1 = 200 kPa) and specific volume (v\_1 = 0.05	m3/kg). At the final state, the specific volume (v\_2 = 0.1 , 	ext\{m\}\textasciicircum{}3/	ext\{kg\}). Analyze the process for polytropic exponents (n = 1.3) and (n = 1.0) (isothermal process).
\begin{enumerate}
\sphinxsetlistlabels{\arabic}{enumi}{enumii}{}{.}%
\item {} 
\sphinxAtStartPar
Sketch the two processes on a P\sphinxhyphen{}v diagram. Which process has a larger specific boundary work?

\item {} 
\sphinxAtStartPar
Calculate the specific boundary work for both processes.

\item {} 
\sphinxAtStartPar
Compare the work for an isobaric (n=0), n=1 (isothermal), and n=1.3 polytropic processes.

\end{enumerate}


\subsection{Solution}
\label{\detokenize{notebooks/Chapter4/Polytropic_Processes_Problem_Extended:solution}}
\begin{sphinxuseclass}{cell}\begin{sphinxVerbatimInput}

\begin{sphinxuseclass}{cell_input}
\begin{sphinxVerbatim}[commandchars=\\\{\}]
\PYG{k+kn}{import} \PYG{n+nn}{matplotlib}\PYG{n+nn}{.}\PYG{n+nn}{pyplot} \PYG{k}{as} \PYG{n+nn}{plt}
\PYG{k+kn}{import} \PYG{n+nn}{numpy} \PYG{k}{as} \PYG{n+nn}{np}
\PYG{c+c1}{\PYGZsh{} Given values}
\PYG{n}{P1} \PYG{o}{=} \PYG{l+m+mi}{200}  \PYG{c+c1}{\PYGZsh{} kPa}
\PYG{n}{v1} \PYG{o}{=} \PYG{l+m+mf}{0.05}  \PYG{c+c1}{\PYGZsh{} m\PYGZca{}3/kg}
\PYG{n}{v2} \PYG{o}{=} \PYG{l+m+mf}{0.1}  \PYG{c+c1}{\PYGZsh{} m\PYGZca{}3/kg}
\PYG{n}{n\PYGZus{}values} \PYG{o}{=} \PYG{p}{[}\PYG{l+m+mi}{0}\PYG{p}{,} \PYG{l+m+mf}{1.0}\PYG{p}{,} \PYG{l+m+mf}{1.3}\PYG{p}{]}  \PYG{c+c1}{\PYGZsh{} Polytropic exponents}

\PYG{k}{def} \PYG{n+nf}{polytropic\PYGZus{}process}\PYG{p}{(}\PYG{n}{P1}\PYG{p}{,} \PYG{n}{v1}\PYG{p}{,} \PYG{n}{v2}\PYG{p}{,} \PYG{n}{n}\PYG{p}{)}\PYG{p}{:}
    \PYG{k}{if} \PYG{n}{n} \PYG{o}{==} \PYG{l+m+mi}{0}\PYG{p}{:}  \PYG{c+c1}{\PYGZsh{} Isobaric}
        \PYG{n}{P2} \PYG{o}{=} \PYG{n}{P1}
    \PYG{k}{else}\PYG{p}{:}  \PYG{c+c1}{\PYGZsh{} Polytropic or Isothermal}
        \PYG{n}{P2} \PYG{o}{=} \PYG{n}{P1} \PYG{o}{*} \PYG{p}{(}\PYG{n}{v1} \PYG{o}{/} \PYG{n}{v2}\PYG{p}{)}\PYG{o}{*}\PYG{o}{*}\PYG{n}{n}
    \PYG{k}{return} \PYG{n}{P2}

\PYG{k}{def} \PYG{n+nf}{specific\PYGZus{}work}\PYG{p}{(}\PYG{n}{P1}\PYG{p}{,} \PYG{n}{v1}\PYG{p}{,} \PYG{n}{v2}\PYG{p}{,} \PYG{n}{n}\PYG{p}{)}\PYG{p}{:}
    \PYG{k}{if} \PYG{n}{n} \PYG{o}{==} \PYG{l+m+mi}{0}\PYG{p}{:}  \PYG{c+c1}{\PYGZsh{} Isobaric}
        \PYG{k}{return} \PYG{n}{P1} \PYG{o}{*} \PYG{p}{(}\PYG{n}{v2} \PYG{o}{\PYGZhy{}} \PYG{n}{v1}\PYG{p}{)}
    \PYG{k}{elif} \PYG{n}{n} \PYG{o}{==} \PYG{l+m+mi}{1}\PYG{p}{:}  \PYG{c+c1}{\PYGZsh{} Isothermal}
        \PYG{k}{return} \PYG{n}{P1} \PYG{o}{*} \PYG{n}{v1} \PYG{o}{*} \PYG{n}{np}\PYG{o}{.}\PYG{n}{log}\PYG{p}{(}\PYG{n}{v2} \PYG{o}{/} \PYG{n}{v1}\PYG{p}{)}
    \PYG{k}{else}\PYG{p}{:}  \PYG{c+c1}{\PYGZsh{} Polytropic}
        \PYG{k}{return} \PYG{p}{(}\PYG{n}{P1} \PYG{o}{*} \PYG{n}{v1} \PYG{o}{\PYGZhy{}} \PYG{n}{polytropic\PYGZus{}process}\PYG{p}{(}\PYG{n}{P1}\PYG{p}{,} \PYG{n}{v1}\PYG{p}{,} \PYG{n}{v2}\PYG{p}{,} \PYG{n}{n}\PYG{p}{)} \PYG{o}{*} \PYG{n}{v2}\PYG{p}{)} \PYG{o}{/} \PYG{p}{(}\PYG{l+m+mi}{1} \PYG{o}{\PYGZhy{}} \PYG{n}{n}\PYG{p}{)}

\PYG{c+c1}{\PYGZsh{} Plotting the P\PYGZhy{}v diagram}
\PYG{n}{plt}\PYG{o}{.}\PYG{n}{figure}\PYG{p}{(}\PYG{n}{figsize}\PYG{o}{=}\PYG{p}{(}\PYG{l+m+mi}{10}\PYG{p}{,} \PYG{l+m+mi}{6}\PYG{p}{)}\PYG{p}{)}
\PYG{n}{v} \PYG{o}{=} \PYG{n}{np}\PYG{o}{.}\PYG{n}{linspace}\PYG{p}{(}\PYG{n}{v1}\PYG{p}{,} \PYG{n}{v2}\PYG{p}{,} \PYG{l+m+mi}{100}\PYG{p}{)}
\PYG{k}{for} \PYG{n}{n} \PYG{o+ow}{in} \PYG{n}{n\PYGZus{}values}\PYG{p}{:}
    \PYG{n}{P} \PYG{o}{=} \PYG{p}{[}\PYG{n}{polytropic\PYGZus{}process}\PYG{p}{(}\PYG{n}{P1}\PYG{p}{,} \PYG{n}{v1}\PYG{p}{,} \PYG{n}{vi}\PYG{p}{,} \PYG{n}{n}\PYG{p}{)} \PYG{k}{for} \PYG{n}{vi} \PYG{o+ow}{in} \PYG{n}{v}\PYG{p}{]}
    \PYG{n}{plt}\PYG{o}{.}\PYG{n}{plot}\PYG{p}{(}\PYG{n}{v}\PYG{p}{,} \PYG{n}{P}\PYG{p}{,} \PYG{n}{label}\PYG{o}{=}\PYG{l+s+sa}{f}\PYG{l+s+s1}{\PYGZsq{}}\PYG{l+s+s1}{n = }\PYG{l+s+si}{\PYGZob{}}\PYG{n}{n}\PYG{l+s+si}{\PYGZcb{}}\PYG{l+s+s1}{\PYGZsq{}}\PYG{p}{)}
    \PYG{n}{work} \PYG{o}{=} \PYG{n}{specific\PYGZus{}work}\PYG{p}{(}\PYG{n}{P1}\PYG{p}{,} \PYG{n}{v1}\PYG{p}{,} \PYG{n}{v2}\PYG{p}{,} \PYG{n}{n}\PYG{p}{)}
    \PYG{n+nb}{print}\PYG{p}{(}\PYG{l+s+sa}{f}\PYG{l+s+s1}{\PYGZsq{}}\PYG{l+s+s1}{Specific boundary work for n = }\PYG{l+s+si}{\PYGZob{}}\PYG{n}{n}\PYG{l+s+si}{\PYGZcb{}}\PYG{l+s+s1}{: }\PYG{l+s+si}{\PYGZob{}}\PYG{n}{work}\PYG{l+s+si}{:}\PYG{l+s+s1}{.1f}\PYG{l+s+si}{\PYGZcb{}}\PYG{l+s+s1}{ kJ/kg}\PYG{l+s+s1}{\PYGZsq{}}\PYG{p}{)}

\PYG{n}{plt}\PYG{o}{.}\PYG{n}{xlabel}\PYG{p}{(}\PYG{l+s+s1}{\PYGZsq{}}\PYG{l+s+s1}{Specific Volume (m\PYGZca{}3/kg)}\PYG{l+s+s1}{\PYGZsq{}}\PYG{p}{)}
\PYG{n}{plt}\PYG{o}{.}\PYG{n}{ylabel}\PYG{p}{(}\PYG{l+s+s1}{\PYGZsq{}}\PYG{l+s+s1}{Pressure (kPa)}\PYG{l+s+s1}{\PYGZsq{}}\PYG{p}{)}
\PYG{n}{plt}\PYG{o}{.}\PYG{n}{title}\PYG{p}{(}\PYG{l+s+s1}{\PYGZsq{}}\PYG{l+s+s1}{P\PYGZhy{}v Diagram for Polytropic Processes}\PYG{l+s+s1}{\PYGZsq{}}\PYG{p}{)}
\PYG{n}{plt}\PYG{o}{.}\PYG{n}{legend}\PYG{p}{(}\PYG{p}{)}
\PYG{n}{plt}\PYG{o}{.}\PYG{n}{grid}\PYG{p}{(}\PYG{k+kc}{True}\PYG{p}{)}
\PYG{n}{plt}\PYG{o}{.}\PYG{n}{show}\PYG{p}{(}\PYG{p}{)}
\end{sphinxVerbatim}

\end{sphinxuseclass}\end{sphinxVerbatimInput}
\begin{sphinxVerbatimOutput}

\begin{sphinxuseclass}{cell_output}
\begin{sphinxVerbatim}[commandchars=\\\{\}]
Specific boundary work for n = 0: 10.0 kJ/kg
Specific boundary work for n = 1.0: 6.9 kJ/kg
Specific boundary work for n = 1.3: \PYGZhy{}6.3 kJ/kg
\end{sphinxVerbatim}

\noindent\sphinxincludegraphics{{66c205a7c192a6abe97eb904a66a1cb301a2c0d14e308b34d7eb177c5c6bb54d}.png}

\end{sphinxuseclass}\end{sphinxVerbatimOutput}

\end{sphinxuseclass}
\sphinxstepscope


\chapter{5. The First Law of Thermodynamics for a Control Volume}
\label{\detokenize{notebooks/Chapter5/chapter5:the-first-law-of-thermodynamics-for-a-control-volume}}\label{\detokenize{notebooks/Chapter5/chapter5::doc}}
\sphinxAtStartPar
Practice problems on \sphinxhref{https://pressbooks.bccampus.ca/thermo1/chapter/5-0-chapter-introduction-and-learning-objectives/}{Chapter 5} from Prof. Claire Yu Yan’s textbook.

\sphinxstepscope


\section{Energy balance for Stream in Control Volume}
\label{\detokenize{notebooks/Chapter5/CH5-Q1:energy-balance-for-stream-in-control-volume}}\label{\detokenize{notebooks/Chapter5/CH5-Q1::doc}}
\sphinxAtStartPar
1kg of steam is confined in a space of the size \(1\:m^3\) where the thermometer measures \(400 ^{\circ}  C\). Using CoolProp as a tool to extract thrmodynamic properties, determine:

\sphinxAtStartPar
a)the pressure reading on a pressure guage at atmospheric pressure

\sphinxAtStartPar
the steam is then heated to double the pressure reading while the space is kept from expanding. determine:

\sphinxAtStartPar
b)the temperature of steam after heating

\sphinxAtStartPar
c)how much heat is require to do so

\sphinxAtStartPar
d)what would be the asnwer if \(C_v\) is used instead of CoolProp? What assumptions are made?


\subsection{Solution Approach for a)}
\label{\detokenize{notebooks/Chapter5/CH5-Q1:solution-approach-for-a}}
\sphinxAtStartPar
Density \(D\) and temperature \(T\) are the two known parameters that can be used to extract properties.

\sphinxAtStartPar
\(D=m/V\)

\begin{sphinxuseclass}{cell}\begin{sphinxVerbatimInput}

\begin{sphinxuseclass}{cell_input}
\begin{sphinxVerbatim}[commandchars=\\\{\}]
\PYG{c+c1}{\PYGZsh{}importing the required library}
\PYG{k+kn}{import} \PYG{n+nn}{CoolProp}\PYG{n+nn}{.}\PYG{n+nn}{CoolProp} \PYG{k}{as} \PYG{n+nn}{CP}

\PYG{c+c1}{\PYGZsh{} define the given inputs:}
\PYG{n}{T} \PYG{o}{=} \PYG{l+m+mi}{400} \PYG{o}{+} \PYG{l+m+mf}{273.15} \PYG{c+c1}{\PYGZsh{}temperature in K}
\PYG{n}{D} \PYG{o}{=} \PYG{l+m+mi}{1} \PYG{o}{/} \PYG{l+m+mi}{1} \PYG{c+c1}{\PYGZsh{}density in kg/m3}
\PYG{n}{P} \PYG{o}{=} \PYG{n}{CP}\PYG{o}{.}\PYG{n}{PropsSI}\PYG{p}{(}\PYG{l+s+s1}{\PYGZsq{}}\PYG{l+s+s1}{P}\PYG{l+s+s1}{\PYGZsq{}}\PYG{p}{,} \PYG{l+s+s1}{\PYGZsq{}}\PYG{l+s+s1}{D}\PYG{l+s+s1}{\PYGZsq{}}\PYG{p}{,} \PYG{n}{D}\PYG{p}{,} \PYG{l+s+s1}{\PYGZsq{}}\PYG{l+s+s1}{T}\PYG{l+s+s1}{\PYGZsq{}}\PYG{p}{,} \PYG{n}{T}\PYG{p}{,} \PYG{l+s+s1}{\PYGZsq{}}\PYG{l+s+s1}{Water}\PYG{l+s+s1}{\PYGZsq{}}\PYG{p}{)} \PYG{c+c1}{\PYGZsh{}calculating pressure using coolprop Pa}
\PYG{n}{P\PYGZus{}g} \PYG{o}{=} \PYG{n}{P} \PYG{o}{\PYGZhy{}} \PYG{l+m+mi}{101325} \PYG{c+c1}{\PYGZsh{}gauge pressure in Pa}
\PYG{n+nb}{print}\PYG{p}{(}\PYG{l+s+s1}{\PYGZsq{}}\PYG{l+s+s1}{The gauage pressure of steam at the given initial properties is}\PYG{l+s+s1}{\PYGZsq{}}\PYG{p}{,}\PYG{l+s+sa}{f}\PYG{l+s+s2}{\PYGZdq{}}\PYG{l+s+si}{\PYGZob{}}\PYG{n}{P\PYGZus{}g}\PYG{l+s+si}{:}\PYG{l+s+s2}{.1f}\PYG{l+s+si}{\PYGZcb{}}\PYG{l+s+s2}{\PYGZdq{}}\PYG{p}{,}\PYG{l+s+s1}{\PYGZsq{}}\PYG{l+s+s1}{Pa}\PYG{l+s+s1}{\PYGZsq{}}\PYG{p}{)}
\end{sphinxVerbatim}

\end{sphinxuseclass}\end{sphinxVerbatimInput}
\begin{sphinxVerbatimOutput}

\begin{sphinxuseclass}{cell_output}
\begin{sphinxVerbatim}[commandchars=\\\{\}]
The gauage pressure of steam at the given initial properties is 208099.6 Pa
\end{sphinxVerbatim}

\end{sphinxuseclass}\end{sphinxVerbatimOutput}

\end{sphinxuseclass}

\subsection{Solution Approach for b)}
\label{\detokenize{notebooks/Chapter5/CH5-Q1:solution-approach-for-b}}
\sphinxAtStartPar
The known parameters from the secondary state are pressure

\sphinxAtStartPar
\(P_{g2} = 2  P_g\)

\sphinxAtStartPar
and density

\sphinxAtStartPar
\(D_2 = D\)

\sphinxAtStartPar
since the mass and the volume of the space remains constant

\begin{sphinxuseclass}{cell}\begin{sphinxVerbatimInput}

\begin{sphinxuseclass}{cell_input}
\begin{sphinxVerbatim}[commandchars=\\\{\}]
\PYG{c+c1}{\PYGZsh{} define the inputs:}
\PYG{n}{P\PYGZus{}g2} \PYG{o}{=} \PYG{n}{P\PYGZus{}g} \PYG{o}{*} \PYG{l+m+mi}{2} \PYG{c+c1}{\PYGZsh{}gauge pressure in secondary state in Pa}
\PYG{n}{P\PYGZus{}2} \PYG{o}{=} \PYG{n}{P\PYGZus{}g2} \PYG{o}{+} \PYG{l+m+mi}{101325} \PYG{c+c1}{\PYGZsh{}absolute pressure in secondary state in Pa}
\PYG{n}{D\PYGZus{}2} \PYG{o}{=} \PYG{n}{D} \PYG{c+c1}{\PYGZsh{}density in the secondary state in kg/m3}
\PYG{n}{T\PYGZus{}2} \PYG{o}{=} \PYG{n}{CP}\PYG{o}{.}\PYG{n}{PropsSI}\PYG{p}{(}\PYG{l+s+s1}{\PYGZsq{}}\PYG{l+s+s1}{T}\PYG{l+s+s1}{\PYGZsq{}}\PYG{p}{,} \PYG{l+s+s1}{\PYGZsq{}}\PYG{l+s+s1}{D}\PYG{l+s+s1}{\PYGZsq{}}\PYG{p}{,} \PYG{n}{D\PYGZus{}2}\PYG{p}{,} \PYG{l+s+s1}{\PYGZsq{}}\PYG{l+s+s1}{P}\PYG{l+s+s1}{\PYGZsq{}}\PYG{p}{,} \PYG{n}{P\PYGZus{}2}\PYG{p}{,} \PYG{l+s+s1}{\PYGZsq{}}\PYG{l+s+s1}{Water}\PYG{l+s+s1}{\PYGZsq{}}\PYG{p}{)} \PYG{c+c1}{\PYGZsh{}calculating secondary temperature using coolprop in K}
\PYG{n}{T\PYGZus{}2C} \PYG{o}{=} \PYG{n}{T\PYGZus{}2} \PYG{o}{\PYGZhy{}} \PYG{l+m+mf}{273.15} \PYG{c+c1}{\PYGZsh{}temperature at secondary state in C}
\PYG{n+nb}{print}\PYG{p}{(}\PYG{l+s+s1}{\PYGZsq{}}\PYG{l+s+s1}{The temperature of steam at the secondary state is}\PYG{l+s+s1}{\PYGZsq{}}\PYG{p}{,}\PYG{l+s+sa}{f}\PYG{l+s+s2}{\PYGZdq{}}\PYG{l+s+si}{\PYGZob{}}\PYG{n}{T\PYGZus{}2C}\PYG{l+s+si}{:}\PYG{l+s+s2}{.1f}\PYG{l+s+si}{\PYGZcb{}}\PYG{l+s+s2}{\PYGZdq{}}\PYG{p}{,}\PYG{l+s+s1}{\PYGZsq{}}\PYG{l+s+s1}{C}\PYG{l+s+s1}{\PYGZsq{}}\PYG{p}{)}
\end{sphinxVerbatim}

\end{sphinxuseclass}\end{sphinxVerbatimInput}
\begin{sphinxVerbatimOutput}

\begin{sphinxuseclass}{cell_output}
\begin{sphinxVerbatim}[commandchars=\\\{\}]
The temperature of steam at the secondary state is 849.0 C
\end{sphinxVerbatim}

\end{sphinxuseclass}\end{sphinxVerbatimOutput}

\end{sphinxuseclass}

\subsection{Solution Approach for c)}
\label{\detokenize{notebooks/Chapter5/CH5-Q1:solution-approach-for-c}}
\sphinxAtStartPar
Given the first law of thermodynamics,

\sphinxAtStartPar
\(Q=\Delta U + W\)

\sphinxAtStartPar
and \(W = 0\) since the boundaries of the space are fixed. Therefore,

\sphinxAtStartPar
\(Q=\Delta U=U_2 - U_1=m(u_2-u_1)\)

\begin{sphinxuseclass}{cell}\begin{sphinxVerbatimInput}

\begin{sphinxuseclass}{cell_input}
\begin{sphinxVerbatim}[commandchars=\\\{\}]
\PYG{c+c1}{\PYGZsh{} define the inputs using coolprop:}
\PYG{n}{m} \PYG{o}{=} \PYG{l+m+mi}{1} \PYG{c+c1}{\PYGZsh{}steam mass in kg}
\PYG{n}{u\PYGZus{}1} \PYG{o}{=} \PYG{n}{CP}\PYG{o}{.}\PYG{n}{PropsSI}\PYG{p}{(}\PYG{l+s+s1}{\PYGZsq{}}\PYG{l+s+s1}{U}\PYG{l+s+s1}{\PYGZsq{}}\PYG{p}{,} \PYG{l+s+s1}{\PYGZsq{}}\PYG{l+s+s1}{D}\PYG{l+s+s1}{\PYGZsq{}}\PYG{p}{,} \PYG{n}{D}\PYG{p}{,} \PYG{l+s+s1}{\PYGZsq{}}\PYG{l+s+s1}{T}\PYG{l+s+s1}{\PYGZsq{}}\PYG{p}{,} \PYG{n}{T}\PYG{p}{,} \PYG{l+s+s1}{\PYGZsq{}}\PYG{l+s+s1}{Water}\PYG{l+s+s1}{\PYGZsq{}}\PYG{p}{)} \PYG{c+c1}{\PYGZsh{}calculating initial internal energy in J/kg}
\PYG{n}{u\PYGZus{}2} \PYG{o}{=} \PYG{n}{CP}\PYG{o}{.}\PYG{n}{PropsSI}\PYG{p}{(}\PYG{l+s+s1}{\PYGZsq{}}\PYG{l+s+s1}{U}\PYG{l+s+s1}{\PYGZsq{}}\PYG{p}{,} \PYG{l+s+s1}{\PYGZsq{}}\PYG{l+s+s1}{D}\PYG{l+s+s1}{\PYGZsq{}}\PYG{p}{,} \PYG{n}{D\PYGZus{}2}\PYG{p}{,} \PYG{l+s+s1}{\PYGZsq{}}\PYG{l+s+s1}{P}\PYG{l+s+s1}{\PYGZsq{}}\PYG{p}{,} \PYG{n}{P\PYGZus{}2}\PYG{p}{,} \PYG{l+s+s1}{\PYGZsq{}}\PYG{l+s+s1}{Water}\PYG{l+s+s1}{\PYGZsq{}}\PYG{p}{)}  \PYG{c+c1}{\PYGZsh{}calculating secondary internal energy in J/kg}
\PYG{n}{Q} \PYG{o}{=} \PYG{n}{m} \PYG{o}{*} \PYG{p}{(}\PYG{n}{u\PYGZus{}2} \PYG{o}{\PYGZhy{}} \PYG{n}{u\PYGZus{}1}\PYG{p}{)} \PYG{o}{/} \PYG{l+m+mi}{1000} \PYG{c+c1}{\PYGZsh{}heat required in kJ}
\PYG{n+nb}{print}\PYG{p}{(}\PYG{l+s+s1}{\PYGZsq{}}\PYG{l+s+s1}{The heat required to double the guage presure is}\PYG{l+s+s1}{\PYGZsq{}}\PYG{p}{,}\PYG{l+s+sa}{f}\PYG{l+s+s2}{\PYGZdq{}}\PYG{l+s+si}{\PYGZob{}}\PYG{n}{Q}\PYG{l+s+si}{:}\PYG{l+s+s2}{.1f}\PYG{l+s+si}{\PYGZcb{}}\PYG{l+s+s2}{\PYGZdq{}}\PYG{p}{,}\PYG{l+s+s1}{\PYGZsq{}}\PYG{l+s+s1}{kJ}\PYG{l+s+s1}{\PYGZsq{}}\PYG{p}{)}
\end{sphinxVerbatim}

\end{sphinxuseclass}\end{sphinxVerbatimInput}
\begin{sphinxVerbatimOutput}

\begin{sphinxuseclass}{cell_output}
\begin{sphinxVerbatim}[commandchars=\\\{\}]
The heat required to double the guage presure is 790.8 kJ
\end{sphinxVerbatim}

\end{sphinxuseclass}\end{sphinxVerbatimOutput}

\end{sphinxuseclass}

\subsection{Solution Approach for d)}
\label{\detokenize{notebooks/Chapter5/CH5-Q1:solution-approach-for-d}}
\sphinxAtStartPar
To use \(C_v\) and \(C_p\) values to calculate changes in enthalpy and internal energy, ideal gas assumption is to be made for steam.

\sphinxAtStartPar
\(Q=\Delta U=m\Delta u=mC_v\Delta T\)

\begin{sphinxuseclass}{cell}\begin{sphinxVerbatimInput}

\begin{sphinxuseclass}{cell_input}
\begin{sphinxVerbatim}[commandchars=\\\{\}]
\PYG{c+c1}{\PYGZsh{}define the constants}
\PYG{n}{C\PYGZus{}v} \PYG{o}{=} \PYG{l+m+mf}{1.4108} \PYG{c+c1}{\PYGZsh{}C\PYGZus{}v of steam in kJ/kg.k}
\PYG{n}{Q} \PYG{o}{=} \PYG{n}{m} \PYG{o}{*} \PYG{n}{C\PYGZus{}v} \PYG{o}{*} \PYG{p}{(}\PYG{n}{T\PYGZus{}2} \PYG{o}{\PYGZhy{}} \PYG{n}{T}\PYG{p}{)} \PYG{c+c1}{\PYGZsh{}heat required in kJ}
\PYG{n+nb}{print}\PYG{p}{(}\PYG{l+s+s1}{\PYGZsq{}}\PYG{l+s+s1}{The heat required to double the guage presure is}\PYG{l+s+s1}{\PYGZsq{}}\PYG{p}{,}\PYG{l+s+sa}{f}\PYG{l+s+s2}{\PYGZdq{}}\PYG{l+s+si}{\PYGZob{}}\PYG{n}{Q}\PYG{l+s+si}{:}\PYG{l+s+s2}{.1f}\PYG{l+s+si}{\PYGZcb{}}\PYG{l+s+s2}{\PYGZdq{}}\PYG{p}{,}\PYG{l+s+s1}{\PYGZsq{}}\PYG{l+s+s1}{kJ}\PYG{l+s+s1}{\PYGZsq{}}\PYG{p}{,}\PYG{l+s+s1}{\PYGZsq{}}\PYG{l+s+s1}{using C\PYGZus{}v}\PYG{l+s+s1}{\PYGZsq{}}\PYG{p}{)}
\end{sphinxVerbatim}

\end{sphinxuseclass}\end{sphinxVerbatimInput}
\begin{sphinxVerbatimOutput}

\begin{sphinxuseclass}{cell_output}
\begin{sphinxVerbatim}[commandchars=\\\{\}]
The heat required to double the guage presure is 633.5 kJ using C\PYGZus{}v
\end{sphinxVerbatim}

\end{sphinxuseclass}\end{sphinxVerbatimOutput}

\end{sphinxuseclass}
\sphinxstepscope


\section{Piston\sphinxhyphen{}cylinder: Carbon dioxide}
\label{\detokenize{notebooks/Chapter5/CH5-Q2:piston-cylinder-carbon-dioxide}}\label{\detokenize{notebooks/Chapter5/CH5-Q2::doc}}
\sphinxAtStartPar
Consider carbon\sphinxhyphen{}dioxide stored in a cylinder\sphinxhyphen{}and\sphinxhyphen{}piston system at \(1\:MPa\) and \(0 ^{\circ}  C\) with a volume of \(1\:m^3\). The piston is free to move keeping the pressure of carbon\sphinxhyphen{}diaxide constant. The system is then heated to \(25 ^{\circ} C\). Using CoolProp as a tool, determine:

\sphinxAtStartPar
a)the mass of carbon\sphinxhyphen{}dioxide

\sphinxAtStartPar
b)heat required to do so

\sphinxAtStartPar
c)the required heat if the weight of the 500g aluminum piston and cylinder are to be considered

\sphinxAtStartPar
d)the required heat in b if \(C_p\) is to be used to calculate \(\Delta h\) instead of CoolProp? what assumptions are made in this case?


\subsection{Solution Approach for a)}
\label{\detokenize{notebooks/Chapter5/CH5-Q2:solution-approach-for-a}}
\sphinxAtStartPar
the two knowns to extract thermodynamic properties are

\sphinxAtStartPar
\(P_1=1\:MPa\)

\sphinxAtStartPar
\(T_1=0 ^{\circ}  C\)

\sphinxAtStartPar
to calculate mass, density is to be extracted

\sphinxAtStartPar
\(m=D_1V_1\)

\begin{sphinxuseclass}{cell}\begin{sphinxVerbatimInput}

\begin{sphinxuseclass}{cell_input}
\begin{sphinxVerbatim}[commandchars=\\\{\}]
\PYG{c+c1}{\PYGZsh{}importing the required library}
\PYG{k+kn}{import} \PYG{n+nn}{CoolProp}\PYG{n+nn}{.}\PYG{n+nn}{CoolProp} \PYG{k}{as} \PYG{n+nn}{CP}

\PYG{c+c1}{\PYGZsh{}define state variables}
\PYG{n}{V\PYGZus{}1} \PYG{o}{=} \PYG{l+m+mi}{1} \PYG{c+c1}{\PYGZsh{}ninitial volume in m3}
\PYG{n}{P\PYGZus{}1} \PYG{o}{=} \PYG{l+m+mf}{1e+6} \PYG{c+c1}{\PYGZsh{}initial pressure in Pa}
\PYG{n}{T\PYGZus{}1} \PYG{o}{=} \PYG{l+m+mi}{0} \PYG{o}{+} \PYG{l+m+mf}{273.15} \PYG{c+c1}{\PYGZsh{}initial temperature in K}
\PYG{n}{D\PYGZus{}1} \PYG{o}{=} \PYG{n}{CP}\PYG{o}{.}\PYG{n}{PropsSI}\PYG{p}{(}\PYG{l+s+s1}{\PYGZsq{}}\PYG{l+s+s1}{D}\PYG{l+s+s1}{\PYGZsq{}}\PYG{p}{,} \PYG{l+s+s1}{\PYGZsq{}}\PYG{l+s+s1}{P}\PYG{l+s+s1}{\PYGZsq{}}\PYG{p}{,} \PYG{n}{P\PYGZus{}1}\PYG{p}{,} \PYG{l+s+s1}{\PYGZsq{}}\PYG{l+s+s1}{T}\PYG{l+s+s1}{\PYGZsq{}}\PYG{p}{,} \PYG{n}{T\PYGZus{}1}\PYG{p}{,} \PYG{l+s+s1}{\PYGZsq{}}\PYG{l+s+s1}{CO2}\PYG{l+s+s1}{\PYGZsq{}}\PYG{p}{)} \PYG{c+c1}{\PYGZsh{}calculating density using coolprop kg/m3}
\PYG{n}{m\PYGZus{}co2} \PYG{o}{=} \PYG{n}{D\PYGZus{}1} \PYG{o}{*} \PYG{n}{V\PYGZus{}1} \PYG{c+c1}{\PYGZsh{}mass of carbon\PYGZhy{}dioxide in kg}
\PYG{n+nb}{print}\PYG{p}{(}\PYG{l+s+s1}{\PYGZsq{}}\PYG{l+s+s1}{The mass of carbon\PYGZhy{}dioxide is}\PYG{l+s+s1}{\PYGZsq{}}\PYG{p}{,}\PYG{l+s+sa}{f}\PYG{l+s+s2}{\PYGZdq{}}\PYG{l+s+si}{\PYGZob{}}\PYG{n}{m\PYGZus{}co2}\PYG{l+s+si}{:}\PYG{l+s+s2}{.1f}\PYG{l+s+si}{\PYGZcb{}}\PYG{l+s+s2}{\PYGZdq{}}\PYG{p}{,}\PYG{l+s+s1}{\PYGZsq{}}\PYG{l+s+s1}{kg}\PYG{l+s+s1}{\PYGZsq{}}\PYG{p}{)}
\end{sphinxVerbatim}

\end{sphinxuseclass}\end{sphinxVerbatimInput}
\begin{sphinxVerbatimOutput}

\begin{sphinxuseclass}{cell_output}
\begin{sphinxVerbatim}[commandchars=\\\{\}]
The mass of carbon\PYGZhy{}dioxide is 20.8 kg
\end{sphinxVerbatim}

\end{sphinxuseclass}\end{sphinxVerbatimOutput}

\end{sphinxuseclass}

\subsection{Solution Approach for b)}
\label{\detokenize{notebooks/Chapter5/CH5-Q2:solution-approach-for-b}}
\sphinxAtStartPar
based on the first law,

\sphinxAtStartPar
\(Q=\Delta U + W=\Delta H=m(h_2-h_1)\)

\begin{sphinxuseclass}{cell}\begin{sphinxVerbatimInput}

\begin{sphinxuseclass}{cell_input}
\begin{sphinxVerbatim}[commandchars=\\\{\}]
\PYG{c+c1}{\PYGZsh{}define state variable}
\PYG{n}{T\PYGZus{}2} \PYG{o}{=} \PYG{l+m+mi}{25} \PYG{o}{+} \PYG{l+m+mf}{273.15} \PYG{c+c1}{\PYGZsh{}secondary temperature in K}
\PYG{n}{P\PYGZus{}2} \PYG{o}{=} \PYG{n}{P\PYGZus{}1} \PYG{c+c1}{\PYGZsh{}constant pressure process}
\PYG{c+c1}{\PYGZsh{}extracting enthalpy}
\PYG{n}{h\PYGZus{}1} \PYG{o}{=} \PYG{n}{CP}\PYG{o}{.}\PYG{n}{PropsSI}\PYG{p}{(}\PYG{l+s+s1}{\PYGZsq{}}\PYG{l+s+s1}{H}\PYG{l+s+s1}{\PYGZsq{}}\PYG{p}{,} \PYG{l+s+s1}{\PYGZsq{}}\PYG{l+s+s1}{P}\PYG{l+s+s1}{\PYGZsq{}}\PYG{p}{,} \PYG{n}{P\PYGZus{}1}\PYG{p}{,} \PYG{l+s+s1}{\PYGZsq{}}\PYG{l+s+s1}{T}\PYG{l+s+s1}{\PYGZsq{}}\PYG{p}{,} \PYG{n}{T\PYGZus{}1}\PYG{p}{,} \PYG{l+s+s1}{\PYGZsq{}}\PYG{l+s+s1}{CO2}\PYG{l+s+s1}{\PYGZsq{}}\PYG{p}{)} \PYG{c+c1}{\PYGZsh{}initial enthalpy in J/kg}
\PYG{n}{h\PYGZus{}2} \PYG{o}{=} \PYG{n}{CP}\PYG{o}{.}\PYG{n}{PropsSI}\PYG{p}{(}\PYG{l+s+s1}{\PYGZsq{}}\PYG{l+s+s1}{H}\PYG{l+s+s1}{\PYGZsq{}}\PYG{p}{,} \PYG{l+s+s1}{\PYGZsq{}}\PYG{l+s+s1}{P}\PYG{l+s+s1}{\PYGZsq{}}\PYG{p}{,} \PYG{n}{P\PYGZus{}2}\PYG{p}{,} \PYG{l+s+s1}{\PYGZsq{}}\PYG{l+s+s1}{T}\PYG{l+s+s1}{\PYGZsq{}}\PYG{p}{,} \PYG{n}{T\PYGZus{}2}\PYG{p}{,} \PYG{l+s+s1}{\PYGZsq{}}\PYG{l+s+s1}{CO2}\PYG{l+s+s1}{\PYGZsq{}}\PYG{p}{)} \PYG{c+c1}{\PYGZsh{}secondary enthalpy in J/kg}
\PYG{n}{Q} \PYG{o}{=} \PYG{n}{m\PYGZus{}co2} \PYG{o}{*} \PYG{p}{(}\PYG{n}{h\PYGZus{}2}\PYG{o}{\PYGZhy{}}\PYG{n}{h\PYGZus{}1}\PYG{p}{)} \PYG{o}{/} \PYG{l+m+mi}{1000} \PYG{c+c1}{\PYGZsh{}heat required in KJ}
\PYG{n+nb}{print}\PYG{p}{(}\PYG{l+s+s1}{\PYGZsq{}}\PYG{l+s+s1}{The required heat for this process is}\PYG{l+s+s1}{\PYGZsq{}}\PYG{p}{,}\PYG{l+s+sa}{f}\PYG{l+s+s2}{\PYGZdq{}}\PYG{l+s+si}{\PYGZob{}}\PYG{n}{Q}\PYG{l+s+si}{:}\PYG{l+s+s2}{.1f}\PYG{l+s+si}{\PYGZcb{}}\PYG{l+s+s2}{\PYGZdq{}}\PYG{p}{,}\PYG{l+s+s1}{\PYGZsq{}}\PYG{l+s+s1}{kJ}\PYG{l+s+s1}{\PYGZsq{}}\PYG{p}{)}
\end{sphinxVerbatim}

\end{sphinxuseclass}\end{sphinxVerbatimInput}
\begin{sphinxVerbatimOutput}

\begin{sphinxuseclass}{cell_output}
\begin{sphinxVerbatim}[commandchars=\\\{\}]
The required heat for this process is 481.1 kJ
\end{sphinxVerbatim}

\end{sphinxuseclass}\end{sphinxVerbatimOutput}

\end{sphinxuseclass}

\subsection{Solution Approach for c)}
\label{\detokenize{notebooks/Chapter5/CH5-Q2:solution-approach-for-c}}
\sphinxAtStartPar
for solids and liquids

\sphinxAtStartPar
\(\Delta u=\Delta h=C_p\Delta T\)

\begin{sphinxuseclass}{cell}\begin{sphinxVerbatimInput}

\begin{sphinxuseclass}{cell_input}
\begin{sphinxVerbatim}[commandchars=\\\{\}]
\PYG{c+c1}{\PYGZsh{}define constants}
\PYG{n}{m\PYGZus{}alm} \PYG{o}{=} \PYG{l+m+mf}{0.5} \PYG{c+c1}{\PYGZsh{}mass of aluminum in kg}
\PYG{n}{C\PYGZus{}p\PYGZus{}alm} \PYG{o}{=} \PYG{l+m+mf}{0.897} \PYG{c+c1}{\PYGZsh{}C\PYGZus{}p of aluminum in kJ/kg.K}
\PYG{n}{Q\PYGZus{}alm} \PYG{o}{=} \PYG{n}{m\PYGZus{}alm} \PYG{o}{*} \PYG{n}{C\PYGZus{}p\PYGZus{}alm} \PYG{o}{*} \PYG{p}{(}\PYG{n}{T\PYGZus{}2}\PYG{o}{\PYGZhy{}}\PYG{n}{T\PYGZus{}1}\PYG{p}{)}
\PYG{n}{Q\PYGZus{}total} \PYG{o}{=} \PYG{n}{Q\PYGZus{}alm} \PYG{o}{+} \PYG{n}{Q} \PYG{c+c1}{\PYGZsh{}total heat required for the process}
\PYG{n+nb}{print}\PYG{p}{(}\PYG{l+s+s1}{\PYGZsq{}}\PYG{l+s+s1}{The required total heat for this process is}\PYG{l+s+s1}{\PYGZsq{}}\PYG{p}{,}\PYG{l+s+sa}{f}\PYG{l+s+s2}{\PYGZdq{}}\PYG{l+s+si}{\PYGZob{}}\PYG{n}{Q\PYGZus{}total}\PYG{l+s+si}{:}\PYG{l+s+s2}{.1f}\PYG{l+s+si}{\PYGZcb{}}\PYG{l+s+s2}{\PYGZdq{}}\PYG{p}{,}\PYG{l+s+s1}{\PYGZsq{}}\PYG{l+s+s1}{kJ}\PYG{l+s+s1}{\PYGZsq{}}\PYG{p}{)}
\end{sphinxVerbatim}

\end{sphinxuseclass}\end{sphinxVerbatimInput}
\begin{sphinxVerbatimOutput}

\begin{sphinxuseclass}{cell_output}
\begin{sphinxVerbatim}[commandchars=\\\{\}]
The required total heat for this process is 492.4 kJ
\end{sphinxVerbatim}

\end{sphinxuseclass}\end{sphinxVerbatimOutput}

\end{sphinxuseclass}

\subsection{Solution Approach for d)}
\label{\detokenize{notebooks/Chapter5/CH5-Q2:solution-approach-for-d}}
\sphinxAtStartPar
in order to use \(C_p\) to calculate \(\Delta h\), carbon\sphinxhyphen{}dioxide is assumed to be ideal gas; then from the first law,

\sphinxAtStartPar
\(Q=\Delta U + W=\Delta H=mC_p\Delta T\)

\begin{sphinxuseclass}{cell}\begin{sphinxVerbatimInput}

\begin{sphinxuseclass}{cell_input}
\begin{sphinxVerbatim}[commandchars=\\\{\}]
\PYG{c+c1}{\PYGZsh{}define constants}
\PYG{n}{C\PYGZus{}p\PYGZus{}co2} \PYG{o}{=} \PYG{l+m+mf}{0.846} \PYG{c+c1}{\PYGZsh{}Cp of CO2 in kJ/kg.K}
\PYG{n}{Q} \PYG{o}{=} \PYG{n}{m\PYGZus{}co2} \PYG{o}{*} \PYG{n}{C\PYGZus{}p\PYGZus{}co2} \PYG{o}{*} \PYG{p}{(}\PYG{n}{T\PYGZus{}2}\PYG{o}{\PYGZhy{}}\PYG{n}{T\PYGZus{}1}\PYG{p}{)}
\PYG{n+nb}{print}\PYG{p}{(}\PYG{l+s+s1}{\PYGZsq{}}\PYG{l+s+s1}{The required heat for this process using Cp assumption is}\PYG{l+s+s1}{\PYGZsq{}}\PYG{p}{,}\PYG{l+s+sa}{f}\PYG{l+s+s2}{\PYGZdq{}}\PYG{l+s+si}{\PYGZob{}}\PYG{n}{Q}\PYG{l+s+si}{:}\PYG{l+s+s2}{.1f}\PYG{l+s+si}{\PYGZcb{}}\PYG{l+s+s2}{\PYGZdq{}}\PYG{p}{,}\PYG{l+s+s1}{\PYGZsq{}}\PYG{l+s+s1}{kJ}\PYG{l+s+s1}{\PYGZsq{}}\PYG{p}{)}
\end{sphinxVerbatim}

\end{sphinxuseclass}\end{sphinxVerbatimInput}
\begin{sphinxVerbatimOutput}

\begin{sphinxuseclass}{cell_output}
\begin{sphinxVerbatim}[commandchars=\\\{\}]
The required heat for this process using Cp assumption is 440.7 kJ
\end{sphinxVerbatim}

\end{sphinxuseclass}\end{sphinxVerbatimOutput}

\end{sphinxuseclass}
\sphinxstepscope


\section{Air Compressor and Turbine}
\label{\detokenize{notebooks/Chapter5/CH5-Q3:air-compressor-and-turbine}}\label{\detokenize{notebooks/Chapter5/CH5-Q3::doc}}
\sphinxAtStartPar
Imagine a turbine whose work output is used to drive a compressor to compress air from atmospheric pressure and temperature to \(1\:MPa\). The turbine operates with steam enerting the turbine at \(1.2\:MPa\) and \(600 ^{\circ}  C\) and exits as satirated vapor at atmospheric pressure. Assuming air to follow ideal\sphinxhyphen{}gas law going through a polytropic process with \(n=1.3\), determine the flow\sphinxhyphen{}rate of steam per \(1\:kg/s\) of air flow\sphinxhyphen{}rate.

\sphinxAtStartPar
\sphinxincludegraphics{{CH5-Q3}.png}


\section{Solution Approach}
\label{\detokenize{notebooks/Chapter5/CH5-Q3:solution-approach}}
\sphinxAtStartPar
Since the turbine and the compressor are couples, the work output from the turbine is considered the work input for the compressor. Therefore, calculating the work output for unit mass of turbine flow\sphinxhyphen{}rate as well as the work input for compressing \(1\:kg\) of air build the bridge to connect the turbune and the compressor and to calculate the required flow\sphinxhyphen{}rate.

\sphinxAtStartPar
From the first law of thermodynamics for the trubine assuming no changes in velocity and elevation:

\sphinxAtStartPar
\(w=\Delta h=h_{in}-h_{out}\)

\begin{sphinxuseclass}{cell}\begin{sphinxVerbatimInput}

\begin{sphinxuseclass}{cell_input}
\begin{sphinxVerbatim}[commandchars=\\\{\}]
\PYG{c+c1}{\PYGZsh{}importing the required library}
\PYG{k+kn}{import} \PYG{n+nn}{CoolProp}\PYG{n+nn}{.}\PYG{n+nn}{CoolProp} \PYG{k}{as} \PYG{n+nn}{CP}

\PYG{c+c1}{\PYGZsh{}define thermodynamic variables}
\PYG{n}{P\PYGZus{}atm} \PYG{o}{=} \PYG{l+m+mi}{101325} \PYG{c+c1}{\PYGZsh{}atmospheric pressure in Pa}
\PYG{n}{P\PYGZus{}it} \PYG{o}{=} \PYG{l+m+mf}{1.2E+6} \PYG{c+c1}{\PYGZsh{}turbine inlet pressure in Pa }
\PYG{n}{T\PYGZus{}it} \PYG{o}{=} \PYG{l+m+mi}{600} \PYG{o}{+} \PYG{l+m+mf}{273.15} \PYG{c+c1}{\PYGZsh{}turbine inlet temperatue in K}
\PYG{n}{P\PYGZus{}ot} \PYG{o}{=} \PYG{n}{P\PYGZus{}atm} \PYG{c+c1}{\PYGZsh{}turbine outlet pressure in Pa}
\PYG{n}{x\PYGZus{}ot} \PYG{o}{=} \PYG{l+m+mi}{1}\PYG{c+c1}{\PYGZsh{}turbine outlet quality}

\PYG{n}{h\PYGZus{}it} \PYG{o}{=} \PYG{n}{CP}\PYG{o}{.}\PYG{n}{PropsSI}\PYG{p}{(}\PYG{l+s+s1}{\PYGZsq{}}\PYG{l+s+s1}{H}\PYG{l+s+s1}{\PYGZsq{}}\PYG{p}{,} \PYG{l+s+s1}{\PYGZsq{}}\PYG{l+s+s1}{P}\PYG{l+s+s1}{\PYGZsq{}}\PYG{p}{,} \PYG{n}{P\PYGZus{}it}\PYG{p}{,} \PYG{l+s+s1}{\PYGZsq{}}\PYG{l+s+s1}{T}\PYG{l+s+s1}{\PYGZsq{}}\PYG{p}{,} \PYG{n}{T\PYGZus{}it}\PYG{p}{,} \PYG{l+s+s1}{\PYGZsq{}}\PYG{l+s+s1}{Water}\PYG{l+s+s1}{\PYGZsq{}}\PYG{p}{)} \PYG{c+c1}{\PYGZsh{}turbine inlet enthapy J/kg}
\PYG{n}{h\PYGZus{}ot} \PYG{o}{=} \PYG{n}{CP}\PYG{o}{.}\PYG{n}{PropsSI}\PYG{p}{(}\PYG{l+s+s1}{\PYGZsq{}}\PYG{l+s+s1}{H}\PYG{l+s+s1}{\PYGZsq{}}\PYG{p}{,} \PYG{l+s+s1}{\PYGZsq{}}\PYG{l+s+s1}{P}\PYG{l+s+s1}{\PYGZsq{}}\PYG{p}{,} \PYG{n}{P\PYGZus{}ot}\PYG{p}{,} \PYG{l+s+s1}{\PYGZsq{}}\PYG{l+s+s1}{Q}\PYG{l+s+s1}{\PYGZsq{}}\PYG{p}{,} \PYG{n}{x\PYGZus{}ot}\PYG{p}{,} \PYG{l+s+s1}{\PYGZsq{}}\PYG{l+s+s1}{Water}\PYG{l+s+s1}{\PYGZsq{}}\PYG{p}{)} \PYG{c+c1}{\PYGZsh{}turbine inlet enthapy J/kg}

\PYG{n}{w\PYGZus{}t} \PYG{o}{=} \PYG{n}{h\PYGZus{}it} \PYG{o}{\PYGZhy{}} \PYG{n}{h\PYGZus{}ot} \PYG{c+c1}{\PYGZsh{}turbine work output per unit mass of steam in J}
\end{sphinxVerbatim}

\end{sphinxuseclass}\end{sphinxVerbatimInput}

\end{sphinxuseclass}
\sphinxAtStartPar
For a polytropic process of an ideal gas with a polytropic constant of \(n\):

\sphinxAtStartPar
\(T_2=T_1(P_2/P_1)^{(n-1)/n}\)

\sphinxAtStartPar
for reference look at \sphinxhref{https://pressbooks.bccampus.ca/thermo1/chapter/5-5-application-of-the-mass-and-energy-conservation-equations-in-steady-flow-devices/}{Example 1} from Chapter 5 in the textbook

\begin{sphinxuseclass}{cell}\begin{sphinxVerbatimInput}

\begin{sphinxuseclass}{cell_input}
\begin{sphinxVerbatim}[commandchars=\\\{\}]
\PYG{n}{n} \PYG{o}{=} \PYG{l+m+mf}{1.3} \PYG{c+c1}{\PYGZsh{}polytropic constant of the process}
\PYG{n}{T\PYGZus{}ic} \PYG{o}{=} \PYG{l+m+mi}{25} \PYG{o}{+} \PYG{l+m+mf}{273.15} \PYG{c+c1}{\PYGZsh{}inlet temperature of compressor in K}
\PYG{n}{P\PYGZus{}ic} \PYG{o}{=} \PYG{n}{P\PYGZus{}atm} \PYG{c+c1}{\PYGZsh{}inlet pressure of compressor in Pa}
\PYG{n}{P\PYGZus{}oc} \PYG{o}{=} \PYG{l+m+mf}{1E+6} \PYG{c+c1}{\PYGZsh{}outlet pressure of compressor in Pa}
\PYG{n}{T\PYGZus{}oc} \PYG{o}{=} \PYG{n}{T\PYGZus{}ic} \PYG{o}{*} \PYG{p}{(}\PYG{n}{P\PYGZus{}oc}\PYG{o}{/}\PYG{n}{P\PYGZus{}ic}\PYG{p}{)}\PYG{o}{*}\PYG{o}{*}\PYG{p}{(}\PYG{p}{(}\PYG{n}{n}\PYG{o}{\PYGZhy{}}\PYG{l+m+mi}{1}\PYG{p}{)}\PYG{o}{/}\PYG{n}{n}\PYG{p}{)} \PYG{c+c1}{\PYGZsh{}outlet temperature of compressor air in K}
\end{sphinxVerbatim}

\end{sphinxuseclass}\end{sphinxVerbatimInput}

\end{sphinxuseclass}
\sphinxAtStartPar
Now assuming air as an ideal gas, its \(C_p\) value can be used to calculate changes in enthalpy

\sphinxAtStartPar
\(w=\Delta h=h_{out}-h_{in}=C_p(T_{out}-T_{in})\)

\begin{sphinxuseclass}{cell}\begin{sphinxVerbatimInput}

\begin{sphinxuseclass}{cell_input}
\begin{sphinxVerbatim}[commandchars=\\\{\}]
\PYG{c+c1}{\PYGZsh{}define constants}
\PYG{n}{C\PYGZus{}p} \PYG{o}{=} \PYG{l+m+mf}{100.5} \PYG{c+c1}{\PYGZsh{}Cp of air in J/kg.K}
\PYG{n}{w\PYGZus{}c} \PYG{o}{=} \PYG{n}{C\PYGZus{}p} \PYG{o}{*} \PYG{p}{(}\PYG{n}{T\PYGZus{}oc} \PYG{o}{\PYGZhy{}} \PYG{n}{T\PYGZus{}ic}\PYG{p}{)} \PYG{c+c1}{\PYGZsh{}specific work input to compress air in J }
\end{sphinxVerbatim}

\end{sphinxuseclass}\end{sphinxVerbatimInput}

\end{sphinxuseclass}
\sphinxAtStartPar
Now, to correlate mass flow\sphinxhyphen{}rates, the rate of work output in the turbine equals to the rate of work input into the compressor

\sphinxAtStartPar
\(\dot W_{turbine}=\dot W_{compressor}\)

\sphinxAtStartPar
\(\dot m_{turbine}w_{turbine}=\dot m_{compressor}w_{compressor}\)

\sphinxAtStartPar
\(\dot m_{turbine}=\dot m_{compressor}w_{compressor}/w_{turbine}\)

\begin{sphinxuseclass}{cell}\begin{sphinxVerbatimInput}

\begin{sphinxuseclass}{cell_input}
\begin{sphinxVerbatim}[commandchars=\\\{\}]
\PYG{n}{m\PYGZus{}c} \PYG{o}{=} \PYG{l+m+mi}{1} \PYG{c+c1}{\PYGZsh{}flow\PYGZhy{}rate of compressor required in kg/s}
\PYG{n}{m\PYGZus{}t} \PYG{o}{=} \PYG{n}{m\PYGZus{}c} \PYG{o}{*} \PYG{n}{w\PYGZus{}c} \PYG{o}{/} \PYG{n}{w\PYGZus{}t}
\PYG{n+nb}{print}\PYG{p}{(}\PYG{l+s+s1}{\PYGZsq{}}\PYG{l+s+s1}{The steam mass flow\PYGZhy{}rate required to compress 1 kg/s of air is}\PYG{l+s+s1}{\PYGZsq{}}\PYG{p}{,}\PYG{l+s+sa}{f}\PYG{l+s+s2}{\PYGZdq{}}\PYG{l+s+si}{\PYGZob{}}\PYG{n}{m\PYGZus{}t}\PYG{l+s+si}{:}\PYG{l+s+s2}{.3f}\PYG{l+s+si}{\PYGZcb{}}\PYG{l+s+s2}{\PYGZdq{}}\PYG{p}{,}\PYG{l+s+s1}{\PYGZsq{}}\PYG{l+s+s1}{kg/s}\PYG{l+s+s1}{\PYGZsq{}}\PYG{p}{)}
\end{sphinxVerbatim}

\end{sphinxuseclass}\end{sphinxVerbatimInput}
\begin{sphinxVerbatimOutput}

\begin{sphinxuseclass}{cell_output}
\begin{sphinxVerbatim}[commandchars=\\\{\}]
The steam mass flow\PYGZhy{}rate required to compress 1 kg/s of air is 0.020 kg/s
\end{sphinxVerbatim}

\end{sphinxuseclass}\end{sphinxVerbatimOutput}

\end{sphinxuseclass}
\sphinxstepscope


\section{P\sphinxhyphen{}h diagram for R\sphinxhyphen{}134a refrigerant}
\label{\detokenize{notebooks/Chapter5/CH5-Q4_v1.2:p-h-diagram-for-r-134a-refrigerant}}\label{\detokenize{notebooks/Chapter5/CH5-Q4_v1.2::doc}}
\sphinxAtStartPar
Consider R134\sphinxhyphen{}a as a refrigrant fluid. Build the P\sphinxhyphen{}h (Pressure vs. Specific Enthalpy) for R134\sphinxhyphen{}a knowing the critical pressure is around \(4.03\:MPa\). Build three constant temperature curves for \(T=120 ^{\circ}  C\), \(0\ ^{\circ}  C\) and \(-20 ^{\circ}  C\).


\section{Solution approach:}
\label{\detokenize{notebooks/Chapter5/CH5-Q4_v1.2:solution-approach}}
\begin{sphinxuseclass}{cell}\begin{sphinxVerbatimInput}

\begin{sphinxuseclass}{cell_input}
\begin{sphinxVerbatim}[commandchars=\\\{\}]
\PYG{c+c1}{\PYGZsh{} Plot a P\PYGZhy{}h diagram for a fluid of choice}


\PYG{c+c1}{\PYGZsh{} import the libraries we\PYGZsq{}ll need}
\PYG{k+kn}{import} \PYG{n+nn}{CoolProp}\PYG{n+nn}{.}\PYG{n+nn}{CoolProp} \PYG{k}{as} \PYG{n+nn}{CP}
\PYG{k+kn}{import} \PYG{n+nn}{numpy} \PYG{k}{as} \PYG{n+nn}{np}
\PYG{k+kn}{import} \PYG{n+nn}{matplotlib}\PYG{n+nn}{.}\PYG{n+nn}{pyplot} \PYG{k}{as} \PYG{n+nn}{plt}


\PYG{c+c1}{\PYGZsh{} define variables}
\PYG{n}{fluid} \PYG{o}{=} \PYG{l+s+s2}{\PYGZdq{}}\PYG{l+s+s2}{R134A}\PYG{l+s+s2}{\PYGZdq{}}  \PYG{c+c1}{\PYGZsh{} define the fluid or material of interest, for full list see CP.Fluidslist()}
\PYG{n}{T\PYGZus{}min} \PYG{o}{=} \PYG{n}{CP}\PYG{o}{.}\PYG{n}{PropsSI}\PYG{p}{(}\PYG{l+s+s2}{\PYGZdq{}}\PYG{l+s+s2}{Tmin}\PYG{l+s+s2}{\PYGZdq{}}\PYG{p}{,} \PYG{n}{fluid}\PYG{p}{)}  \PYG{c+c1}{\PYGZsh{} triple\PYGZhy{}point temp for the fluid}
\PYG{n}{P\PYGZus{}min} \PYG{o}{=} \PYG{n}{CP}\PYG{o}{.}\PYG{n}{PropsSI}\PYG{p}{(}\PYG{l+s+s2}{\PYGZdq{}}\PYG{l+s+s2}{P}\PYG{l+s+s2}{\PYGZdq{}}\PYG{p}{,} \PYG{l+s+s2}{\PYGZdq{}}\PYG{l+s+s2}{T}\PYG{l+s+s2}{\PYGZdq{}}\PYG{p}{,} \PYG{n}{T\PYGZus{}min}\PYG{p}{,} \PYG{l+s+s2}{\PYGZdq{}}\PYG{l+s+s2}{Q}\PYG{l+s+s2}{\PYGZdq{}}\PYG{p}{,} \PYG{l+m+mi}{0}\PYG{p}{,} \PYG{n}{fluid}\PYG{p}{)}  \PYG{c+c1}{\PYGZsh{} triple\PYGZhy{}point pressure for the fluid}
\PYG{n}{P\PYGZus{}max} \PYG{o}{=} \PYG{l+m+mf}{4.03E+6} \PYG{c+c1}{\PYGZsh{}approximate critical pressure}

\PYG{n}{P\PYGZus{}vals} \PYG{o}{=} \PYG{n}{np}\PYG{o}{.}\PYG{n}{linspace}\PYG{p}{(}\PYG{n}{P\PYGZus{}min}\PYG{p}{,} \PYG{n}{P\PYGZus{}max}\PYG{p}{,} \PYG{l+m+mi}{1000}\PYG{p}{)}  \PYG{c+c1}{\PYGZsh{} define an array of values from P\PYGZus{}min to P\PYGZus{}max}
\PYG{n}{Q} \PYG{o}{=} \PYG{l+m+mi}{1}  \PYG{c+c1}{\PYGZsh{} define the fluid quality as 1, which is 100\PYGZpc{} vapor}

\PYG{n}{enthalpy} \PYG{o}{=} \PYG{p}{[}\PYG{n}{CP}\PYG{o}{.}\PYG{n}{PropsSI}\PYG{p}{(}\PYG{l+s+s2}{\PYGZdq{}}\PYG{l+s+s2}{H}\PYG{l+s+s2}{\PYGZdq{}}\PYG{p}{,} \PYG{l+s+s2}{\PYGZdq{}}\PYG{l+s+s2}{P}\PYG{l+s+s2}{\PYGZdq{}}\PYG{p}{,} \PYG{n}{P}\PYG{p}{,} \PYG{l+s+s2}{\PYGZdq{}}\PYG{l+s+s2}{Q}\PYG{l+s+s2}{\PYGZdq{}}\PYG{p}{,} \PYG{n}{Q}\PYG{p}{,} \PYG{n}{fluid}\PYG{p}{)}\PYG{o}{/}\PYG{l+m+mi}{1000} \PYG{k}{for} \PYG{n}{P} \PYG{o+ow}{in} \PYG{n}{P\PYGZus{}vals}\PYG{p}{]}  \PYG{c+c1}{\PYGZsh{} call for enthalpy values using CoolProp}

\PYG{n}{plt}\PYG{o}{.}\PYG{n}{plot}\PYG{p}{(}\PYG{n}{enthalpy}\PYG{p}{,} \PYG{n}{P\PYGZus{}vals}\PYG{p}{,} \PYG{l+s+s2}{\PYGZdq{}}\PYG{l+s+s2}{\PYGZhy{}b}\PYG{l+s+s2}{\PYGZdq{}}\PYG{p}{,} \PYG{n}{label}\PYG{o}{=}\PYG{l+s+s2}{\PYGZdq{}}\PYG{l+s+s2}{Saturation Line}\PYG{l+s+s2}{\PYGZdq{}}\PYG{p}{)}  \PYG{c+c1}{\PYGZsh{} plot pressure vs enthalpy}

\PYG{n}{Q} \PYG{o}{=} \PYG{l+m+mi}{0}  \PYG{c+c1}{\PYGZsh{} define the fluid quality as 0, which is 100\PYGZpc{} liquid}

\PYG{n}{enthalpy} \PYG{o}{=} \PYG{p}{[}\PYG{n}{CP}\PYG{o}{.}\PYG{n}{PropsSI}\PYG{p}{(}\PYG{l+s+s2}{\PYGZdq{}}\PYG{l+s+s2}{H}\PYG{l+s+s2}{\PYGZdq{}}\PYG{p}{,} \PYG{l+s+s2}{\PYGZdq{}}\PYG{l+s+s2}{P}\PYG{l+s+s2}{\PYGZdq{}}\PYG{p}{,} \PYG{n}{P}\PYG{p}{,} \PYG{l+s+s2}{\PYGZdq{}}\PYG{l+s+s2}{Q}\PYG{l+s+s2}{\PYGZdq{}}\PYG{p}{,} \PYG{n}{Q}\PYG{p}{,} \PYG{n}{fluid}\PYG{p}{)}\PYG{o}{/}\PYG{l+m+mi}{1000} \PYG{k}{for} \PYG{n}{P} \PYG{o+ow}{in} \PYG{n}{P\PYGZus{}vals}\PYG{p}{]} \PYG{c+c1}{\PYGZsh{} call for enthalpy values using CoolProp}

\PYG{n}{plt}\PYG{o}{.}\PYG{n}{plot}\PYG{p}{(}\PYG{n}{enthalpy}\PYG{p}{,} \PYG{n}{P\PYGZus{}vals}\PYG{p}{,} \PYG{l+s+s2}{\PYGZdq{}}\PYG{l+s+s2}{\PYGZhy{}b}\PYG{l+s+s2}{\PYGZdq{}}\PYG{p}{)}  \PYG{c+c1}{\PYGZsh{} plot pressure vs enthalpy}


\PYG{n}{plt}\PYG{o}{.}\PYG{n}{yscale}\PYG{p}{(}\PYG{l+s+s2}{\PYGZdq{}}\PYG{l+s+s2}{log}\PYG{l+s+s2}{\PYGZdq{}}\PYG{p}{)}  \PYG{c+c1}{\PYGZsh{} use log scale on y axis}
\PYG{n}{plt}\PYG{o}{.}\PYG{n}{ylabel}\PYG{p}{(}\PYG{l+s+s2}{\PYGZdq{}}\PYG{l+s+s2}{Pressure [Pa]}\PYG{l+s+s2}{\PYGZdq{}}\PYG{p}{)}  \PYG{c+c1}{\PYGZsh{} give y axis a label}
\PYG{n}{plt}\PYG{o}{.}\PYG{n}{xlabel}\PYG{p}{(}\PYG{l+s+s2}{\PYGZdq{}}\PYG{l+s+s2}{Enthalpy [kJ/kg]}\PYG{l+s+s2}{\PYGZdq{}}\PYG{p}{)}  \PYG{c+c1}{\PYGZsh{} give x axis a label}
\PYG{n}{plt}\PYG{o}{.}\PYG{n}{grid}\PYG{p}{(}\PYG{p}{)}
\PYG{n}{plt}\PYG{o}{.}\PYG{n}{legend}\PYG{p}{(}\PYG{p}{)}

\PYG{c+c1}{\PYGZsh{} Building constant temperature curves}

\PYG{n}{T\PYGZus{}up} \PYG{o}{=} \PYG{l+m+mi}{120} \PYG{o}{+} \PYG{l+m+mf}{273.15}
\PYG{n}{T\PYGZus{}mid} \PYG{o}{=} \PYG{l+m+mi}{40} \PYG{o}{+} \PYG{l+m+mf}{273.15}
\PYG{n}{T\PYGZus{}down} \PYG{o}{=} \PYG{o}{\PYGZhy{}}\PYG{l+m+mi}{20} \PYG{o}{+} \PYG{l+m+mf}{273.15}

\PYG{n}{P\PYGZus{}max} \PYG{o}{=} \PYG{l+m+mf}{20E+6}  \PYG{c+c1}{\PYGZsh{} max pressure in the plot set to 20MPa}
\PYG{n}{P\PYGZus{}vals} \PYG{o}{=} \PYG{n}{np}\PYG{o}{.}\PYG{n}{linspace}\PYG{p}{(}\PYG{n}{P\PYGZus{}min}\PYG{p}{,} \PYG{n}{P\PYGZus{}max}\PYG{p}{,} \PYG{l+m+mi}{10000}\PYG{p}{)}  \PYG{c+c1}{\PYGZsh{} define an array of values from P\PYGZus{}min to P\PYGZus{}max}
\PYG{n}{enthalpy\PYGZus{}up} \PYG{o}{=} \PYG{p}{[}\PYG{n}{CP}\PYG{o}{.}\PYG{n}{PropsSI}\PYG{p}{(}\PYG{l+s+s2}{\PYGZdq{}}\PYG{l+s+s2}{H}\PYG{l+s+s2}{\PYGZdq{}}\PYG{p}{,} \PYG{l+s+s2}{\PYGZdq{}}\PYG{l+s+s2}{P}\PYG{l+s+s2}{\PYGZdq{}}\PYG{p}{,} \PYG{n}{P}\PYG{p}{,} \PYG{l+s+s2}{\PYGZdq{}}\PYG{l+s+s2}{T}\PYG{l+s+s2}{\PYGZdq{}}\PYG{p}{,} \PYG{n}{T\PYGZus{}up}\PYG{p}{,} \PYG{n}{fluid}\PYG{p}{)}\PYG{o}{/}\PYG{l+m+mi}{1000} \PYG{k}{for} \PYG{n}{P} \PYG{o+ow}{in} \PYG{n}{P\PYGZus{}vals}\PYG{p}{]} \PYG{c+c1}{\PYGZsh{} call for enthalpy values using CoolProp}
\PYG{n}{enthalpy\PYGZus{}mid} \PYG{o}{=} \PYG{p}{[}\PYG{n}{CP}\PYG{o}{.}\PYG{n}{PropsSI}\PYG{p}{(}\PYG{l+s+s2}{\PYGZdq{}}\PYG{l+s+s2}{H}\PYG{l+s+s2}{\PYGZdq{}}\PYG{p}{,} \PYG{l+s+s2}{\PYGZdq{}}\PYG{l+s+s2}{P}\PYG{l+s+s2}{\PYGZdq{}}\PYG{p}{,} \PYG{n}{P}\PYG{p}{,} \PYG{l+s+s2}{\PYGZdq{}}\PYG{l+s+s2}{T}\PYG{l+s+s2}{\PYGZdq{}}\PYG{p}{,} \PYG{n}{T\PYGZus{}mid}\PYG{p}{,} \PYG{n}{fluid}\PYG{p}{)}\PYG{o}{/}\PYG{l+m+mi}{1000} \PYG{k}{for} \PYG{n}{P} \PYG{o+ow}{in} \PYG{n}{P\PYGZus{}vals}\PYG{p}{]} \PYG{c+c1}{\PYGZsh{} call for enthalpy values using CoolProp}
\PYG{n}{enthalpy\PYGZus{}down} \PYG{o}{=} \PYG{p}{[}\PYG{n}{CP}\PYG{o}{.}\PYG{n}{PropsSI}\PYG{p}{(}\PYG{l+s+s2}{\PYGZdq{}}\PYG{l+s+s2}{H}\PYG{l+s+s2}{\PYGZdq{}}\PYG{p}{,} \PYG{l+s+s2}{\PYGZdq{}}\PYG{l+s+s2}{P}\PYG{l+s+s2}{\PYGZdq{}}\PYG{p}{,} \PYG{n}{P}\PYG{p}{,} \PYG{l+s+s2}{\PYGZdq{}}\PYG{l+s+s2}{T}\PYG{l+s+s2}{\PYGZdq{}}\PYG{p}{,} \PYG{n}{T\PYGZus{}down}\PYG{p}{,} \PYG{n}{fluid}\PYG{p}{)}\PYG{o}{/}\PYG{l+m+mi}{1000} \PYG{k}{for} \PYG{n}{P} \PYG{o+ow}{in} \PYG{n}{P\PYGZus{}vals}\PYG{p}{]} \PYG{c+c1}{\PYGZsh{} call for enthalpy values using CoolProp}

\PYG{n}{plt}\PYG{o}{.}\PYG{n}{plot}\PYG{p}{(}\PYG{n}{enthalpy\PYGZus{}up}\PYG{p}{,} \PYG{n}{P\PYGZus{}vals}\PYG{p}{,} \PYG{l+s+s2}{\PYGZdq{}}\PYG{l+s+s2}{\PYGZhy{}.y}\PYG{l+s+s2}{\PYGZdq{}}\PYG{p}{,} \PYG{n}{label}\PYG{o}{=}\PYG{l+s+s2}{\PYGZdq{}}\PYG{l+s+si}{\PYGZob{}\PYGZcb{}}\PYG{l+s+s2}{ °C}\PYG{l+s+s2}{\PYGZdq{}}\PYG{o}{.}\PYG{n}{format}\PYG{p}{(}\PYG{n}{T\PYGZus{}up}\PYG{o}{\PYGZhy{}}\PYG{l+m+mf}{273.15}\PYG{p}{)}\PYG{p}{)}  \PYG{c+c1}{\PYGZsh{} plot pressure vs enthalpy}
\PYG{n}{plt}\PYG{o}{.}\PYG{n}{plot}\PYG{p}{(}\PYG{n}{enthalpy\PYGZus{}mid}\PYG{p}{,} \PYG{n}{P\PYGZus{}vals}\PYG{p}{,} \PYG{l+s+s2}{\PYGZdq{}}\PYG{l+s+s2}{:r}\PYG{l+s+s2}{\PYGZdq{}}\PYG{p}{,} \PYG{n}{label}\PYG{o}{=}\PYG{l+s+s2}{\PYGZdq{}}\PYG{l+s+si}{\PYGZob{}\PYGZcb{}}\PYG{l+s+s2}{ °C}\PYG{l+s+s2}{\PYGZdq{}}\PYG{o}{.}\PYG{n}{format}\PYG{p}{(}\PYG{n}{T\PYGZus{}mid}\PYG{o}{\PYGZhy{}}\PYG{l+m+mf}{273.15}\PYG{p}{)}\PYG{p}{)}  \PYG{c+c1}{\PYGZsh{} plot pressure vs enthalpy}
\PYG{n}{plt}\PYG{o}{.}\PYG{n}{plot}\PYG{p}{(}\PYG{n}{enthalpy\PYGZus{}down}\PYG{p}{,} \PYG{n}{P\PYGZus{}vals}\PYG{p}{,} \PYG{l+s+s2}{\PYGZdq{}}\PYG{l+s+s2}{\PYGZhy{}\PYGZhy{}m}\PYG{l+s+s2}{\PYGZdq{}}\PYG{p}{,} \PYG{n}{label}\PYG{o}{=}\PYG{l+s+s2}{\PYGZdq{}}\PYG{l+s+si}{\PYGZob{}\PYGZcb{}}\PYG{l+s+s2}{ °C}\PYG{l+s+s2}{\PYGZdq{}}\PYG{o}{.}\PYG{n}{format}\PYG{p}{(}\PYG{n}{T\PYGZus{}down}\PYG{o}{\PYGZhy{}}\PYG{l+m+mf}{273.15}\PYG{p}{)}\PYG{p}{)}  \PYG{c+c1}{\PYGZsh{} plot pressure vs enthalpy}
\PYG{n}{plt}\PYG{o}{.}\PYG{n}{legend}\PYG{p}{(}\PYG{p}{)}
\end{sphinxVerbatim}

\end{sphinxuseclass}\end{sphinxVerbatimInput}
\begin{sphinxVerbatimOutput}

\begin{sphinxuseclass}{cell_output}
\begin{sphinxVerbatim}[commandchars=\\\{\}]
\PYGZlt{}matplotlib.legend.Legend at 0x7f5d64166100\PYGZgt{}
\end{sphinxVerbatim}

\noindent\sphinxincludegraphics{{3416c23ddc2c07f672117ba09ae411f7b5e68ea1576b6efdf565e0620d3b5c48}.png}

\end{sphinxuseclass}\end{sphinxVerbatimOutput}

\end{sphinxuseclass}
\sphinxstepscope


\section{Refigeration cycle: R134a Part 2}
\label{\detokenize{notebooks/Chapter5/CH5-Q5:refigeration-cycle-r134a-part-2}}\label{\detokenize{notebooks/Chapter5/CH5-Q5::doc}}
\sphinxAtStartPar
consider a refrigration cycle working with R134\sphinxhyphen{}a as coolant. The refrigrnt absorbs heat at \(-20^{\circ}  C\) during evaporation and enters the compressor as saturated vapor. The compressor then pressurizes R134\sphinxhyphen{}a while the its temperature increases to \(120^{\circ}  C\). The refrigrant is then cooled down to saturated liquid at \(40^{\circ}  C\) in a condensor in constant pressure before entering a throttling valve. The refrigrant is then throttled to \(-20^{\circ}  C\) to provide refrigrating fluid for the evaporator. Calculate,

\sphinxAtStartPar
a) specific enthalpy of the refrigrant entering the throttling valve

\sphinxAtStartPar
b) specific enthalpy, quality and pressure of the refrigrant exiting the throttling valve

\sphinxAtStartPar
c) specific enthalpy and pressure at compressor inlet

\sphinxAtStartPar
d) specific enthalpy at compressor outlet

\sphinxAtStartPar
e) how much heat is absorbed in the evaporator per unit mass of the refrigrant

\sphinxAtStartPar
f) how much heat is rejected in the condenser per unit mass of the refrigrant

\sphinxAtStartPar
g) how much work is required by the compresor to elevate \(1\:kg\) of the refrigrant’s pressure; what would be the coefficient of performance (COP)?

\sphinxAtStartPar
h) the polytropic constant assuming a polytropic process in the compressor

\sphinxAtStartPar
i) verify the first law of thermodynamics for the system as a whoole

\sphinxAtStartPar
j) draw the process on a P\sphinxhyphen{}h diagram developed previously for Q4 at this chapter

\sphinxAtStartPar
\sphinxincludegraphics{{CH5-Q5}.png}


\section{Solution Approach for a)}
\label{\detokenize{notebooks/Chapter5/CH5-Q5:solution-approach-for-a}}
\sphinxAtStartPar
The refrigrant entering the throttling valve comes from the condenser; therefore the state 3 enthalpy is desired. The temperature at this state is guven as \(T_3=0^{\circ}  C\) and the refrigrant is at its satuarated liquid state with quality equal to \(x=0\).

\begin{sphinxuseclass}{cell}\begin{sphinxVerbatimInput}

\begin{sphinxuseclass}{cell_input}
\begin{sphinxVerbatim}[commandchars=\\\{\}]
\PYG{c+c1}{\PYGZsh{} import the libraries we\PYGZsq{}ll need}
\PYG{k+kn}{import} \PYG{n+nn}{CoolProp}\PYG{n+nn}{.}\PYG{n+nn}{CoolProp} \PYG{k}{as} \PYG{n+nn}{CP}

\PYG{c+c1}{\PYGZsh{} define variables}
\PYG{n}{fluid} \PYG{o}{=} \PYG{l+s+s2}{\PYGZdq{}}\PYG{l+s+s2}{R134A}\PYG{l+s+s2}{\PYGZdq{}}  \PYG{c+c1}{\PYGZsh{} define the fluid or material of interest}
\PYG{n}{T\PYGZus{}3} \PYG{o}{=} \PYG{l+m+mi}{40} \PYG{o}{+} \PYG{l+m+mf}{273.15} \PYG{c+c1}{\PYGZsh{}stete \PYGZsh{}3 temperature in K}
\PYG{n}{h\PYGZus{}3} \PYG{o}{=} \PYG{n}{CP}\PYG{o}{.}\PYG{n}{PropsSI}\PYG{p}{(}\PYG{l+s+s2}{\PYGZdq{}}\PYG{l+s+s2}{H}\PYG{l+s+s2}{\PYGZdq{}}\PYG{p}{,} \PYG{l+s+s2}{\PYGZdq{}}\PYG{l+s+s2}{T}\PYG{l+s+s2}{\PYGZdq{}}\PYG{p}{,} \PYG{n}{T\PYGZus{}3}\PYG{p}{,} \PYG{l+s+s2}{\PYGZdq{}}\PYG{l+s+s2}{Q}\PYG{l+s+s2}{\PYGZdq{}}\PYG{p}{,} \PYG{l+m+mi}{0}\PYG{p}{,} \PYG{n}{fluid}\PYG{p}{)}\PYG{o}{/}\PYG{l+m+mi}{1000}  \PYG{c+c1}{\PYGZsh{} enthalpy of the refrigrant at state \PYGZsh{}3 in kJ/kg}
\PYG{n+nb}{print}\PYG{p}{(}\PYG{l+s+s1}{\PYGZsq{}}\PYG{l+s+s1}{Th specific of the refrigrant at state \PYGZsh{}3 is:}\PYG{l+s+s1}{\PYGZsq{}}\PYG{p}{,} \PYG{l+s+sa}{f}\PYG{l+s+s2}{\PYGZdq{}}\PYG{l+s+si}{\PYGZob{}}\PYG{n}{h\PYGZus{}3}\PYG{l+s+si}{:}\PYG{l+s+s2}{.1f}\PYG{l+s+si}{\PYGZcb{}}\PYG{l+s+s2}{\PYGZdq{}}\PYG{p}{,} \PYG{l+s+s1}{\PYGZsq{}}\PYG{l+s+s1}{kJ/kg}\PYG{l+s+s1}{\PYGZsq{}}\PYG{p}{)}
\end{sphinxVerbatim}

\end{sphinxuseclass}\end{sphinxVerbatimInput}
\begin{sphinxVerbatimOutput}

\begin{sphinxuseclass}{cell_output}
\begin{sphinxVerbatim}[commandchars=\\\{\}]
Th specific of the refrigrant at state \PYGZsh{}3 is: 256.4 kJ/kg
\end{sphinxVerbatim}

\end{sphinxuseclass}\end{sphinxVerbatimOutput}

\end{sphinxuseclass}

\section{Solution Approach for b)}
\label{\detokenize{notebooks/Chapter5/CH5-Q5:solution-approach-for-b}}
\sphinxAtStartPar
A throttling valve is ideally assumed to conserve enthalpy considering the first law of thermodynamics. Therefore,

\sphinxAtStartPar
\(h_3 = h_4\)

\sphinxAtStartPar
The refrigrant enters the evaporator as saturated liquid at \(-20^{\circ}  C\), therefore

\sphinxAtStartPar
\(h_4 = h_{f@-20^{\circ}  C}+xh_{fg@-20^{\circ}  C}\)

\sphinxAtStartPar
\(x = (h_4 - h_{f@-20^{\circ}  C}) / h_{fg@-20^{\circ}  C}\)

\sphinxAtStartPar
The pressure at the throttling valve exit, which is same as evaporator inlet, would be satuartion pressure at the evaporator’s working temperature since the refrigrant enters the evaporator as saturated fluid.

\sphinxAtStartPar
\(P_4 = P_{sat@-20^{\circ}  C}\)

\begin{sphinxuseclass}{cell}\begin{sphinxVerbatimInput}

\begin{sphinxuseclass}{cell_input}
\begin{sphinxVerbatim}[commandchars=\\\{\}]
\PYG{n}{h\PYGZus{}4} \PYG{o}{=} \PYG{n}{h\PYGZus{}3}   \PYG{c+c1}{\PYGZsh{}constant enthalpy through a throttling valve}

\PYG{n}{T\PYGZus{}4} \PYG{o}{=} \PYG{o}{\PYGZhy{}}\PYG{l+m+mi}{20} \PYG{o}{+} \PYG{l+m+mf}{273.15}   \PYG{c+c1}{\PYGZsh{}temperature of refrigrant at state \PYGZsh{}4 in K}
\PYG{n}{h\PYGZus{}gN20} \PYG{o}{=} \PYG{n}{CP}\PYG{o}{.}\PYG{n}{PropsSI}\PYG{p}{(}\PYG{l+s+s2}{\PYGZdq{}}\PYG{l+s+s2}{H}\PYG{l+s+s2}{\PYGZdq{}}\PYG{p}{,} \PYG{l+s+s2}{\PYGZdq{}}\PYG{l+s+s2}{T}\PYG{l+s+s2}{\PYGZdq{}}\PYG{p}{,} \PYG{n}{T\PYGZus{}4}\PYG{p}{,} \PYG{l+s+s2}{\PYGZdq{}}\PYG{l+s+s2}{Q}\PYG{l+s+s2}{\PYGZdq{}}\PYG{p}{,} \PYG{l+m+mi}{1}\PYG{p}{,} \PYG{n}{fluid}\PYG{p}{)}\PYG{o}{/}\PYG{l+m+mi}{1000}   \PYG{c+c1}{\PYGZsh{}enthalpy of sat vap at \PYGZhy{}20C in kJ/kg}
\PYG{n}{h\PYGZus{}fN20} \PYG{o}{=} \PYG{n}{CP}\PYG{o}{.}\PYG{n}{PropsSI}\PYG{p}{(}\PYG{l+s+s2}{\PYGZdq{}}\PYG{l+s+s2}{H}\PYG{l+s+s2}{\PYGZdq{}}\PYG{p}{,} \PYG{l+s+s2}{\PYGZdq{}}\PYG{l+s+s2}{T}\PYG{l+s+s2}{\PYGZdq{}}\PYG{p}{,} \PYG{n}{T\PYGZus{}4}\PYG{p}{,} \PYG{l+s+s2}{\PYGZdq{}}\PYG{l+s+s2}{Q}\PYG{l+s+s2}{\PYGZdq{}}\PYG{p}{,} \PYG{l+m+mi}{0}\PYG{p}{,} \PYG{n}{fluid}\PYG{p}{)}\PYG{o}{/}\PYG{l+m+mi}{1000}   \PYG{c+c1}{\PYGZsh{}enthalpy of sat lig at \PYGZhy{}20C in kJ/kg}
\PYG{n}{h\PYGZus{}fgN20} \PYG{o}{=} \PYG{n}{h\PYGZus{}gN20} \PYG{o}{\PYGZhy{}} \PYG{n}{h\PYGZus{}fN20}   \PYG{c+c1}{\PYGZsh{}h\PYGZus{}fg at \PYGZhy{}20C in kJ/kg}
\PYG{n}{x} \PYG{o}{=} \PYG{p}{(}\PYG{n}{h\PYGZus{}4} \PYG{o}{\PYGZhy{}} \PYG{n}{h\PYGZus{}fN20}\PYG{p}{)}\PYG{o}{/}\PYG{n}{h\PYGZus{}fgN20}  \PYG{c+c1}{\PYGZsh{}qaulity of refrigrant entering the evaporator}

\PYG{n}{P\PYGZus{}4} \PYG{o}{=} \PYG{n}{CP}\PYG{o}{.}\PYG{n}{PropsSI}\PYG{p}{(}\PYG{l+s+s2}{\PYGZdq{}}\PYG{l+s+s2}{P}\PYG{l+s+s2}{\PYGZdq{}}\PYG{p}{,} \PYG{l+s+s2}{\PYGZdq{}}\PYG{l+s+s2}{T}\PYG{l+s+s2}{\PYGZdq{}}\PYG{p}{,} \PYG{n}{T\PYGZus{}4}\PYG{p}{,} \PYG{l+s+s2}{\PYGZdq{}}\PYG{l+s+s2}{Q}\PYG{l+s+s2}{\PYGZdq{}}\PYG{p}{,} \PYG{n}{x}\PYG{p}{,} \PYG{n}{fluid}\PYG{p}{)}  \PYG{c+c1}{\PYGZsh{} pressure of the refrigrant at state \PYGZsh{}1 in Pa}

\PYG{n+nb}{print}\PYG{p}{(}\PYG{l+s+s1}{\PYGZsq{}}\PYG{l+s+s1}{The specific of the refrigrant at state \PYGZsh{}4 is:}\PYG{l+s+s1}{\PYGZsq{}}\PYG{p}{,} \PYG{l+s+sa}{f}\PYG{l+s+s2}{\PYGZdq{}}\PYG{l+s+si}{\PYGZob{}}\PYG{n}{h\PYGZus{}4}\PYG{l+s+si}{:}\PYG{l+s+s2}{.1f}\PYG{l+s+si}{\PYGZcb{}}\PYG{l+s+s2}{\PYGZdq{}}\PYG{p}{,} \PYG{l+s+s1}{\PYGZsq{}}\PYG{l+s+s1}{kJ/kg}\PYG{l+s+s1}{\PYGZsq{}}\PYG{p}{)}
\PYG{n+nb}{print}\PYG{p}{(}\PYG{l+s+s1}{\PYGZsq{}}\PYG{l+s+s1}{The quality of the refrigrant at state \PYGZsh{}4 is:}\PYG{l+s+s1}{\PYGZsq{}}\PYG{p}{,} \PYG{l+s+sa}{f}\PYG{l+s+s2}{\PYGZdq{}}\PYG{l+s+si}{\PYGZob{}}\PYG{n}{x}\PYG{l+s+si}{:}\PYG{l+s+s2}{.3f}\PYG{l+s+si}{\PYGZcb{}}\PYG{l+s+s2}{\PYGZdq{}}\PYG{p}{)}
\PYG{n+nb}{print}\PYG{p}{(}\PYG{l+s+s1}{\PYGZsq{}}\PYG{l+s+s1}{The pressure of the refrigrant at state \PYGZsh{}4 is:}\PYG{l+s+s1}{\PYGZsq{}}\PYG{p}{,} \PYG{l+s+sa}{f}\PYG{l+s+s2}{\PYGZdq{}}\PYG{l+s+si}{\PYGZob{}}\PYG{n}{P\PYGZus{}4}\PYG{l+s+si}{:}\PYG{l+s+s2}{.1f}\PYG{l+s+si}{\PYGZcb{}}\PYG{l+s+s2}{\PYGZdq{}}\PYG{p}{,} \PYG{l+s+s1}{\PYGZsq{}}\PYG{l+s+s1}{Pa}\PYG{l+s+s1}{\PYGZsq{}}\PYG{p}{)}
\end{sphinxVerbatim}

\end{sphinxuseclass}\end{sphinxVerbatimInput}
\begin{sphinxVerbatimOutput}

\begin{sphinxuseclass}{cell_output}
\begin{sphinxVerbatim}[commandchars=\\\{\}]
The specific of the refrigrant at state \PYGZsh{}4 is: 256.4 kJ/kg
The quality of the refrigrant at state \PYGZsh{}4 is: 0.389
The pressure of the refrigrant at state \PYGZsh{}4 is: 132735.0 Pa
\end{sphinxVerbatim}

\end{sphinxuseclass}\end{sphinxVerbatimOutput}

\end{sphinxuseclass}

\section{Solution Approach for c)}
\label{\detokenize{notebooks/Chapter5/CH5-Q5:solution-approach-for-c}}
\sphinxAtStartPar
The refrigrant enters the compressor as saturated vapor, therefore it has gone through constant pressure(and temperature) heating in the evaporator whose temperature is \(-20^{\circ}  C\). So,

\sphinxAtStartPar
\(h_1 = h_v@-20^{\circ}  C\)

\sphinxAtStartPar
\(P_1=P_{sat@-20^{\circ}  C}\)

\begin{sphinxuseclass}{cell}\begin{sphinxVerbatimInput}

\begin{sphinxuseclass}{cell_input}
\begin{sphinxVerbatim}[commandchars=\\\{\}]
\PYG{n}{T\PYGZus{}1} \PYG{o}{=} \PYG{n}{T\PYGZus{}4} \PYG{c+c1}{\PYGZsh{}temperature at state \PYGZsh{}1 in K}
\PYG{n}{h\PYGZus{}1} \PYG{o}{=} \PYG{n}{CP}\PYG{o}{.}\PYG{n}{PropsSI}\PYG{p}{(}\PYG{l+s+s2}{\PYGZdq{}}\PYG{l+s+s2}{H}\PYG{l+s+s2}{\PYGZdq{}}\PYG{p}{,} \PYG{l+s+s2}{\PYGZdq{}}\PYG{l+s+s2}{T}\PYG{l+s+s2}{\PYGZdq{}}\PYG{p}{,} \PYG{n}{T\PYGZus{}1}\PYG{p}{,} \PYG{l+s+s2}{\PYGZdq{}}\PYG{l+s+s2}{Q}\PYG{l+s+s2}{\PYGZdq{}}\PYG{p}{,} \PYG{l+m+mi}{1}\PYG{p}{,} \PYG{n}{fluid}\PYG{p}{)}\PYG{o}{/}\PYG{l+m+mi}{1000}  \PYG{c+c1}{\PYGZsh{} enthalpy of the refrigrant at state \PYGZsh{}1 in kJ/kg}
\PYG{n}{P\PYGZus{}1} \PYG{o}{=} \PYG{n}{CP}\PYG{o}{.}\PYG{n}{PropsSI}\PYG{p}{(}\PYG{l+s+s2}{\PYGZdq{}}\PYG{l+s+s2}{P}\PYG{l+s+s2}{\PYGZdq{}}\PYG{p}{,} \PYG{l+s+s2}{\PYGZdq{}}\PYG{l+s+s2}{T}\PYG{l+s+s2}{\PYGZdq{}}\PYG{p}{,} \PYG{n}{T\PYGZus{}1}\PYG{p}{,} \PYG{l+s+s2}{\PYGZdq{}}\PYG{l+s+s2}{Q}\PYG{l+s+s2}{\PYGZdq{}}\PYG{p}{,} \PYG{l+m+mi}{1}\PYG{p}{,} \PYG{n}{fluid}\PYG{p}{)}  \PYG{c+c1}{\PYGZsh{} pressure of the refrigrant at state \PYGZsh{}1 in Pa}
\PYG{n+nb}{print}\PYG{p}{(}\PYG{l+s+s1}{\PYGZsq{}}\PYG{l+s+s1}{Th specific enthalpy of the refrigrant at state \PYGZsh{}1 is:}\PYG{l+s+s1}{\PYGZsq{}}\PYG{p}{,} \PYG{l+s+sa}{f}\PYG{l+s+s2}{\PYGZdq{}}\PYG{l+s+si}{\PYGZob{}}\PYG{n}{h\PYGZus{}1}\PYG{l+s+si}{:}\PYG{l+s+s2}{.1f}\PYG{l+s+si}{\PYGZcb{}}\PYG{l+s+s2}{\PYGZdq{}}\PYG{p}{,} \PYG{l+s+s1}{\PYGZsq{}}\PYG{l+s+s1}{kJ/kg}\PYG{l+s+s1}{\PYGZsq{}}\PYG{p}{)}
\PYG{n+nb}{print}\PYG{p}{(}\PYG{l+s+s1}{\PYGZsq{}}\PYG{l+s+s1}{Th pressure of the refrigrant at state \PYGZsh{}1 is:}\PYG{l+s+s1}{\PYGZsq{}}\PYG{p}{,} \PYG{l+s+sa}{f}\PYG{l+s+s2}{\PYGZdq{}}\PYG{l+s+si}{\PYGZob{}}\PYG{n}{P\PYGZus{}1}\PYG{l+s+si}{:}\PYG{l+s+s2}{.1f}\PYG{l+s+si}{\PYGZcb{}}\PYG{l+s+s2}{\PYGZdq{}}\PYG{p}{,} \PYG{l+s+s1}{\PYGZsq{}}\PYG{l+s+s1}{Pa}\PYG{l+s+s1}{\PYGZsq{}}\PYG{p}{)}
\end{sphinxVerbatim}

\end{sphinxuseclass}\end{sphinxVerbatimInput}
\begin{sphinxVerbatimOutput}

\begin{sphinxuseclass}{cell_output}
\begin{sphinxVerbatim}[commandchars=\\\{\}]
Th specific enthalpy of the refrigrant at state \PYGZsh{}1 is: 386.6 kJ/kg
Th pressure of the refrigrant at state \PYGZsh{}1 is: 132735.0 Pa
\end{sphinxVerbatim}

\end{sphinxuseclass}\end{sphinxVerbatimOutput}

\end{sphinxuseclass}

\section{Solution Approach for d)}
\label{\detokenize{notebooks/Chapter5/CH5-Q5:solution-approach-for-d}}
\sphinxAtStartPar
The refrigrant is heated to \(T_2=120^{\circ}  C\). The pressure to which the refrigrant is pressurized to, however, is unknown. The condensor is assumed to be operating with constant pressure; therefore, the pressure keeps constant and the pressure at condensor outlet would be same as pressure at its inlet which is same as compressor outlet. The pressure at condensor outlet is calculated based on the refrigrant being at saturation state at \(0^{\circ}  C\).

\sphinxAtStartPar
\(P_2=P_3\)

\sphinxAtStartPar
\(P_3=P_{satR134a@-20^{\circ}  C}\)

\sphinxAtStartPar
The pressure and temperature at the compressor outlet are then used to calculate enthalpy at this state.

\begin{sphinxuseclass}{cell}\begin{sphinxVerbatimInput}

\begin{sphinxuseclass}{cell_input}
\begin{sphinxVerbatim}[commandchars=\\\{\}]
\PYG{n}{T\PYGZus{}2} \PYG{o}{=} \PYG{l+m+mi}{120} \PYG{o}{+} \PYG{l+m+mf}{273.15} \PYG{c+c1}{\PYGZsh{}temperature at state \PYGZsh{}2 in K}
\PYG{n}{P\PYGZus{}3} \PYG{o}{=} \PYG{n}{CP}\PYG{o}{.}\PYG{n}{PropsSI}\PYG{p}{(}\PYG{l+s+s2}{\PYGZdq{}}\PYG{l+s+s2}{P}\PYG{l+s+s2}{\PYGZdq{}}\PYG{p}{,} \PYG{l+s+s2}{\PYGZdq{}}\PYG{l+s+s2}{T}\PYG{l+s+s2}{\PYGZdq{}}\PYG{p}{,} \PYG{n}{T\PYGZus{}3}\PYG{p}{,} \PYG{l+s+s2}{\PYGZdq{}}\PYG{l+s+s2}{Q}\PYG{l+s+s2}{\PYGZdq{}}\PYG{p}{,} \PYG{l+m+mi}{0}\PYG{p}{,} \PYG{n}{fluid}\PYG{p}{)}  \PYG{c+c1}{\PYGZsh{} pressure of the refrigrant at state \PYGZsh{}3 in Pa}
\PYG{n}{P\PYGZus{}2} \PYG{o}{=} \PYG{n}{P\PYGZus{}3} \PYG{c+c1}{\PYGZsh{} pressure of the refrigrant at state \PYGZsh{}2 in Pa}
\PYG{n}{h\PYGZus{}2} \PYG{o}{=} \PYG{n}{CP}\PYG{o}{.}\PYG{n}{PropsSI}\PYG{p}{(}\PYG{l+s+s2}{\PYGZdq{}}\PYG{l+s+s2}{H}\PYG{l+s+s2}{\PYGZdq{}}\PYG{p}{,} \PYG{l+s+s2}{\PYGZdq{}}\PYG{l+s+s2}{T}\PYG{l+s+s2}{\PYGZdq{}}\PYG{p}{,} \PYG{n}{T\PYGZus{}2}\PYG{p}{,} \PYG{l+s+s2}{\PYGZdq{}}\PYG{l+s+s2}{P}\PYG{l+s+s2}{\PYGZdq{}}\PYG{p}{,} \PYG{n}{P\PYGZus{}2}\PYG{p}{,} \PYG{n}{fluid}\PYG{p}{)}\PYG{o}{/}\PYG{l+m+mi}{1000}  \PYG{c+c1}{\PYGZsh{} enthalpy of the refrigrant at state \PYGZsh{}2 in kJ/kg}
\PYG{n+nb}{print}\PYG{p}{(}\PYG{l+s+s1}{\PYGZsq{}}\PYG{l+s+s1}{The specific of the refrigrant at state \PYGZsh{}2 is:}\PYG{l+s+s1}{\PYGZsq{}}\PYG{p}{,} \PYG{l+s+sa}{f}\PYG{l+s+s2}{\PYGZdq{}}\PYG{l+s+si}{\PYGZob{}}\PYG{n}{h\PYGZus{}2}\PYG{l+s+si}{:}\PYG{l+s+s2}{.1f}\PYG{l+s+si}{\PYGZcb{}}\PYG{l+s+s2}{\PYGZdq{}}\PYG{p}{,} \PYG{l+s+s1}{\PYGZsq{}}\PYG{l+s+s1}{kJ/kg}\PYG{l+s+s1}{\PYGZsq{}}\PYG{p}{)}
\end{sphinxVerbatim}

\end{sphinxuseclass}\end{sphinxVerbatimInput}
\begin{sphinxVerbatimOutput}

\begin{sphinxuseclass}{cell_output}
\begin{sphinxVerbatim}[commandchars=\\\{\}]
The specific of the refrigrant at state \PYGZsh{}2 is: 504.1 kJ/kg
\end{sphinxVerbatim}

\end{sphinxuseclass}\end{sphinxVerbatimOutput}

\end{sphinxuseclass}

\section{Solution Approach for e)}
\label{\detokenize{notebooks/Chapter5/CH5-Q5:solution-approach-for-e}}
\sphinxAtStartPar
Considerin the first law of thermodynamics for the evaporator,

\sphinxAtStartPar
\(q_c=h_1-h_4\)

\begin{sphinxuseclass}{cell}\begin{sphinxVerbatimInput}

\begin{sphinxuseclass}{cell_input}
\begin{sphinxVerbatim}[commandchars=\\\{\}]
\PYG{n}{q\PYGZus{}c} \PYG{o}{=} \PYG{n}{h\PYGZus{}1} \PYG{o}{\PYGZhy{}} \PYG{n}{h\PYGZus{}4}
\PYG{n+nb}{print}\PYG{p}{(}\PYG{l+s+sa}{f}\PYG{l+s+s2}{\PYGZdq{}}\PYG{l+s+si}{\PYGZob{}}\PYG{n}{q\PYGZus{}c}\PYG{l+s+si}{:}\PYG{l+s+s2}{.1f}\PYG{l+s+si}{\PYGZcb{}}\PYG{l+s+s2}{\PYGZdq{}}\PYG{p}{,} \PYG{l+s+s1}{\PYGZsq{}}\PYG{l+s+s1}{kJ of heat is absorbed by the refrigrant in the evaporator per kg of refrigrant}\PYG{l+s+s1}{\PYGZsq{}}\PYG{p}{)}
\end{sphinxVerbatim}

\end{sphinxuseclass}\end{sphinxVerbatimInput}
\begin{sphinxVerbatimOutput}

\begin{sphinxuseclass}{cell_output}
\begin{sphinxVerbatim}[commandchars=\\\{\}]
130.1 kJ of heat is absorbed by the refrigrant in the evaporator per kg of refrigrant
\end{sphinxVerbatim}

\end{sphinxuseclass}\end{sphinxVerbatimOutput}

\end{sphinxuseclass}

\section{Solution Approach for f)}
\label{\detokenize{notebooks/Chapter5/CH5-Q5:solution-approach-for-f}}
\sphinxAtStartPar
Considerin the first law of thermodynamics for the condenser,

\sphinxAtStartPar
\(q_h=h_2-h_3\)

\begin{sphinxuseclass}{cell}\begin{sphinxVerbatimInput}

\begin{sphinxuseclass}{cell_input}
\begin{sphinxVerbatim}[commandchars=\\\{\}]
\PYG{n}{q\PYGZus{}h} \PYG{o}{=} \PYG{n}{h\PYGZus{}2} \PYG{o}{\PYGZhy{}} \PYG{n}{h\PYGZus{}3}
\PYG{n+nb}{print}\PYG{p}{(}\PYG{l+s+sa}{f}\PYG{l+s+s2}{\PYGZdq{}}\PYG{l+s+si}{\PYGZob{}}\PYG{n}{q\PYGZus{}h}\PYG{l+s+si}{:}\PYG{l+s+s2}{.1f}\PYG{l+s+si}{\PYGZcb{}}\PYG{l+s+s2}{\PYGZdq{}}\PYG{p}{,} \PYG{l+s+s1}{\PYGZsq{}}\PYG{l+s+s1}{kJ of heat is rejected to the environment in the condenser per kg of refrigrant}\PYG{l+s+s1}{\PYGZsq{}}\PYG{p}{)}
\end{sphinxVerbatim}

\end{sphinxuseclass}\end{sphinxVerbatimInput}
\begin{sphinxVerbatimOutput}

\begin{sphinxuseclass}{cell_output}
\begin{sphinxVerbatim}[commandchars=\\\{\}]
247.6 kJ of heat is rejected to the environment in the condenser per kg of refrigrant
\end{sphinxVerbatim}

\end{sphinxuseclass}\end{sphinxVerbatimOutput}

\end{sphinxuseclass}

\section{Solution Approach for g)}
\label{\detokenize{notebooks/Chapter5/CH5-Q5:solution-approach-for-g}}
\sphinxAtStartPar
Considerin the first law of thermodynamics for the compressor,

\sphinxAtStartPar
\(w=h_2-h_1\)

\sphinxAtStartPar
\(COP=q_c/w\)

\begin{sphinxuseclass}{cell}\begin{sphinxVerbatimInput}

\begin{sphinxuseclass}{cell_input}
\begin{sphinxVerbatim}[commandchars=\\\{\}]
\PYG{n}{w} \PYG{o}{=} \PYG{n}{h\PYGZus{}2} \PYG{o}{\PYGZhy{}} \PYG{n}{h\PYGZus{}1}
\PYG{n}{cop} \PYG{o}{=} \PYG{n}{q\PYGZus{}c} \PYG{o}{/} \PYG{n}{w}
\PYG{n+nb}{print}\PYG{p}{(}\PYG{l+s+sa}{f}\PYG{l+s+s2}{\PYGZdq{}}\PYG{l+s+si}{\PYGZob{}}\PYG{n}{w}\PYG{l+s+si}{:}\PYG{l+s+s2}{.1f}\PYG{l+s+si}{\PYGZcb{}}\PYG{l+s+s2}{\PYGZdq{}}\PYG{p}{,} \PYG{l+s+s1}{\PYGZsq{}}\PYG{l+s+s1}{kJ of energy is required to compress 1kg of R134\PYGZhy{}a to the desired pressure}\PYG{l+s+s1}{\PYGZsq{}}\PYG{p}{)}
\PYG{n+nb}{print}\PYG{p}{(}\PYG{l+s+s1}{\PYGZsq{}}\PYG{l+s+s1}{The COP of the cycle is:}\PYG{l+s+s1}{\PYGZsq{}}\PYG{p}{,} \PYG{l+s+sa}{f}\PYG{l+s+s2}{\PYGZdq{}}\PYG{l+s+si}{\PYGZob{}}\PYG{n}{cop}\PYG{l+s+si}{:}\PYG{l+s+s2}{.1f}\PYG{l+s+si}{\PYGZcb{}}\PYG{l+s+s2}{\PYGZdq{}}\PYG{p}{)}
\end{sphinxVerbatim}

\end{sphinxuseclass}\end{sphinxVerbatimInput}
\begin{sphinxVerbatimOutput}

\begin{sphinxuseclass}{cell_output}
\begin{sphinxVerbatim}[commandchars=\\\{\}]
117.5 kJ of energy is required to compress 1kg of R134\PYGZhy{}a to the desired pressure
The COP of the cycle is: 1.1
\end{sphinxVerbatim}

\end{sphinxuseclass}\end{sphinxVerbatimOutput}

\end{sphinxuseclass}

\section{Solution Approach for h)}
\label{\detokenize{notebooks/Chapter5/CH5-Q5:solution-approach-for-h}}
\sphinxAtStartPar
for a polytropic process,

\sphinxAtStartPar
\(Pv^k=constant\)

\sphinxAtStartPar
\(P_1v_1^k=P_2v_2^k\)

\sphinxAtStartPar
\(P_1/P_2=(v_2/v_1)^k\)

\sphinxAtStartPar
\(d (density)=1/v\)

\sphinxAtStartPar
\(P_1/P_2=(d_1/d_2)^k\)

\sphinxAtStartPar
\(log(P_1/P_2)=k\:log(d_1/d_2)\)

\sphinxAtStartPar
\(k=log(P_1/P_2)/log(d_1/d_2)\)

\begin{sphinxuseclass}{cell}\begin{sphinxVerbatimInput}

\begin{sphinxuseclass}{cell_input}
\begin{sphinxVerbatim}[commandchars=\\\{\}]
\PYG{c+c1}{\PYGZsh{}importing required libraries}
\PYG{k+kn}{import} \PYG{n+nn}{numpy} \PYG{k}{as} \PYG{n+nn}{np}

\PYG{c+c1}{\PYGZsh{}evaluating density }
\PYG{n}{d\PYGZus{}1} \PYG{o}{=} \PYG{n}{CP}\PYG{o}{.}\PYG{n}{PropsSI}\PYG{p}{(}\PYG{l+s+s2}{\PYGZdq{}}\PYG{l+s+s2}{D}\PYG{l+s+s2}{\PYGZdq{}}\PYG{p}{,} \PYG{l+s+s2}{\PYGZdq{}}\PYG{l+s+s2}{T}\PYG{l+s+s2}{\PYGZdq{}}\PYG{p}{,} \PYG{n}{T\PYGZus{}1}\PYG{p}{,} \PYG{l+s+s2}{\PYGZdq{}}\PYG{l+s+s2}{Q}\PYG{l+s+s2}{\PYGZdq{}}\PYG{p}{,} \PYG{l+m+mi}{1}\PYG{p}{,} \PYG{n}{fluid}\PYG{p}{)}  \PYG{c+c1}{\PYGZsh{} density of the refrigrant at state \PYGZsh{}1 in kg/m3 based on sat vepor}
\PYG{n}{d\PYGZus{}2} \PYG{o}{=} \PYG{n}{CP}\PYG{o}{.}\PYG{n}{PropsSI}\PYG{p}{(}\PYG{l+s+s2}{\PYGZdq{}}\PYG{l+s+s2}{D}\PYG{l+s+s2}{\PYGZdq{}}\PYG{p}{,} \PYG{l+s+s2}{\PYGZdq{}}\PYG{l+s+s2}{T}\PYG{l+s+s2}{\PYGZdq{}}\PYG{p}{,} \PYG{n}{T\PYGZus{}2}\PYG{p}{,} \PYG{l+s+s2}{\PYGZdq{}}\PYG{l+s+s2}{P}\PYG{l+s+s2}{\PYGZdq{}}\PYG{p}{,} \PYG{n}{P\PYGZus{}2}\PYG{p}{,} \PYG{n}{fluid}\PYG{p}{)}  \PYG{c+c1}{\PYGZsh{} density of the refrigrant at state \PYGZsh{}2 in kg/m3 based on sat vepor}
\PYG{n}{k} \PYG{o}{=} \PYG{n}{np}\PYG{o}{.}\PYG{n}{log}\PYG{p}{(}\PYG{n}{P\PYGZus{}1}\PYG{o}{/}\PYG{n}{P\PYGZus{}2}\PYG{p}{)}\PYG{o}{/}\PYG{n}{np}\PYG{o}{.}\PYG{n}{log}\PYG{p}{(}\PYG{n}{d\PYGZus{}1}\PYG{o}{/}\PYG{n}{d\PYGZus{}2}\PYG{p}{)}
\PYG{n+nb}{print}\PYG{p}{(}\PYG{l+s+s1}{\PYGZsq{}}\PYG{l+s+s1}{Th polytropic constant for the compression process is:}\PYG{l+s+s1}{\PYGZsq{}}\PYG{p}{,} \PYG{l+s+sa}{f}\PYG{l+s+s2}{\PYGZdq{}}\PYG{l+s+si}{\PYGZob{}}\PYG{n}{k}\PYG{l+s+si}{:}\PYG{l+s+s2}{.3f}\PYG{l+s+si}{\PYGZcb{}}\PYG{l+s+s2}{\PYGZdq{}}\PYG{p}{)}
\end{sphinxVerbatim}

\end{sphinxuseclass}\end{sphinxVerbatimInput}
\begin{sphinxVerbatimOutput}

\begin{sphinxuseclass}{cell_output}
\begin{sphinxVerbatim}[commandchars=\\\{\}]
Th polytropic constant for the compression process is: 1.254
\end{sphinxVerbatim}

\end{sphinxuseclass}\end{sphinxVerbatimOutput}

\end{sphinxuseclass}

\section{Solution Approach for i)}
\label{\detokenize{notebooks/Chapter5/CH5-Q5:solution-approach-for-i}}
\sphinxAtStartPar
Considerin the first law of thermodynamics for the whole system,

\sphinxAtStartPar
\(w=q_h-q_c\)

\begin{sphinxuseclass}{cell}\begin{sphinxVerbatimInput}

\begin{sphinxuseclass}{cell_input}
\begin{sphinxVerbatim}[commandchars=\\\{\}]
\PYG{n}{w\PYGZus{}test} \PYG{o}{=}  \PYG{n}{q\PYGZus{}h} \PYG{o}{\PYGZhy{}} \PYG{n}{q\PYGZus{}c}
\PYG{n}{w\PYGZus{}test} \PYG{o}{==} \PYG{n}{w} \PYG{c+c1}{\PYGZsh{}w\PYGZus{}test is the value calculated from the first law and w is the value calculated using enthalpies}
\end{sphinxVerbatim}

\end{sphinxuseclass}\end{sphinxVerbatimInput}
\begin{sphinxVerbatimOutput}

\begin{sphinxuseclass}{cell_output}
\begin{sphinxVerbatim}[commandchars=\\\{\}]
True
\end{sphinxVerbatim}

\end{sphinxuseclass}\end{sphinxVerbatimOutput}

\end{sphinxuseclass}

\section{Solution Approach for j)}
\label{\detokenize{notebooks/Chapter5/CH5-Q5:solution-approach-for-j}}
\sphinxAtStartPar
Except the compression process, other are straight lines of constant pressure or constant enthalpy processes. Therefore, the curve for the compression is to be built based on polytropic process and thermodynamic properties obtained from CoolProp.

\sphinxAtStartPar
from h)

\sphinxAtStartPar
\(P/P_1=(d/d_1)^k\)

\sphinxAtStartPar
\(P=P1\:(d/d_1)^k\)

\sphinxAtStartPar
values of \(P\) and \(d\) are generic pressure and density values for a polytropic process.

\sphinxAtStartPar
then an array of densities ranging from \(d_1\) to \(d_2\) is used to calculate pressure and enthalpy accordingly.

\begin{sphinxuseclass}{cell}\begin{sphinxVerbatimInput}

\begin{sphinxuseclass}{cell_input}
\begin{sphinxVerbatim}[commandchars=\\\{\}]
\PYG{c+c1}{\PYGZsh{} import the libraries we\PYGZsq{}ll need}
\PYG{k+kn}{import} \PYG{n+nn}{CoolProp}\PYG{n+nn}{.}\PYG{n+nn}{CoolProp} \PYG{k}{as} \PYG{n+nn}{CP}
\PYG{k+kn}{import} \PYG{n+nn}{numpy} \PYG{k}{as} \PYG{n+nn}{np}
\PYG{k+kn}{import} \PYG{n+nn}{matplotlib}\PYG{n+nn}{.}\PYG{n+nn}{pyplot} \PYG{k}{as} \PYG{n+nn}{plt}

\PYG{c+c1}{\PYGZsh{}for the compression process 1\PYGZhy{}2}
\PYG{c+c1}{\PYGZsh{}building an array of densities}
\PYG{n}{e} \PYG{o}{=} \PYG{l+m+mi}{1000}   \PYG{c+c1}{\PYGZsh{}number of data points for the polytropic process}
\PYG{n}{d\PYGZus{}12} \PYG{o}{=} \PYG{n}{np}\PYG{o}{.}\PYG{n}{linspace}\PYG{p}{(}\PYG{n}{d\PYGZus{}1}\PYG{p}{,} \PYG{n}{d\PYGZus{}2}\PYG{p}{,} \PYG{n}{e}\PYG{p}{)}  \PYG{c+c1}{\PYGZsh{} define an array of values from d\PYGZus{}1 to d\PYGZus{}2}
\PYG{n}{P\PYGZus{}12} \PYG{o}{=} \PYG{n}{P\PYGZus{}1} \PYG{o}{*} \PYG{p}{(}\PYG{n}{d\PYGZus{}12}\PYG{o}{/}\PYG{n}{d\PYGZus{}1}\PYG{p}{)} \PYG{o}{*}\PYG{o}{*} \PYG{n}{k}   \PYG{c+c1}{\PYGZsh{}array of pressure based on density for the polytropic compression}
\PYG{n}{h\PYGZus{}12} \PYG{o}{=} \PYG{n}{np}\PYG{o}{.}\PYG{n}{zeros}\PYG{p}{(}\PYG{n}{e}\PYG{p}{)}   \PYG{c+c1}{\PYGZsh{}an empty array to store enthalpy values}

\PYG{k}{for} \PYG{n}{i} \PYG{o+ow}{in} \PYG{n+nb}{range}\PYG{p}{(}\PYG{n}{e}\PYG{p}{)}\PYG{p}{:}
   \PYG{n}{h\PYGZus{}12}\PYG{p}{[}\PYG{n}{i}\PYG{p}{]} \PYG{o}{=} \PYG{n}{CP}\PYG{o}{.}\PYG{n}{PropsSI}\PYG{p}{(}\PYG{l+s+s2}{\PYGZdq{}}\PYG{l+s+s2}{H}\PYG{l+s+s2}{\PYGZdq{}}\PYG{p}{,} \PYG{l+s+s2}{\PYGZdq{}}\PYG{l+s+s2}{P}\PYG{l+s+s2}{\PYGZdq{}}\PYG{p}{,} \PYG{n}{P\PYGZus{}12}\PYG{p}{[}\PYG{n}{i}\PYG{p}{]}\PYG{p}{,} \PYG{l+s+s2}{\PYGZdq{}}\PYG{l+s+s2}{D}\PYG{l+s+s2}{\PYGZdq{}}\PYG{p}{,} \PYG{n}{d\PYGZus{}12}\PYG{p}{[}\PYG{n}{i}\PYG{p}{]}\PYG{p}{,} \PYG{n}{fluid}\PYG{p}{)}\PYG{o}{/}\PYG{l+m+mi}{1000}  \PYG{c+c1}{\PYGZsh{} enthalpy values for the polytropic process in kJ/kg}

\PYG{c+c1}{\PYGZsh{}for the cooling process in the condensor in constant pressure}
\PYG{n}{P\PYGZus{}23} \PYG{o}{=} \PYG{n}{np}\PYG{o}{.}\PYG{n}{linspace}\PYG{p}{(}\PYG{n}{P\PYGZus{}2}\PYG{p}{,} \PYG{n}{P\PYGZus{}3}\PYG{p}{,} \PYG{l+m+mi}{1000}\PYG{p}{)}  \PYG{c+c1}{\PYGZsh{} define an array of pressure values from 2 to 3}
\PYG{n}{h\PYGZus{}23} \PYG{o}{=} \PYG{n}{np}\PYG{o}{.}\PYG{n}{linspace}\PYG{p}{(}\PYG{n}{h\PYGZus{}2}\PYG{p}{,} \PYG{n}{h\PYGZus{}3}\PYG{p}{,} \PYG{l+m+mi}{1000}\PYG{p}{)}  \PYG{c+c1}{\PYGZsh{} define a linear array of enthalpy values from 2 to 3}

\PYG{c+c1}{\PYGZsh{}for the process through the throttling valve at constant enthalpy}
\PYG{n}{P\PYGZus{}34} \PYG{o}{=} \PYG{n}{np}\PYG{o}{.}\PYG{n}{linspace}\PYG{p}{(}\PYG{n}{P\PYGZus{}3}\PYG{p}{,} \PYG{n}{P\PYGZus{}4}\PYG{p}{,} \PYG{l+m+mi}{1000}\PYG{p}{)}  \PYG{c+c1}{\PYGZsh{} define an array of pressure values from 3 to 4}
\PYG{n}{h\PYGZus{}34} \PYG{o}{=} \PYG{n}{np}\PYG{o}{.}\PYG{n}{linspace}\PYG{p}{(}\PYG{n}{h\PYGZus{}3}\PYG{p}{,} \PYG{n}{h\PYGZus{}4}\PYG{p}{,} \PYG{l+m+mi}{1000}\PYG{p}{)}  \PYG{c+c1}{\PYGZsh{} define a linear array of enthalpy values from 3 to 4}

\PYG{c+c1}{\PYGZsh{}for the process through the evaporatot at constant pressure}
\PYG{n}{P\PYGZus{}41} \PYG{o}{=} \PYG{n}{np}\PYG{o}{.}\PYG{n}{linspace}\PYG{p}{(}\PYG{n}{P\PYGZus{}4}\PYG{p}{,} \PYG{n}{P\PYGZus{}1}\PYG{p}{,} \PYG{l+m+mi}{1000}\PYG{p}{)}  \PYG{c+c1}{\PYGZsh{} define an array of pressure values from 4 to 1}
\PYG{n}{h\PYGZus{}41} \PYG{o}{=} \PYG{n}{np}\PYG{o}{.}\PYG{n}{linspace}\PYG{p}{(}\PYG{n}{h\PYGZus{}4}\PYG{p}{,} \PYG{n}{h\PYGZus{}1}\PYG{p}{,} \PYG{l+m+mi}{1000}\PYG{p}{)}  \PYG{c+c1}{\PYGZsh{} define a linear array of enthalpy values from 4 to 1}

\PYG{c+c1}{\PYGZsh{}plotting the process on the P\PYGZhy{}h diagram}
\PYG{n}{plt}\PYG{o}{.}\PYG{n}{plot}\PYG{p}{(}\PYG{n}{h\PYGZus{}23}\PYG{p}{,} \PYG{n}{P\PYGZus{}23}\PYG{p}{,} \PYG{l+s+s2}{\PYGZdq{}}\PYG{l+s+s2}{\PYGZhy{}}\PYG{l+s+s2}{\PYGZdq{}}\PYG{p}{,} \PYG{n}{label}\PYG{o}{=}\PYG{l+s+s2}{\PYGZdq{}}\PYG{l+s+s2}{condensor}\PYG{l+s+s2}{\PYGZdq{}}\PYG{p}{)}
\PYG{n}{plt}\PYG{o}{.}\PYG{n}{plot}\PYG{p}{(}\PYG{n}{h\PYGZus{}34}\PYG{p}{,} \PYG{n}{P\PYGZus{}34}\PYG{p}{,} \PYG{l+s+s2}{\PYGZdq{}}\PYG{l+s+s2}{\PYGZhy{}}\PYG{l+s+s2}{\PYGZdq{}}\PYG{p}{,} \PYG{n}{label}\PYG{o}{=}\PYG{l+s+s2}{\PYGZdq{}}\PYG{l+s+s2}{throttling valve}\PYG{l+s+s2}{\PYGZdq{}}\PYG{p}{)}
\PYG{n}{plt}\PYG{o}{.}\PYG{n}{plot}\PYG{p}{(}\PYG{n}{h\PYGZus{}41}\PYG{p}{,} \PYG{n}{P\PYGZus{}41}\PYG{p}{,} \PYG{l+s+s2}{\PYGZdq{}}\PYG{l+s+s2}{\PYGZhy{}}\PYG{l+s+s2}{\PYGZdq{}}\PYG{p}{,} \PYG{n}{label}\PYG{o}{=}\PYG{l+s+s2}{\PYGZdq{}}\PYG{l+s+s2}{evaporator}\PYG{l+s+s2}{\PYGZdq{}}\PYG{p}{)}
\PYG{n}{plt}\PYG{o}{.}\PYG{n}{plot}\PYG{p}{(}\PYG{n}{h\PYGZus{}12}\PYG{p}{,} \PYG{n}{P\PYGZus{}12}\PYG{p}{,} \PYG{l+s+s2}{\PYGZdq{}}\PYG{l+s+s2}{\PYGZhy{}}\PYG{l+s+s2}{\PYGZdq{}}\PYG{p}{,} \PYG{n}{label}\PYG{o}{=}\PYG{l+s+s2}{\PYGZdq{}}\PYG{l+s+s2}{compressor}\PYG{l+s+s2}{\PYGZdq{}}\PYG{p}{)}
\PYG{n}{plt}\PYG{o}{.}\PYG{n}{legend}\PYG{p}{(}\PYG{p}{)}


\PYG{c+c1}{\PYGZsh{}building the P\PYGZhy{}h diagram}


\PYG{c+c1}{\PYGZsh{} define variables}
\PYG{n}{fluid} \PYG{o}{=} \PYG{l+s+s2}{\PYGZdq{}}\PYG{l+s+s2}{R134A}\PYG{l+s+s2}{\PYGZdq{}}  \PYG{c+c1}{\PYGZsh{} define the fluid or material of interest, for full list see CP.Fluidslist()}
\PYG{n}{T\PYGZus{}min} \PYG{o}{=} \PYG{n}{CP}\PYG{o}{.}\PYG{n}{PropsSI}\PYG{p}{(}\PYG{l+s+s2}{\PYGZdq{}}\PYG{l+s+s2}{Tmin}\PYG{l+s+s2}{\PYGZdq{}}\PYG{p}{,} \PYG{n}{fluid}\PYG{p}{)}  \PYG{c+c1}{\PYGZsh{} triple\PYGZhy{}point temp for the fluid}
\PYG{n}{P\PYGZus{}min} \PYG{o}{=} \PYG{n}{CP}\PYG{o}{.}\PYG{n}{PropsSI}\PYG{p}{(}\PYG{l+s+s2}{\PYGZdq{}}\PYG{l+s+s2}{P}\PYG{l+s+s2}{\PYGZdq{}}\PYG{p}{,} \PYG{l+s+s2}{\PYGZdq{}}\PYG{l+s+s2}{T}\PYG{l+s+s2}{\PYGZdq{}}\PYG{p}{,} \PYG{n}{T\PYGZus{}min}\PYG{p}{,} \PYG{l+s+s2}{\PYGZdq{}}\PYG{l+s+s2}{Q}\PYG{l+s+s2}{\PYGZdq{}}\PYG{p}{,} \PYG{l+m+mi}{0}\PYG{p}{,} \PYG{n}{fluid}\PYG{p}{)}  \PYG{c+c1}{\PYGZsh{} triple\PYGZhy{}point pressure for the fluid}
\PYG{n}{P\PYGZus{}max} \PYG{o}{=} \PYG{l+m+mf}{4.03E+6} \PYG{c+c1}{\PYGZsh{}approximate critical pressure}

\PYG{n}{P\PYGZus{}vals} \PYG{o}{=} \PYG{n}{np}\PYG{o}{.}\PYG{n}{linspace}\PYG{p}{(}\PYG{n}{P\PYGZus{}min}\PYG{p}{,} \PYG{n}{P\PYGZus{}max}\PYG{p}{,} \PYG{l+m+mi}{1000}\PYG{p}{)}  \PYG{c+c1}{\PYGZsh{} define an array of values from P\PYGZus{}min to P\PYGZus{}max}
\PYG{n}{Q} \PYG{o}{=} \PYG{l+m+mi}{1}  \PYG{c+c1}{\PYGZsh{} define the fluid quality as 1, which is 100\PYGZpc{} vapor}

\PYG{n}{enthalpy} \PYG{o}{=} \PYG{p}{[}\PYG{n}{CP}\PYG{o}{.}\PYG{n}{PropsSI}\PYG{p}{(}\PYG{l+s+s2}{\PYGZdq{}}\PYG{l+s+s2}{H}\PYG{l+s+s2}{\PYGZdq{}}\PYG{p}{,} \PYG{l+s+s2}{\PYGZdq{}}\PYG{l+s+s2}{P}\PYG{l+s+s2}{\PYGZdq{}}\PYG{p}{,} \PYG{n}{P}\PYG{p}{,} \PYG{l+s+s2}{\PYGZdq{}}\PYG{l+s+s2}{Q}\PYG{l+s+s2}{\PYGZdq{}}\PYG{p}{,} \PYG{n}{Q}\PYG{p}{,} \PYG{n}{fluid}\PYG{p}{)}\PYG{o}{/}\PYG{l+m+mi}{1000} \PYG{k}{for} \PYG{n}{P} \PYG{o+ow}{in} \PYG{n}{P\PYGZus{}vals}\PYG{p}{]}  \PYG{c+c1}{\PYGZsh{} call for enthalpy values using CoolProp}

\PYG{n}{plt}\PYG{o}{.}\PYG{n}{plot}\PYG{p}{(}\PYG{n}{enthalpy}\PYG{p}{,} \PYG{n}{P\PYGZus{}vals}\PYG{p}{,} \PYG{l+s+s2}{\PYGZdq{}}\PYG{l+s+s2}{\PYGZhy{}b}\PYG{l+s+s2}{\PYGZdq{}}\PYG{p}{,} \PYG{n}{label}\PYG{o}{=}\PYG{l+s+s2}{\PYGZdq{}}\PYG{l+s+s2}{Saturation Line}\PYG{l+s+s2}{\PYGZdq{}}\PYG{p}{)}  \PYG{c+c1}{\PYGZsh{} plot pressure vs enthalpy}

\PYG{n}{Q} \PYG{o}{=} \PYG{l+m+mi}{0}  \PYG{c+c1}{\PYGZsh{} define the fluid quality as 0, which is 100\PYGZpc{} liquid}

\PYG{n}{enthalpy} \PYG{o}{=} \PYG{p}{[}\PYG{n}{CP}\PYG{o}{.}\PYG{n}{PropsSI}\PYG{p}{(}\PYG{l+s+s2}{\PYGZdq{}}\PYG{l+s+s2}{H}\PYG{l+s+s2}{\PYGZdq{}}\PYG{p}{,} \PYG{l+s+s2}{\PYGZdq{}}\PYG{l+s+s2}{P}\PYG{l+s+s2}{\PYGZdq{}}\PYG{p}{,} \PYG{n}{P}\PYG{p}{,} \PYG{l+s+s2}{\PYGZdq{}}\PYG{l+s+s2}{Q}\PYG{l+s+s2}{\PYGZdq{}}\PYG{p}{,} \PYG{n}{Q}\PYG{p}{,} \PYG{n}{fluid}\PYG{p}{)}\PYG{o}{/}\PYG{l+m+mi}{1000} \PYG{k}{for} \PYG{n}{P} \PYG{o+ow}{in} \PYG{n}{P\PYGZus{}vals}\PYG{p}{]} \PYG{c+c1}{\PYGZsh{} call for enthalpy values using CoolProp}

\PYG{n}{plt}\PYG{o}{.}\PYG{n}{plot}\PYG{p}{(}\PYG{n}{enthalpy}\PYG{p}{,} \PYG{n}{P\PYGZus{}vals}\PYG{p}{,} \PYG{l+s+s2}{\PYGZdq{}}\PYG{l+s+s2}{\PYGZhy{}b}\PYG{l+s+s2}{\PYGZdq{}}\PYG{p}{)}  \PYG{c+c1}{\PYGZsh{} plot pressure vs enthalpy}


\PYG{n}{plt}\PYG{o}{.}\PYG{n}{yscale}\PYG{p}{(}\PYG{l+s+s2}{\PYGZdq{}}\PYG{l+s+s2}{log}\PYG{l+s+s2}{\PYGZdq{}}\PYG{p}{)}  \PYG{c+c1}{\PYGZsh{} use log scale on y axis}
\PYG{n}{plt}\PYG{o}{.}\PYG{n}{ylabel}\PYG{p}{(}\PYG{l+s+s2}{\PYGZdq{}}\PYG{l+s+s2}{Pressure [Pa]}\PYG{l+s+s2}{\PYGZdq{}}\PYG{p}{)}  \PYG{c+c1}{\PYGZsh{} give y axis a label}
\PYG{n}{plt}\PYG{o}{.}\PYG{n}{xlabel}\PYG{p}{(}\PYG{l+s+s2}{\PYGZdq{}}\PYG{l+s+s2}{Enthalpy [kJ/kg]}\PYG{l+s+s2}{\PYGZdq{}}\PYG{p}{)}  \PYG{c+c1}{\PYGZsh{} give x axis a label}
\PYG{n}{plt}\PYG{o}{.}\PYG{n}{grid}\PYG{p}{(}\PYG{p}{)}


\PYG{c+c1}{\PYGZsh{} Building constant temperature curves}

\PYG{n}{T\PYGZus{}up} \PYG{o}{=} \PYG{n}{T\PYGZus{}2}
\PYG{n}{T\PYGZus{}mid} \PYG{o}{=} \PYG{n}{T\PYGZus{}3}
\PYG{n}{T\PYGZus{}down} \PYG{o}{=} \PYG{n}{T\PYGZus{}4}

\PYG{n}{P\PYGZus{}max} \PYG{o}{=} \PYG{l+m+mf}{20E+6}  \PYG{c+c1}{\PYGZsh{} max pressure in the plot set to 20MPa}
\PYG{n}{P\PYGZus{}vals} \PYG{o}{=} \PYG{n}{np}\PYG{o}{.}\PYG{n}{linspace}\PYG{p}{(}\PYG{n}{P\PYGZus{}min}\PYG{p}{,} \PYG{n}{P\PYGZus{}max}\PYG{p}{,} \PYG{l+m+mi}{10000}\PYG{p}{)}  \PYG{c+c1}{\PYGZsh{} define an array of values from P\PYGZus{}min to P\PYGZus{}max}
\PYG{n}{enthalpy\PYGZus{}up} \PYG{o}{=} \PYG{p}{[}\PYG{n}{CP}\PYG{o}{.}\PYG{n}{PropsSI}\PYG{p}{(}\PYG{l+s+s2}{\PYGZdq{}}\PYG{l+s+s2}{H}\PYG{l+s+s2}{\PYGZdq{}}\PYG{p}{,} \PYG{l+s+s2}{\PYGZdq{}}\PYG{l+s+s2}{P}\PYG{l+s+s2}{\PYGZdq{}}\PYG{p}{,} \PYG{n}{P}\PYG{p}{,} \PYG{l+s+s2}{\PYGZdq{}}\PYG{l+s+s2}{T}\PYG{l+s+s2}{\PYGZdq{}}\PYG{p}{,} \PYG{n}{T\PYGZus{}up}\PYG{p}{,} \PYG{n}{fluid}\PYG{p}{)}\PYG{o}{/}\PYG{l+m+mi}{1000} \PYG{k}{for} \PYG{n}{P} \PYG{o+ow}{in} \PYG{n}{P\PYGZus{}vals}\PYG{p}{]} \PYG{c+c1}{\PYGZsh{} call for enthalpy values using CoolProp}
\PYG{n}{enthalpy\PYGZus{}mid} \PYG{o}{=} \PYG{p}{[}\PYG{n}{CP}\PYG{o}{.}\PYG{n}{PropsSI}\PYG{p}{(}\PYG{l+s+s2}{\PYGZdq{}}\PYG{l+s+s2}{H}\PYG{l+s+s2}{\PYGZdq{}}\PYG{p}{,} \PYG{l+s+s2}{\PYGZdq{}}\PYG{l+s+s2}{P}\PYG{l+s+s2}{\PYGZdq{}}\PYG{p}{,} \PYG{n}{P}\PYG{p}{,} \PYG{l+s+s2}{\PYGZdq{}}\PYG{l+s+s2}{T}\PYG{l+s+s2}{\PYGZdq{}}\PYG{p}{,} \PYG{n}{T\PYGZus{}mid}\PYG{p}{,} \PYG{n}{fluid}\PYG{p}{)}\PYG{o}{/}\PYG{l+m+mi}{1000} \PYG{k}{for} \PYG{n}{P} \PYG{o+ow}{in} \PYG{n}{P\PYGZus{}vals}\PYG{p}{]} \PYG{c+c1}{\PYGZsh{} call for enthalpy values using CoolProp}
\PYG{n}{enthalpy\PYGZus{}down} \PYG{o}{=} \PYG{p}{[}\PYG{n}{CP}\PYG{o}{.}\PYG{n}{PropsSI}\PYG{p}{(}\PYG{l+s+s2}{\PYGZdq{}}\PYG{l+s+s2}{H}\PYG{l+s+s2}{\PYGZdq{}}\PYG{p}{,} \PYG{l+s+s2}{\PYGZdq{}}\PYG{l+s+s2}{P}\PYG{l+s+s2}{\PYGZdq{}}\PYG{p}{,} \PYG{n}{P}\PYG{p}{,} \PYG{l+s+s2}{\PYGZdq{}}\PYG{l+s+s2}{T}\PYG{l+s+s2}{\PYGZdq{}}\PYG{p}{,} \PYG{n}{T\PYGZus{}down}\PYG{p}{,} \PYG{n}{fluid}\PYG{p}{)}\PYG{o}{/}\PYG{l+m+mi}{1000} \PYG{k}{for} \PYG{n}{P} \PYG{o+ow}{in} \PYG{n}{P\PYGZus{}vals}\PYG{p}{]} \PYG{c+c1}{\PYGZsh{} call for enthalpy values using CoolProp}

\PYG{n}{plt}\PYG{o}{.}\PYG{n}{plot}\PYG{p}{(}\PYG{n}{enthalpy\PYGZus{}up}\PYG{p}{,} \PYG{n}{P\PYGZus{}vals}\PYG{p}{,} \PYG{l+s+s2}{\PYGZdq{}}\PYG{l+s+s2}{\PYGZhy{}.y}\PYG{l+s+s2}{\PYGZdq{}}\PYG{p}{,} \PYG{n}{label}\PYG{o}{=}\PYG{n}{T\PYGZus{}up}\PYG{o}{\PYGZhy{}}\PYG{l+m+mf}{273.15}\PYG{p}{)}  \PYG{c+c1}{\PYGZsh{} plot pressure vs enthalpy}
\PYG{n}{plt}\PYG{o}{.}\PYG{n}{plot}\PYG{p}{(}\PYG{n}{enthalpy\PYGZus{}mid}\PYG{p}{,} \PYG{n}{P\PYGZus{}vals}\PYG{p}{,} \PYG{l+s+s2}{\PYGZdq{}}\PYG{l+s+s2}{:r}\PYG{l+s+s2}{\PYGZdq{}}\PYG{p}{,} \PYG{n}{label}\PYG{o}{=}\PYG{n}{T\PYGZus{}mid}\PYG{o}{\PYGZhy{}}\PYG{l+m+mf}{273.15}\PYG{p}{)}  \PYG{c+c1}{\PYGZsh{} plot pressure vs enthalpy}
\PYG{n}{plt}\PYG{o}{.}\PYG{n}{plot}\PYG{p}{(}\PYG{n}{enthalpy\PYGZus{}down}\PYG{p}{,} \PYG{n}{P\PYGZus{}vals}\PYG{p}{,} \PYG{l+s+s2}{\PYGZdq{}}\PYG{l+s+s2}{\PYGZhy{}\PYGZhy{}m}\PYG{l+s+s2}{\PYGZdq{}}\PYG{p}{,} \PYG{n}{label}\PYG{o}{=}\PYG{n}{T\PYGZus{}down}\PYG{o}{\PYGZhy{}}\PYG{l+m+mf}{273.15}\PYG{p}{)}  \PYG{c+c1}{\PYGZsh{} plot pressure vs enthalpy}
\PYG{n}{plt}\PYG{o}{.}\PYG{n}{legend}\PYG{p}{(}\PYG{p}{)}
\end{sphinxVerbatim}

\end{sphinxuseclass}\end{sphinxVerbatimInput}
\begin{sphinxVerbatimOutput}

\begin{sphinxuseclass}{cell_output}
\begin{sphinxVerbatim}[commandchars=\\\{\}]
\PYGZlt{}matplotlib.legend.Legend at 0x7f90685ef940\PYGZgt{}
\end{sphinxVerbatim}

\noindent\sphinxincludegraphics{{d26e029312123208707e990f73796284bc4d6972ae0efa3ab2655257a311ded2}.png}

\end{sphinxuseclass}\end{sphinxVerbatimOutput}

\end{sphinxuseclass}
\sphinxstepscope


\section{Refrigeration cycle: R134a}
\label{\detokenize{notebooks/Chapter5/CH5-Q6:refrigeration-cycle-r134a}}\label{\detokenize{notebooks/Chapter5/CH5-Q6::doc}}
\sphinxAtStartPar
consider a refrigration cycle working with R134\sphinxhyphen{}a as coolant to support \(100\:kW\) of cooling load to a cold storage. The refrigrnt absorbs heat at \(-20^{\circ}  C\) during evaporation and enters the compressor as saturated vapor. The compressor is coupled to a gas turbine working with compressed air that pressurizes R134\sphinxhyphen{}a increasing the coolant’s temperature to \(120^{\circ}  C\). The refrigrant is then cooled down to saturated liquid at \(40^{\circ}  C\) in a condensor in constant pressure before entering a throttling valve. The refrigrant is then throttled to \(-20^{\circ}  C\) to provide refrigrating fluid for the evaporator. Given the turbine works with compressed air at \(1\:MPa\) guage pressure and room temperature, determine the air flow\sphinxhyphen{}rate required to provide the required cooling load given the air is discharched to the atmosphere at the turbine outlet and goes through a polytropic process with \(n=1.5\).

\sphinxAtStartPar
\sphinxincludegraphics{{CH5-Q6}.png}


\section{Solution Approach}
\label{\detokenize{notebooks/Chapter5/CH5-Q6:solution-approach}}
\sphinxAtStartPar
the enthalpy change through the evaporator per kg of the refrigrant is to be calculated. Afterwards, the flow\sphinxhyphen{}rate of refrigrant is calculated based on the cooling load, followed by calculation of compressor load based on flow\sphinxhyphen{}rate and work calculated per kg of refrigrant. From Q\#5 of this chapter:

\begin{sphinxuseclass}{cell}\begin{sphinxVerbatimInput}

\begin{sphinxuseclass}{cell_input}
\begin{sphinxVerbatim}[commandchars=\\\{\}]
\PYG{c+c1}{\PYGZsh{} import the libraries we\PYGZsq{}ll need}
\PYG{k+kn}{import} \PYG{n+nn}{CoolProp}\PYG{n+nn}{.}\PYG{n+nn}{CoolProp} \PYG{k}{as} \PYG{n+nn}{CP}

\PYG{c+c1}{\PYGZsh{} define variables}
\PYG{n}{fluid} \PYG{o}{=} \PYG{l+s+s2}{\PYGZdq{}}\PYG{l+s+s2}{R134A}\PYG{l+s+s2}{\PYGZdq{}}  \PYG{c+c1}{\PYGZsh{} define the fluid or material of interest}
\PYG{n}{T\PYGZus{}3} \PYG{o}{=} \PYG{l+m+mi}{40} \PYG{o}{+} \PYG{l+m+mf}{273.15} \PYG{c+c1}{\PYGZsh{}stete \PYGZsh{}3 temperature in K}
\PYG{n}{h\PYGZus{}3} \PYG{o}{=} \PYG{n}{CP}\PYG{o}{.}\PYG{n}{PropsSI}\PYG{p}{(}\PYG{l+s+s2}{\PYGZdq{}}\PYG{l+s+s2}{H}\PYG{l+s+s2}{\PYGZdq{}}\PYG{p}{,} \PYG{l+s+s2}{\PYGZdq{}}\PYG{l+s+s2}{T}\PYG{l+s+s2}{\PYGZdq{}}\PYG{p}{,} \PYG{n}{T\PYGZus{}3}\PYG{p}{,} \PYG{l+s+s2}{\PYGZdq{}}\PYG{l+s+s2}{Q}\PYG{l+s+s2}{\PYGZdq{}}\PYG{p}{,} \PYG{l+m+mi}{0}\PYG{p}{,} \PYG{n}{fluid}\PYG{p}{)}\PYG{o}{/}\PYG{l+m+mi}{1000}  \PYG{c+c1}{\PYGZsh{} enthalpy of the refrigrant at state \PYGZsh{}3 in kJ/kg}

\PYG{n}{h\PYGZus{}4} \PYG{o}{=} \PYG{n}{h\PYGZus{}3}   \PYG{c+c1}{\PYGZsh{}constant enthalpy through a throttling valve}

\PYG{n}{T\PYGZus{}4} \PYG{o}{=} \PYG{o}{\PYGZhy{}}\PYG{l+m+mi}{20} \PYG{o}{+} \PYG{l+m+mf}{273.15}   \PYG{c+c1}{\PYGZsh{}temperature of refrigrant at state \PYGZsh{}4 in K}

\PYG{n}{P\PYGZus{}4} \PYG{o}{=} \PYG{n}{CP}\PYG{o}{.}\PYG{n}{PropsSI}\PYG{p}{(}\PYG{l+s+s2}{\PYGZdq{}}\PYG{l+s+s2}{P}\PYG{l+s+s2}{\PYGZdq{}}\PYG{p}{,} \PYG{l+s+s2}{\PYGZdq{}}\PYG{l+s+s2}{T}\PYG{l+s+s2}{\PYGZdq{}}\PYG{p}{,} \PYG{n}{T\PYGZus{}4}\PYG{p}{,} \PYG{l+s+s2}{\PYGZdq{}}\PYG{l+s+s2}{Q}\PYG{l+s+s2}{\PYGZdq{}}\PYG{p}{,} \PYG{l+m+mi}{1}\PYG{p}{,} \PYG{n}{fluid}\PYG{p}{)}  \PYG{c+c1}{\PYGZsh{} pressure of the refrigrant at state \PYGZsh{}1 in Pa (quality is set to 1 as the pressure keeps constant in sat region)}

\PYG{n}{T\PYGZus{}1} \PYG{o}{=} \PYG{n}{T\PYGZus{}4} \PYG{c+c1}{\PYGZsh{}temperature at state \PYGZsh{}1 in K}
\PYG{n}{h\PYGZus{}1} \PYG{o}{=} \PYG{n}{CP}\PYG{o}{.}\PYG{n}{PropsSI}\PYG{p}{(}\PYG{l+s+s2}{\PYGZdq{}}\PYG{l+s+s2}{H}\PYG{l+s+s2}{\PYGZdq{}}\PYG{p}{,} \PYG{l+s+s2}{\PYGZdq{}}\PYG{l+s+s2}{T}\PYG{l+s+s2}{\PYGZdq{}}\PYG{p}{,} \PYG{n}{T\PYGZus{}1}\PYG{p}{,} \PYG{l+s+s2}{\PYGZdq{}}\PYG{l+s+s2}{Q}\PYG{l+s+s2}{\PYGZdq{}}\PYG{p}{,} \PYG{l+m+mi}{1}\PYG{p}{,} \PYG{n}{fluid}\PYG{p}{)}\PYG{o}{/}\PYG{l+m+mi}{1000}  \PYG{c+c1}{\PYGZsh{} enthalpy of the refrigrant at state \PYGZsh{}1 in kJ/kg}

\PYG{n}{T\PYGZus{}2} \PYG{o}{=} \PYG{l+m+mi}{120} \PYG{o}{+} \PYG{l+m+mf}{273.15} \PYG{c+c1}{\PYGZsh{}temperature at state \PYGZsh{}2 in K}
\PYG{n}{P\PYGZus{}3} \PYG{o}{=} \PYG{n}{CP}\PYG{o}{.}\PYG{n}{PropsSI}\PYG{p}{(}\PYG{l+s+s2}{\PYGZdq{}}\PYG{l+s+s2}{P}\PYG{l+s+s2}{\PYGZdq{}}\PYG{p}{,} \PYG{l+s+s2}{\PYGZdq{}}\PYG{l+s+s2}{T}\PYG{l+s+s2}{\PYGZdq{}}\PYG{p}{,} \PYG{n}{T\PYGZus{}3}\PYG{p}{,} \PYG{l+s+s2}{\PYGZdq{}}\PYG{l+s+s2}{Q}\PYG{l+s+s2}{\PYGZdq{}}\PYG{p}{,} \PYG{l+m+mi}{0}\PYG{p}{,} \PYG{n}{fluid}\PYG{p}{)}  \PYG{c+c1}{\PYGZsh{} pressure of the refrigrant at state \PYGZsh{}3 in Pa}
\PYG{n}{P\PYGZus{}2} \PYG{o}{=} \PYG{n}{P\PYGZus{}3} \PYG{c+c1}{\PYGZsh{} pressure of the refrigrant at state \PYGZsh{}2 in Pa}
\PYG{n}{h\PYGZus{}2} \PYG{o}{=} \PYG{n}{CP}\PYG{o}{.}\PYG{n}{PropsSI}\PYG{p}{(}\PYG{l+s+s2}{\PYGZdq{}}\PYG{l+s+s2}{H}\PYG{l+s+s2}{\PYGZdq{}}\PYG{p}{,} \PYG{l+s+s2}{\PYGZdq{}}\PYG{l+s+s2}{T}\PYG{l+s+s2}{\PYGZdq{}}\PYG{p}{,} \PYG{n}{T\PYGZus{}2}\PYG{p}{,} \PYG{l+s+s2}{\PYGZdq{}}\PYG{l+s+s2}{P}\PYG{l+s+s2}{\PYGZdq{}}\PYG{p}{,} \PYG{n}{P\PYGZus{}2}\PYG{p}{,} \PYG{n}{fluid}\PYG{p}{)}\PYG{o}{/}\PYG{l+m+mi}{1000}  \PYG{c+c1}{\PYGZsh{} enthalpy of the refrigrant at state \PYGZsh{}2 in kJ/kg}

\PYG{n}{q\PYGZus{}c} \PYG{o}{=} \PYG{n}{h\PYGZus{}1} \PYG{o}{\PYGZhy{}} \PYG{n}{h\PYGZus{}4}

\PYG{n}{q\PYGZus{}h} \PYG{o}{=} \PYG{n}{h\PYGZus{}2} \PYG{o}{\PYGZhy{}} \PYG{n}{h\PYGZus{}3}

\PYG{n}{w} \PYG{o}{=} \PYG{n}{h\PYGZus{}2} \PYG{o}{\PYGZhy{}} \PYG{n}{h\PYGZus{}1}
\end{sphinxVerbatim}

\end{sphinxuseclass}\end{sphinxVerbatimInput}

\end{sphinxuseclass}
\sphinxAtStartPar
The total cooling load \(\dot Q_c\) is given to be \(100\:kW\); therefore the refrigrant flow\sphinxhyphen{}rate is calculated as

\sphinxAtStartPar
\(\dot m_{R134a}=\dot Q_c/q_c\)

\begin{sphinxuseclass}{cell}\begin{sphinxVerbatimInput}

\begin{sphinxuseclass}{cell_input}
\begin{sphinxVerbatim}[commandchars=\\\{\}]
\PYG{n}{Q\PYGZus{}c} \PYG{o}{=} \PYG{l+m+mi}{100}    \PYG{c+c1}{\PYGZsh{}cooling load in kW}
\PYG{n}{m\PYGZus{}r134a} \PYG{o}{=} \PYG{n}{Q\PYGZus{}c} \PYG{o}{/} \PYG{n}{q\PYGZus{}c}   \PYG{c+c1}{\PYGZsh{}refrigrant flow\PYGZhy{}rate in kg/s}
\end{sphinxVerbatim}

\end{sphinxuseclass}\end{sphinxVerbatimInput}

\end{sphinxuseclass}
\sphinxAtStartPar
The total work done by the compressor then would be,

\sphinxAtStartPar
\(\dot W_{R134a}=\dot m_{R134a}\:w\)

\begin{sphinxuseclass}{cell}\begin{sphinxVerbatimInput}

\begin{sphinxuseclass}{cell_input}
\begin{sphinxVerbatim}[commandchars=\\\{\}]
\PYG{n}{W\PYGZus{}r134a} \PYG{o}{=} \PYG{n}{m\PYGZus{}r134a} \PYG{o}{*} \PYG{n}{w}   \PYG{c+c1}{\PYGZsh{}the work input by the compressor in kW}
\end{sphinxVerbatim}

\end{sphinxuseclass}\end{sphinxVerbatimInput}

\end{sphinxuseclass}
\sphinxAtStartPar
Now, looking at the turbine\sphinxhyphen{}compressor coupling, the work required by the compressor is supported by the turbine in which air goes through a polytropic process. For a polytropic process,

\sphinxAtStartPar
\(Pv^n=constant\)

\sphinxAtStartPar
\(P_1v_1^k=P_2v_2^k\)

\sphinxAtStartPar
\(d (density)=1/v\)

\sphinxAtStartPar
\(d_2/d_1=(P_2/P_1)^{(1/k)}\)

\sphinxAtStartPar
\(d_2=d_1\:(P_2/P_1)^{(1/k)}\)

\begin{sphinxuseclass}{cell}\begin{sphinxVerbatimInput}

\begin{sphinxuseclass}{cell_input}
\begin{sphinxVerbatim}[commandchars=\\\{\}]
\PYG{c+c1}{\PYGZsh{}thermodynamic properties and constants for air at state \PYGZsh{}5}
\PYG{n}{fluid} \PYG{o}{=} \PYG{l+s+s2}{\PYGZdq{}}\PYG{l+s+s2}{Air}\PYG{l+s+s2}{\PYGZdq{}}
\PYG{n}{R} \PYG{o}{=} \PYG{l+m+mf}{0.287}    \PYG{c+c1}{\PYGZsh{}air gas constant in kJ/kg.k}
\PYG{n}{P\PYGZus{}atm} \PYG{o}{=} \PYG{l+m+mf}{101.325}   \PYG{c+c1}{\PYGZsh{}atmospheric pressure in kPa}
\PYG{n}{P\PYGZus{}guage} \PYG{o}{=} \PYG{l+m+mi}{1000}   \PYG{c+c1}{\PYGZsh{}guage presssure at turbine inlet in kPa}
\PYG{n}{P\PYGZus{}5} \PYG{o}{=} \PYG{n}{P\PYGZus{}guage} \PYG{o}{+} \PYG{n}{P\PYGZus{}atm}
\PYG{n}{T\PYGZus{}5} \PYG{o}{=} \PYG{l+m+mi}{25} \PYG{o}{+} \PYG{l+m+mf}{273.15}   \PYG{c+c1}{\PYGZsh{}compressed air temperature in K}
\PYG{n}{D\PYGZus{}5} \PYG{o}{=} \PYG{n}{CP}\PYG{o}{.}\PYG{n}{PropsSI}\PYG{p}{(}\PYG{l+s+s2}{\PYGZdq{}}\PYG{l+s+s2}{D}\PYG{l+s+s2}{\PYGZdq{}}\PYG{p}{,} \PYG{l+s+s2}{\PYGZdq{}}\PYG{l+s+s2}{T}\PYG{l+s+s2}{\PYGZdq{}}\PYG{p}{,} \PYG{n}{T\PYGZus{}5}\PYG{p}{,} \PYG{l+s+s2}{\PYGZdq{}}\PYG{l+s+s2}{P}\PYG{l+s+s2}{\PYGZdq{}}\PYG{p}{,} \PYG{n}{P\PYGZus{}5}\PYG{p}{,} \PYG{n}{fluid}\PYG{p}{)}  \PYG{c+c1}{\PYGZsh{}air density at turbine inlet}

\PYG{n}{P\PYGZus{}6} \PYG{o}{=} \PYG{n}{P\PYGZus{}atm}   \PYG{c+c1}{\PYGZsh{}pressure at turbine outlet in kPa}
\PYG{n}{n} \PYG{o}{=} \PYG{l+m+mf}{1.5}
\PYG{n}{D\PYGZus{}6} \PYG{o}{=} \PYG{n}{D\PYGZus{}5} \PYG{o}{*} \PYG{p}{(}\PYG{n}{P\PYGZus{}6}\PYG{o}{/}\PYG{n}{P\PYGZus{}5}\PYG{p}{)} \PYG{o}{*}\PYG{o}{*} \PYG{p}{(}\PYG{l+m+mi}{1}\PYG{o}{/}\PYG{n}{n}\PYG{p}{)}
\end{sphinxVerbatim}

\end{sphinxuseclass}\end{sphinxVerbatimInput}

\end{sphinxuseclass}
\sphinxAtStartPar
Now, considering the first law of thermodynamics,

\sphinxAtStartPar
\(w_turbine = h_5 - h_6\)

\sphinxAtStartPar
\(\dot W_{turbine}=\dot m_{air} \:(h_5-h_6)\)

\sphinxAtStartPar
and from the coupling

\sphinxAtStartPar
\(\dot W_{turbine}=\dot W_{R134a}\)

\sphinxAtStartPar
\(\dot m_{air}=\dot W_{R134a}/(h_5-h_6)\)

\begin{sphinxuseclass}{cell}\begin{sphinxVerbatimInput}

\begin{sphinxuseclass}{cell_input}
\begin{sphinxVerbatim}[commandchars=\\\{\}]
\PYG{n}{fluid} \PYG{o}{=} \PYG{l+s+s2}{\PYGZdq{}}\PYG{l+s+s2}{Air}\PYG{l+s+s2}{\PYGZdq{}}
\PYG{n}{h\PYGZus{}5} \PYG{o}{=} \PYG{n}{CP}\PYG{o}{.}\PYG{n}{PropsSI}\PYG{p}{(}\PYG{l+s+s2}{\PYGZdq{}}\PYG{l+s+s2}{H}\PYG{l+s+s2}{\PYGZdq{}}\PYG{p}{,} \PYG{l+s+s2}{\PYGZdq{}}\PYG{l+s+s2}{T}\PYG{l+s+s2}{\PYGZdq{}}\PYG{p}{,} \PYG{n}{T\PYGZus{}5}\PYG{p}{,} \PYG{l+s+s2}{\PYGZdq{}}\PYG{l+s+s2}{P}\PYG{l+s+s2}{\PYGZdq{}}\PYG{p}{,} \PYG{n}{P\PYGZus{}5}\PYG{p}{,} \PYG{n}{fluid}\PYG{p}{)}\PYG{o}{/}\PYG{l+m+mi}{1000}   \PYG{c+c1}{\PYGZsh{}air enthalpy at turbine inlet in kJ/kg}
\PYG{n}{h\PYGZus{}6} \PYG{o}{=} \PYG{n}{CP}\PYG{o}{.}\PYG{n}{PropsSI}\PYG{p}{(}\PYG{l+s+s2}{\PYGZdq{}}\PYG{l+s+s2}{H}\PYG{l+s+s2}{\PYGZdq{}}\PYG{p}{,} \PYG{l+s+s2}{\PYGZdq{}}\PYG{l+s+s2}{D}\PYG{l+s+s2}{\PYGZdq{}}\PYG{p}{,} \PYG{n}{D\PYGZus{}6}\PYG{p}{,} \PYG{l+s+s2}{\PYGZdq{}}\PYG{l+s+s2}{P}\PYG{l+s+s2}{\PYGZdq{}}\PYG{p}{,} \PYG{n}{P\PYGZus{}6}\PYG{p}{,} \PYG{n}{fluid}\PYG{p}{)}\PYG{o}{/}\PYG{l+m+mi}{1000}   \PYG{c+c1}{\PYGZsh{}air enthalpy at turbine outlet in kJ/kg}
\PYG{n}{m\PYGZus{}air} \PYG{o}{=} \PYG{n}{W\PYGZus{}r134a} \PYG{o}{/} \PYG{p}{(}\PYG{n}{h\PYGZus{}5} \PYG{o}{\PYGZhy{}} \PYG{n}{h\PYGZus{}6}\PYG{p}{)}   \PYG{c+c1}{\PYGZsh{}required air flow\PYGZhy{}rate in kg/s}

\PYG{n+nb}{print}\PYG{p}{(}\PYG{l+s+s1}{\PYGZsq{}}\PYG{l+s+s1}{The required air flow\PYGZhy{}rate to run the compressor is:}\PYG{l+s+s1}{\PYGZsq{}}\PYG{p}{,} \PYG{l+s+sa}{f}\PYG{l+s+s2}{\PYGZdq{}}\PYG{l+s+si}{\PYGZob{}}\PYG{n}{m\PYGZus{}air}\PYG{l+s+si}{:}\PYG{l+s+s2}{.1f}\PYG{l+s+si}{\PYGZcb{}}\PYG{l+s+s2}{\PYGZdq{}}\PYG{p}{,} \PYG{l+s+s1}{\PYGZsq{}}\PYG{l+s+s1}{kg/s}\PYG{l+s+s1}{\PYGZsq{}}\PYG{p}{)}
\end{sphinxVerbatim}

\end{sphinxuseclass}\end{sphinxVerbatimInput}
\begin{sphinxVerbatimOutput}

\begin{sphinxuseclass}{cell_output}
\begin{sphinxVerbatim}[commandchars=\\\{\}]
The required air flow\PYGZhy{}rate to run the compressor is: 0.6 kg/s
\end{sphinxVerbatim}

\end{sphinxuseclass}\end{sphinxVerbatimOutput}

\end{sphinxuseclass}
\sphinxstepscope


\section{Refrigeration cycle: Water flow\sphinxhyphen{}rate}
\label{\detokenize{notebooks/Chapter5/CH5-Q7:refrigeration-cycle-water-flow-rate}}\label{\detokenize{notebooks/Chapter5/CH5-Q7::doc}}
\sphinxAtStartPar
consider a refrigration cycle working with R134\sphinxhyphen{}a as coolant. The refrigrnt absorbs heat at \(-20 ^{\circ}  C\) during evaporation and enters the compressor as saturated vapor. The compressor then pressurizes R134\sphinxhyphen{}a while the its temperature increases to \(120 ^{\circ}  C\). The refrigrant is then cooled down to saturated liquid at \(40 ^{\circ}  C\) in constant pressure in a heat exchanger before entering a throttling valve. The refrigrant is then throttled to \(-20 ^{\circ}  C\) to provide refrigrating fluid for the evaporator. Assuming the refrigrant exchanges heat with pressurized water in the heat exchanger to heat it up from room temperature to saturated vapor at \(110 ^{\circ}  C\), calculate how much water is required per kg of R134a to support cooling power in the heat exhanger.

\sphinxAtStartPar
\sphinxincludegraphics{{CH5-Q7}.png}


\section{Solution Approach}
\label{\detokenize{notebooks/Chapter5/CH5-Q7:solution-approach}}
\sphinxAtStartPar
based on the first law of thermodynamics, for the heat exchanger

\sphinxAtStartPar
\(Q+\dot m_ih_i=\dot m_eh_e\)

\sphinxAtStartPar
Assuming an isolated heat exchanger,

\sphinxAtStartPar
\(\dot m_5h_5+\dot m_2h_2=\dot m_3h_3+\dot m_6h_6\)

\sphinxAtStartPar
given

\sphinxAtStartPar
\(\dot m_5=\dot m_6=\dot m_{water}\)

\sphinxAtStartPar
and

\sphinxAtStartPar
\(\dot m_2=\dot m_3=\dot m_{R134a}\)

\sphinxAtStartPar
water flow\sphinxhyphen{}rate per \(1\:kg/s\) of \(R134a\) flow\sphinxhyphen{}rate would be

\sphinxAtStartPar
\(\dot m_{water}=(h_2-h_3)/(h_6-h_5)\)

\sphinxAtStartPar
from Q\#5 of this chapter

\begin{sphinxuseclass}{cell}\begin{sphinxVerbatimInput}

\begin{sphinxuseclass}{cell_input}
\begin{sphinxVerbatim}[commandchars=\\\{\}]
\PYG{c+c1}{\PYGZsh{}\PYGZsh{} import the libraries we\PYGZsq{}ll need}
\PYG{k+kn}{import} \PYG{n+nn}{CoolProp}\PYG{n+nn}{.}\PYG{n+nn}{CoolProp} \PYG{k}{as} \PYG{n+nn}{CP}

\PYG{c+c1}{\PYGZsh{} define variables}
\PYG{n}{fluid} \PYG{o}{=} \PYG{l+s+s2}{\PYGZdq{}}\PYG{l+s+s2}{R134A}\PYG{l+s+s2}{\PYGZdq{}}  \PYG{c+c1}{\PYGZsh{} define the fluid or material of interest}
\PYG{n}{T\PYGZus{}3} \PYG{o}{=} \PYG{l+m+mi}{40} \PYG{o}{+} \PYG{l+m+mf}{273.15} \PYG{c+c1}{\PYGZsh{}stete \PYGZsh{}3 temperature in K}
\PYG{n}{h\PYGZus{}3} \PYG{o}{=} \PYG{n}{CP}\PYG{o}{.}\PYG{n}{PropsSI}\PYG{p}{(}\PYG{l+s+s2}{\PYGZdq{}}\PYG{l+s+s2}{H}\PYG{l+s+s2}{\PYGZdq{}}\PYG{p}{,} \PYG{l+s+s2}{\PYGZdq{}}\PYG{l+s+s2}{T}\PYG{l+s+s2}{\PYGZdq{}}\PYG{p}{,} \PYG{n}{T\PYGZus{}3}\PYG{p}{,} \PYG{l+s+s2}{\PYGZdq{}}\PYG{l+s+s2}{Q}\PYG{l+s+s2}{\PYGZdq{}}\PYG{p}{,} \PYG{l+m+mi}{0}\PYG{p}{,} \PYG{n}{fluid}\PYG{p}{)}\PYG{o}{/}\PYG{l+m+mi}{1000}  \PYG{c+c1}{\PYGZsh{} enthalpy of the refrigrant at state \PYGZsh{}3 in kJ/kg}

\PYG{n}{h\PYGZus{}4} \PYG{o}{=} \PYG{n}{h\PYGZus{}3}   \PYG{c+c1}{\PYGZsh{}constant enthalpy through a throttling valve}

\PYG{n}{T\PYGZus{}4} \PYG{o}{=} \PYG{o}{\PYGZhy{}}\PYG{l+m+mi}{20} \PYG{o}{+} \PYG{l+m+mf}{273.15}   \PYG{c+c1}{\PYGZsh{}temperature of refrigrant at state \PYGZsh{}4 in K}

\PYG{n}{P\PYGZus{}4} \PYG{o}{=} \PYG{n}{CP}\PYG{o}{.}\PYG{n}{PropsSI}\PYG{p}{(}\PYG{l+s+s2}{\PYGZdq{}}\PYG{l+s+s2}{P}\PYG{l+s+s2}{\PYGZdq{}}\PYG{p}{,} \PYG{l+s+s2}{\PYGZdq{}}\PYG{l+s+s2}{T}\PYG{l+s+s2}{\PYGZdq{}}\PYG{p}{,} \PYG{n}{T\PYGZus{}4}\PYG{p}{,} \PYG{l+s+s2}{\PYGZdq{}}\PYG{l+s+s2}{Q}\PYG{l+s+s2}{\PYGZdq{}}\PYG{p}{,} \PYG{l+m+mi}{1}\PYG{p}{,} \PYG{n}{fluid}\PYG{p}{)}  \PYG{c+c1}{\PYGZsh{} pressure of the refrigrant at state \PYGZsh{}1 in Pa (quality is set to 1 as the pressure keeps constant in sat region)}

\PYG{n}{T\PYGZus{}1} \PYG{o}{=} \PYG{n}{T\PYGZus{}4} \PYG{c+c1}{\PYGZsh{}temperature at state \PYGZsh{}1 in K}
\PYG{n}{h\PYGZus{}1} \PYG{o}{=} \PYG{n}{CP}\PYG{o}{.}\PYG{n}{PropsSI}\PYG{p}{(}\PYG{l+s+s2}{\PYGZdq{}}\PYG{l+s+s2}{H}\PYG{l+s+s2}{\PYGZdq{}}\PYG{p}{,} \PYG{l+s+s2}{\PYGZdq{}}\PYG{l+s+s2}{T}\PYG{l+s+s2}{\PYGZdq{}}\PYG{p}{,} \PYG{n}{T\PYGZus{}1}\PYG{p}{,} \PYG{l+s+s2}{\PYGZdq{}}\PYG{l+s+s2}{Q}\PYG{l+s+s2}{\PYGZdq{}}\PYG{p}{,} \PYG{l+m+mi}{1}\PYG{p}{,} \PYG{n}{fluid}\PYG{p}{)}\PYG{o}{/}\PYG{l+m+mi}{1000}  \PYG{c+c1}{\PYGZsh{} enthalpy of the refrigrant at state \PYGZsh{}1 in kJ/kg}

\PYG{n}{T\PYGZus{}2} \PYG{o}{=} \PYG{l+m+mi}{120} \PYG{o}{+} \PYG{l+m+mf}{273.15} \PYG{c+c1}{\PYGZsh{}temperature at state \PYGZsh{}2 in K}
\PYG{n}{P\PYGZus{}3} \PYG{o}{=} \PYG{n}{CP}\PYG{o}{.}\PYG{n}{PropsSI}\PYG{p}{(}\PYG{l+s+s2}{\PYGZdq{}}\PYG{l+s+s2}{P}\PYG{l+s+s2}{\PYGZdq{}}\PYG{p}{,} \PYG{l+s+s2}{\PYGZdq{}}\PYG{l+s+s2}{T}\PYG{l+s+s2}{\PYGZdq{}}\PYG{p}{,} \PYG{n}{T\PYGZus{}3}\PYG{p}{,} \PYG{l+s+s2}{\PYGZdq{}}\PYG{l+s+s2}{Q}\PYG{l+s+s2}{\PYGZdq{}}\PYG{p}{,} \PYG{l+m+mi}{0}\PYG{p}{,} \PYG{n}{fluid}\PYG{p}{)}  \PYG{c+c1}{\PYGZsh{} pressure of the refrigrant at state \PYGZsh{}3 in Pa}
\PYG{n}{P\PYGZus{}2} \PYG{o}{=} \PYG{n}{P\PYGZus{}3} \PYG{c+c1}{\PYGZsh{} pressure of the refrigrant at state \PYGZsh{}2 in Pa}
\PYG{n}{h\PYGZus{}2} \PYG{o}{=} \PYG{n}{CP}\PYG{o}{.}\PYG{n}{PropsSI}\PYG{p}{(}\PYG{l+s+s2}{\PYGZdq{}}\PYG{l+s+s2}{H}\PYG{l+s+s2}{\PYGZdq{}}\PYG{p}{,} \PYG{l+s+s2}{\PYGZdq{}}\PYG{l+s+s2}{T}\PYG{l+s+s2}{\PYGZdq{}}\PYG{p}{,} \PYG{n}{T\PYGZus{}2}\PYG{p}{,} \PYG{l+s+s2}{\PYGZdq{}}\PYG{l+s+s2}{P}\PYG{l+s+s2}{\PYGZdq{}}\PYG{p}{,} \PYG{n}{P\PYGZus{}2}\PYG{p}{,} \PYG{n}{fluid}\PYG{p}{)}\PYG{o}{/}\PYG{l+m+mi}{1000}  \PYG{c+c1}{\PYGZsh{} enthalpy of the refrigrant at state \PYGZsh{}2 in kJ/kg}

\PYG{c+c1}{\PYGZsh{}for the water side}
\PYG{n}{fluid} \PYG{o}{=} \PYG{l+s+s2}{\PYGZdq{}}\PYG{l+s+s2}{water}\PYG{l+s+s2}{\PYGZdq{}}
\PYG{n}{T\PYGZus{}6} \PYG{o}{=} \PYG{l+m+mi}{110} \PYG{o}{+} \PYG{l+m+mf}{273.15}   \PYG{c+c1}{\PYGZsh{}temperature of water at heat exchanger exit in K}
\PYG{n}{T\PYGZus{}5} \PYG{o}{=} \PYG{l+m+mi}{25} \PYG{o}{+} \PYG{l+m+mf}{273.15}   \PYG{c+c1}{\PYGZsh{}temperature of water at heat exchanger inlet in K}
\PYG{n}{h\PYGZus{}6} \PYG{o}{=} \PYG{n}{CP}\PYG{o}{.}\PYG{n}{PropsSI}\PYG{p}{(}\PYG{l+s+s2}{\PYGZdq{}}\PYG{l+s+s2}{H}\PYG{l+s+s2}{\PYGZdq{}}\PYG{p}{,} \PYG{l+s+s2}{\PYGZdq{}}\PYG{l+s+s2}{T}\PYG{l+s+s2}{\PYGZdq{}}\PYG{p}{,} \PYG{n}{T\PYGZus{}6}\PYG{p}{,} \PYG{l+s+s2}{\PYGZdq{}}\PYG{l+s+s2}{Q}\PYG{l+s+s2}{\PYGZdq{}}\PYG{p}{,} \PYG{l+m+mi}{1}\PYG{p}{,} \PYG{n}{fluid}\PYG{p}{)}\PYG{o}{/}\PYG{l+m+mi}{1000}   \PYG{c+c1}{\PYGZsh{}enthalpy of water at heat exchanger exit in kJ/kg}
\PYG{n}{P\PYGZus{}6} \PYG{o}{=} \PYG{n}{CP}\PYG{o}{.}\PYG{n}{PropsSI}\PYG{p}{(}\PYG{l+s+s2}{\PYGZdq{}}\PYG{l+s+s2}{P}\PYG{l+s+s2}{\PYGZdq{}}\PYG{p}{,} \PYG{l+s+s2}{\PYGZdq{}}\PYG{l+s+s2}{T}\PYG{l+s+s2}{\PYGZdq{}}\PYG{p}{,} \PYG{n}{T\PYGZus{}6}\PYG{p}{,} \PYG{l+s+s2}{\PYGZdq{}}\PYG{l+s+s2}{Q}\PYG{l+s+s2}{\PYGZdq{}}\PYG{p}{,} \PYG{l+m+mi}{1}\PYG{p}{,} \PYG{n}{fluid}\PYG{p}{)}   \PYG{c+c1}{\PYGZsh{}pressure of water at heat exchanger exit in Pa}
\PYG{n}{P\PYGZus{}5} \PYG{o}{=} \PYG{n}{P\PYGZus{}6}   \PYG{c+c1}{\PYGZsh{}pressure of water at heat exchanger inlet in Pa assuming constant pressure heating}
\PYG{n}{h\PYGZus{}5} \PYG{o}{=} \PYG{n}{CP}\PYG{o}{.}\PYG{n}{PropsSI}\PYG{p}{(}\PYG{l+s+s2}{\PYGZdq{}}\PYG{l+s+s2}{H}\PYG{l+s+s2}{\PYGZdq{}}\PYG{p}{,} \PYG{l+s+s2}{\PYGZdq{}}\PYG{l+s+s2}{T}\PYG{l+s+s2}{\PYGZdq{}}\PYG{p}{,} \PYG{n}{T\PYGZus{}5}\PYG{p}{,} \PYG{l+s+s2}{\PYGZdq{}}\PYG{l+s+s2}{P}\PYG{l+s+s2}{\PYGZdq{}}\PYG{p}{,} \PYG{n}{P\PYGZus{}5}\PYG{p}{,} \PYG{n}{fluid}\PYG{p}{)}\PYG{o}{/}\PYG{l+m+mi}{1000}   \PYG{c+c1}{\PYGZsh{}enthalpy of water at heat exchanger exit in kJ/kg}

\PYG{n}{m\PYGZus{}water} \PYG{o}{=} \PYG{p}{(}\PYG{n}{h\PYGZus{}2} \PYG{o}{\PYGZhy{}} \PYG{n}{h\PYGZus{}3}\PYG{p}{)} \PYG{o}{/} \PYG{p}{(}\PYG{n}{h\PYGZus{}6} \PYG{o}{\PYGZhy{}} \PYG{n}{h\PYGZus{}5}\PYG{p}{)}

\PYG{n+nb}{print}\PYG{p}{(}\PYG{l+s+s1}{\PYGZsq{}}\PYG{l+s+s1}{The required water flow\PYGZhy{}rate to support cooling for condensor per kg/s of R134a:}\PYG{l+s+s1}{\PYGZsq{}}\PYG{p}{,} \PYG{l+s+sa}{f}\PYG{l+s+s2}{\PYGZdq{}}\PYG{l+s+si}{\PYGZob{}}\PYG{n}{m\PYGZus{}water}\PYG{l+s+si}{:}\PYG{l+s+s2}{.1f}\PYG{l+s+si}{\PYGZcb{}}\PYG{l+s+s2}{\PYGZdq{}}\PYG{p}{,} \PYG{l+s+s1}{\PYGZsq{}}\PYG{l+s+s1}{kg/s}\PYG{l+s+s1}{\PYGZsq{}}\PYG{p}{)}
\end{sphinxVerbatim}

\end{sphinxuseclass}\end{sphinxVerbatimInput}
\begin{sphinxVerbatimOutput}

\begin{sphinxuseclass}{cell_output}
\begin{sphinxVerbatim}[commandchars=\\\{\}]
The required water flow\PYGZhy{}rate to support cooling for condensor per kg/s of R134a: 0.1 kg/s
\end{sphinxVerbatim}

\end{sphinxuseclass}\end{sphinxVerbatimOutput}

\end{sphinxuseclass}
\sphinxstepscope


\section{Chapter 5}
\label{\detokenize{notebooks/Chapter5/CH5-Q8:chapter-5}}\label{\detokenize{notebooks/Chapter5/CH5-Q8::doc}}

\subsection{Question \#8}
\label{\detokenize{notebooks/Chapter5/CH5-Q8:question-8}}
\sphinxAtStartPar
Consider a multi\sphinxhyphen{}evaporator refrigration system operating with two evaporators to support cooling powers for two seperate compartments set at different temperatures. The low\sphinxhyphen{}temperature evaporator1 operates at a constant \(-20^{\circ} C\) and the high\sphinxhyphen{}temperature evaporator2 operates at \(0^{\circ} C\). To optimize the size of the condensor and the evaporators, the refrigrant R134\sphinxhyphen{}a is cooled down to saturated liquid at \(40^{\circ} C\) in the condensor and is heated up to saturated vapor in the evaporators. The evaporator outputs are then mixed after regulating the pressure for the high\sphinxhyphen{}temperature evaporator using a valve and then pressurized in the compressor while increasing to \(120^{\circ} C\) in temperature. Given the cooling loads are \(150\:kW\) for both cooling compartments seperately, calculate,

\sphinxAtStartPar
a) the specific enthalpy of the refrigrant exiting the condenser

\sphinxAtStartPar
b) the specific enthalpy of the refrigrant exiting each of the evaporators

\sphinxAtStartPar
c) the flow\sphinxhyphen{}rate of the refrigrant through each evaporator

\sphinxAtStartPar
d) the specific enthalpy of the refrigrant entering the compressor

\sphinxAtStartPar
e) the pressure ratio of the compressor

\sphinxAtStartPar
f) the specific enthalpy of the refrigrant exiting the compressor

\sphinxAtStartPar
g) the work required in the compressor

\sphinxAtStartPar
h) the coefficient of performance (COP)

\sphinxAtStartPar
\sphinxincludegraphics{{CH5-Q8}.png}


\subsubsection{Solution Approach for a)}
\label{\detokenize{notebooks/Chapter5/CH5-Q8:solution-approach-for-a}}
\sphinxAtStartPar
the refrigrant exiting the condenser would be a sat liquid at \(40^{\circ} C\), therefore

\sphinxAtStartPar
\(h_3=h_{f@40^{\circ} C}\)

\begin{sphinxuseclass}{cell}\begin{sphinxVerbatimInput}

\begin{sphinxuseclass}{cell_input}
\begin{sphinxVerbatim}[commandchars=\\\{\}]
\PYG{c+c1}{\PYGZsh{} import the libraries we\PYGZsq{}ll need}
\PYG{k+kn}{import} \PYG{n+nn}{CoolProp}\PYG{n+nn}{.}\PYG{n+nn}{CoolProp} \PYG{k}{as} \PYG{n+nn}{CP}

\PYG{c+c1}{\PYGZsh{} define variables}
\PYG{n}{fluid} \PYG{o}{=} \PYG{l+s+s2}{\PYGZdq{}}\PYG{l+s+s2}{R134A}\PYG{l+s+s2}{\PYGZdq{}}  \PYG{c+c1}{\PYGZsh{} define the fluid or material of interest}
\PYG{n}{T\PYGZus{}3} \PYG{o}{=} \PYG{l+m+mi}{40} \PYG{o}{+} \PYG{l+m+mf}{273.15} \PYG{c+c1}{\PYGZsh{}stete \PYGZsh{}3 temperature in K}
\PYG{n}{h\PYGZus{}3} \PYG{o}{=} \PYG{n}{CP}\PYG{o}{.}\PYG{n}{PropsSI}\PYG{p}{(}\PYG{l+s+s2}{\PYGZdq{}}\PYG{l+s+s2}{H}\PYG{l+s+s2}{\PYGZdq{}}\PYG{p}{,} \PYG{l+s+s2}{\PYGZdq{}}\PYG{l+s+s2}{T}\PYG{l+s+s2}{\PYGZdq{}}\PYG{p}{,} \PYG{n}{T\PYGZus{}3}\PYG{p}{,} \PYG{l+s+s2}{\PYGZdq{}}\PYG{l+s+s2}{Q}\PYG{l+s+s2}{\PYGZdq{}}\PYG{p}{,} \PYG{l+m+mi}{0}\PYG{p}{,} \PYG{n}{fluid}\PYG{p}{)}\PYG{o}{/}\PYG{l+m+mi}{1000}  \PYG{c+c1}{\PYGZsh{} enthalpy of the refrigrant at state \PYGZsh{}3 in kJ/kg}
\PYG{n+nb}{print}\PYG{p}{(}\PYG{l+s+s1}{\PYGZsq{}}\PYG{l+s+s1}{The specific enthalpy of the refrigrant exiting the condenser is:}\PYG{l+s+s1}{\PYGZsq{}}\PYG{p}{,} \PYG{l+s+sa}{f}\PYG{l+s+s2}{\PYGZdq{}}\PYG{l+s+si}{\PYGZob{}}\PYG{n}{h\PYGZus{}3}\PYG{l+s+si}{:}\PYG{l+s+s2}{.1f}\PYG{l+s+si}{\PYGZcb{}}\PYG{l+s+s2}{\PYGZdq{}}\PYG{p}{,} \PYG{l+s+s1}{\PYGZsq{}}\PYG{l+s+s1}{kJ/kg}\PYG{l+s+s1}{\PYGZsq{}}\PYG{p}{)}
\end{sphinxVerbatim}

\end{sphinxuseclass}\end{sphinxVerbatimInput}
\begin{sphinxVerbatimOutput}

\begin{sphinxuseclass}{cell_output}
\begin{sphinxVerbatim}[commandchars=\\\{\}]
The specific enthalpy of the refrigrant exiting the condenser is: 256.4 kJ/kg
\end{sphinxVerbatim}

\end{sphinxuseclass}\end{sphinxVerbatimOutput}

\end{sphinxuseclass}

\subsubsection{Solution Approach for b)}
\label{\detokenize{notebooks/Chapter5/CH5-Q8:solution-approach-for-b}}
\sphinxAtStartPar
the low\sphinxhyphen{}temperature evaporator operates at \(-20^{\circ} C\) and the high\sphinxhyphen{}temperature one operates at \(0^{\circ} C\) and the refrigrant exiting those are at their saturated liquid state to optimize evaporators’ performance, therefore

\sphinxAtStartPar
\(h_6=h_{g@-20^{\circ} C}\)

\sphinxAtStartPar
\(h_9=h_{g@0^{\circ} C}\)

\begin{sphinxuseclass}{cell}\begin{sphinxVerbatimInput}

\begin{sphinxuseclass}{cell_input}
\begin{sphinxVerbatim}[commandchars=\\\{\}]
\PYG{n}{T\PYGZus{}6} \PYG{o}{=} \PYG{o}{\PYGZhy{}}\PYG{l+m+mi}{20} \PYG{o}{+} \PYG{l+m+mf}{273.15} \PYG{c+c1}{\PYGZsh{}stete \PYGZsh{}6 temperature in K}
\PYG{n}{T\PYGZus{}9} \PYG{o}{=} \PYG{l+m+mi}{0} \PYG{o}{+} \PYG{l+m+mf}{273.15} \PYG{c+c1}{\PYGZsh{}stete \PYGZsh{}9 temperature in K}

\PYG{n}{h\PYGZus{}6} \PYG{o}{=} \PYG{n}{CP}\PYG{o}{.}\PYG{n}{PropsSI}\PYG{p}{(}\PYG{l+s+s2}{\PYGZdq{}}\PYG{l+s+s2}{H}\PYG{l+s+s2}{\PYGZdq{}}\PYG{p}{,} \PYG{l+s+s2}{\PYGZdq{}}\PYG{l+s+s2}{T}\PYG{l+s+s2}{\PYGZdq{}}\PYG{p}{,} \PYG{n}{T\PYGZus{}6}\PYG{p}{,} \PYG{l+s+s2}{\PYGZdq{}}\PYG{l+s+s2}{Q}\PYG{l+s+s2}{\PYGZdq{}}\PYG{p}{,} \PYG{l+m+mi}{1}\PYG{p}{,} \PYG{n}{fluid}\PYG{p}{)}\PYG{o}{/}\PYG{l+m+mi}{1000}  \PYG{c+c1}{\PYGZsh{} enthalpy of the refrigrant at state \PYGZsh{}6 in kJ/kg}
\PYG{n}{h\PYGZus{}9} \PYG{o}{=} \PYG{n}{CP}\PYG{o}{.}\PYG{n}{PropsSI}\PYG{p}{(}\PYG{l+s+s2}{\PYGZdq{}}\PYG{l+s+s2}{H}\PYG{l+s+s2}{\PYGZdq{}}\PYG{p}{,} \PYG{l+s+s2}{\PYGZdq{}}\PYG{l+s+s2}{T}\PYG{l+s+s2}{\PYGZdq{}}\PYG{p}{,} \PYG{n}{T\PYGZus{}9}\PYG{p}{,} \PYG{l+s+s2}{\PYGZdq{}}\PYG{l+s+s2}{Q}\PYG{l+s+s2}{\PYGZdq{}}\PYG{p}{,} \PYG{l+m+mi}{1}\PYG{p}{,} \PYG{n}{fluid}\PYG{p}{)}\PYG{o}{/}\PYG{l+m+mi}{1000}  \PYG{c+c1}{\PYGZsh{} enthalpy of the refrigrant at state \PYGZsh{}9 in kJ/kg}

\PYG{n+nb}{print}\PYG{p}{(}\PYG{l+s+s1}{\PYGZsq{}}\PYG{l+s+s1}{The specific enthalpy of the refrigrant exiting the evaporator 1 is:}\PYG{l+s+s1}{\PYGZsq{}}\PYG{p}{,} \PYG{l+s+sa}{f}\PYG{l+s+s2}{\PYGZdq{}}\PYG{l+s+si}{\PYGZob{}}\PYG{n}{h\PYGZus{}6}\PYG{l+s+si}{:}\PYG{l+s+s2}{.1f}\PYG{l+s+si}{\PYGZcb{}}\PYG{l+s+s2}{\PYGZdq{}}\PYG{p}{,} \PYG{l+s+s1}{\PYGZsq{}}\PYG{l+s+s1}{kJ/kg}\PYG{l+s+s1}{\PYGZsq{}}\PYG{p}{)}
\PYG{n+nb}{print}\PYG{p}{(}\PYG{l+s+s1}{\PYGZsq{}}\PYG{l+s+s1}{The specific enthalpy of the refrigrant exiting the evaporator 2 is:}\PYG{l+s+s1}{\PYGZsq{}}\PYG{p}{,} \PYG{l+s+sa}{f}\PYG{l+s+s2}{\PYGZdq{}}\PYG{l+s+si}{\PYGZob{}}\PYG{n}{h\PYGZus{}9}\PYG{l+s+si}{:}\PYG{l+s+s2}{.1f}\PYG{l+s+si}{\PYGZcb{}}\PYG{l+s+s2}{\PYGZdq{}}\PYG{p}{,} \PYG{l+s+s1}{\PYGZsq{}}\PYG{l+s+s1}{kJ/kg}\PYG{l+s+s1}{\PYGZsq{}}\PYG{p}{)}
\end{sphinxVerbatim}

\end{sphinxuseclass}\end{sphinxVerbatimInput}
\begin{sphinxVerbatimOutput}

\begin{sphinxuseclass}{cell_output}
\begin{sphinxVerbatim}[commandchars=\\\{\}]
The specific enthalpy of the refrigrant exiting the evaporator 1 is: 386.6 kJ/kg
The specific enthalpy of the refrigrant exiting the evaporator 2 is: 398.6 kJ/kg
\end{sphinxVerbatim}

\end{sphinxuseclass}\end{sphinxVerbatimOutput}

\end{sphinxuseclass}

\subsubsection{Solution Approach for c)}
\label{\detokenize{notebooks/Chapter5/CH5-Q8:solution-approach-for-c}}
\sphinxAtStartPar
to calculate the flow\sphinxhyphen{}rate based on cooling load, the cooling capacities for the evaporators are to be calculated first. For Evaporator 1,

\sphinxAtStartPar
\(h_5=h_4\) assuming a constant enthalpy expansion valve

\sphinxAtStartPar
and

\sphinxAtStartPar
\(h_4=h_3\)

\sphinxAtStartPar
therefore,

\sphinxAtStartPar
\(q_{c1}=h_6-h_5\) the cooling capacity for evaporator 1

\sphinxAtStartPar
and

\sphinxAtStartPar
\(\dot m_{e1}=\dot Q_{c1}/q_{c1}\) the flow\sphinxhyphen{}rate for evaporator 1

\sphinxAtStartPar
similarly for evaporator 2,

\sphinxAtStartPar
\(h_8=h_7\) assuming a constant enthalpy expansion valve

\sphinxAtStartPar
and

\sphinxAtStartPar
\(h_7=h_3\)

\sphinxAtStartPar
therefore,

\sphinxAtStartPar
\(q_{c2}=h_9-h_8\) the cooling capacity for evaporator 1

\sphinxAtStartPar
and

\sphinxAtStartPar
\(\dot m_{e2}=\dot Q_{c2}/q_{c2}\) the flow\sphinxhyphen{}rate for evaporator 1

\begin{sphinxuseclass}{cell}\begin{sphinxVerbatimInput}

\begin{sphinxuseclass}{cell_input}
\begin{sphinxVerbatim}[commandchars=\\\{\}]
\PYG{c+c1}{\PYGZsh{}cooling loads }
\PYG{n}{Q\PYGZus{}c1} \PYG{o}{=} \PYG{l+m+mi}{100}   \PYG{c+c1}{\PYGZsh{}cooling load for evaporator 1 in kW}
\PYG{n}{Q\PYGZus{}c2} \PYG{o}{=} \PYG{l+m+mi}{100}   \PYG{c+c1}{\PYGZsh{}cooling load for evaporator 1 in kW}

\PYG{c+c1}{\PYGZsh{}for evaporator 1}
\PYG{n}{h\PYGZus{}4} \PYG{o}{=} \PYG{n}{h\PYGZus{}3}
\PYG{n}{h\PYGZus{}5} \PYG{o}{=} \PYG{n}{h\PYGZus{}4}
\PYG{n}{q\PYGZus{}c1} \PYG{o}{=} \PYG{n}{h\PYGZus{}6} \PYG{o}{\PYGZhy{}} \PYG{n}{h\PYGZus{}5}   \PYG{c+c1}{\PYGZsh{}cooling capacity for evaporator 1}
\PYG{n}{m\PYGZus{}e1} \PYG{o}{=} \PYG{n}{Q\PYGZus{}c1} \PYG{o}{/} \PYG{n}{q\PYGZus{}c1}   \PYG{c+c1}{\PYGZsh{}refrigrant flow\PYGZhy{}rate in evaporator 1 in kg/s}

\PYG{c+c1}{\PYGZsh{}for evaporator 2}
\PYG{n}{h\PYGZus{}7} \PYG{o}{=} \PYG{n}{h\PYGZus{}3}
\PYG{n}{h\PYGZus{}8} \PYG{o}{=} \PYG{n}{h\PYGZus{}7}
\PYG{n}{q\PYGZus{}c2} \PYG{o}{=} \PYG{n}{h\PYGZus{}9} \PYG{o}{\PYGZhy{}} \PYG{n}{h\PYGZus{}8}   \PYG{c+c1}{\PYGZsh{}cooling capacity for evaporator 2}
\PYG{n}{m\PYGZus{}e2} \PYG{o}{=} \PYG{n}{Q\PYGZus{}c2} \PYG{o}{/} \PYG{n}{q\PYGZus{}c2}   \PYG{c+c1}{\PYGZsh{}refrigrant flow\PYGZhy{}rate in evaporator 2 in kg/s}

\PYG{n+nb}{print}\PYG{p}{(}\PYG{l+s+s1}{\PYGZsq{}}\PYG{l+s+s1}{The refrigrant flow\PYGZhy{}rate in evaporator 1 is:}\PYG{l+s+s1}{\PYGZsq{}}\PYG{p}{,} \PYG{l+s+sa}{f}\PYG{l+s+s2}{\PYGZdq{}}\PYG{l+s+si}{\PYGZob{}}\PYG{n}{m\PYGZus{}e1}\PYG{l+s+si}{:}\PYG{l+s+s2}{.1f}\PYG{l+s+si}{\PYGZcb{}}\PYG{l+s+s2}{\PYGZdq{}}\PYG{p}{,} \PYG{l+s+s1}{\PYGZsq{}}\PYG{l+s+s1}{kg/s}\PYG{l+s+s1}{\PYGZsq{}}\PYG{p}{)}
\PYG{n+nb}{print}\PYG{p}{(}\PYG{l+s+s1}{\PYGZsq{}}\PYG{l+s+s1}{The refrigrant flow\PYGZhy{}rate in evaporator 2 is:}\PYG{l+s+s1}{\PYGZsq{}}\PYG{p}{,} \PYG{l+s+sa}{f}\PYG{l+s+s2}{\PYGZdq{}}\PYG{l+s+si}{\PYGZob{}}\PYG{n}{m\PYGZus{}e2}\PYG{l+s+si}{:}\PYG{l+s+s2}{.1f}\PYG{l+s+si}{\PYGZcb{}}\PYG{l+s+s2}{\PYGZdq{}}\PYG{p}{,} \PYG{l+s+s1}{\PYGZsq{}}\PYG{l+s+s1}{kg/s}\PYG{l+s+s1}{\PYGZsq{}}\PYG{p}{)}
\end{sphinxVerbatim}

\end{sphinxuseclass}\end{sphinxVerbatimInput}
\begin{sphinxVerbatimOutput}

\begin{sphinxuseclass}{cell_output}
\begin{sphinxVerbatim}[commandchars=\\\{\}]
The refrigrant flow\PYGZhy{}rate in evaporator 1 is: 0.8 kg/s
The refrigrant flow\PYGZhy{}rate in evaporator 2 is: 0.7 kg/s
\end{sphinxVerbatim}

\end{sphinxuseclass}\end{sphinxVerbatimOutput}

\end{sphinxuseclass}

\subsubsection{Solution Approach for d)}
\label{\detokenize{notebooks/Chapter5/CH5-Q8:solution-approach-for-d}}
\sphinxAtStartPar
applying the first law and conservation of mass to the mixing point where two outlets from evaporators come together,

\sphinxAtStartPar
\(\dot m_{10}+\dot m_6=\dot m_1\)

\sphinxAtStartPar
\(\dot m_{10}h_{10}+\dot m_6h_6=\dot m_1h_1\)

\sphinxAtStartPar
so

\sphinxAtStartPar
\((\dot m_{10}h_{10}+\dot m_6h_6)/\dot m_1=h_1\)

\sphinxAtStartPar
while

\sphinxAtStartPar
\(h_{10}=h_9\) assuming a constant enthalpy expansion

\sphinxAtStartPar
and

\sphinxAtStartPar
\(\dot m_{10}=\dot m_{e2}\)

\sphinxAtStartPar
\(\dot m_6=\dot m_{e1}\)

\begin{sphinxuseclass}{cell}\begin{sphinxVerbatimInput}

\begin{sphinxuseclass}{cell_input}
\begin{sphinxVerbatim}[commandchars=\\\{\}]
\PYG{c+c1}{\PYGZsh{}mass flow\PYGZhy{}rate}
\PYG{n}{m\PYGZus{}10} \PYG{o}{=} \PYG{n}{m\PYGZus{}e2}
\PYG{n}{m\PYGZus{}6} \PYG{o}{=} \PYG{n}{m\PYGZus{}e1}
\PYG{n}{m\PYGZus{}1} \PYG{o}{=} \PYG{n}{m\PYGZus{}10} \PYG{o}{+} \PYG{n}{m\PYGZus{}6}

\PYG{n}{h\PYGZus{}10}\PYG{o}{=}\PYG{n}{h\PYGZus{}9}

\PYG{n}{h\PYGZus{}1} \PYG{o}{=} \PYG{p}{(}\PYG{n}{m\PYGZus{}10} \PYG{o}{*} \PYG{n}{h\PYGZus{}10} \PYG{o}{+} \PYG{n}{m\PYGZus{}6} \PYG{o}{*} \PYG{n}{h\PYGZus{}6}\PYG{p}{)}\PYG{o}{/}\PYG{n}{m\PYGZus{}1}
\PYG{n+nb}{print}\PYG{p}{(}\PYG{l+s+s1}{\PYGZsq{}}\PYG{l+s+s1}{The specific enthalpy of the refrigrant entering the compressor is:}\PYG{l+s+s1}{\PYGZsq{}}\PYG{p}{,} \PYG{l+s+sa}{f}\PYG{l+s+s2}{\PYGZdq{}}\PYG{l+s+si}{\PYGZob{}}\PYG{n}{h\PYGZus{}1}\PYG{l+s+si}{:}\PYG{l+s+s2}{.1f}\PYG{l+s+si}{\PYGZcb{}}\PYG{l+s+s2}{\PYGZdq{}}\PYG{p}{,} \PYG{l+s+s1}{\PYGZsq{}}\PYG{l+s+s1}{kJ/kg}\PYG{l+s+s1}{\PYGZsq{}}\PYG{p}{)}
\end{sphinxVerbatim}

\end{sphinxuseclass}\end{sphinxVerbatimInput}
\begin{sphinxVerbatimOutput}

\begin{sphinxuseclass}{cell_output}
\begin{sphinxVerbatim}[commandchars=\\\{\}]
The specific enthalpy of the refrigrant entering the compressor is: 392.3 kJ/kg
\end{sphinxVerbatim}

\end{sphinxuseclass}\end{sphinxVerbatimOutput}

\end{sphinxuseclass}

\subsubsection{Solution Approach for e)}
\label{\detokenize{notebooks/Chapter5/CH5-Q8:solution-approach-for-e}}
\sphinxAtStartPar
For compressor inlet, the pressure would be same pressure as is in evaporator 1 which is the saturation pressure at the low temperature(\(-20^{\circ} C\))

\sphinxAtStartPar
\(P_1 = P_6 = P_{sat@-20^{\circ} C}\)

\sphinxAtStartPar
For compressor outlet, the pressure would be same as working pressure for the condenser which is the saturation pressure at \(40^{\circ} C\)

\sphinxAtStartPar
\(P_2 = P_3 = P_{sat@40^{\circ} C}\)

\sphinxAtStartPar
\(R(pressure\:ratio)=P_2/P_1\)

\begin{sphinxuseclass}{cell}\begin{sphinxVerbatimInput}

\begin{sphinxuseclass}{cell_input}
\begin{sphinxVerbatim}[commandchars=\\\{\}]
\PYG{n}{P\PYGZus{}6} \PYG{o}{=} \PYG{n}{CP}\PYG{o}{.}\PYG{n}{PropsSI}\PYG{p}{(}\PYG{l+s+s2}{\PYGZdq{}}\PYG{l+s+s2}{P}\PYG{l+s+s2}{\PYGZdq{}}\PYG{p}{,} \PYG{l+s+s2}{\PYGZdq{}}\PYG{l+s+s2}{T}\PYG{l+s+s2}{\PYGZdq{}}\PYG{p}{,} \PYG{n}{T\PYGZus{}6}\PYG{p}{,} \PYG{l+s+s2}{\PYGZdq{}}\PYG{l+s+s2}{Q}\PYG{l+s+s2}{\PYGZdq{}}\PYG{p}{,} \PYG{l+m+mi}{1}\PYG{p}{,} \PYG{n}{fluid}\PYG{p}{)}  \PYG{c+c1}{\PYGZsh{} sat pressure at state \PYGZsh{}6 in Pa}
\PYG{n}{P\PYGZus{}1} \PYG{o}{=} \PYG{n}{P\PYGZus{}6}

\PYG{n}{P\PYGZus{}3} \PYG{o}{=} \PYG{n}{CP}\PYG{o}{.}\PYG{n}{PropsSI}\PYG{p}{(}\PYG{l+s+s2}{\PYGZdq{}}\PYG{l+s+s2}{P}\PYG{l+s+s2}{\PYGZdq{}}\PYG{p}{,} \PYG{l+s+s2}{\PYGZdq{}}\PYG{l+s+s2}{T}\PYG{l+s+s2}{\PYGZdq{}}\PYG{p}{,} \PYG{n}{T\PYGZus{}3}\PYG{p}{,} \PYG{l+s+s2}{\PYGZdq{}}\PYG{l+s+s2}{Q}\PYG{l+s+s2}{\PYGZdq{}}\PYG{p}{,} \PYG{l+m+mi}{1}\PYG{p}{,} \PYG{n}{fluid}\PYG{p}{)}  \PYG{c+c1}{\PYGZsh{} sat pressure at state \PYGZsh{}3 in Pa}
\PYG{n}{P\PYGZus{}2} \PYG{o}{=} \PYG{n}{P\PYGZus{}3}

\PYG{n}{R} \PYG{o}{=} \PYG{n}{P\PYGZus{}2} \PYG{o}{/} \PYG{n}{P\PYGZus{}1}   \PYG{c+c1}{\PYGZsh{}the compressor pressure ratio}
\PYG{n+nb}{print}\PYG{p}{(}\PYG{l+s+s1}{\PYGZsq{}}\PYG{l+s+s1}{The pressure ratio of the compressor is:}\PYG{l+s+s1}{\PYGZsq{}}\PYG{p}{,} \PYG{l+s+sa}{f}\PYG{l+s+s2}{\PYGZdq{}}\PYG{l+s+si}{\PYGZob{}}\PYG{n}{R}\PYG{l+s+si}{:}\PYG{l+s+s2}{.1f}\PYG{l+s+si}{\PYGZcb{}}\PYG{l+s+s2}{\PYGZdq{}}\PYG{p}{)}
\end{sphinxVerbatim}

\end{sphinxuseclass}\end{sphinxVerbatimInput}
\begin{sphinxVerbatimOutput}

\begin{sphinxuseclass}{cell_output}
\begin{sphinxVerbatim}[commandchars=\\\{\}]
The pressure ratio of the compressor is: 7.7
\end{sphinxVerbatim}

\end{sphinxuseclass}\end{sphinxVerbatimOutput}

\end{sphinxuseclass}

\subsubsection{Solution Approach for f)}
\label{\detokenize{notebooks/Chapter5/CH5-Q8:solution-approach-for-f}}
\sphinxAtStartPar
The specific enthalpy of he refrigrant at the compressor outlet is evaluated based on temperature and pressure

\begin{sphinxuseclass}{cell}\begin{sphinxVerbatimInput}

\begin{sphinxuseclass}{cell_input}
\begin{sphinxVerbatim}[commandchars=\\\{\}]
\PYG{c+c1}{\PYGZsh{}define variables}
\PYG{n}{T\PYGZus{}2} \PYG{o}{=} \PYG{l+m+mi}{120} \PYG{o}{+} \PYG{l+m+mf}{273.15} \PYG{c+c1}{\PYGZsh{}temperature of R134a at compressor outlet}
\PYG{n}{h\PYGZus{}2} \PYG{o}{=} \PYG{n}{CP}\PYG{o}{.}\PYG{n}{PropsSI}\PYG{p}{(}\PYG{l+s+s2}{\PYGZdq{}}\PYG{l+s+s2}{H}\PYG{l+s+s2}{\PYGZdq{}}\PYG{p}{,} \PYG{l+s+s2}{\PYGZdq{}}\PYG{l+s+s2}{T}\PYG{l+s+s2}{\PYGZdq{}}\PYG{p}{,} \PYG{n}{T\PYGZus{}2}\PYG{p}{,} \PYG{l+s+s2}{\PYGZdq{}}\PYG{l+s+s2}{P}\PYG{l+s+s2}{\PYGZdq{}}\PYG{p}{,} \PYG{n}{P\PYGZus{}2}\PYG{p}{,} \PYG{n}{fluid}\PYG{p}{)}\PYG{o}{/}\PYG{l+m+mi}{1000}  \PYG{c+c1}{\PYGZsh{} refrigrant enthalpy at compressor outlet in kJ/kg}
\PYG{n+nb}{print}\PYG{p}{(}\PYG{l+s+s1}{\PYGZsq{}}\PYG{l+s+s1}{The specific enthalpy of the refrigrant exiting the compressor is:}\PYG{l+s+s1}{\PYGZsq{}}\PYG{p}{,} \PYG{l+s+sa}{f}\PYG{l+s+s2}{\PYGZdq{}}\PYG{l+s+si}{\PYGZob{}}\PYG{n}{h\PYGZus{}2}\PYG{l+s+si}{:}\PYG{l+s+s2}{.1f}\PYG{l+s+si}{\PYGZcb{}}\PYG{l+s+s2}{\PYGZdq{}}\PYG{p}{,} \PYG{l+s+s1}{\PYGZsq{}}\PYG{l+s+s1}{kJ/kg}\PYG{l+s+s1}{\PYGZsq{}}\PYG{p}{)}
\end{sphinxVerbatim}

\end{sphinxuseclass}\end{sphinxVerbatimInput}
\begin{sphinxVerbatimOutput}

\begin{sphinxuseclass}{cell_output}
\begin{sphinxVerbatim}[commandchars=\\\{\}]
The specific enthalpy of the refrigrant exiting the compressor is: 504.1 kJ/kg
\end{sphinxVerbatim}

\end{sphinxuseclass}\end{sphinxVerbatimOutput}

\end{sphinxuseclass}

\subsubsection{Solution Approach for g)}
\label{\detokenize{notebooks/Chapter5/CH5-Q8:solution-approach-for-g}}
\sphinxAtStartPar
Based on the first law,

\sphinxAtStartPar
\(\dot W= \dot m\:(h_2-h_1)\)e

\begin{sphinxuseclass}{cell}\begin{sphinxVerbatimInput}

\begin{sphinxuseclass}{cell_input}
\begin{sphinxVerbatim}[commandchars=\\\{\}]
\PYG{n}{W} \PYG{o}{=} \PYG{n}{m\PYGZus{}1} \PYG{o}{*} \PYG{p}{(}\PYG{n}{h\PYGZus{}2} \PYG{o}{\PYGZhy{}} \PYG{n}{h\PYGZus{}1}\PYG{p}{)}   \PYG{c+c1}{\PYGZsh{}work input for compressor in kW}
\PYG{n+nb}{print}\PYG{p}{(}\PYG{l+s+s1}{\PYGZsq{}}\PYG{l+s+s1}{The compressor work input is:}\PYG{l+s+s1}{\PYGZsq{}}\PYG{p}{,} \PYG{l+s+sa}{f}\PYG{l+s+s2}{\PYGZdq{}}\PYG{l+s+si}{\PYGZob{}}\PYG{n}{W}\PYG{l+s+si}{:}\PYG{l+s+s2}{.1f}\PYG{l+s+si}{\PYGZcb{}}\PYG{l+s+s2}{\PYGZdq{}}\PYG{p}{,}\PYG{l+s+s1}{\PYGZsq{}}\PYG{l+s+s1}{kW}\PYG{l+s+s1}{\PYGZsq{}}\PYG{p}{)}
\end{sphinxVerbatim}

\end{sphinxuseclass}\end{sphinxVerbatimInput}
\begin{sphinxVerbatimOutput}

\begin{sphinxuseclass}{cell_output}
\begin{sphinxVerbatim}[commandchars=\\\{\}]
The compressor work input is: 164.4 kW
\end{sphinxVerbatim}

\end{sphinxuseclass}\end{sphinxVerbatimOutput}

\end{sphinxuseclass}

\subsubsection{Solution Approach for h)}
\label{\detokenize{notebooks/Chapter5/CH5-Q8:solution-approach-for-h}}
\sphinxAtStartPar
\(COP = Q_c(cooling\:load) / W(compressor\:work\:input)\)

\begin{sphinxuseclass}{cell}\begin{sphinxVerbatimInput}

\begin{sphinxuseclass}{cell_input}
\begin{sphinxVerbatim}[commandchars=\\\{\}]
\PYG{n}{cop} \PYG{o}{=} \PYG{p}{(}\PYG{n}{Q\PYGZus{}c1} \PYG{o}{+} \PYG{n}{Q\PYGZus{}c2}\PYG{p}{)} \PYG{o}{/} \PYG{n}{W}
\PYG{n+nb}{print}\PYG{p}{(}\PYG{l+s+s1}{\PYGZsq{}}\PYG{l+s+s1}{The COP of the cycle is:}\PYG{l+s+s1}{\PYGZsq{}}\PYG{p}{,} \PYG{l+s+sa}{f}\PYG{l+s+s2}{\PYGZdq{}}\PYG{l+s+si}{\PYGZob{}}\PYG{n}{cop}\PYG{l+s+si}{:}\PYG{l+s+s2}{.1f}\PYG{l+s+si}{\PYGZcb{}}\PYG{l+s+s2}{\PYGZdq{}}\PYG{p}{)}
\end{sphinxVerbatim}

\end{sphinxuseclass}\end{sphinxVerbatimInput}
\begin{sphinxVerbatimOutput}

\begin{sphinxuseclass}{cell_output}
\begin{sphinxVerbatim}[commandchars=\\\{\}]
The COP of the cycle is: 1.2
\end{sphinxVerbatim}

\end{sphinxuseclass}\end{sphinxVerbatimOutput}

\end{sphinxuseclass}
\sphinxstepscope


\section{Chapter 5}
\label{\detokenize{notebooks/Chapter5/CH5-Q9:chapter-5}}\label{\detokenize{notebooks/Chapter5/CH5-Q9::doc}}

\subsection{Question \#9}
\label{\detokenize{notebooks/Chapter5/CH5-Q9:question-9}}
\sphinxAtStartPar
\(5\:g/s\) of hydrogen at \(-20^{\circ} C\) and \(5\:bar\) enters a nozzle with inlet and outlet diameters of \(40\:mm\) and \(15\:mm\). Given the outlet pressure of nozzle to be \(2\:bar\), calculate

\sphinxAtStartPar
a) inlet velocity of hydrogen

\sphinxAtStartPar
b) outlet temperature and velocity of hydrogen assuming ideal gas application

\sphinxAtStartPar
c) outlet temperature and velocity of hydrogen using CoolProp module using the answers from b as the first guess

\sphinxAtStartPar
\sphinxincludegraphics{{CH5-Q9}.png}


\subsubsection{Solution Approach for a)}
\label{\detokenize{notebooks/Chapter5/CH5-Q9:solution-approach-for-a}}
\sphinxAtStartPar
from mass flow\sphinxhyphen{}rate correlation

\sphinxAtStartPar
\(\dot m=\rho_1 A_1V_1\)

\sphinxAtStartPar
so

\sphinxAtStartPar
\(V_1=\dot m/(\rho_1A_1)\)

\begin{sphinxuseclass}{cell}\begin{sphinxVerbatimInput}

\begin{sphinxuseclass}{cell_input}
\begin{sphinxVerbatim}[commandchars=\\\{\}]
\PYG{c+c1}{\PYGZsh{} import the libraries we\PYGZsq{}ll need}
\PYG{k+kn}{import} \PYG{n+nn}{CoolProp}\PYG{n+nn}{.}\PYG{n+nn}{CoolProp} \PYG{k}{as} \PYG{n+nn}{CP}
\PYG{k+kn}{import} \PYG{n+nn}{numpy} \PYG{k}{as} \PYG{n+nn}{np}

\PYG{c+c1}{\PYGZsh{} define variables}
\PYG{n}{fluid} \PYG{o}{=} \PYG{l+s+s2}{\PYGZdq{}}\PYG{l+s+s2}{hydrogen}\PYG{l+s+s2}{\PYGZdq{}}  \PYG{c+c1}{\PYGZsh{} define the fluid or material of interest}
\PYG{n}{R} \PYG{o}{=} \PYG{l+m+mi}{4124}   \PYG{c+c1}{\PYGZsh{}fluid gas constant in J/kg.K}
\PYG{n}{C\PYGZus{}p} \PYG{o}{=} \PYG{l+m+mi}{14307}   \PYG{c+c1}{\PYGZsh{}fluid Cp in J/kg.K}
\PYG{n}{T\PYGZus{}1} \PYG{o}{=} \PYG{o}{\PYGZhy{}}\PYG{l+m+mi}{20} \PYG{o}{+} \PYG{l+m+mf}{273.15}   \PYG{c+c1}{\PYGZsh{}inlet temperature in K}
\PYG{n}{P\PYGZus{}1} \PYG{o}{=} \PYG{l+m+mf}{5e+5}   \PYG{c+c1}{\PYGZsh{}inlet pressure in Pa}
\PYG{n}{P\PYGZus{}2} \PYG{o}{=} \PYG{l+m+mf}{2e+5}   \PYG{c+c1}{\PYGZsh{}outlet pressure in Pa}
\PYG{n}{m} \PYG{o}{=} \PYG{l+m+mf}{0.005}   \PYG{c+c1}{\PYGZsh{}fluid mass flow rate in kg/s}
\PYG{n}{D\PYGZus{}1} \PYG{o}{=} \PYG{l+m+mf}{0.04}   \PYG{c+c1}{\PYGZsh{}inlet diameter in m}
\PYG{n}{A\PYGZus{}1} \PYG{o}{=} \PYG{n}{np}\PYG{o}{.}\PYG{n}{pi} \PYG{o}{*} \PYG{n}{D\PYGZus{}1} \PYG{o}{*}\PYG{o}{*}\PYG{l+m+mi}{2} \PYG{o}{/}\PYG{l+m+mi}{4}   \PYG{c+c1}{\PYGZsh{}inlet area in m2}
\PYG{n}{rho\PYGZus{}1} \PYG{o}{=} \PYG{n}{CP}\PYG{o}{.}\PYG{n}{PropsSI}\PYG{p}{(}\PYG{l+s+s2}{\PYGZdq{}}\PYG{l+s+s2}{D}\PYG{l+s+s2}{\PYGZdq{}}\PYG{p}{,} \PYG{l+s+s2}{\PYGZdq{}}\PYG{l+s+s2}{T}\PYG{l+s+s2}{\PYGZdq{}}\PYG{p}{,} \PYG{n}{T\PYGZus{}1}\PYG{p}{,} \PYG{l+s+s2}{\PYGZdq{}}\PYG{l+s+s2}{P}\PYG{l+s+s2}{\PYGZdq{}}\PYG{p}{,} \PYG{n}{P\PYGZus{}1}\PYG{p}{,} \PYG{n}{fluid}\PYG{p}{)}   \PYG{c+c1}{\PYGZsh{}density of fluid at inlet in kg/m3}

\PYG{n}{V\PYGZus{}1} \PYG{o}{=} \PYG{n}{m} \PYG{o}{/} \PYG{p}{(}\PYG{n}{rho\PYGZus{}1} \PYG{o}{*} \PYG{n}{A\PYGZus{}1}\PYG{p}{)}   \PYG{c+c1}{\PYGZsh{}velocity of fluid at inlet}

\PYG{n+nb}{print}\PYG{p}{(}\PYG{l+s+s1}{\PYGZsq{}}\PYG{l+s+s1}{The velocity of the fluid at inlet is:}\PYG{l+s+s1}{\PYGZsq{}}\PYG{p}{,} \PYG{l+s+sa}{f}\PYG{l+s+s2}{\PYGZdq{}}\PYG{l+s+si}{\PYGZob{}}\PYG{n}{V\PYGZus{}1}\PYG{l+s+si}{:}\PYG{l+s+s2}{.1f}\PYG{l+s+si}{\PYGZcb{}}\PYG{l+s+s2}{\PYGZdq{}}\PYG{p}{,} \PYG{l+s+s1}{\PYGZsq{}}\PYG{l+s+s1}{m/s}\PYG{l+s+s1}{\PYGZsq{}}\PYG{p}{)}
\end{sphinxVerbatim}

\end{sphinxuseclass}\end{sphinxVerbatimInput}
\begin{sphinxVerbatimOutput}

\begin{sphinxuseclass}{cell_output}
\begin{sphinxVerbatim}[commandchars=\\\{\}]
The velocity of the fluid at inlet is: 8.3 m/s
\end{sphinxVerbatim}

\end{sphinxuseclass}\end{sphinxVerbatimOutput}

\end{sphinxuseclass}

\subsubsection{Solution Approach for b)}
\label{\detokenize{notebooks/Chapter5/CH5-Q9:solution-approach-for-b}}
\sphinxAtStartPar
from mass conservation

\sphinxAtStartPar
\(\dot m=\rho_2 A_2V_2\)

\sphinxAtStartPar
so

\sphinxAtStartPar
\(V_2=\dot m/(\rho_2A_2)\)

\sphinxAtStartPar
while from ideal gas assumption

\sphinxAtStartPar
\(P=\rho RT\)

\sphinxAtStartPar
therefore

\sphinxAtStartPar
\(\rho_2=P_2/(RT_2)\)

\sphinxAtStartPar
so

\sphinxAtStartPar
\(V_2=\dot m RT_2/(P_2A_2)=\alpha T_2\)

\sphinxAtStartPar
while

\sphinxAtStartPar
\(\alpha=\dot m R/(P_2A_2)\)

\sphinxAtStartPar
from energy conservation

\sphinxAtStartPar
\(h_1+1/2V_1^2=h_2+1/2V_2^2\)

\sphinxAtStartPar
and from ideal gas assumption

\sphinxAtStartPar
\(\Delta h=C_p(T_2-T_1)\)

\sphinxAtStartPar
therefore

\sphinxAtStartPar
\(C_pT_1+1/2V_1^2=C_pT_2+1/2V_2^2\)

\sphinxAtStartPar
substituting \(V_2\)

\sphinxAtStartPar
\(C_pT_1+1/2V_1^2=C_pT_2+1/2\alpha^2T_2^2\)

\sphinxAtStartPar
organizing for \(T_2\)

\sphinxAtStartPar
\((1/2\alpha^2)T_2^2+C_pT_2-(C_pT_1+1/2V_1^2)=0\)

\begin{sphinxuseclass}{cell}\begin{sphinxVerbatimInput}

\begin{sphinxuseclass}{cell_input}
\begin{sphinxVerbatim}[commandchars=\\\{\}]
\PYG{c+c1}{\PYGZsh{} define variables}
\PYG{n}{D\PYGZus{}2} \PYG{o}{=} \PYG{l+m+mf}{0.015}   \PYG{c+c1}{\PYGZsh{}outlet diameter in m}
\PYG{n}{A\PYGZus{}2} \PYG{o}{=} \PYG{n}{np}\PYG{o}{.}\PYG{n}{pi} \PYG{o}{*} \PYG{n}{D\PYGZus{}2} \PYG{o}{*}\PYG{o}{*} \PYG{l+m+mi}{2} \PYG{o}{/}\PYG{l+m+mi}{4}   \PYG{c+c1}{\PYGZsh{}outlet area in m2}
\PYG{n}{alpha} \PYG{o}{=} \PYG{n}{m} \PYG{o}{*} \PYG{n}{R} \PYG{o}{/} \PYG{p}{(}\PYG{n}{P\PYGZus{}2} \PYG{o}{*} \PYG{n}{A\PYGZus{}2}\PYG{p}{)}

\PYG{c+c1}{\PYGZsh{} Coefficients of the quadratic equation ax\PYGZca{}2 + bx + c = 0}
\PYG{n}{a} \PYG{o}{=} \PYG{l+m+mf}{0.5} \PYG{o}{*} \PYG{n}{alpha} \PYG{o}{*}\PYG{o}{*}\PYG{l+m+mi}{2}
\PYG{n}{b} \PYG{o}{=} \PYG{n}{C\PYGZus{}p}
\PYG{n}{c} \PYG{o}{=} \PYG{o}{\PYGZhy{}}\PYG{l+m+mi}{1} \PYG{o}{*} \PYG{p}{(}\PYG{n}{C\PYGZus{}p} \PYG{o}{*} \PYG{n}{T\PYGZus{}1} \PYG{o}{+} \PYG{l+m+mf}{0.5} \PYG{o}{*} \PYG{n}{V\PYGZus{}1} \PYG{o}{*}\PYG{o}{*} \PYG{l+m+mi}{2}\PYG{p}{)}

\PYG{c+c1}{\PYGZsh{} Calculate the discriminant (the value inside the square root)}
\PYG{n}{discriminant} \PYG{o}{=} \PYG{n}{b}\PYG{o}{*}\PYG{o}{*}\PYG{l+m+mi}{2} \PYG{o}{\PYGZhy{}} \PYG{l+m+mi}{4}\PYG{o}{*}\PYG{n}{a}\PYG{o}{*}\PYG{n}{c}
\PYG{c+c1}{\PYGZsh{} Two real solutions}
\PYG{n}{x1} \PYG{o}{=} \PYG{p}{(}\PYG{o}{\PYGZhy{}}\PYG{l+m+mi}{1}\PYG{o}{*}\PYG{n}{b} \PYG{o}{+} \PYG{n}{np}\PYG{o}{.}\PYG{n}{sqrt}\PYG{p}{(}\PYG{n}{discriminant}\PYG{p}{)}\PYG{p}{)} \PYG{o}{/} \PYG{p}{(}\PYG{l+m+mi}{2}\PYG{o}{*}\PYG{n}{a}\PYG{p}{)}
\PYG{n}{x2} \PYG{o}{=} \PYG{p}{(}\PYG{o}{\PYGZhy{}}\PYG{l+m+mi}{1}\PYG{o}{*}\PYG{n}{b} \PYG{o}{\PYGZhy{}} \PYG{n}{np}\PYG{o}{.}\PYG{n}{sqrt}\PYG{p}{(}\PYG{n}{discriminant}\PYG{p}{)}\PYG{p}{)} \PYG{o}{/} \PYG{p}{(}\PYG{l+m+mi}{2}\PYG{o}{*}\PYG{n}{a}\PYG{p}{)}

\PYG{c+c1}{\PYGZsh{} to pick the correct positive value for temperature at outlet}
\PYG{k}{if} \PYG{n}{x1} \PYG{o}{\PYGZgt{}} \PYG{l+m+mi}{0}\PYG{p}{:}
   \PYG{n}{T\PYGZus{}2} \PYG{o}{=} \PYG{n}{x1}
\PYG{k}{else}\PYG{p}{:}
   \PYG{n}{T\PYGZus{}2} \PYG{o}{=} \PYG{n}{x2}

\PYG{n}{V\PYGZus{}2} \PYG{o}{=} \PYG{n}{alpha} \PYG{o}{*} \PYG{n}{T\PYGZus{}2}   \PYG{c+c1}{\PYGZsh{} velocity at outlet}
\PYG{n}{T\PYGZus{}2C} \PYG{o}{=} \PYG{n}{T\PYGZus{}2} \PYG{o}{\PYGZhy{}} \PYG{l+m+mf}{273.15}   \PYG{c+c1}{\PYGZsh{}temperature at outlet in celcius}
\PYG{n+nb}{print}\PYG{p}{(}\PYG{l+s+s1}{\PYGZsq{}}\PYG{l+s+s1}{The velocity of the fluid at outlet is:}\PYG{l+s+s1}{\PYGZsq{}}\PYG{p}{,} \PYG{l+s+sa}{f}\PYG{l+s+s2}{\PYGZdq{}}\PYG{l+s+si}{\PYGZob{}}\PYG{n}{V\PYGZus{}2}\PYG{l+s+si}{:}\PYG{l+s+s2}{.2f}\PYG{l+s+si}{\PYGZcb{}}\PYG{l+s+s2}{\PYGZdq{}}\PYG{p}{,} \PYG{l+s+s1}{\PYGZsq{}}\PYG{l+s+s1}{m/s}\PYG{l+s+s1}{\PYGZsq{}}\PYG{p}{)}
\PYG{n+nb}{print}\PYG{p}{(}\PYG{l+s+s1}{\PYGZsq{}}\PYG{l+s+s1}{The temperature of the fluid at outlet is:}\PYG{l+s+s1}{\PYGZsq{}}\PYG{p}{,} \PYG{l+s+sa}{f}\PYG{l+s+s2}{\PYGZdq{}}\PYG{l+s+si}{\PYGZob{}}\PYG{n}{T\PYGZus{}2C}\PYG{l+s+si}{:}\PYG{l+s+s2}{.2f}\PYG{l+s+si}{\PYGZcb{}}\PYG{l+s+s2}{\PYGZdq{}}\PYG{p}{,} \PYG{l+s+s1}{\PYGZsq{}}\PYG{l+s+s1}{celcius}\PYG{l+s+s1}{\PYGZsq{}}\PYG{p}{)}
\end{sphinxVerbatim}

\end{sphinxuseclass}\end{sphinxVerbatimInput}
\begin{sphinxVerbatimOutput}

\begin{sphinxuseclass}{cell_output}
\begin{sphinxVerbatim}[commandchars=\\\{\}]
The velocity of the fluid at outlet is: 147.25 m/s
The temperature of the fluid at outlet is: \PYGZhy{}20.76 celcius
\end{sphinxVerbatim}

\end{sphinxuseclass}\end{sphinxVerbatimOutput}

\end{sphinxuseclass}

\subsubsection{Solution Approach for b)}
\label{\detokenize{notebooks/Chapter5/CH5-Q9:id1}}
\sphinxAtStartPar
again from energy conservarion

\sphinxAtStartPar
\(h_1+1/2V_1^2=h_2+1/2V_2^2\)

\sphinxAtStartPar
while \(h_2\) is obtained from coolprop based on pressure and temperature and

\sphinxAtStartPar
\(V_2=\dot m/(\rho_2A_2)\)

\sphinxAtStartPar
while \(\rho_2\) is obtained from coolprop likewise, therefore

\sphinxAtStartPar
\(h_2+(\dot m^2/(2A_2^2))(1/\rho_2^2)=h_1+1/2V_1^2\)

\sphinxAtStartPar
setting \(\beta\) as

\sphinxAtStartPar
\(\beta=h_1+1/2V_1^2\)

\sphinxAtStartPar
the difference between \(\beta\) and the left hand side of the previous equation calculated by CoolProp would be the error of calculations used along with a tunning parameter \((z)\) in a tial\sphinxhyphen{}error method.

\sphinxAtStartPar
now, a trial and error method is to be used to obtain temperature (and veocity) using coolprop and the previous equation

\begin{sphinxuseclass}{cell}\begin{sphinxVerbatimInput}

\begin{sphinxuseclass}{cell_input}
\begin{sphinxVerbatim}[commandchars=\\\{\}]
\PYG{n}{h\PYGZus{}1} \PYG{o}{=} \PYG{n}{CP}\PYG{o}{.}\PYG{n}{PropsSI}\PYG{p}{(}\PYG{l+s+s2}{\PYGZdq{}}\PYG{l+s+s2}{H}\PYG{l+s+s2}{\PYGZdq{}}\PYG{p}{,} \PYG{l+s+s2}{\PYGZdq{}}\PYG{l+s+s2}{T}\PYG{l+s+s2}{\PYGZdq{}}\PYG{p}{,} \PYG{n}{T\PYGZus{}1}\PYG{p}{,} \PYG{l+s+s2}{\PYGZdq{}}\PYG{l+s+s2}{P}\PYG{l+s+s2}{\PYGZdq{}}\PYG{p}{,} \PYG{n}{P\PYGZus{}1}\PYG{p}{,} \PYG{n}{fluid}\PYG{p}{)}   \PYG{c+c1}{\PYGZsh{}enthalpy of fluid at inlet in J/kg}
\PYG{n}{betha} \PYG{o}{=} \PYG{n}{h\PYGZus{}1} \PYG{o}{+} \PYG{l+m+mf}{0.5} \PYG{o}{*} \PYG{n}{V\PYGZus{}1} \PYG{o}{*}\PYG{o}{*}\PYG{l+m+mi}{2}

\PYG{c+c1}{\PYGZsh{}obtaining enthalpy and density based on the guess value}
\PYG{n}{h\PYGZus{}2} \PYG{o}{=} \PYG{n}{CP}\PYG{o}{.}\PYG{n}{PropsSI}\PYG{p}{(}\PYG{l+s+s2}{\PYGZdq{}}\PYG{l+s+s2}{H}\PYG{l+s+s2}{\PYGZdq{}}\PYG{p}{,} \PYG{l+s+s2}{\PYGZdq{}}\PYG{l+s+s2}{T}\PYG{l+s+s2}{\PYGZdq{}}\PYG{p}{,} \PYG{n}{T\PYGZus{}2}\PYG{p}{,} \PYG{l+s+s2}{\PYGZdq{}}\PYG{l+s+s2}{P}\PYG{l+s+s2}{\PYGZdq{}}\PYG{p}{,} \PYG{n}{P\PYGZus{}2}\PYG{p}{,} \PYG{n}{fluid}\PYG{p}{)}   \PYG{c+c1}{\PYGZsh{}guess enthalpy of fluid at outlet in J/kg}
\PYG{n}{rho\PYGZus{}2} \PYG{o}{=} \PYG{n}{CP}\PYG{o}{.}\PYG{n}{PropsSI}\PYG{p}{(}\PYG{l+s+s2}{\PYGZdq{}}\PYG{l+s+s2}{D}\PYG{l+s+s2}{\PYGZdq{}}\PYG{p}{,} \PYG{l+s+s2}{\PYGZdq{}}\PYG{l+s+s2}{T}\PYG{l+s+s2}{\PYGZdq{}}\PYG{p}{,} \PYG{n}{T\PYGZus{}2}\PYG{p}{,} \PYG{l+s+s2}{\PYGZdq{}}\PYG{l+s+s2}{P}\PYG{l+s+s2}{\PYGZdq{}}\PYG{p}{,} \PYG{n}{P\PYGZus{}2}\PYG{p}{,} \PYG{n}{fluid}\PYG{p}{)}   \PYG{c+c1}{\PYGZsh{}guess density of fluid at outlet in kg/m3}
\PYG{n}{e} \PYG{o}{=} \PYG{n}{betha} \PYG{o}{\PYGZhy{}} \PYG{n}{h\PYGZus{}2} \PYG{o}{\PYGZhy{}} \PYG{l+m+mf}{0.5} \PYG{o}{*} \PYG{p}{(}\PYG{n}{m} \PYG{o}{/} \PYG{p}{(}\PYG{n}{rho\PYGZus{}2} \PYG{o}{*} \PYG{n}{A\PYGZus{}2}\PYG{p}{)}\PYG{p}{)} \PYG{o}{*}\PYG{o}{*} \PYG{l+m+mi}{2}   \PYG{c+c1}{\PYGZsh{}error based on the guess}

\PYG{c+c1}{\PYGZsh{}now starting a loop to implement the trila and error method}
\PYG{n}{i} \PYG{o}{=} \PYG{l+m+mi}{1}
\PYG{n}{z} \PYG{o}{=} \PYG{l+m+mf}{0.0001}    \PYG{c+c1}{\PYGZsh{}tunning parameter to tune calculation error and apply to the new temperature guess}

\PYG{k}{while} \PYG{n}{np}\PYG{o}{.}\PYG{n}{absolute}\PYG{p}{(}\PYG{n}{e}\PYG{p}{)} \PYG{o}{\PYGZgt{}} \PYG{l+m+mi}{1} \PYG{p}{:}
    \PYG{n}{T\PYGZus{}2} \PYG{o}{=} \PYG{n}{T\PYGZus{}2} \PYG{o}{+} \PYG{n}{e} \PYG{o}{*} \PYG{n}{z}    \PYG{c+c1}{\PYGZsh{}new temperature guess based on the error (e) and the tunning parameter (z) }
    \PYG{n}{h\PYGZus{}2} \PYG{o}{=} \PYG{n}{CP}\PYG{o}{.}\PYG{n}{PropsSI}\PYG{p}{(}\PYG{l+s+s2}{\PYGZdq{}}\PYG{l+s+s2}{H}\PYG{l+s+s2}{\PYGZdq{}}\PYG{p}{,} \PYG{l+s+s2}{\PYGZdq{}}\PYG{l+s+s2}{T}\PYG{l+s+s2}{\PYGZdq{}}\PYG{p}{,} \PYG{n}{T\PYGZus{}2}\PYG{p}{,} \PYG{l+s+s2}{\PYGZdq{}}\PYG{l+s+s2}{P}\PYG{l+s+s2}{\PYGZdq{}}\PYG{p}{,} \PYG{n}{P\PYGZus{}2}\PYG{p}{,} \PYG{n}{fluid}\PYG{p}{)}   \PYG{c+c1}{\PYGZsh{}new guess enthalpy of fluid at outlet in J/kg}
    \PYG{n}{rho\PYGZus{}2} \PYG{o}{=} \PYG{n}{CP}\PYG{o}{.}\PYG{n}{PropsSI}\PYG{p}{(}\PYG{l+s+s2}{\PYGZdq{}}\PYG{l+s+s2}{D}\PYG{l+s+s2}{\PYGZdq{}}\PYG{p}{,} \PYG{l+s+s2}{\PYGZdq{}}\PYG{l+s+s2}{T}\PYG{l+s+s2}{\PYGZdq{}}\PYG{p}{,} \PYG{n}{T\PYGZus{}2}\PYG{p}{,} \PYG{l+s+s2}{\PYGZdq{}}\PYG{l+s+s2}{P}\PYG{l+s+s2}{\PYGZdq{}}\PYG{p}{,} \PYG{n}{P\PYGZus{}2}\PYG{p}{,} \PYG{n}{fluid}\PYG{p}{)}   \PYG{c+c1}{\PYGZsh{}new guess density of fluid at outlet in kg/m3}
    \PYG{n}{e} \PYG{o}{=} \PYG{n}{betha} \PYG{o}{\PYGZhy{}} \PYG{n}{h\PYGZus{}2} \PYG{o}{\PYGZhy{}} \PYG{l+m+mf}{0.5} \PYG{o}{*} \PYG{p}{(}\PYG{n}{m} \PYG{o}{/} \PYG{p}{(}\PYG{n}{rho\PYGZus{}2} \PYG{o}{*} \PYG{n}{A\PYGZus{}2}\PYG{p}{)}\PYG{p}{)} \PYG{o}{*}\PYG{o}{*} \PYG{l+m+mi}{2}   \PYG{c+c1}{\PYGZsh{}new error based on the new guess}
    \PYG{n}{i} \PYG{o}{+}\PYG{o}{=} \PYG{l+m+mi}{1}
    \PYG{k}{if} \PYG{n}{i} \PYG{o}{==} \PYG{l+m+mi}{1000}\PYG{p}{:}
        \PYG{k}{break}

\PYG{c+c1}{\PYGZsh{} note the difference between two consecutive temperature guesses is e * z therefore an e \PYGZlt{} 1 means a diffrence lower than 0.0001 K}

\PYG{n}{V\PYGZus{}2} \PYG{o}{=} \PYG{n}{m} \PYG{o}{/} \PYG{p}{(}\PYG{n}{rho\PYGZus{}2} \PYG{o}{*} \PYG{n}{A\PYGZus{}2}\PYG{p}{)}   \PYG{c+c1}{\PYGZsh{}velocity calculated based on temperature }
\PYG{n}{T\PYGZus{}2C} \PYG{o}{=} \PYG{n}{T\PYGZus{}2} \PYG{o}{\PYGZhy{}} \PYG{l+m+mf}{273.15}   \PYG{c+c1}{\PYGZsh{}temperature at outlet in celcius}

\PYG{n+nb}{print}\PYG{p}{(}\PYG{l+s+s1}{\PYGZsq{}}\PYG{l+s+s1}{The velocity of the fluid at outlet based on CoolProp is:}\PYG{l+s+s1}{\PYGZsq{}}\PYG{p}{,} \PYG{l+s+sa}{f}\PYG{l+s+s2}{\PYGZdq{}}\PYG{l+s+si}{\PYGZob{}}\PYG{n}{V\PYGZus{}2}\PYG{l+s+si}{:}\PYG{l+s+s2}{.2f}\PYG{l+s+si}{\PYGZcb{}}\PYG{l+s+s2}{\PYGZdq{}}\PYG{p}{,} \PYG{l+s+s1}{\PYGZsq{}}\PYG{l+s+s1}{m/s}\PYG{l+s+s1}{\PYGZsq{}}\PYG{p}{)}
\PYG{n+nb}{print}\PYG{p}{(}\PYG{l+s+s1}{\PYGZsq{}}\PYG{l+s+s1}{The temperature of the fluid at outlet based on CoolProp is:}\PYG{l+s+s1}{\PYGZsq{}}\PYG{p}{,} \PYG{l+s+sa}{f}\PYG{l+s+s2}{\PYGZdq{}}\PYG{l+s+si}{\PYGZob{}}\PYG{n}{T\PYGZus{}2C}\PYG{l+s+si}{:}\PYG{l+s+s2}{.2f}\PYG{l+s+si}{\PYGZcb{}}\PYG{l+s+s2}{\PYGZdq{}}\PYG{p}{,} \PYG{l+s+s1}{\PYGZsq{}}\PYG{l+s+s1}{celcius}\PYG{l+s+s1}{\PYGZsq{}}\PYG{p}{)}
\end{sphinxVerbatim}

\end{sphinxuseclass}\end{sphinxVerbatimInput}
\begin{sphinxVerbatimOutput}

\begin{sphinxuseclass}{cell_output}
\begin{sphinxVerbatim}[commandchars=\\\{\}]
The velocity of the fluid at outlet based on CoolProp is: 147.49 m/s
The temperature of the fluid at outlet based on CoolProp is: \PYGZhy{}20.71 celcius
\end{sphinxVerbatim}

\end{sphinxuseclass}\end{sphinxVerbatimOutput}

\end{sphinxuseclass}
\sphinxstepscope


\chapter{6. Entropy and the Second Law of Thermodynamics}
\label{\detokenize{notebooks/Chapter6/chapter6:entropy-and-the-second-law-of-thermodynamics}}\label{\detokenize{notebooks/Chapter6/chapter6::doc}}
\sphinxAtStartPar
Let’s do some problems based on concepts on \sphinxhref{https://pressbooks.bccampus.ca/thermo1/chapter/6-0-chapter-introduction-and-learning-objectives/}{Chapter 6} from the e\sphinxhyphen{}textbook.

\sphinxstepscope

\begin{sphinxuseclass}{cell}\begin{sphinxVerbatimInput}

\begin{sphinxuseclass}{cell_input}
\begin{sphinxVerbatim}[commandchars=\\\{\}]
\PYG{k+kn}{import} \PYG{n+nn}{numpy} \PYG{k}{as} \PYG{n+nn}{np}
\end{sphinxVerbatim}

\end{sphinxuseclass}\end{sphinxVerbatimInput}

\end{sphinxuseclass}






\renewcommand{\indexname}{Index}
\printindex
\end{document}